\documentclass[a4paper]{article}

\begin{document}

Menemukan solusi $x$ dari persamaan $f(x) = 0$, yang pada umumnya non-linear.
Solusi $x$ ini juga sering disebut sebagai akar dari persamaan $f(x) = 0$.

\section{Metode bisection}

Merupakan salah satu metode yang paling sederhana untuk mencari akar
dari suatu persamaan non-linear.

Metode ini memerlukan memerlukan dua tebakan awal $x_{1}$ dan $x_{2}$
dengan syarat:
\begin{equation}
f(x_1)f(x_2) < 0.
\end{equation}
Hal ini berarti bahwa $f(x_1)$ dan $f(x_2)$ memiliki tanda yang berbeda, satu positif
dan yang lain negatif.



\section{Metode regula-falsi}



\end{document}