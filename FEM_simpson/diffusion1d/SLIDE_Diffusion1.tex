\documentclass[english,10pt,aspectratio=169,fleqn]{beamer}

\usepackage{amsmath} % load this before unicode-math
\usepackage{amssymb}
\usepackage{mathabx}
%\usepackage{unicode-math}

\usepackage{fontspec}
\setmonofont{Julia Mono}
%\setmathfont{STIXMath}
%\setmathfont{TeX Gyre Termes Math}

\usefonttheme[onlymath]{serif}

\setlength{\parskip}{\smallskipamount}
\setlength{\parindent}{0pt}

%\setbeamersize{text margin left=5pt, text margin right=5pt}

\usepackage{amsmath}
\usepackage{amssymb}
\usepackage{braket}

\usepackage{minted}
\newminted{julia}{breaklines,fontsize=\scriptsize,texcomments=true}
\newminted{python}{breaklines,fontsize=\scriptsize,texcomments=true}
\newminted{bash}{breaklines,fontsize=\scriptsize,texcomments=true}
\newminted{text}{breaklines,fontsize=\scriptsize,texcomments=true}

\newcommand{\txtinline}[1]{\mintinline[fontsize=\scriptsize]{text}{#1}}
\newcommand{\jlinline}[1]{\mintinline[fontsize=\scriptsize]{julia}{#1}}

\definecolor{mintedbg}{rgb}{0.95,0.95,0.95}
\usepackage{mdframed}

%\BeforeBeginEnvironment{minted}{\begin{mdframed}[backgroundcolor=mintedbg]}
%\AfterEndEnvironment{minted}{\end{mdframed}}

\setcounter{secnumdepth}{3}
\setcounter{tocdepth}{3}

\makeatletter

 \newcommand\makebeamertitle{\frame{\maketitle}}%
 % (ERT) argument for the TOC
 \AtBeginDocument{%
   \let\origtableofcontents=\tableofcontents
   \def\tableofcontents{\@ifnextchar[{\origtableofcontents}{\gobbletableofcontents}}
   \def\gobbletableofcontents#1{\origtableofcontents}
 }

\makeatother

\usepackage{babel}

\newcommand{\highlighteq}[1]{\colorbox{blue!25}{$\displaystyle#1$}}
\newcommand{\highlight}[1]{\colorbox{red!25}{#1}}

\begin{document}

\title{Introduction to Finite Element Method}
\subtitle{TF40XX}
\author{Fadjar Fathurrahman}
\institute{
Program Studi Teknik Fisika\\
Institut Teknologi Bandung
}
\date{}


\frame{\titlepage}


\begin{frame} % ---------------------------------------------------------------
\frametitle{1d Diffusion}

Governing PDE
\begin{equation*}
\frac{\partial T}{\partial t} = \kappa \frac{\partial^2 T}{\partial x^2} + H
\end{equation*}

Initial conditions:
\begin{equation*}
T(x,t=0) = 0 \,\, \forall x \in [0,L_{x}]
\end{equation*}

Boundary conditions:
\begin{equation*}
T(x=0,t) = 0 \,\,\text{and} T(x=L_{x},t) = 0
\end{equation*}

\end{frame} % -----------------------------------------------------------------



\begin{frame} % ---------------------------------------------------------------
\frametitle{Discretized equations}

\begin{equation*}
\mathbf{L} \mathbf{T}^{n+1} = \mathbf{R} \mathbf{T}^{n} + \mathbf{F}
\end{equation*}

\begin{equation*}
\mathbf{L} = \frac{\mathbf{M}}{\Delta t} + \mathbf{K}
\end{equation*}

\end{frame}


\begin{frame} % ----------------------------------------
\title{Mass matrix}

\begin{equation*}
\mathbf{M} = \begin{bmatrix}
\dfrac{\Delta x}{3} & \dfrac{\Delta x}{6} \\
\dfrac{\Delta x}{6} & \dfrac{\Delta x}{3}
\end{bmatrix}
\end{equation*}

\begin{equation*}
\mathbf{R} = \frac{\mathbf{M}}{\Delta t}
\end{equation*}

\end{frame}


\begin{frame}
\frametitle{Stiffness matrix}
\begin{equation*}
\mathbf{K} = \kappa\begin{bmatrix}
\dfrac{1}{\Delta x} & -\dfrac{1}{\Delta x} \\[0.4cm]
-\dfrac{1}{\Delta x} & \dfrac{1}{\Delta x}
\end{bmatrix}
\end{equation*}

\end{frame}


\begin{frame}
\frametitle{Load vector}

\begin{equation*}
\mathbf{F} = H\begin{bmatrix}
\dfrac{\Delta x}{2} \\
\dfrac{\Delta x}{2}
\end{bmatrix}
\end{equation*}

\end{frame}


\begin{frame}
$$
\mathbf{T} = \begin{bmatrix}
T_1 \\ T_2
\end{bmatrix}
$$
\end{frame}

\begin{frame} % ---------------------------------------------------------------
Sistem global (menggunakan matriks dan vektor global):
$$
\tilde{\mathbf{L}} \tilde{\mathbf{T}}^{n+1} = \tilde{\mathbf{R}}\ 
\tilde{\mathbf{T}}^{n} + \tilde{\mathbf{F}} = \tilde{\mathbf{b}}
$$
\end{frame} % -----------------------------------------------------------------


%\begin{frame} % ---------------------------------------------------------------
%\frametitle{Title of the slide}
%\end{frame} % -----------------------------------------------------------------

\end{document}

