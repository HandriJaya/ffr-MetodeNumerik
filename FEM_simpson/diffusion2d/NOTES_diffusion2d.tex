%\RequirePackage{luatex85} % tuftebook not yet compatible with recent luatex

\documentclass[a4paper,english]{tufte-handout}

\usepackage{fontspec}
\setmonofont{JuliaMono-Regular}

\usepackage{hyperref}
\usepackage{url}
\usepackage{xcolor}

\usepackage{amsmath}
\usepackage{amssymb}

\usepackage{graphicx}
\usepackage{float}

\usepackage{minted}
\usepackage{enumitem}

\newminted{julia}{breaklines,fontsize=\footnotesize}
\newminted{python}{breaklines,fontsize=\footnotesize}

\newminted{bash}{breaklines,fontsize=\footnotesize}
\newminted{text}{breaklines,fontsize=\footnotesize}

\newcommand{\txtinline}[1]{\mintinline[breaklines,fontsize=\footnotesize]{text}{#1}}
\newcommand{\jlinline}[1]{\mintinline[breaklines,fontsize=\footnotesize]{julia}{#1}}
\newcommand{\pyinline}[1]{\mintinline[breaklines,fontsize=\footnotesize]{python}{#1}}

\newmintedfile[juliafile]{julia}{breaklines,fontsize=\footnotesize}
\newmintedfile[pythonfile]{python}{breaklines,fontsize=\footnotesize}
\newmintedfile[fortranfile]{fortran}{breaklines,fontsize=\footnotesize}
% f-o-otnotesize

\usepackage{mdframed}
\usepackage{setspace}
\onehalfspacing

\usepackage{mhchem}
\usepackage{appendix}

\newcommand{\highlighteq}[1]{\colorbox{blue!25}{$\displaystyle#1$}}
\newcommand{\highlight}[1]{\colorbox{red!25}{#1}}

\newcounter{soal}%[section]
\newenvironment{soal}[1][]{\refstepcounter{soal}\par\medskip
   \noindent {\color{blue}\textbf{Soal~\thesoal. #1}} \sffamily}{\medskip}


\definecolor{mintedbg}{rgb}{0.95,0.95,0.95}
\BeforeBeginEnvironment{minted}{
  \begin{mdframed}[ %backgroundcolor=mintedbg,%
    topline=false,bottomline=false,%
    leftline=false,rightline=false]
}
\AfterEndEnvironment{minted}{\end{mdframed}}


%\BeforeBeginEnvironment{soal}{
%  \begin{mdframed}[%
%    topline=true,bottomline=true,%
%    leftline=true,rightline=true]
%}
%\AfterEndEnvironment{soal}{\end{mdframed}}


\setcounter{secnumdepth}{1}

% Noindent for paragraph
% https://tex.stackexchange.com/questions/77999/remove-indent-of-paragraph-and-add-line-skip-with-tufte-latex

\makeatletter
% Paragraph indentation and separation for normal text
\renewcommand{\@tufte@reset@par}{%
  \setlength{\RaggedRightParindent}{0pt}%
  \setlength{\JustifyingParindent}{0pt}%
  \setlength{\parindent}{0pt}%
  \setlength{\parskip}{1pt}%
}
\@tufte@reset@par

% Paragraph indentation and separation for marginal text
\renewcommand{\@tufte@margin@par}{%
  \setlength{\RaggedRightParindent}{0pt}%
  \setlength{\JustifyingParindent}{0pt}%
  \setlength{\parindent}{0pt}%
  \setlength{\parskip}{1pt}%
}
\makeatother

% You can copy those commands to the preamble of your document and fill in the values that you prefer (e.g., 0pt for the indents and \baselineskip for the \parskip).
\documentclass[a4paper,bahasa]{article}
\usepackage{babel}

\usepackage[a4paper]{geometry}
\geometry{verbose,tmargin=2.0cm,bmargin=2.0cm,lmargin=3.0cm,rmargin=3.0cm}

\setlength{\parskip}{\smallskipamount}
\setlength{\parindent}{0pt}

\usepackage{fontspec}
\setmonofont{JuliaMono-Regular}

\usepackage{hyperref}
\usepackage{url}
\usepackage{xcolor}

\usepackage{amsmath}
\usepackage{amssymb}

\usepackage{graphicx}
\usepackage{float}

\usepackage{minted}
\usepackage{enumitem}

\newminted{julia}{breaklines,fontsize=\footnotesize}
\newminted{python}{breaklines,fontsize=\footnotesize}

\newminted{bash}{breaklines,fontsize=\footnotesize}
\newminted{text}{breaklines,fontsize=\footnotesize}

\newcommand{\txtinline}[1]{\mintinline[breaklines,fontsize=\footnotesize]{text}{#1}}
\newcommand{\jlinline}[1]{\mintinline[breaklines,fontsize=\footnotesize]{julia}{#1}}
\newcommand{\pyinline}[1]{\mintinline[breaklines,fontsize=\footnotesize]{python}{#1}}

\newmintedfile[juliafile]{julia}{breaklines,fontsize=\footnotesize}
\newmintedfile[pythonfile]{python}{breaklines,fontsize=\footnotesize}
\newmintedfile[fortranfile]{fortran}{breaklines,fontsize=\footnotesize}
% f-o-otnotesize

\usepackage{mdframed}
\usepackage{setspace}
\onehalfspacing

\usepackage{mhchem}
\usepackage{appendix}

\newcommand{\highlighteq}[1]{\colorbox{blue!25}{$\displaystyle#1$}}
\newcommand{\highlight}[1]{\colorbox{red!25}{#1}}

\newcounter{soal}%[section]
\newenvironment{soal}[1][]{\refstepcounter{soal}\par\medskip
   \noindent {\color{blue}\textbf{Soal~\thesoal. #1}} \sffamily}{\medskip}


\definecolor{mintedbg}{rgb}{0.95,0.95,0.95}
\BeforeBeginEnvironment{minted}{
  \begin{mdframed}[ %backgroundcolor=mintedbg,%
    topline=false,bottomline=false,%
    leftline=false,rightline=false]
}
\AfterEndEnvironment{minted}{\end{mdframed}}


%\BeforeBeginEnvironment{soal}{
%  \begin{mdframed}[%
%    topline=true,bottomline=true,%
%    leftline=true,rightline=true]
%}
%\AfterEndEnvironment{soal}{\end{mdframed}}



\begin{document}

\title{Pengenalan Metode Elemen Hingga: Difusi Transien 2d}
\author{Fadjar Fathurrahman}
\date{}
\maketitle

\begin{equation*}
\rho c \frac{\partial T}{\partial t} = -\nabla \cdot \mathbf{q} + A = 
\begin{bmatrix}\frac{\partial}{\partial x} \\
\frac{\partial}{\partial y} \end{bmatrix} \cdot
\begin{bmatrix} q_x \\ q_y \end{bmatrix}
\end{equation*}

Hubungan konstitutif (Hukum Fourier):
\begin{equation*}
\mathbf{q} = \begin{bmatrix}
q_x \\ q_y \end{bmatrix} = 
-\begin{bmatrix} k & 0 \\ 0 & k \end{bmatrix}
\begin{bmatrix} \frac{\partial}{\partial x} \\ \frac{\partial}{\partial y} \end{bmatrix} T =
-\mathbf{K} \nabla T
\end{equation*}

diperoleh:
\begin{equation*}
\frac{\partial T}{\partial t} = \nabla \cdot (\mathbf{K} \nabla T) + \frac{A}{\rho c}
\end{equation*}

$\mathbf{K}$: tensor difusivitas termal, dengan entri diagonal $k_x/\rho c$ dan
$k_y/\rho c$, dan nol pada entri lainnya.

Jika difusivitas termal adalah konstan, persamaan ini dapat ditulis menjadi:
\begin{equation*}
\frac{\partial T}{\partial t} = \kappa \left(
\frac{\partial^2 T}{\partial x^2} + \frac{\partial^2 T}{\partial y^2}
\right) + H
\end{equation*}

Syarat awal:
\begin{equation*}
T(x,y,t=0) = 1 \,\, \forall x \in [0,L_x], \forall y \in [0,L_y]
\end{equation*}

Kondisi batas Dirichlet:
\begin{align*}
T(x=0,y,t)   & = 0 \\
T(x=L_x,y,t) & = 0 \\
T(x,y=0,t)   & = 0 \\
T(x,y=L_y,t) & = 0 \\
\end{align*}

Diskritisasi

Fungsi basis:
\begin{equation*}
\mathbf{N}^{\mathsf{T}} = \begin{bmatrix}
N_1 & N_2 & N_3 & N_4
\end{bmatrix}
\end{equation*}

\begin{equation*}
T \approx \begin{bmatrix}
N_1 & N_2 & N_3 & N_4
\end{bmatrix}
\begin{bmatrix}
T_1 \\ T_2 \\ T_3 \\ T_4
\end{bmatrix} = 
\mathbf{N}^{\mathsf{T}} \mathbf{T}
\end{equation*}


Substitusi, untuk menghitung residual:
\begin{equation*}
R =
\frac{\partial}{\partial t} \mathbf{N}^{\mathsf{T}} \mathbf{T} -
\nabla \cdot (\mathbf{K} \nabla \mathbf{N}^{\mathsf{T}} \mathbf{T}) - H
\end{equation*}


\begin{equation*}
\nabla \mathbf{N}^{\mathsf{T}} = 
\begin{bmatrix}
\frac{\partial}{\partial x} \\ \frac{\partial}{\partial y}
\end{bmatrix}
\begin{bmatrix}
N_1 & N_2 & N_3 & N_4
\end{bmatrix} = 
\begin{bmatrix}
\frac{\partial N_1}{\partial x} & \frac{\partial N_2}{\partial x} &
\frac{\partial N_3}{\partial x} & \frac{\partial N_4}{\partial x} \\
\frac{\partial N_1}{\partial y} & \frac{\partial N_2}{\partial y} &
\frac{\partial N_3}{\partial y} & \frac{\partial N_4}{\partial y} \\
\end{bmatrix}
\end{equation*}


Galerkin:
\begin{equation*}
\int \mathbf{N} \frac{\partial}{\partial t} \mathbf{N}^{\mathsf{T}}
\mathbf{T} \, \mathrm{d}x \, \mathrm{d}y - 
\int \mathbf{N} \nabla \cdot
(\mathbf{K} \nabla \mathbf{N}^{\mathsf{T}} \mathbf{T}
\, \mathrm{d}x \, \mathrm{d}y = 
\int H \mathbf{N} \, \mathrm{d}x \, \mathrm{d}y
\end{equation*}

Menggunakan integral parsial dan mengabaikan integral batas (boundary integral):
\begin{equation*}
\int \mathbf{N} \mathbf{N}^{\mathsf{T}} \, \mathrm{d}x \, \mathrm{d}y
\frac{\partial}{\partial t} \mathbf{T} +
\int ( \nabla \mathbf{N}^{\mathsf{T}} )^{\mathsf{T}} \mathbf{K}
( \nabla \mathbf{N}^{\mathsf{T}}) \, \mathrm{d}x \, \mathrm{d}y \mathbf{T}
\end{equation*}

\end{document}
