\section{Chapra Contoh 3.2}
Fungsi eksponensial dapat dihitung dengan menggunakan deret sebagai berikut:
\begin{equation}
e^{x} = 1 + x + \frac{x^2}{2} + \frac{x^3}{3!} + \cdots + \frac{x^n}{n!}
\label{eq:exp-deret}
\end{equation}
Kita ingin menggunakan Persamaan \eqref{eq:exp-deret} untuk menghitung estimasi
dari $e^{0.5}$.
Dengan menggunakan kriteria dari Scarborough:
\begin{equation*}
\epsilon_{s} = (0.5 \times 10^{2-3}) \% = 0.05 \%
\end{equation*}
Kita akan menambahkan suku-suku pada Persamaan \eqref{eq:exp-deret} sampai $\epsilon_{a}$
lebih kecil dari $\epsilon_{s}$.

Program Python berikut ini dapat digunakan untuk melakukan perhitungan yang ada pada buku teks.
\begin{pythoncode}
from math import factorial, exp

def approx_exp(x, N):
  assert(N >= 0)
  if N == 0:
    return 1
  s = 0.0
  for i in range(N+1):
    s = s + x**i/factorial(i)
  return s
  
x = 0.5
true_val = exp(x) # from math module
  
n_digit = 3
# Equation 3.7
ε_s_percent = 0.5*10**(2-n_digit)
  
prev_approx = 0.0
for N in range(50):
  approx_val = approx_exp(x, N)
  ε_t_percent = abs(approx_val - true_val)/true_val * 100
  if N > 0:
    ε_a_percent = abs(approx_val - prev_approx)/approx_val * 100
  else:
    ε_a_percent = float('nan')
  prev_approx = approx_val
  print("%3d %18.10f %10.5f%% %10.5f%%" % (N+1, approx_val, ε_t_percent, ε_a_percent))
  if ε_a_percent < ε_s_percent:
    print("Converged within %d significant digits" % n_digit)
    break

print("true_val   is %18.10f" % true_val)
print("approx_val is %18.10f" % approx_val)
\end{pythoncode}

Catatan: Pada program di atas, \pyinline{for}-loop digunakan dengan jumlah iterasi yang relatif
besar. Anda dapat menggunakan \pyinline{while}-loop sebagai gantinya.

Contoh output:
\begin{textcode}
  1       1.0000000000   39.34693%        nan%
  2       1.5000000000    9.02040%   33.33333%
  3       1.6250000000    1.43877%    7.69231%
  4       1.6458333333    0.17516%    1.26582%
  5       1.6484375000    0.01721%    0.15798%
  6       1.6486979167    0.00142%    0.01580%
Converged within 3 significant digits
true_val   is       1.6487212707
approx_val is       1.6486979167
\end{textcode}


\begin{soal}
Ulangi perhitungan ini untuk jumlah digit signifikan yang berbeda, misalnya 5, 8, dan 10
digit signifikan. Silakan melakukan modifikasi terhadap program yang diberikan.
\end{soal}