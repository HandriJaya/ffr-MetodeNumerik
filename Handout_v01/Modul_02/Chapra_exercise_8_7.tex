\begin{soal}[Chapra Latihan 8.7]
Persamaan keadan Redlich-Kwong dinyatakan sebagai:
\begin{equation*}
p = \frac{RT}{v - b} - \frac{a}{v(v + b)\sqrt{T}}
\end{equation*}
di mana $R$ = konstanta gas universal = 0.518 kJ/(kg K),
$T$ adalah temperature absolut (K), $p$ adalah tekanan absolut (kPa),
dan $v$ = volume spesifik gas ($\mathrm{m}^3/\mathrm{kg}$).
Parameter $a$ dan $b$ dihitung dengan persamaan:
\begin{align*}
a & = -0.427 \frac{R^{2} T_{c}^{2.5}}{p_{c}} \\
b & = 0.0866 R\frac{T_c}{p_c}
\end{align*}
di mana $p_c$ dan $T_c$ menyatakan tekanan dan temperatur kritis.
Tentukan jumlah methana (yaitu $v$) ($p_c$ = 4600 kPa and $T_c$ = 191 K)
yang dapat disimpan dalam tangki dengan volumne 3 $\mathrm{m}^{3}$
pada temperatur -40$^{\circ}\,\mathrm{C}$ dengan tekanan 65000 MPa.
\end{soal}
