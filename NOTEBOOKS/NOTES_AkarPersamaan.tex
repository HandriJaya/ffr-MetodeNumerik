
% Default to the notebook output style

    


% Inherit from the specified cell style.




    
\documentclass[11pt]{article}

    
    
    \usepackage[T1]{fontenc}
    % Nicer default font (+ math font) than Computer Modern for most use cases
    \usepackage{mathpazo}

    % Basic figure setup, for now with no caption control since it's done
    % automatically by Pandoc (which extracts ![](path) syntax from Markdown).
    \usepackage{graphicx}
    % We will generate all images so they have a width \maxwidth. This means
    % that they will get their normal width if they fit onto the page, but
    % are scaled down if they would overflow the margins.
    \makeatletter
    \def\maxwidth{\ifdim\Gin@nat@width>\linewidth\linewidth
    \else\Gin@nat@width\fi}
    \makeatother
    \let\Oldincludegraphics\includegraphics
    % Set max figure width to be 80% of text width, for now hardcoded.
    \renewcommand{\includegraphics}[1]{\Oldincludegraphics[width=.8\maxwidth]{#1}}
    % Ensure that by default, figures have no caption (until we provide a
    % proper Figure object with a Caption API and a way to capture that
    % in the conversion process - todo).
    \usepackage{caption}
    \DeclareCaptionLabelFormat{nolabel}{}
    \captionsetup{labelformat=nolabel}

    \usepackage{adjustbox} % Used to constrain images to a maximum size 
    \usepackage{xcolor} % Allow colors to be defined
    \usepackage{enumerate} % Needed for markdown enumerations to work
    \usepackage{geometry} % Used to adjust the document margins
    \usepackage{amsmath} % Equations
    \usepackage{amssymb} % Equations
    \usepackage{textcomp} % defines textquotesingle
    % Hack from http://tex.stackexchange.com/a/47451/13684:
    \AtBeginDocument{%
        \def\PYZsq{\textquotesingle}% Upright quotes in Pygmentized code
    }
    \usepackage{upquote} % Upright quotes for verbatim code
    \usepackage{eurosym} % defines \euro
    \usepackage[mathletters]{ucs} % Extended unicode (utf-8) support
    \usepackage[utf8x]{inputenc} % Allow utf-8 characters in the tex document
    \usepackage{fancyvrb} % verbatim replacement that allows latex
    \usepackage{grffile} % extends the file name processing of package graphics 
                         % to support a larger range 
    % The hyperref package gives us a pdf with properly built
    % internal navigation ('pdf bookmarks' for the table of contents,
    % internal cross-reference links, web links for URLs, etc.)
    \usepackage{hyperref}
    \usepackage{longtable} % longtable support required by pandoc >1.10
    \usepackage{booktabs}  % table support for pandoc > 1.12.2
    \usepackage[inline]{enumitem} % IRkernel/repr support (it uses the enumerate* environment)
    \usepackage[normalem]{ulem} % ulem is needed to support strikethroughs (\sout)
                                % normalem makes italics be italics, not underlines
    

    
    
    % Colors for the hyperref package
    \definecolor{urlcolor}{rgb}{0,.145,.698}
    \definecolor{linkcolor}{rgb}{.71,0.21,0.01}
    \definecolor{citecolor}{rgb}{.12,.54,.11}

    % ANSI colors
    \definecolor{ansi-black}{HTML}{3E424D}
    \definecolor{ansi-black-intense}{HTML}{282C36}
    \definecolor{ansi-red}{HTML}{E75C58}
    \definecolor{ansi-red-intense}{HTML}{B22B31}
    \definecolor{ansi-green}{HTML}{00A250}
    \definecolor{ansi-green-intense}{HTML}{007427}
    \definecolor{ansi-yellow}{HTML}{DDB62B}
    \definecolor{ansi-yellow-intense}{HTML}{B27D12}
    \definecolor{ansi-blue}{HTML}{208FFB}
    \definecolor{ansi-blue-intense}{HTML}{0065CA}
    \definecolor{ansi-magenta}{HTML}{D160C4}
    \definecolor{ansi-magenta-intense}{HTML}{A03196}
    \definecolor{ansi-cyan}{HTML}{60C6C8}
    \definecolor{ansi-cyan-intense}{HTML}{258F8F}
    \definecolor{ansi-white}{HTML}{C5C1B4}
    \definecolor{ansi-white-intense}{HTML}{A1A6B2}

    % commands and environments needed by pandoc snippets
    % extracted from the output of `pandoc -s`
    \providecommand{\tightlist}{%
      \setlength{\itemsep}{0pt}\setlength{\parskip}{0pt}}
    \DefineVerbatimEnvironment{Highlighting}{Verbatim}{commandchars=\\\{\}}
    % Add ',fontsize=\small' for more characters per line
    \newenvironment{Shaded}{}{}
    \newcommand{\KeywordTok}[1]{\textcolor[rgb]{0.00,0.44,0.13}{\textbf{{#1}}}}
    \newcommand{\DataTypeTok}[1]{\textcolor[rgb]{0.56,0.13,0.00}{{#1}}}
    \newcommand{\DecValTok}[1]{\textcolor[rgb]{0.25,0.63,0.44}{{#1}}}
    \newcommand{\BaseNTok}[1]{\textcolor[rgb]{0.25,0.63,0.44}{{#1}}}
    \newcommand{\FloatTok}[1]{\textcolor[rgb]{0.25,0.63,0.44}{{#1}}}
    \newcommand{\CharTok}[1]{\textcolor[rgb]{0.25,0.44,0.63}{{#1}}}
    \newcommand{\StringTok}[1]{\textcolor[rgb]{0.25,0.44,0.63}{{#1}}}
    \newcommand{\CommentTok}[1]{\textcolor[rgb]{0.38,0.63,0.69}{\textit{{#1}}}}
    \newcommand{\OtherTok}[1]{\textcolor[rgb]{0.00,0.44,0.13}{{#1}}}
    \newcommand{\AlertTok}[1]{\textcolor[rgb]{1.00,0.00,0.00}{\textbf{{#1}}}}
    \newcommand{\FunctionTok}[1]{\textcolor[rgb]{0.02,0.16,0.49}{{#1}}}
    \newcommand{\RegionMarkerTok}[1]{{#1}}
    \newcommand{\ErrorTok}[1]{\textcolor[rgb]{1.00,0.00,0.00}{\textbf{{#1}}}}
    \newcommand{\NormalTok}[1]{{#1}}
    
    % Additional commands for more recent versions of Pandoc
    \newcommand{\ConstantTok}[1]{\textcolor[rgb]{0.53,0.00,0.00}{{#1}}}
    \newcommand{\SpecialCharTok}[1]{\textcolor[rgb]{0.25,0.44,0.63}{{#1}}}
    \newcommand{\VerbatimStringTok}[1]{\textcolor[rgb]{0.25,0.44,0.63}{{#1}}}
    \newcommand{\SpecialStringTok}[1]{\textcolor[rgb]{0.73,0.40,0.53}{{#1}}}
    \newcommand{\ImportTok}[1]{{#1}}
    \newcommand{\DocumentationTok}[1]{\textcolor[rgb]{0.73,0.13,0.13}{\textit{{#1}}}}
    \newcommand{\AnnotationTok}[1]{\textcolor[rgb]{0.38,0.63,0.69}{\textbf{\textit{{#1}}}}}
    \newcommand{\CommentVarTok}[1]{\textcolor[rgb]{0.38,0.63,0.69}{\textbf{\textit{{#1}}}}}
    \newcommand{\VariableTok}[1]{\textcolor[rgb]{0.10,0.09,0.49}{{#1}}}
    \newcommand{\ControlFlowTok}[1]{\textcolor[rgb]{0.00,0.44,0.13}{\textbf{{#1}}}}
    \newcommand{\OperatorTok}[1]{\textcolor[rgb]{0.40,0.40,0.40}{{#1}}}
    \newcommand{\BuiltInTok}[1]{{#1}}
    \newcommand{\ExtensionTok}[1]{{#1}}
    \newcommand{\PreprocessorTok}[1]{\textcolor[rgb]{0.74,0.48,0.00}{{#1}}}
    \newcommand{\AttributeTok}[1]{\textcolor[rgb]{0.49,0.56,0.16}{{#1}}}
    \newcommand{\InformationTok}[1]{\textcolor[rgb]{0.38,0.63,0.69}{\textbf{\textit{{#1}}}}}
    \newcommand{\WarningTok}[1]{\textcolor[rgb]{0.38,0.63,0.69}{\textbf{\textit{{#1}}}}}
    
    
    % Define a nice break command that doesn't care if a line doesn't already
    % exist.
    \def\br{\hspace*{\fill} \\* }
    % Math Jax compatability definitions
    \def\gt{>}
    \def\lt{<}
    % Document parameters
    \title{NOTES\_AkarPersamaan}
    
    
    

    % Pygments definitions
    
\makeatletter
\def\PY@reset{\let\PY@it=\relax \let\PY@bf=\relax%
    \let\PY@ul=\relax \let\PY@tc=\relax%
    \let\PY@bc=\relax \let\PY@ff=\relax}
\def\PY@tok#1{\csname PY@tok@#1\endcsname}
\def\PY@toks#1+{\ifx\relax#1\empty\else%
    \PY@tok{#1}\expandafter\PY@toks\fi}
\def\PY@do#1{\PY@bc{\PY@tc{\PY@ul{%
    \PY@it{\PY@bf{\PY@ff{#1}}}}}}}
\def\PY#1#2{\PY@reset\PY@toks#1+\relax+\PY@do{#2}}

\expandafter\def\csname PY@tok@w\endcsname{\def\PY@tc##1{\textcolor[rgb]{0.73,0.73,0.73}{##1}}}
\expandafter\def\csname PY@tok@c\endcsname{\let\PY@it=\textit\def\PY@tc##1{\textcolor[rgb]{0.25,0.50,0.50}{##1}}}
\expandafter\def\csname PY@tok@cp\endcsname{\def\PY@tc##1{\textcolor[rgb]{0.74,0.48,0.00}{##1}}}
\expandafter\def\csname PY@tok@k\endcsname{\let\PY@bf=\textbf\def\PY@tc##1{\textcolor[rgb]{0.00,0.50,0.00}{##1}}}
\expandafter\def\csname PY@tok@kp\endcsname{\def\PY@tc##1{\textcolor[rgb]{0.00,0.50,0.00}{##1}}}
\expandafter\def\csname PY@tok@kt\endcsname{\def\PY@tc##1{\textcolor[rgb]{0.69,0.00,0.25}{##1}}}
\expandafter\def\csname PY@tok@o\endcsname{\def\PY@tc##1{\textcolor[rgb]{0.40,0.40,0.40}{##1}}}
\expandafter\def\csname PY@tok@ow\endcsname{\let\PY@bf=\textbf\def\PY@tc##1{\textcolor[rgb]{0.67,0.13,1.00}{##1}}}
\expandafter\def\csname PY@tok@nb\endcsname{\def\PY@tc##1{\textcolor[rgb]{0.00,0.50,0.00}{##1}}}
\expandafter\def\csname PY@tok@nf\endcsname{\def\PY@tc##1{\textcolor[rgb]{0.00,0.00,1.00}{##1}}}
\expandafter\def\csname PY@tok@nc\endcsname{\let\PY@bf=\textbf\def\PY@tc##1{\textcolor[rgb]{0.00,0.00,1.00}{##1}}}
\expandafter\def\csname PY@tok@nn\endcsname{\let\PY@bf=\textbf\def\PY@tc##1{\textcolor[rgb]{0.00,0.00,1.00}{##1}}}
\expandafter\def\csname PY@tok@ne\endcsname{\let\PY@bf=\textbf\def\PY@tc##1{\textcolor[rgb]{0.82,0.25,0.23}{##1}}}
\expandafter\def\csname PY@tok@nv\endcsname{\def\PY@tc##1{\textcolor[rgb]{0.10,0.09,0.49}{##1}}}
\expandafter\def\csname PY@tok@no\endcsname{\def\PY@tc##1{\textcolor[rgb]{0.53,0.00,0.00}{##1}}}
\expandafter\def\csname PY@tok@nl\endcsname{\def\PY@tc##1{\textcolor[rgb]{0.63,0.63,0.00}{##1}}}
\expandafter\def\csname PY@tok@ni\endcsname{\let\PY@bf=\textbf\def\PY@tc##1{\textcolor[rgb]{0.60,0.60,0.60}{##1}}}
\expandafter\def\csname PY@tok@na\endcsname{\def\PY@tc##1{\textcolor[rgb]{0.49,0.56,0.16}{##1}}}
\expandafter\def\csname PY@tok@nt\endcsname{\let\PY@bf=\textbf\def\PY@tc##1{\textcolor[rgb]{0.00,0.50,0.00}{##1}}}
\expandafter\def\csname PY@tok@nd\endcsname{\def\PY@tc##1{\textcolor[rgb]{0.67,0.13,1.00}{##1}}}
\expandafter\def\csname PY@tok@s\endcsname{\def\PY@tc##1{\textcolor[rgb]{0.73,0.13,0.13}{##1}}}
\expandafter\def\csname PY@tok@sd\endcsname{\let\PY@it=\textit\def\PY@tc##1{\textcolor[rgb]{0.73,0.13,0.13}{##1}}}
\expandafter\def\csname PY@tok@si\endcsname{\let\PY@bf=\textbf\def\PY@tc##1{\textcolor[rgb]{0.73,0.40,0.53}{##1}}}
\expandafter\def\csname PY@tok@se\endcsname{\let\PY@bf=\textbf\def\PY@tc##1{\textcolor[rgb]{0.73,0.40,0.13}{##1}}}
\expandafter\def\csname PY@tok@sr\endcsname{\def\PY@tc##1{\textcolor[rgb]{0.73,0.40,0.53}{##1}}}
\expandafter\def\csname PY@tok@ss\endcsname{\def\PY@tc##1{\textcolor[rgb]{0.10,0.09,0.49}{##1}}}
\expandafter\def\csname PY@tok@sx\endcsname{\def\PY@tc##1{\textcolor[rgb]{0.00,0.50,0.00}{##1}}}
\expandafter\def\csname PY@tok@m\endcsname{\def\PY@tc##1{\textcolor[rgb]{0.40,0.40,0.40}{##1}}}
\expandafter\def\csname PY@tok@gh\endcsname{\let\PY@bf=\textbf\def\PY@tc##1{\textcolor[rgb]{0.00,0.00,0.50}{##1}}}
\expandafter\def\csname PY@tok@gu\endcsname{\let\PY@bf=\textbf\def\PY@tc##1{\textcolor[rgb]{0.50,0.00,0.50}{##1}}}
\expandafter\def\csname PY@tok@gd\endcsname{\def\PY@tc##1{\textcolor[rgb]{0.63,0.00,0.00}{##1}}}
\expandafter\def\csname PY@tok@gi\endcsname{\def\PY@tc##1{\textcolor[rgb]{0.00,0.63,0.00}{##1}}}
\expandafter\def\csname PY@tok@gr\endcsname{\def\PY@tc##1{\textcolor[rgb]{1.00,0.00,0.00}{##1}}}
\expandafter\def\csname PY@tok@ge\endcsname{\let\PY@it=\textit}
\expandafter\def\csname PY@tok@gs\endcsname{\let\PY@bf=\textbf}
\expandafter\def\csname PY@tok@gp\endcsname{\let\PY@bf=\textbf\def\PY@tc##1{\textcolor[rgb]{0.00,0.00,0.50}{##1}}}
\expandafter\def\csname PY@tok@go\endcsname{\def\PY@tc##1{\textcolor[rgb]{0.53,0.53,0.53}{##1}}}
\expandafter\def\csname PY@tok@gt\endcsname{\def\PY@tc##1{\textcolor[rgb]{0.00,0.27,0.87}{##1}}}
\expandafter\def\csname PY@tok@err\endcsname{\def\PY@bc##1{\setlength{\fboxsep}{0pt}\fcolorbox[rgb]{1.00,0.00,0.00}{1,1,1}{\strut ##1}}}
\expandafter\def\csname PY@tok@kc\endcsname{\let\PY@bf=\textbf\def\PY@tc##1{\textcolor[rgb]{0.00,0.50,0.00}{##1}}}
\expandafter\def\csname PY@tok@kd\endcsname{\let\PY@bf=\textbf\def\PY@tc##1{\textcolor[rgb]{0.00,0.50,0.00}{##1}}}
\expandafter\def\csname PY@tok@kn\endcsname{\let\PY@bf=\textbf\def\PY@tc##1{\textcolor[rgb]{0.00,0.50,0.00}{##1}}}
\expandafter\def\csname PY@tok@kr\endcsname{\let\PY@bf=\textbf\def\PY@tc##1{\textcolor[rgb]{0.00,0.50,0.00}{##1}}}
\expandafter\def\csname PY@tok@bp\endcsname{\def\PY@tc##1{\textcolor[rgb]{0.00,0.50,0.00}{##1}}}
\expandafter\def\csname PY@tok@fm\endcsname{\def\PY@tc##1{\textcolor[rgb]{0.00,0.00,1.00}{##1}}}
\expandafter\def\csname PY@tok@vc\endcsname{\def\PY@tc##1{\textcolor[rgb]{0.10,0.09,0.49}{##1}}}
\expandafter\def\csname PY@tok@vg\endcsname{\def\PY@tc##1{\textcolor[rgb]{0.10,0.09,0.49}{##1}}}
\expandafter\def\csname PY@tok@vi\endcsname{\def\PY@tc##1{\textcolor[rgb]{0.10,0.09,0.49}{##1}}}
\expandafter\def\csname PY@tok@vm\endcsname{\def\PY@tc##1{\textcolor[rgb]{0.10,0.09,0.49}{##1}}}
\expandafter\def\csname PY@tok@sa\endcsname{\def\PY@tc##1{\textcolor[rgb]{0.73,0.13,0.13}{##1}}}
\expandafter\def\csname PY@tok@sb\endcsname{\def\PY@tc##1{\textcolor[rgb]{0.73,0.13,0.13}{##1}}}
\expandafter\def\csname PY@tok@sc\endcsname{\def\PY@tc##1{\textcolor[rgb]{0.73,0.13,0.13}{##1}}}
\expandafter\def\csname PY@tok@dl\endcsname{\def\PY@tc##1{\textcolor[rgb]{0.73,0.13,0.13}{##1}}}
\expandafter\def\csname PY@tok@s2\endcsname{\def\PY@tc##1{\textcolor[rgb]{0.73,0.13,0.13}{##1}}}
\expandafter\def\csname PY@tok@sh\endcsname{\def\PY@tc##1{\textcolor[rgb]{0.73,0.13,0.13}{##1}}}
\expandafter\def\csname PY@tok@s1\endcsname{\def\PY@tc##1{\textcolor[rgb]{0.73,0.13,0.13}{##1}}}
\expandafter\def\csname PY@tok@mb\endcsname{\def\PY@tc##1{\textcolor[rgb]{0.40,0.40,0.40}{##1}}}
\expandafter\def\csname PY@tok@mf\endcsname{\def\PY@tc##1{\textcolor[rgb]{0.40,0.40,0.40}{##1}}}
\expandafter\def\csname PY@tok@mh\endcsname{\def\PY@tc##1{\textcolor[rgb]{0.40,0.40,0.40}{##1}}}
\expandafter\def\csname PY@tok@mi\endcsname{\def\PY@tc##1{\textcolor[rgb]{0.40,0.40,0.40}{##1}}}
\expandafter\def\csname PY@tok@il\endcsname{\def\PY@tc##1{\textcolor[rgb]{0.40,0.40,0.40}{##1}}}
\expandafter\def\csname PY@tok@mo\endcsname{\def\PY@tc##1{\textcolor[rgb]{0.40,0.40,0.40}{##1}}}
\expandafter\def\csname PY@tok@ch\endcsname{\let\PY@it=\textit\def\PY@tc##1{\textcolor[rgb]{0.25,0.50,0.50}{##1}}}
\expandafter\def\csname PY@tok@cm\endcsname{\let\PY@it=\textit\def\PY@tc##1{\textcolor[rgb]{0.25,0.50,0.50}{##1}}}
\expandafter\def\csname PY@tok@cpf\endcsname{\let\PY@it=\textit\def\PY@tc##1{\textcolor[rgb]{0.25,0.50,0.50}{##1}}}
\expandafter\def\csname PY@tok@c1\endcsname{\let\PY@it=\textit\def\PY@tc##1{\textcolor[rgb]{0.25,0.50,0.50}{##1}}}
\expandafter\def\csname PY@tok@cs\endcsname{\let\PY@it=\textit\def\PY@tc##1{\textcolor[rgb]{0.25,0.50,0.50}{##1}}}

\def\PYZbs{\char`\\}
\def\PYZus{\char`\_}
\def\PYZob{\char`\{}
\def\PYZcb{\char`\}}
\def\PYZca{\char`\^}
\def\PYZam{\char`\&}
\def\PYZlt{\char`\<}
\def\PYZgt{\char`\>}
\def\PYZsh{\char`\#}
\def\PYZpc{\char`\%}
\def\PYZdl{\char`\$}
\def\PYZhy{\char`\-}
\def\PYZsq{\char`\'}
\def\PYZdq{\char`\"}
\def\PYZti{\char`\~}
% for compatibility with earlier versions
\def\PYZat{@}
\def\PYZlb{[}
\def\PYZrb{]}
\makeatother


    % Exact colors from NB
    \definecolor{incolor}{rgb}{0.0, 0.0, 0.5}
    \definecolor{outcolor}{rgb}{0.545, 0.0, 0.0}



    
    % Prevent overflowing lines due to hard-to-break entities
    \sloppy 
    % Setup hyperref package
    \hypersetup{
      breaklinks=true,  % so long urls are correctly broken across lines
      colorlinks=true,
      urlcolor=urlcolor,
      linkcolor=linkcolor,
      citecolor=citecolor,
      }
    % Slightly bigger margins than the latex defaults
    
    \geometry{verbose,tmargin=1in,bmargin=1in,lmargin=1in,rmargin=1in}
    
    

    \begin{document}
    
    
    \maketitle
    
    

    
    \hypertarget{tf2202-teknik-komputasi---akar-persamaan-nonlinear}{%
\section{TF2202 Teknik Komputasi - Akar Persamaan
Nonlinear}\label{tf2202-teknik-komputasi---akar-persamaan-nonlinear}}

    Fadjar Fathurrahman

    \begin{Verbatim}[commandchars=\\\{\}]
{\color{incolor}In [{\color{incolor}1}]:} \PY{k+kn}{import} \PY{n+nn}{numpy} \PY{k}{as} \PY{n+nn}{np}
        \PY{k+kn}{import} \PY{n+nn}{matplotlib}\PY{n+nn}{.}\PY{n+nn}{pyplot} \PY{k}{as} \PY{n+nn}{plt}
\end{Verbatim}


    \begin{Verbatim}[commandchars=\\\{\}]
{\color{incolor}In [{\color{incolor}2}]:} \PY{k+kn}{from} \PY{n+nn}{IPython}\PY{n+nn}{.}\PY{n+nn}{display} \PY{k}{import} \PY{n}{set\PYZus{}matplotlib\PYZus{}formats}
        \PY{n}{set\PYZus{}matplotlib\PYZus{}formats}\PY{p}{(}\PY{l+s+s2}{\PYZdq{}}\PY{l+s+s2}{svg}\PY{l+s+s2}{\PYZdq{}}\PY{p}{)}
        \PY{o}{\PYZpc{}}\PY{k}{matplotlib} inline
\end{Verbatim}


    \begin{Verbatim}[commandchars=\\\{\}]
{\color{incolor}In [{\color{incolor}3}]:} \PY{k+kn}{import} \PY{n+nn}{matplotlib}
        \PY{n}{matplotlib}\PY{o}{.}\PY{n}{style}\PY{o}{.}\PY{n}{use}\PY{p}{(}\PY{l+s+s2}{\PYZdq{}}\PY{l+s+s2}{dark\PYZus{}background}\PY{l+s+s2}{\PYZdq{}}\PY{p}{)}
        \PY{c+c1}{\PYZsh{}matplotlib.style.use(\PYZdq{}default\PYZdq{})}
\end{Verbatim}


    \hypertarget{metode-bisection}{%
\subsection{Metode Bisection}\label{metode-bisection}}

    Ide dari metode bisection fakta bahwa jika tanda dari \(f(x_{1})\) dan
\(f(x_{2})\) berbeda, maka setidaknya ada satu akar dari persamaan
\(f(x) = 0\) ada dalam selang \((x_1,x_2)\).

    Misalkan kita ingin mencari akar dari fungsi berikut: \[
f(x) = 5x^3 - 5x^2 + 6x - 2
\] Untuk memperoleh gambaran mengenai akar dari persamaan \(f(x)=0\),
kita akan membuat plot dari dari \(f(x)\) terlebih dahulu.

    \begin{Verbatim}[commandchars=\\\{\}]
{\color{incolor}In [{\color{incolor}4}]:} \PY{k}{def} \PY{n+nf}{func\PYZus{}01}\PY{p}{(}\PY{n}{x}\PY{p}{)}\PY{p}{:}
            \PY{k}{return} \PY{l+m+mi}{5}\PY{o}{*}\PY{n}{x}\PY{o}{*}\PY{o}{*}\PY{l+m+mi}{3} \PY{o}{\PYZhy{}} \PY{l+m+mi}{5}\PY{o}{*}\PY{n}{x}\PY{o}{*}\PY{o}{*}\PY{l+m+mi}{2} \PY{o}{+} \PY{l+m+mi}{6}\PY{o}{*}\PY{n}{x} \PY{o}{\PYZhy{}} \PY{l+m+mi}{2}
\end{Verbatim}


    \begin{Verbatim}[commandchars=\\\{\}]
{\color{incolor}In [{\color{incolor}7}]:} \PY{n}{x} \PY{o}{=} \PY{n}{np}\PY{o}{.}\PY{n}{linspace}\PY{p}{(}\PY{l+m+mi}{0}\PY{p}{,}\PY{l+m+mi}{1}\PY{p}{,}\PY{l+m+mi}{100}\PY{p}{)}
        \PY{n}{y} \PY{o}{=}  \PY{n}{func\PYZus{}01}\PY{p}{(}\PY{n}{x}\PY{p}{)}
        \PY{n}{plt}\PY{o}{.}\PY{n}{clf}\PY{p}{(}\PY{p}{)}
        \PY{n}{plt}\PY{o}{.}\PY{n}{plot}\PY{p}{(}\PY{n}{x}\PY{p}{,} \PY{n}{y}\PY{p}{,} \PY{n}{label}\PY{o}{=}\PY{l+s+s2}{\PYZdq{}}\PY{l+s+s2}{f(x)}\PY{l+s+s2}{\PYZdq{}}\PY{p}{)}
        \PY{n}{plt}\PY{o}{.}\PY{n}{legend}\PY{p}{(}\PY{p}{)}
        \PY{n}{plt}\PY{o}{.}\PY{n}{ylim}\PY{p}{(}\PY{p}{[}\PY{o}{\PYZhy{}}\PY{l+m+mi}{1}\PY{p}{,} \PY{l+m+mi}{1}\PY{p}{]}\PY{p}{)}
        \PY{n}{plt}\PY{o}{.}\PY{n}{grid}\PY{p}{(}\PY{p}{)}
\end{Verbatim}


    \begin{center}
    \adjustimage{max size={0.9\linewidth}{0.9\paperheight}}{NOTES_AkarPersamaan_files/NOTES_AkarPersamaan_9_0.pdf}
    \end{center}
    { \hspace*{\fill} \\}
    
    Dari plot di atas dapat dilihat bahwa akar dari \(f(x)\) terletak di
sekitar \(x = 0.5\). Sebagai ilustrasi untuk metode bisection kita akan
menggunakan interval \(x_1 = 0\) dan \(x_2 = 1\).

    Mari kita cek bahwa tanda \(f(x_1)\) dan \(f(x_2)\) memiliki tanda yang
berbeda, atau \(f(x_1)f(x_2) < 0\).

    \begin{Verbatim}[commandchars=\\\{\}]
{\color{incolor}In [{\color{incolor}10}]:} \PY{n}{x1} \PY{o}{=} \PY{l+m+mf}{0.0}
         \PY{n}{x2} \PY{o}{=} \PY{l+m+mf}{1.0}
         \PY{n}{f1} \PY{o}{=} \PY{n}{func\PYZus{}01}\PY{p}{(}\PY{n}{x1}\PY{p}{)}
         \PY{n}{f2} \PY{o}{=} \PY{n}{func\PYZus{}01}\PY{p}{(}\PY{n}{x2}\PY{p}{)}
         \PY{n+nb}{print}\PY{p}{(}\PY{l+s+s2}{\PYZdq{}}\PY{l+s+s2}{f1 = }\PY{l+s+s2}{\PYZdq{}}\PY{p}{,} \PY{n}{f1}\PY{p}{)}
         \PY{n+nb}{print}\PY{p}{(}\PY{l+s+s2}{\PYZdq{}}\PY{l+s+s2}{f2 = }\PY{l+s+s2}{\PYZdq{}}\PY{p}{,} \PY{n}{f2}\PY{p}{)}
         \PY{n+nb}{print}\PY{p}{(}\PY{n}{f1}\PY{o}{*}\PY{n}{f2} \PY{o}{\PYZlt{}} \PY{l+m+mi}{0}\PY{p}{)}
\end{Verbatim}


    \begin{Verbatim}[commandchars=\\\{\}]
f1 =  -2.0
f2 =  4.0
True

    \end{Verbatim}

    Fungsi \texttt{np.sign()} juga bisa digunakan untuk mengecek tanda
positif atau negatif dari suatu bilangan.

    \begin{Verbatim}[commandchars=\\\{\}]
{\color{incolor}In [{\color{incolor}11}]:} \PY{n}{np}\PY{o}{.}\PY{n}{sign}\PY{p}{(}\PY{o}{\PYZhy{}}\PY{l+m+mf}{2.0}\PY{p}{)}\PY{p}{,} \PY{n}{np}\PY{o}{.}\PY{n}{sign}\PY{p}{(}\PY{l+m+mf}{2.1}\PY{p}{)}
\end{Verbatim}


\begin{Verbatim}[commandchars=\\\{\}]
{\color{outcolor}Out[{\color{outcolor}11}]:} (-1.0, 1.0)
\end{Verbatim}
            
    Sekarang, kita perlu menentukan tebakan akar dari selang \(x_1\) dan
\(x_2\). Dengan metode bisection, tebakan akar dihitung tepat berada di
tengah selang yaitu: \[
x_{r} = \frac{x_1 + x_2}{2}
\]

    \begin{Verbatim}[commandchars=\\\{\}]
{\color{incolor}In [{\color{incolor}12}]:} \PY{n}{xr} \PY{o}{=} \PY{l+m+mf}{0.5}\PY{o}{*}\PY{p}{(}\PY{n}{x1} \PY{o}{+} \PY{n}{x2}\PY{p}{)}
         \PY{n}{xr}
\end{Verbatim}


\begin{Verbatim}[commandchars=\\\{\}]
{\color{outcolor}Out[{\color{outcolor}12}]:} 0.5
\end{Verbatim}
            
    OK, sepertinya kita sudah berada dekat di akar sebenarnya. Mari kita cek
nilai \(f(x_r)\):

    \begin{Verbatim}[commandchars=\\\{\}]
{\color{incolor}In [{\color{incolor}14}]:} \PY{n}{fxr} \PY{o}{=} \PY{n}{func\PYZus{}01}\PY{p}{(}\PY{n}{xr}\PY{p}{)}
         \PY{n}{fxr}
\end{Verbatim}


\begin{Verbatim}[commandchars=\\\{\}]
{\color{outcolor}Out[{\color{outcolor}14}]:} 0.375
\end{Verbatim}
            
    Ternyata nilai dari \(f(x_r)\) tidak tepat pada 0. Mari buat plot
\(f(x)\) pada domain (interval) yang lebih sempit.

    \begin{Verbatim}[commandchars=\\\{\}]
{\color{incolor}In [{\color{incolor}15}]:} \PY{n}{x} \PY{o}{=} \PY{n}{np}\PY{o}{.}\PY{n}{linspace}\PY{p}{(}\PY{l+m+mf}{0.0}\PY{p}{,} \PY{l+m+mf}{1.0}\PY{p}{,} \PY{l+m+mi}{500}\PY{p}{)}
         \PY{n}{y} \PY{o}{=} \PY{n}{func\PYZus{}01}\PY{p}{(}\PY{n}{x}\PY{p}{)}
         \PY{n}{plt}\PY{o}{.}\PY{n}{clf}\PY{p}{(}\PY{p}{)}
         \PY{n}{plt}\PY{o}{.}\PY{n}{plot}\PY{p}{(}\PY{n}{x}\PY{p}{,} \PY{n}{y}\PY{p}{)}
         \PY{n}{plt}\PY{o}{.}\PY{n}{plot}\PY{p}{(}\PY{n}{xr}\PY{p}{,} \PY{n}{fxr}\PY{p}{,} \PY{n}{marker}\PY{o}{=}\PY{l+s+s2}{\PYZdq{}}\PY{l+s+s2}{o}\PY{l+s+s2}{\PYZdq{}}\PY{p}{)} \PY{c+c1}{\PYZsh{} Tandai nilai fx pada xr, yaitu (xr,fxr)}
         \PY{n}{plt}\PY{o}{.}\PY{n}{grid}\PY{p}{(}\PY{p}{)}
\end{Verbatim}


    \begin{center}
    \adjustimage{max size={0.9\linewidth}{0.9\paperheight}}{NOTES_AkarPersamaan_files/NOTES_AkarPersamaan_20_0.pdf}
    \end{center}
    { \hspace*{\fill} \\}
    
    Kita dapat memperbaiki tebakan akar dengan memilih rentang baru di mana
kita akan mengaplikasikan lagi metode bisection. Kita sekarang memiliki
3 titik yaitu \(x_1\), \(x_2\), dan \(x_r\), dengan nilai fungsi pada
titik-titik tersebut adalah:

    \begin{Verbatim}[commandchars=\\\{\}]
{\color{incolor}In [{\color{incolor}16}]:} \PY{n}{f1}\PY{p}{,} \PY{n}{f2}\PY{p}{,} \PY{n}{fxr}
\end{Verbatim}


\begin{Verbatim}[commandchars=\\\{\}]
{\color{outcolor}Out[{\color{outcolor}16}]:} (-2.0, 4.0, 0.375)
\end{Verbatim}
            
    Dengan informasi tersebut, kita dapat menggunakan \(x_r\) sebagai
pengganti dari \(x_2\) karena selang ini lebih kecil dan diharapkan
nilai tebakan akar dapat menjadi lebih dekat dengan akar sebenarnya.

    \begin{Verbatim}[commandchars=\\\{\}]
{\color{incolor}In [{\color{incolor}17}]:} \PY{n}{x2} \PY{o}{=} \PY{n}{xr}
         \PY{n}{f2} \PY{o}{=} \PY{n}{fxr}
\end{Verbatim}


    Cek apakah nilai fungsi pada interval baru ini berbeda tanda (hasil kali
\(f_(x_{1})\) dan \(f(x_2)\) adalah negatif.

    \begin{Verbatim}[commandchars=\\\{\}]
{\color{incolor}In [{\color{incolor}18}]:} \PY{n+nb}{print}\PY{p}{(}\PY{n}{f1}\PY{o}{*}\PY{n}{f2} \PY{o}{\PYZlt{}} \PY{l+m+mi}{0}\PY{p}{)}
\end{Verbatim}


    \begin{Verbatim}[commandchars=\\\{\}]
True

    \end{Verbatim}

    Kita hitung lagi tebakan akar \(x_r\) sebagai nilai tengah antara
\(x_{1}\) dan \(x_2\):

    \begin{Verbatim}[commandchars=\\\{\}]
{\color{incolor}In [{\color{incolor}19}]:} \PY{n}{xr} \PY{o}{=} \PY{l+m+mf}{0.5}\PY{o}{*}\PY{p}{(}\PY{n}{x1} \PY{o}{+} \PY{n}{x2}\PY{p}{)}
         \PY{n}{fxr} \PY{o}{=} \PY{n}{func\PYZus{}01}\PY{p}{(}\PY{n}{xr}\PY{p}{)}
         \PY{n}{fxr}
\end{Verbatim}


\begin{Verbatim}[commandchars=\\\{\}]
{\color{outcolor}Out[{\color{outcolor}19}]:} -0.734375
\end{Verbatim}
            
    Sepertinya nilai akar yang kita dapatkan masih belum baik.

    \begin{Verbatim}[commandchars=\\\{\}]
{\color{incolor}In [{\color{incolor}20}]:} \PY{n}{x} \PY{o}{=} \PY{n}{np}\PY{o}{.}\PY{n}{linspace}\PY{p}{(}\PY{n}{x1}\PY{p}{,} \PY{n}{x2}\PY{p}{,} \PY{l+m+mi}{500}\PY{p}{)}
         \PY{n}{y} \PY{o}{=} \PY{n}{func\PYZus{}01}\PY{p}{(}\PY{n}{x}\PY{p}{)}
         \PY{n}{plt}\PY{o}{.}\PY{n}{clf}\PY{p}{(}\PY{p}{)}
         \PY{n}{plt}\PY{o}{.}\PY{n}{plot}\PY{p}{(}\PY{n}{x}\PY{p}{,} \PY{n}{y}\PY{p}{)}
         \PY{n}{plt}\PY{o}{.}\PY{n}{plot}\PY{p}{(}\PY{n}{xr}\PY{p}{,} \PY{n}{fxr}\PY{p}{,} \PY{n}{marker}\PY{o}{=}\PY{l+s+s2}{\PYZdq{}}\PY{l+s+s2}{o}\PY{l+s+s2}{\PYZdq{}}\PY{p}{)} \PY{c+c1}{\PYZsh{} Tandai nilai fx pada xr, yaitu (xr,fxr)}
         \PY{n}{plt}\PY{o}{.}\PY{n}{grid}\PY{p}{(}\PY{p}{)}
\end{Verbatim}


    \begin{center}
    \adjustimage{max size={0.9\linewidth}{0.9\paperheight}}{NOTES_AkarPersamaan_files/NOTES_AkarPersamaan_30_0.pdf}
    \end{center}
    { \hspace*{\fill} \\}
    
    \begin{Verbatim}[commandchars=\\\{\}]
{\color{incolor}In [{\color{incolor}21}]:} \PY{n}{f1}\PY{p}{,} \PY{n}{f2}\PY{p}{,} \PY{n}{fxr}
\end{Verbatim}


\begin{Verbatim}[commandchars=\\\{\}]
{\color{outcolor}Out[{\color{outcolor}21}]:} (-2.0, 0.375, -0.734375)
\end{Verbatim}
            
    Kita akan melakukan kembali prosedur bisection. Untuk interval baru kita
akan ganti \(x_1\) dengan \(x_r\).

    \begin{Verbatim}[commandchars=\\\{\}]
{\color{incolor}In [{\color{incolor}22}]:} \PY{n}{x1} \PY{o}{=} \PY{n}{xr}
         \PY{n}{f1} \PY{o}{=} \PY{n}{fxr}
\end{Verbatim}


    \begin{Verbatim}[commandchars=\\\{\}]
{\color{incolor}In [{\color{incolor}23}]:} \PY{n+nb}{print}\PY{p}{(}\PY{n}{f1}\PY{o}{*}\PY{n}{f2} \PY{o}{\PYZlt{}} \PY{l+m+mi}{0}\PY{p}{)}
\end{Verbatim}


    \begin{Verbatim}[commandchars=\\\{\}]
True

    \end{Verbatim}

    Hitung kembali tebakan akar pada selang \(x_1\) dan \(x_2\)

    \begin{Verbatim}[commandchars=\\\{\}]
{\color{incolor}In [{\color{incolor}24}]:} \PY{n}{xr} \PY{o}{=} \PY{l+m+mf}{0.5}\PY{o}{*}\PY{p}{(}\PY{n}{x1} \PY{o}{+} \PY{n}{x2}\PY{p}{)}
         \PY{n}{fxr} \PY{o}{=} \PY{n}{func\PYZus{}01}\PY{p}{(}\PY{n}{xr}\PY{p}{)}
         \PY{n}{fxr}
\end{Verbatim}


\begin{Verbatim}[commandchars=\\\{\}]
{\color{outcolor}Out[{\color{outcolor}24}]:} -0.189453125
\end{Verbatim}
            
    Nilai ini sudah lebih dekat dari tebakan-tebakan kita sebelumnya.

Untuk mendapatkan tebakan akar yang lebih baik kita akan lakukan sekali
lagi metode bisection.

    \begin{Verbatim}[commandchars=\\\{\}]
{\color{incolor}In [{\color{incolor}25}]:} \PY{n}{f1}\PY{p}{,} \PY{n}{f2}\PY{p}{,} \PY{n}{fxr}
\end{Verbatim}


\begin{Verbatim}[commandchars=\\\{\}]
{\color{outcolor}Out[{\color{outcolor}25}]:} (-0.734375, 0.375, -0.189453125)
\end{Verbatim}
            
    \begin{Verbatim}[commandchars=\\\{\}]
{\color{incolor}In [{\color{incolor}26}]:} \PY{n}{x1} \PY{o}{=} \PY{n}{xr}
         \PY{n}{f1} \PY{o}{=} \PY{n}{fxr}
         \PY{n+nb}{print}\PY{p}{(}\PY{n}{f1}\PY{o}{*}\PY{n}{f2} \PY{o}{\PYZlt{}} \PY{l+m+mi}{0}\PY{p}{)}
\end{Verbatim}


    \begin{Verbatim}[commandchars=\\\{\}]
True

    \end{Verbatim}

    \begin{Verbatim}[commandchars=\\\{\}]
{\color{incolor}In [{\color{incolor}27}]:} \PY{n}{xr} \PY{o}{=} \PY{l+m+mf}{0.5}\PY{o}{*}\PY{p}{(}\PY{n}{x1} \PY{o}{+} \PY{n}{x2}\PY{p}{)}
         \PY{n}{fxr} \PY{o}{=} \PY{n}{func\PYZus{}01}\PY{p}{(}\PY{n}{xr}\PY{p}{)}
         \PY{n}{fxr}
\end{Verbatim}


\begin{Verbatim}[commandchars=\\\{\}]
{\color{outcolor}Out[{\color{outcolor}27}]:} 0.086669921875
\end{Verbatim}
            
    Tebakan ini lebih baik dari tebakan sebelumnya karena \(f(x_r)\) yang
diperoleh lebih dekat dengan 0. Kita dapat melakukan prosedur bisection
sekali lagi.

    \begin{Verbatim}[commandchars=\\\{\}]
{\color{incolor}In [{\color{incolor}28}]:} \PY{n}{f1}\PY{p}{,} \PY{n}{f2}\PY{p}{,} \PY{n}{fxr}
\end{Verbatim}


\begin{Verbatim}[commandchars=\\\{\}]
{\color{outcolor}Out[{\color{outcolor}28}]:} (-0.189453125, 0.375, 0.086669921875)
\end{Verbatim}
            
    \begin{Verbatim}[commandchars=\\\{\}]
{\color{incolor}In [{\color{incolor}29}]:} \PY{n}{x2} \PY{o}{=} \PY{n}{xr}
         \PY{n}{f2} \PY{o}{=} \PY{n}{fxr}
\end{Verbatim}


    \begin{Verbatim}[commandchars=\\\{\}]
{\color{incolor}In [{\color{incolor}30}]:} \PY{n}{xr} \PY{o}{=} \PY{l+m+mf}{0.5}\PY{o}{*}\PY{p}{(}\PY{n}{x1} \PY{o}{+} \PY{n}{x2}\PY{p}{)}
         \PY{n}{fxr} \PY{o}{=} \PY{n}{func\PYZus{}01}\PY{p}{(}\PY{n}{xr}\PY{p}{)}
         \PY{n}{fxr}
\end{Verbatim}


\begin{Verbatim}[commandchars=\\\{\}]
{\color{outcolor}Out[{\color{outcolor}30}]:} -0.052459716796875
\end{Verbatim}
            
    Setelah melakukan iterasi metode bisection secara manual, sekarang kita
akan membuat prosedur bisection dalam suatu subrutin (fungsi). Fungsi
ini menerima masukan \texttt{f} sebagai fungsi yang akan dicari akarnya,
\texttt{x1} dan \texttt{x2} sebagai input selang di mana akar akan
dicari. Fungsi ini juga menggunakan \texttt{TOL} dengan nilai default
\texttt{1e-10} untuk menentukan akurasi hasil yang diperoleh dan juga
\texttt{NiterMax} dengan nilai default \texttt{100} sebagai jumlah
maksimum iterasi yang dilakukan.

    \begin{Verbatim}[commandchars=\\\{\}]
{\color{incolor}In [{\color{incolor}22}]:} \PY{k}{def} \PY{n+nf}{bisection}\PY{p}{(}\PY{n}{f}\PY{p}{,} \PY{n}{x1}\PY{p}{,} \PY{n}{x2}\PY{p}{,} \PY{n}{TOL}\PY{o}{=}\PY{l+m+mf}{1e\PYZhy{}10}\PY{p}{,} \PY{n}{NiterMax}\PY{o}{=}\PY{l+m+mi}{100}\PY{p}{)}\PY{p}{:}
             
             \PY{n}{f1} \PY{o}{=} \PY{n}{f}\PY{p}{(}\PY{n}{x1}\PY{p}{)}
             \PY{n}{f2} \PY{o}{=} \PY{n}{f}\PY{p}{(}\PY{n}{x2}\PY{p}{)}
             
             \PY{k}{if} \PY{n}{f1}\PY{o}{*}\PY{n}{f2} \PY{o}{\PYZgt{}} \PY{l+m+mi}{0}\PY{p}{:}
                 \PY{k}{raise} \PY{n+ne}{RuntimeError}\PY{p}{(}\PY{l+s+s2}{\PYZdq{}}\PY{l+s+s2}{f1 dan f2 memiliki tanda yang sama}\PY{l+s+s2}{\PYZdq{}}\PY{p}{)}
                 
             \PY{k}{for} \PY{n}{i} \PY{o+ow}{in} \PY{n+nb}{range}\PY{p}{(}\PY{l+m+mi}{1}\PY{p}{,}\PY{n}{NiterMax}\PY{o}{+}\PY{l+m+mi}{1}\PY{p}{)}\PY{p}{:}
                 
                 \PY{n}{xr} \PY{o}{=} \PY{l+m+mf}{0.5}\PY{o}{*}\PY{p}{(}\PY{n}{x1} \PY{o}{+} \PY{n}{x2}\PY{p}{)}
                 \PY{n}{fxr} \PY{o}{=} \PY{n}{f}\PY{p}{(}\PY{n}{xr}\PY{p}{)}
                 
                 \PY{k}{if} \PY{n+nb}{abs}\PY{p}{(}\PY{n}{fxr}\PY{p}{)} \PY{o}{\PYZlt{}}\PY{o}{=} \PY{n}{TOL}\PY{p}{:}
                     \PY{n+nb}{print}\PY{p}{(}\PY{l+s+s2}{\PYZdq{}}\PY{l+s+s2}{Iterasi konvergen: akar ditemukan}\PY{l+s+s2}{\PYZdq{}}\PY{p}{)}
                     \PY{k}{return} \PY{n}{xr}
                 
                 \PY{n+nb}{print}\PY{p}{(}\PY{l+s+s2}{\PYZdq{}}\PY{l+s+s2}{Iter = }\PY{l+s+si}{\PYZpc{}5d}\PY{l+s+s2}{, xr = }\PY{l+s+si}{\PYZpc{}18.10f}\PY{l+s+s2}{, abs(fxr) = }\PY{l+s+si}{\PYZpc{}15.5e}\PY{l+s+s2}{\PYZdq{}} \PY{o}{\PYZpc{}} \PY{p}{(}\PY{n}{i}\PY{p}{,} \PY{n}{xr}\PY{p}{,} \PY{n+nb}{abs}\PY{p}{(}\PY{n}{fxr}\PY{p}{)}\PY{p}{)}\PY{p}{)}
             
                 \PY{c+c1}{\PYZsh{} f1 dan fxr berbeda tanda}
                 \PY{k}{if} \PY{n}{f1}\PY{o}{*}\PY{n}{fxr} \PY{o}{\PYZlt{}} \PY{l+m+mf}{0.0}\PY{p}{:}
                     \PY{n}{x2} \PY{o}{=} \PY{n}{xr}
                     \PY{n}{f2} \PY{o}{=} \PY{n}{fxr}
                 \PY{k}{else}\PY{p}{:}
                     \PY{n}{x1} \PY{o}{=} \PY{n}{xr}
                     \PY{n}{f1} \PY{o}{=} \PY{n}{fxr}
                 
             \PY{n+nb}{print}\PY{p}{(}\PY{l+s+s2}{\PYZdq{}}\PY{l+s+s2}{WARNING: Konvergensi tidak diperleh setelah }\PY{l+s+si}{\PYZpc{}d}\PY{l+s+s2}{ iterasi}\PY{l+s+s2}{\PYZdq{}} \PY{o}{\PYZpc{}} \PY{n}{NiterMax}\PY{p}{)}
             \PY{n+nb}{print}\PY{p}{(}\PY{l+s+s2}{\PYZdq{}}\PY{l+s+s2}{WARNING: Nilai tebakan akhir akan dikembalikan}\PY{l+s+s2}{\PYZdq{}}\PY{p}{)}
             \PY{k}{return} \PY{n}{xr}
\end{Verbatim}


    Contoh penggunaan:

    \begin{Verbatim}[commandchars=\\\{\}]
{\color{incolor}In [{\color{incolor}37}]:} \PY{n}{xr} \PY{o}{=} \PY{n}{bisection}\PY{p}{(}\PY{n}{func\PYZus{}01}\PY{p}{,} \PY{l+m+mf}{0.0}\PY{p}{,} \PY{l+m+mf}{1.0}\PY{p}{)}
\end{Verbatim}


    \begin{Verbatim}[commandchars=\\\{\}]
Iter =     1, xr =       0.5000000000, abs(fxr) =     3.75000e-01
Iter =     2, xr =       0.2500000000, abs(fxr) =     7.34375e-01
Iter =     3, xr =       0.3750000000, abs(fxr) =     1.89453e-01
Iter =     4, xr =       0.4375000000, abs(fxr) =     8.66699e-02
Iter =     5, xr =       0.4062500000, abs(fxr) =     5.24597e-02
Iter =     6, xr =       0.4218750000, abs(fxr) =     1.67809e-02
Iter =     7, xr =       0.4140625000, abs(fxr) =     1.79133e-02
Iter =     8, xr =       0.4179687500, abs(fxr) =     5.85616e-04
Iter =     9, xr =       0.4199218750, abs(fxr) =     8.09266e-03
Iter =    10, xr =       0.4189453125, abs(fxr) =     3.75230e-03
Iter =    11, xr =       0.4184570312, abs(fxr) =     1.58304e-03
Iter =    12, xr =       0.4182128906, abs(fxr) =     4.98635e-04
Iter =    13, xr =       0.4180908203, abs(fxr) =     4.35092e-05
Iter =    14, xr =       0.4181518555, abs(fxr) =     2.27558e-04
Iter =    15, xr =       0.4181213379, abs(fxr) =     9.20233e-05
Iter =    16, xr =       0.4181060791, abs(fxr) =     2.42567e-05
Iter =    17, xr =       0.4180984497, abs(fxr) =     9.62632e-06
Iter =    18, xr =       0.4181022644, abs(fxr) =     7.31519e-06
Iter =    19, xr =       0.4181003571, abs(fxr) =     1.15557e-06
Iter =    20, xr =       0.4181013107, abs(fxr) =     3.07981e-06
Iter =    21, xr =       0.4181008339, abs(fxr) =     9.62119e-07
Iter =    22, xr =       0.4181005955, abs(fxr) =     9.67255e-08
Iter =    23, xr =       0.4181007147, abs(fxr) =     4.32697e-07
Iter =    24, xr =       0.4181006551, abs(fxr) =     1.67986e-07
Iter =    25, xr =       0.4181006253, abs(fxr) =     3.56301e-08
Iter =    26, xr =       0.4181006104, abs(fxr) =     3.05477e-08
Iter =    27, xr =       0.4181006178, abs(fxr) =     2.54121e-09
Iter =    28, xr =       0.4181006141, abs(fxr) =     1.40032e-08
Iter =    29, xr =       0.4181006160, abs(fxr) =     5.73101e-09
Iter =    30, xr =       0.4181006169, abs(fxr) =     1.59490e-09
Iter =    31, xr =       0.4181006174, abs(fxr) =     4.73154e-10
Iter =    32, xr =       0.4181006171, abs(fxr) =     5.60874e-10
Iterasi konvergen: akar ditemukan

    \end{Verbatim}

    \begin{Verbatim}[commandchars=\\\{\}]
{\color{incolor}In [{\color{incolor}38}]:} \PY{n}{xr} \PY{o}{=} \PY{n}{bisection}\PY{p}{(}\PY{n}{func\PYZus{}01}\PY{p}{,} \PY{l+m+mf}{0.0}\PY{p}{,} \PY{l+m+mf}{0.5}\PY{p}{,} \PY{n}{TOL}\PY{o}{=}\PY{l+m+mf}{1e\PYZhy{}9}\PY{p}{,} \PY{n}{NiterMax}\PY{o}{=}\PY{l+m+mi}{10}\PY{p}{)}
\end{Verbatim}


    \begin{Verbatim}[commandchars=\\\{\}]
Iter =     1, xr =       0.2500000000, abs(fxr) =     7.34375e-01
Iter =     2, xr =       0.3750000000, abs(fxr) =     1.89453e-01
Iter =     3, xr =       0.4375000000, abs(fxr) =     8.66699e-02
Iter =     4, xr =       0.4062500000, abs(fxr) =     5.24597e-02
Iter =     5, xr =       0.4218750000, abs(fxr) =     1.67809e-02
Iter =     6, xr =       0.4140625000, abs(fxr) =     1.79133e-02
Iter =     7, xr =       0.4179687500, abs(fxr) =     5.85616e-04
Iter =     8, xr =       0.4199218750, abs(fxr) =     8.09266e-03
Iter =     9, xr =       0.4189453125, abs(fxr) =     3.75230e-03
Iter =    10, xr =       0.4184570312, abs(fxr) =     1.58304e-03
WARNING: Konvergensi tidak diperleh setelah 10 iterasi
WARNING: Nilai tebakan akhir akan dikembalikan

    \end{Verbatim}

    \begin{Verbatim}[commandchars=\\\{\}]
{\color{incolor}In [{\color{incolor}39}]:} \PY{n}{xr} \PY{o}{=} \PY{n}{bisection}\PY{p}{(}\PY{n}{func\PYZus{}01}\PY{p}{,} \PY{l+m+mf}{0.0}\PY{p}{,} \PY{l+m+mf}{0.5}\PY{p}{,} \PY{n}{TOL}\PY{o}{=}\PY{l+m+mf}{1e\PYZhy{}9}\PY{p}{)}
\end{Verbatim}


    \begin{Verbatim}[commandchars=\\\{\}]
Iter =     1, xr =       0.2500000000, abs(fxr) =     7.34375e-01
Iter =     2, xr =       0.3750000000, abs(fxr) =     1.89453e-01
Iter =     3, xr =       0.4375000000, abs(fxr) =     8.66699e-02
Iter =     4, xr =       0.4062500000, abs(fxr) =     5.24597e-02
Iter =     5, xr =       0.4218750000, abs(fxr) =     1.67809e-02
Iter =     6, xr =       0.4140625000, abs(fxr) =     1.79133e-02
Iter =     7, xr =       0.4179687500, abs(fxr) =     5.85616e-04
Iter =     8, xr =       0.4199218750, abs(fxr) =     8.09266e-03
Iter =     9, xr =       0.4189453125, abs(fxr) =     3.75230e-03
Iter =    10, xr =       0.4184570312, abs(fxr) =     1.58304e-03
Iter =    11, xr =       0.4182128906, abs(fxr) =     4.98635e-04
Iter =    12, xr =       0.4180908203, abs(fxr) =     4.35092e-05
Iter =    13, xr =       0.4181518555, abs(fxr) =     2.27558e-04
Iter =    14, xr =       0.4181213379, abs(fxr) =     9.20233e-05
Iter =    15, xr =       0.4181060791, abs(fxr) =     2.42567e-05
Iter =    16, xr =       0.4180984497, abs(fxr) =     9.62632e-06
Iter =    17, xr =       0.4181022644, abs(fxr) =     7.31519e-06
Iter =    18, xr =       0.4181003571, abs(fxr) =     1.15557e-06
Iter =    19, xr =       0.4181013107, abs(fxr) =     3.07981e-06
Iter =    20, xr =       0.4181008339, abs(fxr) =     9.62119e-07
Iter =    21, xr =       0.4181005955, abs(fxr) =     9.67255e-08
Iter =    22, xr =       0.4181007147, abs(fxr) =     4.32697e-07
Iter =    23, xr =       0.4181006551, abs(fxr) =     1.67986e-07
Iter =    24, xr =       0.4181006253, abs(fxr) =     3.56301e-08
Iter =    25, xr =       0.4181006104, abs(fxr) =     3.05477e-08
Iter =    26, xr =       0.4181006178, abs(fxr) =     2.54121e-09
Iter =    27, xr =       0.4181006141, abs(fxr) =     1.40032e-08
Iter =    28, xr =       0.4181006160, abs(fxr) =     5.73101e-09
Iter =    29, xr =       0.4181006169, abs(fxr) =     1.59490e-09
Iterasi konvergen: akar ditemukan

    \end{Verbatim}

    \begin{Verbatim}[commandchars=\\\{\}]
{\color{incolor}In [{\color{incolor}42}]:} \PY{n}{xr} \PY{o}{=} \PY{n}{bisection}\PY{p}{(}\PY{n}{func\PYZus{}01}\PY{p}{,} \PY{l+m+mf}{0.3}\PY{p}{,} \PY{l+m+mf}{0.5}\PY{p}{,} \PY{n}{NiterMax}\PY{o}{=}\PY{l+m+mi}{100}\PY{p}{)}
\end{Verbatim}


    \begin{Verbatim}[commandchars=\\\{\}]
Iter =     1, xr =       0.4000000000, abs(fxr) =     8.00000e-02
Iter =     2, xr =       0.4500000000, abs(fxr) =     1.43125e-01
Iter =     3, xr =       0.4250000000, abs(fxr) =     3.07031e-02
Iter =     4, xr =       0.4125000000, abs(fxr) =     2.48340e-02
Iter =     5, xr =       0.4187500000, abs(fxr) =     2.88452e-03
Iter =     6, xr =       0.4156250000, abs(fxr) =     1.09868e-02
Iter =     7, xr =       0.4171875000, abs(fxr) =     4.05420e-03
Iter =     8, xr =       0.4179687500, abs(fxr) =     5.85616e-04
Iter =     9, xr =       0.4183593750, abs(fxr) =     1.14926e-03
Iter =    10, xr =       0.4181640625, abs(fxr) =     2.81773e-04
Iter =    11, xr =       0.4180664063, abs(fxr) =     1.51934e-04
Iter =    12, xr =       0.4181152344, abs(fxr) =     6.49166e-05
Iter =    13, xr =       0.4180908203, abs(fxr) =     4.35092e-05
Iter =    14, xr =       0.4181030273, abs(fxr) =     1.07035e-05
Iter =    15, xr =       0.4180969238, abs(fxr) =     1.64029e-05
Iter =    16, xr =       0.4180999756, abs(fxr) =     2.84972e-06
Iter =    17, xr =       0.4181015015, abs(fxr) =     3.92688e-06
Iter =    18, xr =       0.4181007385, abs(fxr) =     5.38581e-07
Iter =    19, xr =       0.4181003571, abs(fxr) =     1.15557e-06
Iter =    20, xr =       0.4181005478, abs(fxr) =     3.08494e-07
Iter =    21, xr =       0.4181006432, abs(fxr) =     1.15043e-07
Iter =    22, xr =       0.4181005955, abs(fxr) =     9.67255e-08
Iter =    23, xr =       0.4181006193, abs(fxr) =     9.15899e-09
Iter =    24, xr =       0.4181006074, abs(fxr) =     4.37832e-08
Iter =    25, xr =       0.4181006134, abs(fxr) =     1.73121e-08
Iter =    26, xr =       0.4181006163, abs(fxr) =     4.07657e-09
Iter =    27, xr =       0.4181006178, abs(fxr) =     2.54121e-09
Iter =    28, xr =       0.4181006171, abs(fxr) =     7.67679e-10
Iter =    29, xr =       0.4181006175, abs(fxr) =     8.86765e-10
Iterasi konvergen: akar ditemukan

    \end{Verbatim}

    \begin{Verbatim}[commandchars=\\\{\}]
{\color{incolor}In [{\color{incolor}43}]:} \PY{n}{xr} \PY{o}{=} \PY{n}{bisection}\PY{p}{(}\PY{n}{func\PYZus{}01}\PY{p}{,} \PY{l+m+mf}{0.3}\PY{p}{,} \PY{l+m+mf}{0.4}\PY{p}{)}
\end{Verbatim}


    \begin{Verbatim}[commandchars=\\\{\}]

        ---------------------------------------------------------------------------

        RuntimeError                              Traceback (most recent call last)

        <ipython-input-43-2f68945ac062> in <module>()
    ----> 1 xr = bisection(func\_01, 0.3, 0.4)
    

        <ipython-input-36-3b4f53df89d5> in bisection(f, x1, x2, TOL, NiterMax)
          5 
          6     if f1*f2 > 0:
    ----> 7         raise RuntimeError("f1 dan f2 memiliki tanda yang sama")
          8 
          9     for i in range(1,NiterMax+1):


        RuntimeError: f1 dan f2 memiliki tanda yang sama

    \end{Verbatim}

    Sebagai contoh lain, kita akan mencari akar persamaan berikut: \[
x^2 \left| \cos\left(\sqrt{x}\right) \right| = 5
\] Untuk menggunakan fungsi \texttt{bisection} kita perlu mengubah
persamaan tersebut dalam bentuk \(f(x) = 0\): \[
f(x) = x^2 \left| \cos\left(\sqrt{x}\right) \right| - 5 = 0
\]

    \begin{Verbatim}[commandchars=\\\{\}]
{\color{incolor}In [{\color{incolor}44}]:} \PY{k}{def} \PY{n+nf}{func\PYZus{}02}\PY{p}{(}\PY{n}{x}\PY{p}{)}\PY{p}{:}
             \PY{k}{return} \PY{n}{x}\PY{o}{*}\PY{o}{*}\PY{l+m+mi}{2} \PY{o}{*} \PY{n}{np}\PY{o}{.}\PY{n}{abs}\PY{p}{(}\PY{n}{np}\PY{o}{.}\PY{n}{cos}\PY{p}{(}\PY{n}{np}\PY{o}{.}\PY{n}{sqrt}\PY{p}{(}\PY{n}{x}\PY{p}{)}\PY{p}{)}\PY{p}{)} \PY{o}{\PYZhy{}} \PY{l+m+mi}{5}
\end{Verbatim}


    Kita perlu selang untuk tebakan awal akar. Untuk memperoleh informasi
tersebut kita akan buat plot dari \(f(x)\) terlebih dahulu.

    \begin{Verbatim}[commandchars=\\\{\}]
{\color{incolor}In [{\color{incolor}45}]:} \PY{n}{x} \PY{o}{=} \PY{n}{np}\PY{o}{.}\PY{n}{linspace}\PY{p}{(}\PY{l+m+mi}{0}\PY{p}{,}\PY{l+m+mi}{5}\PY{p}{,}\PY{l+m+mi}{500}\PY{p}{)}
         \PY{n}{y} \PY{o}{=} \PY{n}{func\PYZus{}02}\PY{p}{(}\PY{n}{x}\PY{p}{)}
         \PY{n}{plt}\PY{o}{.}\PY{n}{clf}\PY{p}{(}\PY{p}{)}
         \PY{n}{plt}\PY{o}{.}\PY{n}{plot}\PY{p}{(}\PY{n}{x}\PY{p}{,} \PY{n}{y}\PY{p}{)}
         \PY{n}{plt}\PY{o}{.}\PY{n}{grid}\PY{p}{(}\PY{p}{)}
\end{Verbatim}


    \begin{center}
    \adjustimage{max size={0.9\linewidth}{0.9\paperheight}}{NOTES_AkarPersamaan_files/NOTES_AkarPersamaan_56_0.pdf}
    \end{center}
    { \hspace*{\fill} \\}
    
    Akar terletak antara \(x = 0\) dan \(x = 5\).

    \begin{Verbatim}[commandchars=\\\{\}]
{\color{incolor}In [{\color{incolor}46}]:} \PY{n}{xr} \PY{o}{=} \PY{n}{bisection}\PY{p}{(}\PY{n}{func\PYZus{}02}\PY{p}{,} \PY{l+m+mi}{0}\PY{p}{,} \PY{l+m+mi}{5}\PY{p}{)}
\end{Verbatim}


    \begin{Verbatim}[commandchars=\\\{\}]
Iter =     1, xr =       2.5000000000, abs(fxr) =     4.93536e+00
Iter =     2, xr =       3.7500000000, abs(fxr) =     2.87324e-02
Iter =     3, xr =       3.1250000000, abs(fxr) =     3.08887e+00
Iter =     4, xr =       3.4375000000, abs(fxr) =     1.69754e+00
Iter =     5, xr =       3.5937500000, abs(fxr) =     8.77063e-01
Iter =     6, xr =       3.6718750000, abs(fxr) =     4.34912e-01
Iter =     7, xr =       3.7109375000, abs(fxr) =     2.05785e-01
Iter =     8, xr =       3.7304687500, abs(fxr) =     8.92016e-02
Iter =     9, xr =       3.7402343750, abs(fxr) =     3.04035e-02
Iter =    10, xr =       3.7451171875, abs(fxr) =     8.77797e-04
Iter =    11, xr =       3.7475585938, abs(fxr) =     1.39167e-02
Iter =    12, xr =       3.7463378906, abs(fxr) =     6.51683e-03
Iter =    13, xr =       3.7457275391, abs(fxr) =     2.81886e-03
Iter =    14, xr =       3.7454223633, abs(fxr) =     9.70364e-04
Iter =    15, xr =       3.7452697754, abs(fxr) =     4.62426e-05
Iter =    16, xr =       3.7451934814, abs(fxr) =     4.15787e-04
Iter =    17, xr =       3.7452316284, abs(fxr) =     1.84775e-04
Iter =    18, xr =       3.7452507019, abs(fxr) =     6.92668e-05
Iter =    19, xr =       3.7452602386, abs(fxr) =     1.15123e-05
Iter =    20, xr =       3.7452650070, abs(fxr) =     1.73651e-05
Iter =    21, xr =       3.7452626228, abs(fxr) =     2.92641e-06
Iter =    22, xr =       3.7452614307, abs(fxr) =     4.29294e-06
Iter =    23, xr =       3.7452620268, abs(fxr) =     6.83264e-07
Iter =    24, xr =       3.7452623248, abs(fxr) =     1.12157e-06
Iter =    25, xr =       3.7452621758, abs(fxr) =     2.19155e-07
Iter =    26, xr =       3.7452621013, abs(fxr) =     2.32055e-07
Iter =    27, xr =       3.7452621385, abs(fxr) =     6.44991e-09
Iter =    28, xr =       3.7452621572, abs(fxr) =     1.06352e-07
Iter =    29, xr =       3.7452621479, abs(fxr) =     4.99512e-08
Iter =    30, xr =       3.7452621432, abs(fxr) =     2.17507e-08
Iter =    31, xr =       3.7452621409, abs(fxr) =     7.65038e-09
Iter =    32, xr =       3.7452621397, abs(fxr) =     6.00232e-10
Iter =    33, xr =       3.7452621391, abs(fxr) =     2.92484e-09
Iter =    34, xr =       3.7452621394, abs(fxr) =     1.16230e-09
Iter =    35, xr =       3.7452621396, abs(fxr) =     2.81036e-10
Iter =    36, xr =       3.7452621396, abs(fxr) =     1.59599e-10
Iterasi konvergen: akar ditemukan

    \end{Verbatim}

    \begin{Verbatim}[commandchars=\\\{\}]
{\color{incolor}In [{\color{incolor}47}]:} \PY{n}{x} \PY{o}{=} \PY{n}{np}\PY{o}{.}\PY{n}{linspace}\PY{p}{(}\PY{l+m+mi}{0}\PY{p}{,}\PY{l+m+mi}{5}\PY{p}{,}\PY{l+m+mi}{500}\PY{p}{)}
         \PY{n}{y} \PY{o}{=} \PY{n}{func\PYZus{}02}\PY{p}{(}\PY{n}{x}\PY{p}{)}
         \PY{n}{plt}\PY{o}{.}\PY{n}{clf}\PY{p}{(}\PY{p}{)}
         \PY{n}{plt}\PY{o}{.}\PY{n}{plot}\PY{p}{(}\PY{n}{x}\PY{p}{,} \PY{n}{y}\PY{p}{)}
         \PY{n}{plt}\PY{o}{.}\PY{n}{plot}\PY{p}{(}\PY{n}{xr}\PY{p}{,} \PY{n}{func\PYZus{}02}\PY{p}{(}\PY{n}{xr}\PY{p}{)}\PY{p}{,} \PY{n}{marker}\PY{o}{=}\PY{l+s+s2}{\PYZdq{}}\PY{l+s+s2}{o}\PY{l+s+s2}{\PYZdq{}}\PY{p}{)}
         \PY{n}{plt}\PY{o}{.}\PY{n}{grid}\PY{p}{(}\PY{p}{)}
\end{Verbatim}


    \begin{center}
    \adjustimage{max size={0.9\linewidth}{0.9\paperheight}}{NOTES_AkarPersamaan_files/NOTES_AkarPersamaan_59_0.pdf}
    \end{center}
    { \hspace*{\fill} \\}
    
    Mari kita coba gunakan selang \(x = 3\) dan \(x = 4\).

    \begin{Verbatim}[commandchars=\\\{\}]
{\color{incolor}In [{\color{incolor}48}]:} \PY{n}{xr} \PY{o}{=} \PY{n}{bisection}\PY{p}{(}\PY{n}{func\PYZus{}02}\PY{p}{,} \PY{l+m+mi}{3}\PY{p}{,} \PY{l+m+mi}{4}\PY{p}{)}
\end{Verbatim}


    \begin{Verbatim}[commandchars=\\\{\}]
Iter =     1, xr =       3.5000000000, abs(fxr) =     1.37950e+00
Iter =     2, xr =       3.7500000000, abs(fxr) =     2.87324e-02
Iter =     3, xr =       3.6250000000, abs(fxr) =     7.02772e-01
Iter =     4, xr =       3.6875000000, abs(fxr) =     3.43907e-01
Iter =     5, xr =       3.7187500000, abs(fxr) =     1.59314e-01
Iter =     6, xr =       3.7343750000, abs(fxr) =     6.57229e-02
Iter =     7, xr =       3.7421875000, abs(fxr) =     1.86034e-02
Iter =     8, xr =       3.7460937500, abs(fxr) =     5.03748e-03
Iter =     9, xr =       3.7441406250, abs(fxr) =     6.78970e-03
Iter =    10, xr =       3.7451171875, abs(fxr) =     8.77797e-04
Iter =    11, xr =       3.7456054688, abs(fxr) =     2.07942e-03
Iter =    12, xr =       3.7453613281, abs(fxr) =     6.00706e-04
Iter =    13, xr =       3.7452392578, abs(fxr) =     1.38572e-04
Iter =    14, xr =       3.7453002930, abs(fxr) =     2.31060e-04
Iter =    15, xr =       3.7452697754, abs(fxr) =     4.62426e-05
Iter =    16, xr =       3.7452545166, abs(fxr) =     4.61651e-05
Iter =    17, xr =       3.7452621460, abs(fxr) =     3.86710e-08
Iter =    18, xr =       3.7452583313, abs(fxr) =     2.30632e-05
Iter =    19, xr =       3.7452602386, abs(fxr) =     1.15123e-05
Iter =    20, xr =       3.7452611923, abs(fxr) =     5.73681e-06
Iter =    21, xr =       3.7452616692, abs(fxr) =     2.84907e-06
Iter =    22, xr =       3.7452619076, abs(fxr) =     1.40520e-06
Iter =    23, xr =       3.7452620268, abs(fxr) =     6.83264e-07
Iter =    24, xr =       3.7452620864, abs(fxr) =     3.22296e-07
Iter =    25, xr =       3.7452621162, abs(fxr) =     1.41813e-07
Iter =    26, xr =       3.7452621311, abs(fxr) =     5.15708e-08
Iter =    27, xr =       3.7452621385, abs(fxr) =     6.44991e-09
Iter =    28, xr =       3.7452621423, abs(fxr) =     1.61106e-08
Iter =    29, xr =       3.7452621404, abs(fxr) =     4.83032e-09
Iter =    30, xr =       3.7452621395, abs(fxr) =     8.09795e-10
Iter =    31, xr =       3.7452621399, abs(fxr) =     2.01026e-09
Iter =    32, xr =       3.7452621397, abs(fxr) =     6.00232e-10
Iter =    33, xr =       3.7452621396, abs(fxr) =     1.04781e-10
Iter =    34, xr =       3.7452621397, abs(fxr) =     2.47725e-10
Iterasi konvergen: akar ditemukan

    \end{Verbatim}

    \hypertarget{metode-regula-falsi}{%
\subsection{Metode Regula-Falsi}\label{metode-regula-falsi}}

    Metode regula-falsi mirip dengan metode bisection, namun dengan
persamaan yang berbeda untuk menentukan aproksimasi akar. Pada metode
bisection, tebakan akar diberikan sebagai nilai tengah dari \(x_1\) dan
\(x_2\) sedangkan pada metode regula-falsi digunakan interpolasi linear
antara \(f(x_1)\) dan \(f(x_2)\). Tebakan akar adalah perpotongan antara
garis interpolasi linear ini dengan sumbu \(x\). Hasil akhirnya adalah:
\[
x_r = x_2 - \frac{f(x_2)}{f(x_1) - f(x_2)}(x_1 - x_2)
\]

    \begin{Verbatim}[commandchars=\\\{\}]
{\color{incolor}In [{\color{incolor}24}]:} \PY{k}{def} \PY{n+nf}{regula\PYZus{}falsi}\PY{p}{(}\PY{n}{f}\PY{p}{,} \PY{n}{x1}\PY{p}{,} \PY{n}{x2}\PY{p}{,} \PY{n}{TOL}\PY{o}{=}\PY{l+m+mf}{1e\PYZhy{}10}\PY{p}{,} \PY{n}{NiterMax}\PY{o}{=}\PY{l+m+mi}{100}\PY{p}{)}\PY{p}{:}
             
             \PY{n}{f1} \PY{o}{=} \PY{n}{f}\PY{p}{(}\PY{n}{x1}\PY{p}{)}
             \PY{n}{f2} \PY{o}{=} \PY{n}{f}\PY{p}{(}\PY{n}{x2}\PY{p}{)}
             
             \PY{k}{if} \PY{n}{f1}\PY{o}{*}\PY{n}{f2} \PY{o}{\PYZgt{}} \PY{l+m+mi}{0}\PY{p}{:}
                 \PY{k}{raise} \PY{n+ne}{RuntimeError}\PY{p}{(}\PY{l+s+s2}{\PYZdq{}}\PY{l+s+s2}{f1 dan f2 memiliki tanda yang sama}\PY{l+s+s2}{\PYZdq{}}\PY{p}{)}
                 
             \PY{k}{for} \PY{n}{i} \PY{o+ow}{in} \PY{n+nb}{range}\PY{p}{(}\PY{l+m+mi}{1}\PY{p}{,}\PY{n}{NiterMax}\PY{o}{+}\PY{l+m+mi}{1}\PY{p}{)}\PY{p}{:}
                 
                 \PY{n}{xr} \PY{o}{=} \PY{n}{x2} \PY{o}{\PYZhy{}} \PY{n}{f2}\PY{o}{*}\PY{p}{(}\PY{n}{x1} \PY{o}{\PYZhy{}} \PY{n}{x2}\PY{p}{)}\PY{o}{/}\PY{p}{(}\PY{n}{f1} \PY{o}{\PYZhy{}} \PY{n}{f2}\PY{p}{)}
                 \PY{n}{fxr} \PY{o}{=} \PY{n}{f}\PY{p}{(}\PY{n}{xr}\PY{p}{)}
                 
                 \PY{k}{if} \PY{n+nb}{abs}\PY{p}{(}\PY{n}{fxr}\PY{p}{)} \PY{o}{\PYZlt{}}\PY{o}{=} \PY{n}{TOL}\PY{p}{:}
                     \PY{n+nb}{print}\PY{p}{(}\PY{l+s+s2}{\PYZdq{}}\PY{l+s+s2}{Iterasi konvergen: akar ditemukan}\PY{l+s+s2}{\PYZdq{}}\PY{p}{)}
                     \PY{k}{return} \PY{n}{xr}
                 
                 \PY{n+nb}{print}\PY{p}{(}\PY{l+s+s2}{\PYZdq{}}\PY{l+s+s2}{Iter = }\PY{l+s+si}{\PYZpc{}5d}\PY{l+s+s2}{, xr = }\PY{l+s+si}{\PYZpc{}18.10f}\PY{l+s+s2}{, abs(fxr) = }\PY{l+s+si}{\PYZpc{}15.5e}\PY{l+s+s2}{\PYZdq{}} \PY{o}{\PYZpc{}} \PY{p}{(}\PY{n}{i}\PY{p}{,} \PY{n}{xr}\PY{p}{,} \PY{n+nb}{abs}\PY{p}{(}\PY{n}{fxr}\PY{p}{)}\PY{p}{)}\PY{p}{)}
             
                 \PY{c+c1}{\PYZsh{} f1 dan fxr berbeda tanda}
                 \PY{k}{if} \PY{n}{f1}\PY{o}{*}\PY{n}{fxr} \PY{o}{\PYZlt{}} \PY{l+m+mf}{0.0}\PY{p}{:}
                     \PY{n}{x2} \PY{o}{=} \PY{n}{xr}
                     \PY{n}{f2} \PY{o}{=} \PY{n}{fxr}
                 \PY{k}{else}\PY{p}{:}
                     \PY{n}{x1} \PY{o}{=} \PY{n}{xr}
                     \PY{n}{f1} \PY{o}{=} \PY{n}{fxr}
                 
             \PY{n+nb}{print}\PY{p}{(}\PY{l+s+s2}{\PYZdq{}}\PY{l+s+s2}{WARNING: Konvergensi tidak diperleh setelah }\PY{l+s+si}{\PYZpc{}d}\PY{l+s+s2}{ iterasi}\PY{l+s+s2}{\PYZdq{}} \PY{o}{\PYZpc{}} \PY{n}{NiterMax}\PY{p}{)}
             \PY{n+nb}{print}\PY{p}{(}\PY{l+s+s2}{\PYZdq{}}\PY{l+s+s2}{WARNING: Nilai tebakan akhir akan dikembalikan}\PY{l+s+s2}{\PYZdq{}}\PY{p}{)}
             \PY{k}{return} \PY{n}{xr}
\end{Verbatim}


    \begin{Verbatim}[commandchars=\\\{\}]
{\color{incolor}In [{\color{incolor}50}]:} \PY{n}{xr} \PY{o}{=} \PY{n}{regula\PYZus{}falsi}\PY{p}{(}\PY{n}{func\PYZus{}01}\PY{p}{,} \PY{l+m+mf}{0.0}\PY{p}{,} \PY{l+m+mf}{1.0}\PY{p}{)}
\end{Verbatim}


    \begin{Verbatim}[commandchars=\\\{\}]
Iter =     1, xr =       0.3333333333, abs(fxr) =     3.70370e-01
Iter =     2, xr =       0.3898305085, abs(fxr) =     1.24648e-01
Iter =     3, xr =       0.4082699418, abs(fxr) =     4.35410e-02
Iter =     4, xr =       0.4146417183, abs(fxr) =     1.53464e-02
Iter =     5, xr =       0.4168789156, abs(fxr) =     5.42383e-03
Iter =     6, xr =       0.4176685323, abs(fxr) =     1.91870e-03
Iter =     7, xr =       0.4179477285, abs(fxr) =     6.78967e-04
Iter =     8, xr =       0.4180465103, abs(fxr) =     2.40291e-04
Iter =     9, xr =       0.4180814678, abs(fxr) =     8.50444e-05
Iter =    10, xr =       0.4180938398, abs(fxr) =     3.00995e-05
Iter =    11, xr =       0.4180982185, abs(fxr) =     1.06531e-05
Iter =    12, xr =       0.4180997683, abs(fxr) =     3.77043e-06
Iter =    13, xr =       0.4181003168, abs(fxr) =     1.33446e-06
Iter =    14, xr =       0.4181005109, abs(fxr) =     4.72306e-07
Iter =    15, xr =       0.4181005796, abs(fxr) =     1.67163e-07
Iter =    16, xr =       0.4181006039, abs(fxr) =     5.91639e-08
Iter =    17, xr =       0.4181006125, abs(fxr) =     2.09398e-08
Iter =    18, xr =       0.4181006156, abs(fxr) =     7.41122e-09
Iter =    19, xr =       0.4181006167, abs(fxr) =     2.62305e-09
Iter =    20, xr =       0.4181006170, abs(fxr) =     9.28374e-10
Iter =    21, xr =       0.4181006172, abs(fxr) =     3.28579e-10
Iter =    22, xr =       0.4181006172, abs(fxr) =     1.16294e-10
Iterasi konvergen: akar ditemukan

    \end{Verbatim}

    \begin{Verbatim}[commandchars=\\\{\}]
{\color{incolor}In [{\color{incolor}51}]:} \PY{n}{xr} \PY{o}{=} \PY{n}{regula\PYZus{}falsi}\PY{p}{(}\PY{n}{func\PYZus{}02}\PY{p}{,} \PY{l+m+mi}{3}\PY{p}{,} \PY{l+m+mi}{4}\PY{p}{)}
\end{Verbatim}


    \begin{Verbatim}[commandchars=\\\{\}]
Iter =     1, xr =       3.6819027315, abs(fxr) =     3.76606e-01
Iter =     2, xr =       3.7407725061, abs(fxr) =     2.71536e-02
Iter =     3, xr =       3.7449486892, abs(fxr) =     1.89809e-03
Iter =     4, xr =       3.7452402785, abs(fxr) =     1.32391e-04
Iter =     5, xr =       3.7452606151, abs(fxr) =     9.23275e-06
Iter =     6, xr =       3.7452620333, abs(fxr) =     6.43873e-07
Iter =     7, xr =       3.7452621322, abs(fxr) =     4.49024e-08
Iter =     8, xr =       3.7452621391, abs(fxr) =     3.13140e-09
Iter =     9, xr =       3.7452621396, abs(fxr) =     2.18379e-10
Iterasi konvergen: akar ditemukan

    \end{Verbatim}

    \hypertarget{metode-fixed-point}{%
\subsection{Metode fixed-point}\label{metode-fixed-point}}

    Pada metode ini, persamaan \(f(x)\) yang ingin kita cari akarnya diubah
menjadi \(x = g(x)\). Contoh: untuk mencari akar dari persamaan
\(f(x) = e^{-x} - x = 0\) kita mengubah persamaan tersebut menjadi
\(x = g(x) = e^{-x}\).

Iterasi dimulai dengan suatu tebakan awal \(x_0\). Nilai tebakan akar
berikutnya dihitung dengan persamaan \[
x_{i+1} = g(x_{i})
\] Jika iterasi ini konvergen, maka \(x_{i+1}\) adalah akar dari
persamaan \(f(x) = 0\).

Perhatikan bahwa metode ini tidak selalu konvergen.

    \begin{Verbatim}[commandchars=\\\{\}]
{\color{incolor}In [{\color{incolor}52}]:} \PY{c+c1}{\PYZsh{} definisi fungsi g(x) = exp(\PYZhy{}x)}
         \PY{k}{def} \PY{n+nf}{func\PYZus{}03}\PY{p}{(}\PY{n}{x}\PY{p}{)}\PY{p}{:}
             \PY{k}{return} \PY{n}{np}\PY{o}{.}\PY{n}{exp}\PY{p}{(}\PY{o}{\PYZhy{}}\PY{n}{x}\PY{p}{)}
\end{Verbatim}


    \begin{Verbatim}[commandchars=\\\{\}]
{\color{incolor}In [{\color{incolor}53}]:} \PY{n}{plt}\PY{o}{.}\PY{n}{clf}\PY{p}{(}\PY{p}{)}
         \PY{n}{x} \PY{o}{=} \PY{n}{np}\PY{o}{.}\PY{n}{linspace}\PY{p}{(}\PY{l+m+mi}{0}\PY{p}{,}\PY{l+m+mi}{1}\PY{p}{,}\PY{l+m+mi}{500}\PY{p}{)}
         \PY{n}{plt}\PY{o}{.}\PY{n}{plot}\PY{p}{(}\PY{n}{x}\PY{p}{,} \PY{n}{x}\PY{p}{,} \PY{n}{label}\PY{o}{=}\PY{l+s+s2}{\PYZdq{}}\PY{l+s+s2}{\PYZdl{}f(x) = x\PYZdl{}}\PY{l+s+s2}{\PYZdq{}}\PY{p}{)}
         \PY{n}{plt}\PY{o}{.}\PY{n}{plot}\PY{p}{(}\PY{n}{x}\PY{p}{,} \PY{n}{func\PYZus{}03}\PY{p}{(}\PY{n}{x}\PY{p}{)}\PY{p}{,} \PY{n}{label}\PY{o}{=}\PY{l+s+s2}{\PYZdq{}}\PY{l+s+s2}{\PYZdl{}g(x) = exp(\PYZhy{}x)\PYZdl{}}\PY{l+s+s2}{\PYZdq{}}\PY{p}{)}
         \PY{n}{plt}\PY{o}{.}\PY{n}{legend}\PY{p}{(}\PY{p}{)}
         \PY{n}{plt}\PY{o}{.}\PY{n}{grid}\PY{p}{(}\PY{p}{)}
\end{Verbatim}


    \begin{center}
    \adjustimage{max size={0.9\linewidth}{0.9\paperheight}}{NOTES_AkarPersamaan_files/NOTES_AkarPersamaan_70_0.pdf}
    \end{center}
    { \hspace*{\fill} \\}
    
    \begin{Verbatim}[commandchars=\\\{\}]
{\color{incolor}In [{\color{incolor}14}]:} \PY{k}{def} \PY{n+nf}{fixed\PYZus{}point}\PY{p}{(}\PY{n}{g}\PY{p}{,} \PY{n}{x}\PY{p}{,} \PY{n}{TOL}\PY{o}{=}\PY{l+m+mf}{1e\PYZhy{}10}\PY{p}{,} \PY{n}{NiterMax}\PY{o}{=}\PY{l+m+mi}{100}\PY{p}{)}\PY{p}{:}
             
             \PY{k}{for} \PY{n}{i} \PY{o+ow}{in} \PY{n+nb}{range}\PY{p}{(}\PY{l+m+mi}{1}\PY{p}{,}\PY{n}{NiterMax}\PY{o}{+}\PY{l+m+mi}{1}\PY{p}{)}\PY{p}{:}
                 \PY{n}{gx} \PY{o}{=} \PY{n}{g}\PY{p}{(}\PY{n}{x}\PY{p}{)}
                 \PY{n+nb}{print}\PY{p}{(}\PY{l+s+s2}{\PYZdq{}}\PY{l+s+s2}{Iter = }\PY{l+s+si}{\PYZpc{}5d}\PY{l+s+s2}{, x = }\PY{l+s+si}{\PYZpc{}18.10f}\PY{l+s+s2}{, g(x) = }\PY{l+s+si}{\PYZpc{}18.10f}\PY{l+s+s2}{ abs(x\PYZhy{}g(x)) }\PY{l+s+si}{\PYZpc{}15.5e}\PY{l+s+s2}{\PYZdq{}} \PY{o}{\PYZpc{}} \PY{p}{(}\PY{n}{i}\PY{p}{,} \PY{n}{x}\PY{p}{,} \PY{n}{g}\PY{p}{(}\PY{n}{x}\PY{p}{)}\PY{p}{,} \PY{n+nb}{abs}\PY{p}{(}\PY{n}{x}\PY{o}{\PYZhy{}}\PY{n}{gx}\PY{p}{)}\PY{p}{)}\PY{p}{)}
                 \PY{k}{if} \PY{n+nb}{abs}\PY{p}{(}\PY{n}{x} \PY{o}{\PYZhy{}} \PY{n}{gx}\PY{p}{)} \PY{o}{\PYZlt{}}\PY{o}{=} \PY{n}{TOL}\PY{p}{:}
                     \PY{n+nb}{print}\PY{p}{(}\PY{l+s+s2}{\PYZdq{}}\PY{l+s+s2}{Iterasi konvergen: akar ditemukan}\PY{l+s+s2}{\PYZdq{}}\PY{p}{)}
                     \PY{k}{return} \PY{n}{x}
                 \PY{n}{x} \PY{o}{=} \PY{n}{gx}
             
             \PY{n+nb}{print}\PY{p}{(}\PY{l+s+s2}{\PYZdq{}}\PY{l+s+s2}{WARNING: Konvergensi tidak diperleh setelah }\PY{l+s+si}{\PYZpc{}d}\PY{l+s+s2}{ iterasi}\PY{l+s+s2}{\PYZdq{}} \PY{o}{\PYZpc{}} \PY{n}{NiterMax}\PY{p}{)}
             \PY{n+nb}{print}\PY{p}{(}\PY{l+s+s2}{\PYZdq{}}\PY{l+s+s2}{WARNING: Nilai tebakan akhir akan dikembalikan}\PY{l+s+s2}{\PYZdq{}}\PY{p}{)}
             \PY{k}{return} \PY{n}{x}
\end{Verbatim}


    \begin{Verbatim}[commandchars=\\\{\}]
{\color{incolor}In [{\color{incolor}57}]:} \PY{n}{x0} \PY{o}{=} \PY{l+m+mf}{0.0}
         \PY{n}{xr} \PY{o}{=} \PY{n}{fixed\PYZus{}point}\PY{p}{(}\PY{n}{func\PYZus{}03}\PY{p}{,} \PY{n}{x0}\PY{p}{)}
\end{Verbatim}


    \begin{Verbatim}[commandchars=\\\{\}]
Iter =     1, x =       0.0000000000, g(x) =       1.0000000000 abs(x-g(x))     1.00000e+00
Iter =     2, x =       1.0000000000, g(x) =       0.3678794412 abs(x-g(x))     6.32121e-01
Iter =     3, x =       0.3678794412, g(x) =       0.6922006276 abs(x-g(x))     3.24321e-01
Iter =     4, x =       0.6922006276, g(x) =       0.5004735006 abs(x-g(x))     1.91727e-01
Iter =     5, x =       0.5004735006, g(x) =       0.6062435351 abs(x-g(x))     1.05770e-01
Iter =     6, x =       0.6062435351, g(x) =       0.5453957860 abs(x-g(x))     6.08477e-02
Iter =     7, x =       0.5453957860, g(x) =       0.5796123355 abs(x-g(x))     3.42165e-02
Iter =     8, x =       0.5796123355, g(x) =       0.5601154614 abs(x-g(x))     1.94969e-02
Iter =     9, x =       0.5601154614, g(x) =       0.5711431151 abs(x-g(x))     1.10277e-02
Iter =    10, x =       0.5711431151, g(x) =       0.5648793474 abs(x-g(x))     6.26377e-03
Iter =    11, x =       0.5648793474, g(x) =       0.5684287250 abs(x-g(x))     3.54938e-03
Iter =    12, x =       0.5684287250, g(x) =       0.5664147331 abs(x-g(x))     2.01399e-03
Iter =    13, x =       0.5664147331, g(x) =       0.5675566373 abs(x-g(x))     1.14190e-03
Iter =    14, x =       0.5675566373, g(x) =       0.5669089119 abs(x-g(x))     6.47725e-04
Iter =    15, x =       0.5669089119, g(x) =       0.5672762322 abs(x-g(x))     3.67320e-04
Iter =    16, x =       0.5672762322, g(x) =       0.5670678984 abs(x-g(x))     2.08334e-04
Iter =    17, x =       0.5670678984, g(x) =       0.5671860501 abs(x-g(x))     1.18152e-04
Iter =    18, x =       0.5671860501, g(x) =       0.5671190401 abs(x-g(x))     6.70100e-05
Iter =    19, x =       0.5671190401, g(x) =       0.5671570440 abs(x-g(x))     3.80039e-05
Iter =    20, x =       0.5671570440, g(x) =       0.5671354902 abs(x-g(x))     2.15538e-05
Iter =    21, x =       0.5671354902, g(x) =       0.5671477143 abs(x-g(x))     1.22241e-05
Iter =    22, x =       0.5671477143, g(x) =       0.5671407815 abs(x-g(x))     6.93280e-06
Iter =    23, x =       0.5671407815, g(x) =       0.5671447133 abs(x-g(x))     3.93189e-06
Iter =    24, x =       0.5671447133, g(x) =       0.5671424834 abs(x-g(x))     2.22995e-06
Iter =    25, x =       0.5671424834, g(x) =       0.5671437481 abs(x-g(x))     1.26470e-06
Iter =    26, x =       0.5671437481, g(x) =       0.5671430308 abs(x-g(x))     7.17265e-07
Iter =    27, x =       0.5671430308, g(x) =       0.5671434376 abs(x-g(x))     4.06792e-07
Iter =    28, x =       0.5671434376, g(x) =       0.5671432069 abs(x-g(x))     2.30709e-07
Iter =    29, x =       0.5671432069, g(x) =       0.5671433378 abs(x-g(x))     1.30845e-07
Iter =    30, x =       0.5671433378, g(x) =       0.5671432636 abs(x-g(x))     7.42080e-08
Iter =    31, x =       0.5671432636, g(x) =       0.5671433056 abs(x-g(x))     4.20866e-08
Iter =    32, x =       0.5671433056, g(x) =       0.5671432818 abs(x-g(x))     2.38691e-08
Iter =    33, x =       0.5671432818, g(x) =       0.5671432953 abs(x-g(x))     1.35372e-08
Iter =    34, x =       0.5671432953, g(x) =       0.5671432876 abs(x-g(x))     7.67754e-09
Iter =    35, x =       0.5671432876, g(x) =       0.5671432920 abs(x-g(x))     4.35427e-09
Iter =    36, x =       0.5671432920, g(x) =       0.5671432895 abs(x-g(x))     2.46949e-09
Iter =    37, x =       0.5671432895, g(x) =       0.5671432909 abs(x-g(x))     1.40056e-09
Iter =    38, x =       0.5671432909, g(x) =       0.5671432901 abs(x-g(x))     7.94316e-10
Iter =    39, x =       0.5671432901, g(x) =       0.5671432906 abs(x-g(x))     4.50491e-10
Iter =    40, x =       0.5671432906, g(x) =       0.5671432903 abs(x-g(x))     2.55493e-10
Iter =    41, x =       0.5671432903, g(x) =       0.5671432905 abs(x-g(x))     1.44901e-10
Iter =    42, x =       0.5671432905, g(x) =       0.5671432904 abs(x-g(x))     8.21796e-11
Iterasi konvergen: akar ditemukan

    \end{Verbatim}

    \begin{Verbatim}[commandchars=\\\{\}]
{\color{incolor}In [{\color{incolor}58}]:} \PY{n}{plt}\PY{o}{.}\PY{n}{clf}\PY{p}{(}\PY{p}{)}
         \PY{n}{x} \PY{o}{=} \PY{n}{np}\PY{o}{.}\PY{n}{linspace}\PY{p}{(}\PY{l+m+mi}{0}\PY{p}{,}\PY{l+m+mi}{1}\PY{p}{,}\PY{l+m+mi}{500}\PY{p}{)}
         \PY{n}{plt}\PY{o}{.}\PY{n}{plot}\PY{p}{(}\PY{n}{x}\PY{p}{,} \PY{n}{x}\PY{p}{,} \PY{n}{label}\PY{o}{=}\PY{l+s+s2}{\PYZdq{}}\PY{l+s+s2}{\PYZdl{}f(x) = x\PYZdl{}}\PY{l+s+s2}{\PYZdq{}}\PY{p}{)}
         \PY{n}{plt}\PY{o}{.}\PY{n}{plot}\PY{p}{(}\PY{n}{x}\PY{p}{,} \PY{n}{func\PYZus{}03}\PY{p}{(}\PY{n}{x}\PY{p}{)}\PY{p}{,} \PY{n}{label}\PY{o}{=}\PY{l+s+s2}{\PYZdq{}}\PY{l+s+s2}{\PYZdl{}g(x) = exp(\PYZhy{}x)\PYZdl{}}\PY{l+s+s2}{\PYZdq{}}\PY{p}{)}
         \PY{n}{plt}\PY{o}{.}\PY{n}{plot}\PY{p}{(}\PY{n}{xr}\PY{p}{,} \PY{n}{func\PYZus{}03}\PY{p}{(}\PY{n}{xr}\PY{p}{)}\PY{p}{,} \PY{n}{marker}\PY{o}{=}\PY{l+s+s2}{\PYZdq{}}\PY{l+s+s2}{o}\PY{l+s+s2}{\PYZdq{}}\PY{p}{)}
         \PY{n}{plt}\PY{o}{.}\PY{n}{legend}\PY{p}{(}\PY{p}{)}
         \PY{n}{plt}\PY{o}{.}\PY{n}{grid}\PY{p}{(}\PY{p}{)}
\end{Verbatim}


    \begin{center}
    \adjustimage{max size={0.9\linewidth}{0.9\paperheight}}{NOTES_AkarPersamaan_files/NOTES_AkarPersamaan_73_0.pdf}
    \end{center}
    { \hspace*{\fill} \\}
    
    \begin{Verbatim}[commandchars=\\\{\}]
{\color{incolor}In [{\color{incolor}59}]:} \PY{n}{x0} \PY{o}{=} \PY{l+m+mf}{1.0}
         \PY{n}{xr} \PY{o}{=} \PY{n}{fixed\PYZus{}point}\PY{p}{(}\PY{n}{func\PYZus{}03}\PY{p}{,} \PY{n}{x0}\PY{p}{)}
\end{Verbatim}


    \begin{Verbatim}[commandchars=\\\{\}]
Iter =     1, x =       1.0000000000, g(x) =       0.3678794412 abs(x-g(x))     6.32121e-01
Iter =     2, x =       0.3678794412, g(x) =       0.6922006276 abs(x-g(x))     3.24321e-01
Iter =     3, x =       0.6922006276, g(x) =       0.5004735006 abs(x-g(x))     1.91727e-01
Iter =     4, x =       0.5004735006, g(x) =       0.6062435351 abs(x-g(x))     1.05770e-01
Iter =     5, x =       0.6062435351, g(x) =       0.5453957860 abs(x-g(x))     6.08477e-02
Iter =     6, x =       0.5453957860, g(x) =       0.5796123355 abs(x-g(x))     3.42165e-02
Iter =     7, x =       0.5796123355, g(x) =       0.5601154614 abs(x-g(x))     1.94969e-02
Iter =     8, x =       0.5601154614, g(x) =       0.5711431151 abs(x-g(x))     1.10277e-02
Iter =     9, x =       0.5711431151, g(x) =       0.5648793474 abs(x-g(x))     6.26377e-03
Iter =    10, x =       0.5648793474, g(x) =       0.5684287250 abs(x-g(x))     3.54938e-03
Iter =    11, x =       0.5684287250, g(x) =       0.5664147331 abs(x-g(x))     2.01399e-03
Iter =    12, x =       0.5664147331, g(x) =       0.5675566373 abs(x-g(x))     1.14190e-03
Iter =    13, x =       0.5675566373, g(x) =       0.5669089119 abs(x-g(x))     6.47725e-04
Iter =    14, x =       0.5669089119, g(x) =       0.5672762322 abs(x-g(x))     3.67320e-04
Iter =    15, x =       0.5672762322, g(x) =       0.5670678984 abs(x-g(x))     2.08334e-04
Iter =    16, x =       0.5670678984, g(x) =       0.5671860501 abs(x-g(x))     1.18152e-04
Iter =    17, x =       0.5671860501, g(x) =       0.5671190401 abs(x-g(x))     6.70100e-05
Iter =    18, x =       0.5671190401, g(x) =       0.5671570440 abs(x-g(x))     3.80039e-05
Iter =    19, x =       0.5671570440, g(x) =       0.5671354902 abs(x-g(x))     2.15538e-05
Iter =    20, x =       0.5671354902, g(x) =       0.5671477143 abs(x-g(x))     1.22241e-05
Iter =    21, x =       0.5671477143, g(x) =       0.5671407815 abs(x-g(x))     6.93280e-06
Iter =    22, x =       0.5671407815, g(x) =       0.5671447133 abs(x-g(x))     3.93189e-06
Iter =    23, x =       0.5671447133, g(x) =       0.5671424834 abs(x-g(x))     2.22995e-06
Iter =    24, x =       0.5671424834, g(x) =       0.5671437481 abs(x-g(x))     1.26470e-06
Iter =    25, x =       0.5671437481, g(x) =       0.5671430308 abs(x-g(x))     7.17265e-07
Iter =    26, x =       0.5671430308, g(x) =       0.5671434376 abs(x-g(x))     4.06792e-07
Iter =    27, x =       0.5671434376, g(x) =       0.5671432069 abs(x-g(x))     2.30709e-07
Iter =    28, x =       0.5671432069, g(x) =       0.5671433378 abs(x-g(x))     1.30845e-07
Iter =    29, x =       0.5671433378, g(x) =       0.5671432636 abs(x-g(x))     7.42080e-08
Iter =    30, x =       0.5671432636, g(x) =       0.5671433056 abs(x-g(x))     4.20866e-08
Iter =    31, x =       0.5671433056, g(x) =       0.5671432818 abs(x-g(x))     2.38691e-08
Iter =    32, x =       0.5671432818, g(x) =       0.5671432953 abs(x-g(x))     1.35372e-08
Iter =    33, x =       0.5671432953, g(x) =       0.5671432876 abs(x-g(x))     7.67754e-09
Iter =    34, x =       0.5671432876, g(x) =       0.5671432920 abs(x-g(x))     4.35427e-09
Iter =    35, x =       0.5671432920, g(x) =       0.5671432895 abs(x-g(x))     2.46949e-09
Iter =    36, x =       0.5671432895, g(x) =       0.5671432909 abs(x-g(x))     1.40056e-09
Iter =    37, x =       0.5671432909, g(x) =       0.5671432901 abs(x-g(x))     7.94316e-10
Iter =    38, x =       0.5671432901, g(x) =       0.5671432906 abs(x-g(x))     4.50491e-10
Iter =    39, x =       0.5671432906, g(x) =       0.5671432903 abs(x-g(x))     2.55493e-10
Iter =    40, x =       0.5671432903, g(x) =       0.5671432905 abs(x-g(x))     1.44901e-10
Iter =    41, x =       0.5671432905, g(x) =       0.5671432904 abs(x-g(x))     8.21796e-11
Iterasi konvergen: akar ditemukan

    \end{Verbatim}

    \begin{Verbatim}[commandchars=\\\{\}]
{\color{incolor}In [{\color{incolor}61}]:} \PY{n}{x0} \PY{o}{=} \PY{o}{\PYZhy{}}\PY{l+m+mf}{10.0}
         \PY{n}{xr} \PY{o}{=} \PY{n}{fixed\PYZus{}point}\PY{p}{(}\PY{n}{func\PYZus{}03}\PY{p}{,} \PY{n}{x0}\PY{p}{)}
\end{Verbatim}


    \begin{Verbatim}[commandchars=\\\{\}]
Iter =     1, x =     -10.0000000000, g(x) =   22026.4657948067 abs(x-g(x))     2.20365e+04
Iter =     2, x =   22026.4657948067, g(x) =       0.0000000000 abs(x-g(x))     2.20265e+04
Iter =     3, x =       0.0000000000, g(x) =       1.0000000000 abs(x-g(x))     1.00000e+00
Iter =     4, x =       1.0000000000, g(x) =       0.3678794412 abs(x-g(x))     6.32121e-01
Iter =     5, x =       0.3678794412, g(x) =       0.6922006276 abs(x-g(x))     3.24321e-01
Iter =     6, x =       0.6922006276, g(x) =       0.5004735006 abs(x-g(x))     1.91727e-01
Iter =     7, x =       0.5004735006, g(x) =       0.6062435351 abs(x-g(x))     1.05770e-01
Iter =     8, x =       0.6062435351, g(x) =       0.5453957860 abs(x-g(x))     6.08477e-02
Iter =     9, x =       0.5453957860, g(x) =       0.5796123355 abs(x-g(x))     3.42165e-02
Iter =    10, x =       0.5796123355, g(x) =       0.5601154614 abs(x-g(x))     1.94969e-02
Iter =    11, x =       0.5601154614, g(x) =       0.5711431151 abs(x-g(x))     1.10277e-02
Iter =    12, x =       0.5711431151, g(x) =       0.5648793474 abs(x-g(x))     6.26377e-03
Iter =    13, x =       0.5648793474, g(x) =       0.5684287250 abs(x-g(x))     3.54938e-03
Iter =    14, x =       0.5684287250, g(x) =       0.5664147331 abs(x-g(x))     2.01399e-03
Iter =    15, x =       0.5664147331, g(x) =       0.5675566373 abs(x-g(x))     1.14190e-03
Iter =    16, x =       0.5675566373, g(x) =       0.5669089119 abs(x-g(x))     6.47725e-04
Iter =    17, x =       0.5669089119, g(x) =       0.5672762322 abs(x-g(x))     3.67320e-04
Iter =    18, x =       0.5672762322, g(x) =       0.5670678984 abs(x-g(x))     2.08334e-04
Iter =    19, x =       0.5670678984, g(x) =       0.5671860501 abs(x-g(x))     1.18152e-04
Iter =    20, x =       0.5671860501, g(x) =       0.5671190401 abs(x-g(x))     6.70100e-05
Iter =    21, x =       0.5671190401, g(x) =       0.5671570440 abs(x-g(x))     3.80039e-05
Iter =    22, x =       0.5671570440, g(x) =       0.5671354902 abs(x-g(x))     2.15538e-05
Iter =    23, x =       0.5671354902, g(x) =       0.5671477143 abs(x-g(x))     1.22241e-05
Iter =    24, x =       0.5671477143, g(x) =       0.5671407815 abs(x-g(x))     6.93280e-06
Iter =    25, x =       0.5671407815, g(x) =       0.5671447133 abs(x-g(x))     3.93189e-06
Iter =    26, x =       0.5671447133, g(x) =       0.5671424834 abs(x-g(x))     2.22995e-06
Iter =    27, x =       0.5671424834, g(x) =       0.5671437481 abs(x-g(x))     1.26470e-06
Iter =    28, x =       0.5671437481, g(x) =       0.5671430308 abs(x-g(x))     7.17265e-07
Iter =    29, x =       0.5671430308, g(x) =       0.5671434376 abs(x-g(x))     4.06792e-07
Iter =    30, x =       0.5671434376, g(x) =       0.5671432069 abs(x-g(x))     2.30709e-07
Iter =    31, x =       0.5671432069, g(x) =       0.5671433378 abs(x-g(x))     1.30845e-07
Iter =    32, x =       0.5671433378, g(x) =       0.5671432636 abs(x-g(x))     7.42080e-08
Iter =    33, x =       0.5671432636, g(x) =       0.5671433056 abs(x-g(x))     4.20866e-08
Iter =    34, x =       0.5671433056, g(x) =       0.5671432818 abs(x-g(x))     2.38691e-08
Iter =    35, x =       0.5671432818, g(x) =       0.5671432953 abs(x-g(x))     1.35372e-08
Iter =    36, x =       0.5671432953, g(x) =       0.5671432876 abs(x-g(x))     7.67754e-09
Iter =    37, x =       0.5671432876, g(x) =       0.5671432920 abs(x-g(x))     4.35427e-09
Iter =    38, x =       0.5671432920, g(x) =       0.5671432895 abs(x-g(x))     2.46949e-09
Iter =    39, x =       0.5671432895, g(x) =       0.5671432909 abs(x-g(x))     1.40056e-09
Iter =    40, x =       0.5671432909, g(x) =       0.5671432901 abs(x-g(x))     7.94316e-10
Iter =    41, x =       0.5671432901, g(x) =       0.5671432906 abs(x-g(x))     4.50491e-10
Iter =    42, x =       0.5671432906, g(x) =       0.5671432903 abs(x-g(x))     2.55493e-10
Iter =    43, x =       0.5671432903, g(x) =       0.5671432905 abs(x-g(x))     1.44901e-10
Iter =    44, x =       0.5671432905, g(x) =       0.5671432904 abs(x-g(x))     8.21796e-11
Iterasi konvergen: akar ditemukan

    \end{Verbatim}

    \hypertarget{metode-newton-raphson}{%
\subsection{Metode Newton-Raphson}\label{metode-newton-raphson}}

    Metode Newton-Raphson adalah salah satu metode yang paling sering
digunakan untuk mencari akar persamaan nonlinear. Metode ini memerlukan
informasi tebakan awal akar dan turunan pertama dari fungsi yang akan
dicari akarnya.

Metode Newton-Rapshon dapat diturunkan dari deret Taylor untuk \(f(x)\)
disekitar \(x\): \[
f(x_{i+1}) = f(x_{i}) +
f'(x_{i})(x_{i+1} - x_{i}) +
\mathcal{O}(x_{i+1} - x_{i})^2
\]

Jika \(x_{i+1}\) adalah akar dari \(f(x)=0\) maka diperoleh: \[
0 = f(x_{i}) +
f'(x_{i})(x_{i+1} - x_{i}) +
\mathcal{O}(x_{i+1} - x_{i})^2
\]

Dengan mengasumsikan \(x_{i}\) dekat dengan \(x_{i+1}\), suku
\(\mathcal{O}(x_{i+1} - x_{i})\) dapat dianggap nol sehingga diperoleh:
\[
x_{i+1} = x_{i} - \frac{f(x_{i})}{f'(x_{i})}
\]

    Sebagai contoh, kita akan menghitung akar dari persamaan
\(f(x) = x^3 - 35\)

    \begin{Verbatim}[commandchars=\\\{\}]
{\color{incolor}In [{\color{incolor} }]:} \PY{k}{def} \PY{n+nf}{func\PYZus{}04}\PY{p}{(}\PY{n}{x}\PY{p}{)}\PY{p}{:}
            \PY{k}{return} \PY{n}{x}\PY{o}{*}\PY{o}{*}\PY{l+m+mi}{3} \PY{o}{\PYZhy{}} \PY{l+m+mf}{35.0}
        
        \PY{k}{def} \PY{n+nf}{dfunc\PYZus{}04}\PY{p}{(}\PY{n}{x}\PY{p}{)}\PY{p}{:}
            \PY{k}{return} \PY{l+m+mi}{3}\PY{o}{*}\PY{n}{x}\PY{o}{*}\PY{o}{*}\PY{l+m+mi}{2}
\end{Verbatim}


    \begin{Verbatim}[commandchars=\\\{\}]
{\color{incolor}In [{\color{incolor} }]:} \PY{n}{plt}\PY{o}{.}\PY{n}{clf}\PY{p}{(}\PY{p}{)}
        \PY{n}{x} \PY{o}{=} \PY{n}{np}\PY{o}{.}\PY{n}{linspace}\PY{p}{(}\PY{l+m+mi}{2}\PY{p}{,}\PY{l+m+mi}{4}\PY{p}{,}\PY{l+m+mi}{500}\PY{p}{)}
        \PY{n}{plt}\PY{o}{.}\PY{n}{plot}\PY{p}{(}\PY{n}{x}\PY{p}{,} \PY{n}{func\PYZus{}04}\PY{p}{(}\PY{n}{x}\PY{p}{)}\PY{p}{)}
        \PY{n}{plt}\PY{o}{.}\PY{n}{grid}\PY{p}{(}\PY{p}{)}
\end{Verbatim}


    \begin{Verbatim}[commandchars=\\\{\}]
{\color{incolor}In [{\color{incolor}15}]:} \PY{k}{def} \PY{n+nf}{newton\PYZus{}raphson}\PY{p}{(}\PY{n}{f}\PY{p}{,} \PY{n}{df}\PY{p}{,} \PY{n}{x}\PY{p}{,} \PY{n}{TOL}\PY{o}{=}\PY{l+m+mf}{1e\PYZhy{}10}\PY{p}{,} \PY{n}{NiterMax}\PY{o}{=}\PY{l+m+mi}{100}\PY{p}{)}\PY{p}{:}
             
             \PY{n}{SMALL} \PY{o}{=} \PY{n}{np}\PY{o}{.}\PY{n}{finfo}\PY{p}{(}\PY{n+nb}{float}\PY{p}{)}\PY{o}{.}\PY{n}{eps}
             
             \PY{k}{for} \PY{n}{i} \PY{o+ow}{in} \PY{n+nb}{range}\PY{p}{(}\PY{l+m+mi}{1}\PY{p}{,}\PY{n}{NiterMax}\PY{o}{+}\PY{l+m+mi}{1}\PY{p}{)}\PY{p}{:}
                 \PY{n}{fx} \PY{o}{=} \PY{n}{f}\PY{p}{(}\PY{n}{x}\PY{p}{)}
                 \PY{n}{dfx} \PY{o}{=} \PY{n}{df}\PY{p}{(}\PY{n}{x}\PY{p}{)}
                 
                 \PY{k}{if} \PY{n+nb}{abs}\PY{p}{(}\PY{n}{dfx}\PY{p}{)} \PY{o}{\PYZlt{}}\PY{o}{=} \PY{n}{SMALL}\PY{p}{:}
                     \PY{k}{raise} \PY{n+ne}{RuntimeError}\PY{p}{(}\PY{l+s+s2}{\PYZdq{}}\PY{l+s+s2}{Turunan f(x) sangat kecil}\PY{l+s+s2}{\PYZdq{}}\PY{p}{)}
                     
                 \PY{n}{xr} \PY{o}{=} \PY{n}{x} \PY{o}{\PYZhy{}} \PY{n}{fx}\PY{o}{/}\PY{n}{dfx}
                 
                 \PY{n+nb}{print}\PY{p}{(}\PY{l+s+s2}{\PYZdq{}}\PY{l+s+s2}{Iter = }\PY{l+s+si}{\PYZpc{}5d}\PY{l+s+s2}{, x = }\PY{l+s+si}{\PYZpc{}18.10f}\PY{l+s+s2}{, abs(f(x)) }\PY{l+s+si}{\PYZpc{}15.5e}\PY{l+s+s2}{\PYZdq{}} \PY{o}{\PYZpc{}} \PY{p}{(}\PY{n}{i}\PY{p}{,} \PY{n}{x}\PY{p}{,} \PY{n+nb}{abs}\PY{p}{(}\PY{n}{fx}\PY{p}{)}\PY{p}{)}\PY{p}{)}
                 \PY{k}{if} \PY{n+nb}{abs}\PY{p}{(}\PY{n}{fx}\PY{p}{)} \PY{o}{\PYZlt{}}\PY{o}{=} \PY{n}{TOL}\PY{p}{:}
                     \PY{n+nb}{print}\PY{p}{(}\PY{l+s+s2}{\PYZdq{}}\PY{l+s+s2}{Iterasi konvergen: akar ditemukan}\PY{l+s+s2}{\PYZdq{}}\PY{p}{)}
                     \PY{k}{return} \PY{n}{x}
                 
                 \PY{n}{x} \PY{o}{=} \PY{n}{xr}
             
             \PY{n+nb}{print}\PY{p}{(}\PY{l+s+s2}{\PYZdq{}}\PY{l+s+s2}{WARNING: Konvergensi tidak diperleh setelah }\PY{l+s+si}{\PYZpc{}d}\PY{l+s+s2}{ iterasi}\PY{l+s+s2}{\PYZdq{}} \PY{o}{\PYZpc{}} \PY{n}{NiterMax}\PY{p}{)}
             \PY{n+nb}{print}\PY{p}{(}\PY{l+s+s2}{\PYZdq{}}\PY{l+s+s2}{WARNING: Nilai tebakan akhir akan dikembalikan}\PY{l+s+s2}{\PYZdq{}}\PY{p}{)}
             \PY{k}{return} \PY{n}{x}
\end{Verbatim}


    Kita akan coba mencari akar persamaan \texttt{func\_04} dengan beberapa
tebakan awal.

    \begin{Verbatim}[commandchars=\\\{\}]
{\color{incolor}In [{\color{incolor} }]:} \PY{n}{x0} \PY{o}{=} \PY{l+m+mf}{3.0}
        \PY{n}{xr} \PY{o}{=} \PY{n}{newton\PYZus{}raphson}\PY{p}{(}\PY{n}{func\PYZus{}04}\PY{p}{,} \PY{n}{dfunc\PYZus{}04}\PY{p}{,} \PY{n}{x0}\PY{p}{)}
\end{Verbatim}


    \begin{Verbatim}[commandchars=\\\{\}]
{\color{incolor}In [{\color{incolor} }]:} \PY{n}{x0} \PY{o}{=} \PY{l+m+mf}{4.0}
        \PY{n}{xr} \PY{o}{=} \PY{n}{newton\PYZus{}raphson}\PY{p}{(}\PY{n}{func\PYZus{}04}\PY{p}{,} \PY{n}{dfunc\PYZus{}04}\PY{p}{,} \PY{n}{x0}\PY{p}{)}
\end{Verbatim}


    \begin{Verbatim}[commandchars=\\\{\}]
{\color{incolor}In [{\color{incolor} }]:} \PY{n}{x0} \PY{o}{=} \PY{l+m+mf}{10.0}
        \PY{n}{xr} \PY{o}{=} \PY{n}{newton\PYZus{}raphson}\PY{p}{(}\PY{n}{func\PYZus{}04}\PY{p}{,} \PY{n}{dfunc\PYZus{}04}\PY{p}{,} \PY{n}{x0}\PY{p}{)}
\end{Verbatim}


    Kita coba mencari akar dari \texttt{func\_01}. Kita perlu mendefinisikan
dulu turunan dari \texttt{func\_01}.

    \begin{Verbatim}[commandchars=\\\{\}]
{\color{incolor}In [{\color{incolor} }]:} \PY{k}{def} \PY{n+nf}{dfunc\PYZus{}01}\PY{p}{(}\PY{n}{x}\PY{p}{)}\PY{p}{:}
            \PY{k}{return} \PY{l+m+mi}{15}\PY{o}{*}\PY{n}{x}\PY{o}{*}\PY{o}{*}\PY{l+m+mi}{2} \PY{o}{\PYZhy{}} \PY{l+m+mi}{10}\PY{o}{*}\PY{n}{x} \PY{o}{+} \PY{l+m+mi}{6}
\end{Verbatim}


    \begin{Verbatim}[commandchars=\\\{\}]
{\color{incolor}In [{\color{incolor} }]:} \PY{n}{xr} \PY{o}{=} \PY{n}{newton\PYZus{}raphson}\PY{p}{(}\PY{n}{func\PYZus{}01}\PY{p}{,} \PY{n}{dfunc\PYZus{}01}\PY{p}{,} \PY{l+m+mf}{0.0}\PY{p}{)}
\end{Verbatim}


    \begin{Verbatim}[commandchars=\\\{\}]
{\color{incolor}In [{\color{incolor} }]:} \PY{n}{xr} \PY{o}{=} \PY{n}{newton\PYZus{}raphson}\PY{p}{(}\PY{n}{func\PYZus{}01}\PY{p}{,} \PY{n}{dfunc\PYZus{}01}\PY{p}{,} \PY{l+m+mf}{1.0}\PY{p}{)}
\end{Verbatim}


    \begin{Verbatim}[commandchars=\\\{\}]
{\color{incolor}In [{\color{incolor} }]:} \PY{n}{xr} \PY{o}{=} \PY{n}{newton\PYZus{}raphson}\PY{p}{(}\PY{n}{func\PYZus{}01}\PY{p}{,} \PY{n}{dfunc\PYZus{}01}\PY{p}{,} \PY{l+m+mf}{10.0}\PY{p}{)}
\end{Verbatim}


    Sebagai perbandingan dengan metode fixed-point, kita akan menghitung
akar dari persamaan \(f(x) = e^{-x} - x\). Turunan pertama dari fungsi
ini adalah \(f'(x) = -e^{-x} - 1\)

    \begin{Verbatim}[commandchars=\\\{\}]
{\color{incolor}In [{\color{incolor} }]:} \PY{k}{def} \PY{n+nf}{func\PYZus{}05}\PY{p}{(}\PY{n}{x}\PY{p}{)}\PY{p}{:}
            \PY{k}{return} \PY{n}{np}\PY{o}{.}\PY{n}{exp}\PY{p}{(}\PY{o}{\PYZhy{}}\PY{n}{x}\PY{p}{)} \PY{o}{\PYZhy{}} \PY{n}{x}
        
        \PY{k}{def} \PY{n+nf}{dfunc\PYZus{}05}\PY{p}{(}\PY{n}{x}\PY{p}{)}\PY{p}{:}
            \PY{k}{return} \PY{o}{\PYZhy{}}\PY{n}{np}\PY{o}{.}\PY{n}{exp}\PY{p}{(}\PY{o}{\PYZhy{}}\PY{n}{x}\PY{p}{)} \PY{o}{\PYZhy{}} \PY{l+m+mi}{1}
\end{Verbatim}


    \begin{Verbatim}[commandchars=\\\{\}]
{\color{incolor}In [{\color{incolor} }]:} \PY{n}{x0} \PY{o}{=} \PY{l+m+mf}{0.0}
        \PY{n}{xr} \PY{o}{=} \PY{n}{newton\PYZus{}raphson}\PY{p}{(}\PY{n}{func\PYZus{}05}\PY{p}{,} \PY{n}{dfunc\PYZus{}05}\PY{p}{,} \PY{n}{x0}\PY{p}{)}
\end{Verbatim}


    Dapat diamati bahwa metode Newton-Raphson konvergen dengan cepat
dibandingkan dengan metode fixed-point.

    \hypertarget{metode-secant}{%
\subsection{Metode secant}\label{metode-secant}}

    Metode secant menggunakan ide yang sama dengan metode Newton-Raphson.
Perbedaannya adalah metode secant menggunakan aproksimasi terhadap
turunan pertama dari \(f(x)\). \[
f'(x) \approx \frac{f(x_{i-1}) - f(x_{i})}{x_{i-1} - x_{i}}
\]

    \begin{Verbatim}[commandchars=\\\{\}]
{\color{incolor}In [{\color{incolor}16}]:} \PY{k}{def} \PY{n+nf}{secant}\PY{p}{(}\PY{n}{f}\PY{p}{,} \PY{n}{x}\PY{p}{,} \PY{n}{TOL}\PY{o}{=}\PY{l+m+mf}{1e\PYZhy{}10}\PY{p}{,} \PY{n}{NiterMax}\PY{o}{=}\PY{l+m+mi}{100}\PY{p}{,} \PY{n}{DELTA}\PY{o}{=}\PY{l+m+mf}{0.001}\PY{p}{)}\PY{p}{:}
             
             \PY{n}{SMALL} \PY{o}{=} \PY{n}{np}\PY{o}{.}\PY{n}{finfo}\PY{p}{(}\PY{n+nb}{float}\PY{p}{)}\PY{o}{.}\PY{n}{eps}
             
             \PY{c+c1}{\PYZsh{} Untuk aproksimasi turunan pertama}
             \PY{n}{x\PYZus{}old} \PY{o}{=} \PY{n}{x} \PY{o}{+} \PY{n}{DELTA}
                 
             \PY{k}{for} \PY{n}{i} \PY{o+ow}{in} \PY{n+nb}{range}\PY{p}{(}\PY{l+m+mi}{1}\PY{p}{,}\PY{n}{NiterMax}\PY{o}{+}\PY{l+m+mi}{1}\PY{p}{)}\PY{p}{:}
                 
                 \PY{n}{fx} \PY{o}{=} \PY{n}{f}\PY{p}{(}\PY{n}{x}\PY{p}{)}
                 \PY{n}{fx\PYZus{}old} \PY{o}{=} \PY{n}{f}\PY{p}{(}\PY{n}{x\PYZus{}old}\PY{p}{)}
         
                 \PY{n}{dfx} \PY{o}{=} \PY{p}{(}\PY{n}{fx\PYZus{}old} \PY{o}{\PYZhy{}} \PY{n}{fx}\PY{p}{)}\PY{o}{/}\PY{p}{(}\PY{n}{x\PYZus{}old} \PY{o}{\PYZhy{}} \PY{n}{x}\PY{p}{)}
                 
                 \PY{k}{if} \PY{n+nb}{abs}\PY{p}{(}\PY{n}{dfx}\PY{p}{)} \PY{o}{\PYZlt{}}\PY{o}{=} \PY{n}{SMALL}\PY{p}{:}
                     \PY{k}{raise} \PY{n+ne}{RuntimeError}\PY{p}{(}\PY{l+s+s2}{\PYZdq{}}\PY{l+s+s2}{Turunan f(x) sangat kecil}\PY{l+s+s2}{\PYZdq{}}\PY{p}{)}
                     
                 \PY{n}{xr} \PY{o}{=} \PY{n}{x} \PY{o}{\PYZhy{}} \PY{n}{fx}\PY{o}{/}\PY{n}{dfx}
                 
                 \PY{n+nb}{print}\PY{p}{(}\PY{l+s+s2}{\PYZdq{}}\PY{l+s+s2}{Iter = }\PY{l+s+si}{\PYZpc{}5d}\PY{l+s+s2}{, x = }\PY{l+s+si}{\PYZpc{}18.10f}\PY{l+s+s2}{, abs(f(x)) }\PY{l+s+si}{\PYZpc{}15.5e}\PY{l+s+s2}{\PYZdq{}} \PY{o}{\PYZpc{}} \PY{p}{(}\PY{n}{i}\PY{p}{,} \PY{n}{x}\PY{p}{,} \PY{n+nb}{abs}\PY{p}{(}\PY{n}{fx}\PY{p}{)}\PY{p}{)}\PY{p}{)}
                 \PY{k}{if} \PY{n+nb}{abs}\PY{p}{(}\PY{n}{fx}\PY{p}{)} \PY{o}{\PYZlt{}}\PY{o}{=} \PY{n}{TOL}\PY{p}{:}
                     \PY{n+nb}{print}\PY{p}{(}\PY{l+s+s2}{\PYZdq{}}\PY{l+s+s2}{Iterasi konvergen: akar ditemukan}\PY{l+s+s2}{\PYZdq{}}\PY{p}{)}
                     \PY{k}{return} \PY{n}{x}
                 
                 \PY{n}{x\PYZus{}old} \PY{o}{=} \PY{n}{x}
                 \PY{n}{x} \PY{o}{=} \PY{n}{xr}
             
             \PY{n+nb}{print}\PY{p}{(}\PY{l+s+s2}{\PYZdq{}}\PY{l+s+s2}{WARNING: Konvergensi tidak diperleh setelah }\PY{l+s+si}{\PYZpc{}d}\PY{l+s+s2}{ iterasi}\PY{l+s+s2}{\PYZdq{}} \PY{o}{\PYZpc{}} \PY{n}{NiterMax}\PY{p}{)}
             \PY{n+nb}{print}\PY{p}{(}\PY{l+s+s2}{\PYZdq{}}\PY{l+s+s2}{WARNING: Nilai tebakan akhir akan dikembalikan}\PY{l+s+s2}{\PYZdq{}}\PY{p}{)}
             \PY{k}{return} \PY{n}{x}
\end{Verbatim}


    \begin{Verbatim}[commandchars=\\\{\}]
{\color{incolor}In [{\color{incolor} }]:} \PY{n}{x0} \PY{o}{=} \PY{l+m+mf}{0.0}
        \PY{n}{xr} \PY{o}{=} \PY{n}{secant}\PY{p}{(}\PY{n}{func\PYZus{}05}\PY{p}{,} \PY{n}{x0}\PY{p}{)}
\end{Verbatim}


    \begin{Verbatim}[commandchars=\\\{\}]
{\color{incolor}In [{\color{incolor} }]:} \PY{n}{x0} \PY{o}{=} \PY{l+m+mf}{0.0}
        \PY{n}{xr} \PY{o}{=} \PY{n}{secant}\PY{p}{(}\PY{n}{func\PYZus{}01}\PY{p}{,} \PY{n}{x0}\PY{p}{)}
\end{Verbatim}


    \begin{Verbatim}[commandchars=\\\{\}]
{\color{incolor}In [{\color{incolor} }]:} \PY{n}{x0} \PY{o}{=} \PY{l+m+mf}{1.0}
        \PY{n}{xr} \PY{o}{=} \PY{n}{secant}\PY{p}{(}\PY{n}{func\PYZus{}04}\PY{p}{,} \PY{n}{x0}\PY{p}{)}
\end{Verbatim}


    \hypertarget{latihan-soal}{%
\section{Latihan Soal}\label{latihan-soal}}

    \hypertarget{soal-1}{%
\subsection{Soal 1}\label{soal-1}}

    Impedansi dari rangkaian paralel RLC dinyatakan oleh persamaan \[
\frac{1}{Z} = \sqrt{\frac{1}{R^2} +
\left( \omega C - \frac{1}{\omega L} \right)^2 }
\] Cari frekuensi angular w untuk Z = 75 ohm, R = 225 ohm,
\(C = 0.6\times10^{−6}\) F, and L = 0.5 H. Untuk metoda grafis kerjakan
sampai ketelitian 2 angka dibelakang koma. (pentunjuk: akar berada di
sekitar 160)

    \hypertarget{jawaban-soal-1}{%
\subsection{Jawaban Soal 1}\label{jawaban-soal-1}}

    \begin{Verbatim}[commandchars=\\\{\}]
{\color{incolor}In [{\color{incolor}7}]:} \PY{k}{def} \PY{n+nf}{func\PYZus{}soal\PYZus{}01}\PY{p}{(}\PY{n}{omega}\PY{p}{)}\PY{p}{:}
            \PY{n}{Z} \PY{o}{=} \PY{l+m+mf}{75.0}
            \PY{n}{R} \PY{o}{=} \PY{l+m+mf}{225.0}
            \PY{n}{C} \PY{o}{=} \PY{l+m+mf}{0.6e\PYZhy{}6}
            \PY{n}{L} \PY{o}{=} \PY{l+m+mf}{0.5}
            \PY{n}{term1} \PY{o}{=} \PY{n}{np}\PY{o}{.}\PY{n}{sqrt}\PY{p}{(} \PY{l+m+mi}{1}\PY{o}{/}\PY{n}{R}\PY{o}{*}\PY{o}{*}\PY{l+m+mi}{2} \PY{o}{+} \PY{p}{(}\PY{n}{omega}\PY{o}{*}\PY{n}{C} \PY{o}{\PYZhy{}} \PY{l+m+mi}{1}\PY{o}{/}\PY{p}{(}\PY{n}{omega}\PY{o}{*}\PY{n}{L}\PY{p}{)}\PY{p}{)}\PY{o}{*}\PY{o}{*}\PY{l+m+mi}{2} \PY{p}{)}
            \PY{k}{return} \PY{n}{term1} \PY{o}{\PYZhy{}} \PY{l+m+mi}{1}\PY{o}{/}\PY{n}{Z}
\end{Verbatim}


    \begin{Verbatim}[commandchars=\\\{\}]
{\color{incolor}In [{\color{incolor}10}]:} \PY{n}{plt}\PY{o}{.}\PY{n}{clf}\PY{p}{(}\PY{p}{)}
         \PY{n}{omega} \PY{o}{=} \PY{n}{np}\PY{o}{.}\PY{n}{linspace}\PY{p}{(}\PY{l+m+mi}{100}\PY{p}{,}\PY{l+m+mi}{200}\PY{p}{,}\PY{l+m+mi}{1000}\PY{p}{)}
         \PY{n}{f} \PY{o}{=} \PY{n}{func\PYZus{}soal\PYZus{}01}\PY{p}{(}\PY{n}{omega}\PY{p}{)}
         \PY{n}{plt}\PY{o}{.}\PY{n}{grid}\PY{p}{(}\PY{p}{)}
         \PY{n}{plt}\PY{o}{.}\PY{n}{plot}\PY{p}{(}\PY{n}{omega}\PY{p}{,} \PY{n}{f}\PY{p}{)}
\end{Verbatim}


\begin{Verbatim}[commandchars=\\\{\}]
{\color{outcolor}Out[{\color{outcolor}10}]:} [<matplotlib.lines.Line2D at 0x7f19d17cd160>]
\end{Verbatim}
            
    \begin{center}
    \adjustimage{max size={0.9\linewidth}{0.9\paperheight}}{NOTES_AkarPersamaan_files/NOTES_AkarPersamaan_106_1.pdf}
    \end{center}
    { \hspace*{\fill} \\}
    
    \hypertarget{solusi-dengan-metode-bisection}{%
\subsubsection{Solusi dengan metode
bisection}\label{solusi-dengan-metode-bisection}}

    \begin{Verbatim}[commandchars=\\\{\}]
{\color{incolor}In [{\color{incolor}12}]:} \PY{n}{omega\PYZus{}root} \PY{o}{=} \PY{n}{bisection}\PY{p}{(}\PY{n}{func\PYZus{}soal\PYZus{}01}\PY{p}{,}\PY{l+m+mi}{150}\PY{p}{,}\PY{l+m+mi}{170}\PY{p}{)}
\end{Verbatim}


    \begin{Verbatim}[commandchars=\\\{\}]
Iter =     1, xr =     160.0000000000, abs(fxr) =     1.57131e-04
Iter =     2, xr =     155.0000000000, abs(fxr) =     2.25980e-04
Iter =     3, xr =     157.5000000000, abs(fxr) =     3.12356e-05
Iter =     4, xr =     158.7500000000, abs(fxr) =     6.37261e-05
Iter =     5, xr =     158.1250000000, abs(fxr) =     1.64421e-05
Iter =     6, xr =     157.8125000000, abs(fxr) =     7.34719e-06
Iter =     7, xr =     157.9687500000, abs(fxr) =     4.55982e-06
Iter =     8, xr =     157.8906250000, abs(fxr) =     1.39060e-06
Iter =     9, xr =     157.9296875000, abs(fxr) =     1.58538e-06
Iter =    10, xr =     157.9101562500, abs(fxr) =     9.75869e-08
Iter =    11, xr =     157.9003906250, abs(fxr) =     6.46456e-07
Iter =    12, xr =     157.9052734375, abs(fxr) =     2.74423e-07
Iter =    13, xr =     157.9077148438, abs(fxr) =     8.84149e-08
Iter =    14, xr =     157.9089355469, abs(fxr) =     4.58675e-09
Iter =    15, xr =     157.9083251953, abs(fxr) =     4.19139e-08
Iter =    16, xr =     157.9086303711, abs(fxr) =     1.86635e-08
Iter =    17, xr =     157.9087829590, abs(fxr) =     7.03837e-09
Iter =    18, xr =     157.9088592529, abs(fxr) =     1.22581e-09
Iter =    19, xr =     157.9088973999, abs(fxr) =     1.68047e-09
Iter =    20, xr =     157.9088783264, abs(fxr) =     2.27334e-10
Iter =    21, xr =     157.9088687897, abs(fxr) =     4.99236e-10
Iter =    22, xr =     157.9088735580, abs(fxr) =     1.35951e-10
Iterasi konvergen: akar ditemukan

    \end{Verbatim}

    \hypertarget{solusi-dengan-metode-regula-falsi}{%
\subsubsection{Solusi dengan metode regula
falsi}\label{solusi-dengan-metode-regula-falsi}}

    \begin{Verbatim}[commandchars=\\\{\}]
{\color{incolor}In [{\color{incolor}17}]:} \PY{n}{omega\PYZus{}root} \PY{o}{=} \PY{n}{regula\PYZus{}falsi}\PY{p}{(}\PY{n}{func\PYZus{}soal\PYZus{}01}\PY{p}{,}\PY{l+m+mi}{150}\PY{p}{,}\PY{l+m+mi}{170}\PY{p}{)}
\end{Verbatim}


    \begin{Verbatim}[commandchars=\\\{\}]
Iter =     1, xr =     158.5447406311, abs(fxr) =     4.82405e-05
Iter =     2, xr =     157.9422131525, abs(fxr) =     2.53932e-06
Iter =     3, xr =     157.9106229293, abs(fxr) =     1.33141e-07
Iter =     4, xr =     157.9089669512, abs(fxr) =     6.97933e-09
Iter =     5, xr =     157.9088801446, abs(fxr) =     3.65857e-10
Iterasi konvergen: akar ditemukan

    \end{Verbatim}

    \hypertarget{solusi-dengan-metode-secant}{%
\subsubsection{Solusi dengan metode
secant}\label{solusi-dengan-metode-secant}}

    \begin{Verbatim}[commandchars=\\\{\}]
{\color{incolor}In [{\color{incolor}18}]:} \PY{n}{omega\PYZus{}root} \PY{o}{=} \PY{n}{secant}\PY{p}{(}\PY{n}{func\PYZus{}soal\PYZus{}01}\PY{p}{,} \PY{l+m+mi}{150}\PY{p}{)}
\end{Verbatim}


    \begin{Verbatim}[commandchars=\\\{\}]
Iter =     1, x =     150.0000000000, abs(f(x))     6.35882e-04
Iter =     2, x =     157.4952274036, abs(f(x))     3.16012e-05
Iter =     3, x =     157.8871942885, abs(f(x))     1.65204e-06
Iter =     4, x =     157.9088157744, abs(f(x))     4.53828e-09
Iter =     5, x =     157.9088753339, abs(f(x))     6.53620e-13
Iterasi konvergen: akar ditemukan

    \end{Verbatim}

    \begin{Verbatim}[commandchars=\\\{\}]
{\color{incolor}In [{\color{incolor}19}]:} \PY{n}{func\PYZus{}soal\PYZus{}01}\PY{p}{(}\PY{n}{omega\PYZus{}root}\PY{p}{)}
\end{Verbatim}


\begin{Verbatim}[commandchars=\\\{\}]
{\color{outcolor}Out[{\color{outcolor}19}]:} 6.536195196193972e-13
\end{Verbatim}
            
    \hypertarget{soal-6}{%
\subsection{Soal 6}\label{soal-6}}

    Gaya \(F\) yang bekerja antara partikel bermuatan \(q\) dengan piringan
bulat dengan jari-jari \(R\) dan rapat muatan \(Q\) diberikan oleh
persamaan: \[
F = \frac{Qq}{2\epsilon_{0}}\left(
1 - \frac{z}{\sqrt{z^2 + R^2}}
\right)
\] dimana \(\epsilon_{0} = 0.885 \times 10^{-12}\)
\(\mathrm{C}^{2}/(\mathrm{Nm}^{2})\) adalah konstanta permitivitas dan
\(z\) adalah jarak partikel terhadap piringan. Tentukan jarak \(z\) jika
\(F = 0.3\) newton \(Q = 9.4 \times 10^{-6}\,\mathrm{C/m}^2\)
\(q = 2.4 \times 10^{-5}\)C dan \(R=0.1\) m.

    \hypertarget{jawaban-soal-6}{%
\subsection{Jawaban Soal 6}\label{jawaban-soal-6}}

    \begin{Verbatim}[commandchars=\\\{\}]
{\color{incolor}In [{\color{incolor}55}]:} \PY{k}{def} \PY{n+nf}{func\PYZus{}soal\PYZus{}06}\PY{p}{(}\PY{n}{z}\PY{p}{)}\PY{p}{:}
             \PY{n}{F} \PY{o}{=} \PY{l+m+mf}{0.3}
             \PY{n}{Q} \PY{o}{=} \PY{l+m+mf}{9.4e\PYZhy{}6}
             \PY{n}{q} \PY{o}{=} \PY{l+m+mf}{2.4e\PYZhy{}5}
             \PY{n}{R} \PY{o}{=} \PY{l+m+mf}{0.1}
             \PY{n}{ϵ0} \PY{o}{=} \PY{l+m+mf}{0.885e\PYZhy{}12}
             \PY{n}{RHS} \PY{o}{=} \PY{n}{Q}\PY{o}{*}\PY{n}{q}\PY{o}{/}\PY{p}{(}\PY{l+m+mi}{2}\PY{o}{*}\PY{n}{ϵ0}\PY{p}{)}\PY{o}{*}\PY{p}{(}\PY{l+m+mi}{1} \PY{o}{\PYZhy{}} \PY{n}{z}\PY{o}{/}\PY{n}{np}\PY{o}{.}\PY{n}{sqrt}\PY{p}{(}\PY{n}{z}\PY{o}{*}\PY{o}{*}\PY{l+m+mi}{2} \PY{o}{+} \PY{n}{R}\PY{o}{*}\PY{o}{*}\PY{l+m+mi}{2}\PY{p}{)}\PY{p}{)}
             \PY{k}{return} \PY{n}{F} \PY{o}{\PYZhy{}} \PY{n}{RHS}
\end{Verbatim}


    \begin{Verbatim}[commandchars=\\\{\}]
{\color{incolor}In [{\color{incolor}56}]:} \PY{k+kn}{import} \PY{n+nn}{matplotlib}\PY{n+nn}{.}\PY{n+nn}{pyplot} \PY{k}{as} \PY{n+nn}{plt}
         \PY{n}{z} \PY{o}{=} \PY{n}{np}\PY{o}{.}\PY{n}{linspace}\PY{p}{(}\PY{l+m+mf}{0.5}\PY{p}{,}\PY{l+m+mf}{10.0}\PY{p}{,}\PY{l+m+mi}{1000}\PY{p}{)}
         \PY{n}{ff} \PY{o}{=} \PY{n}{func\PYZus{}soal\PYZus{}06}\PY{p}{(}\PY{n}{z}\PY{p}{)}
         \PY{n}{plt}\PY{o}{.}\PY{n}{clf}\PY{p}{(}\PY{p}{)}
         \PY{n}{plt}\PY{o}{.}\PY{n}{grid}\PY{p}{(}\PY{p}{)}
         \PY{n}{plt}\PY{o}{.}\PY{n}{plot}\PY{p}{(}\PY{n}{z}\PY{p}{,} \PY{n}{ff}\PY{p}{)}
\end{Verbatim}


\begin{Verbatim}[commandchars=\\\{\}]
{\color{outcolor}Out[{\color{outcolor}56}]:} [<matplotlib.lines.Line2D at 0x7f84076e72e8>]
\end{Verbatim}
            
    \begin{center}
    \adjustimage{max size={0.9\linewidth}{0.9\paperheight}}{NOTES_AkarPersamaan_files/NOTES_AkarPersamaan_118_1.pdf}
    \end{center}
    { \hspace*{\fill} \\}
    
    \begin{Verbatim}[commandchars=\\\{\}]
{\color{incolor}In [{\color{incolor}57}]:} \PY{n}{func\PYZus{}soal\PYZus{}06}\PY{p}{(}\PY{l+m+mf}{4.0}\PY{p}{)}
\end{Verbatim}

\texttt{\color{outcolor}Out[{\color{outcolor}57}]:}
    
    $$0.26018815235731824$$

    

    \begin{Verbatim}[commandchars=\\\{\}]
{\color{incolor}In [{\color{incolor}23}]:} \PY{n}{xroot} \PY{o}{=} \PY{n}{bisection}\PY{p}{(}\PY{n}{func\PYZus{}soal\PYZus{}06}\PY{p}{,} \PY{l+m+mf}{0.5}\PY{p}{,} \PY{l+m+mf}{1.5}\PY{p}{)}
\end{Verbatim}


    \begin{Verbatim}[commandchars=\\\{\}]
Iter =     1, xr =       1.0000000000, abs(fxr) =     3.32548e-01
Iter =     2, xr =       1.2500000000, abs(fxr) =     1.05917e-01
Iter =     3, xr =       1.3750000000, abs(fxr) =     3.57467e-02
Iter =     4, xr =       1.4375000000, abs(fxr) =     7.28923e-03
Iter =     5, xr =       1.4687500000, abs(fxr) =     5.60301e-03
Iter =     6, xr =       1.4531250000, abs(fxr) =     7.39637e-04
Iter =     7, xr =       1.4609375000, abs(fxr) =     2.45701e-03
Iter =     8, xr =       1.4570312500, abs(fxr) =     8.65082e-04
Iter =     9, xr =       1.4550781250, abs(fxr) =     6.43303e-05
Iter =    10, xr =       1.4541015625, abs(fxr) =     3.37250e-04
Iter =    11, xr =       1.4545898438, abs(fxr) =     1.36359e-04
Iter =    12, xr =       1.4548339844, abs(fxr) =     3.59893e-05
Iter =    13, xr =       1.4549560547, abs(fxr) =     1.41768e-05
Iter =    14, xr =       1.4548950195, abs(fxr) =     1.09047e-05
Iter =    15, xr =       1.4549255371, abs(fxr) =     1.63643e-06
Iter =    16, xr =       1.4549102783, abs(fxr) =     4.63404e-06
Iter =    17, xr =       1.4549179077, abs(fxr) =     1.49878e-06
Iter =    18, xr =       1.4549217224, abs(fxr) =     6.88345e-08
Iter =    19, xr =       1.4549198151, abs(fxr) =     7.14970e-07
Iter =    20, xr =       1.4549207687, abs(fxr) =     3.23067e-07
Iter =    21, xr =       1.4549212456, abs(fxr) =     1.27116e-07
Iter =    22, xr =       1.4549214840, abs(fxr) =     2.91409e-08
Iter =    23, xr =       1.4549216032, abs(fxr) =     1.98468e-08
Iter =    24, xr =       1.4549215436, abs(fxr) =     4.64703e-09
Iter =    25, xr =       1.4549215734, abs(fxr) =     7.59991e-09
Iter =    26, xr =       1.4549215585, abs(fxr) =     1.47643e-09
Iter =    27, xr =       1.4549215510, abs(fxr) =     1.58530e-09
Iterasi konvergen: akar ditemukan

    \end{Verbatim}

    \begin{Verbatim}[commandchars=\\\{\}]
{\color{incolor}In [{\color{incolor}25}]:} \PY{n}{xroot} \PY{o}{=} \PY{n}{regula\PYZus{}falsi}\PY{p}{(}\PY{n}{func\PYZus{}soal\PYZus{}06}\PY{p}{,} \PY{l+m+mf}{0.5}\PY{p}{,} \PY{l+m+mf}{1.5}\PY{p}{)}
\end{Verbatim}


    \begin{Verbatim}[commandchars=\\\{\}]
Iter =     1, xr =       1.4919276117, abs(fxr) =     1.46486e-02
Iter =     2, xr =       1.4852921021, abs(fxr) =     1.21020e-02
Iter =     3, xr =       1.4798404998, abs(fxr) =     9.98413e-03
Iter =     4, xr =       1.4753634762, abs(fxr) =     8.22739e-03
Iter =     5, xr =       1.4716881019, abs(fxr) =     6.77326e-03
Iter =     6, xr =       1.4686717138, abs(fxr) =     5.57173e-03
Iter =     7, xr =       1.4661967529, abs(fxr) =     4.58035e-03
Iter =     8, xr =       1.4641664384, abs(fxr) =     3.76334e-03
Iter =     9, xr =       1.4625011583, abs(fxr) =     3.09069e-03
Iter =    10, xr =       1.4611354660, abs(fxr) =     2.53734e-03
Iter =    11, xr =       1.4600155888, abs(fxr) =     2.08244e-03
Iter =    12, xr =       1.4590973653, abs(fxr) =     1.70868e-03
Iter =    13, xr =       1.4583445403, abs(fxr) =     1.40171e-03
Iter =    14, xr =       1.4577273586, abs(fxr) =     1.14970e-03
Iter =    15, xr =       1.4572214057, abs(fxr) =     9.42872e-04
Iter =    16, xr =       1.4568066529, abs(fxr) =     7.73163e-04
Iter =    17, xr =       1.4564666726, abs(fxr) =     6.33943e-04
Iter =    18, xr =       1.4561879924, abs(fxr) =     5.19752e-04
Iter =    19, xr =       1.4559595648, abs(fxr) =     4.26104e-04
Iter =    20, xr =       1.4557723315, abs(fxr) =     3.49312e-04
Iter =    21, xr =       1.4556188660, abs(fxr) =     2.86347e-04
Iter =    22, xr =       1.4554930799, abs(fxr) =     2.34724e-04
Iter =    23, xr =       1.4553899818, abs(fxr) =     1.92402e-04
Iter =    24, xr =       1.4553054802, abs(fxr) =     1.57708e-04
Iter =    25, xr =       1.4552362212, abs(fxr) =     1.29267e-04
Iter =    26, xr =       1.4551794557, abs(fxr) =     1.05953e-04
Iter =    27, xr =       1.4551329302, abs(fxr) =     8.68434e-05
Iter =    28, xr =       1.4550947977, abs(fxr) =     7.11795e-05
Iter =    29, xr =       1.4550635441, abs(fxr) =     5.83403e-05
Iter =    30, xr =       1.4550379286, abs(fxr) =     4.78167e-05
Iter =    31, xr =       1.4550169342, abs(fxr) =     3.91912e-05
Iter =    32, xr =       1.4549997273, abs(fxr) =     3.21215e-05
Iter =    33, xr =       1.4549856245, abs(fxr) =     2.63269e-05
Iter =    34, xr =       1.4549740659, abs(fxr) =     2.15776e-05
Iter =    35, xr =       1.4549645926, abs(fxr) =     1.76850e-05
Iter =    36, xr =       1.4549568283, abs(fxr) =     1.44947e-05
Iter =    37, xr =       1.4549504647, abs(fxr) =     1.18798e-05
Iter =    38, xr =       1.4549452491, abs(fxr) =     9.73663e-06
Iter =    39, xr =       1.4549409745, abs(fxr) =     7.98010e-06
Iter =    40, xr =       1.4549374711, abs(fxr) =     6.54045e-06
Iter =    41, xr =       1.4549345997, abs(fxr) =     5.36052e-06
Iter =    42, xr =       1.4549322463, abs(fxr) =     4.39345e-06
Iter =    43, xr =       1.4549303175, abs(fxr) =     3.60084e-06
Iter =    44, xr =       1.4549287366, abs(fxr) =     2.95122e-06
Iter =    45, xr =       1.4549274410, abs(fxr) =     2.41880e-06
Iter =    46, xr =       1.4549263791, abs(fxr) =     1.98243e-06
Iter =    47, xr =       1.4549255088, abs(fxr) =     1.62479e-06
Iter =    48, xr =       1.4549247955, abs(fxr) =     1.33166e-06
Iter =    49, xr =       1.4549242108, abs(fxr) =     1.09142e-06
Iter =    50, xr =       1.4549237317, abs(fxr) =     8.94518e-07
Iter =    51, xr =       1.4549233390, abs(fxr) =     7.33140e-07
Iter =    52, xr =       1.4549230171, abs(fxr) =     6.00876e-07
Iter =    53, xr =       1.4549227533, abs(fxr) =     4.92473e-07
Iter =    54, xr =       1.4549225371, abs(fxr) =     4.03626e-07
Iter =    55, xr =       1.4549223599, abs(fxr) =     3.30809e-07
Iter =    56, xr =       1.4549222147, abs(fxr) =     2.71128e-07
Iter =    57, xr =       1.4549220957, abs(fxr) =     2.22214e-07
Iter =    58, xr =       1.4549219981, abs(fxr) =     1.82125e-07
Iter =    59, xr =       1.4549219181, abs(fxr) =     1.49268e-07
Iter =    60, xr =       1.4549218526, abs(fxr) =     1.22339e-07
Iter =    61, xr =       1.4549217989, abs(fxr) =     1.00268e-07
Iter =    62, xr =       1.4549217549, abs(fxr) =     8.21787e-08
Iter =    63, xr =       1.4549217188, abs(fxr) =     6.73530e-08
Iter =    64, xr =       1.4549216892, abs(fxr) =     5.52019e-08
Iter =    65, xr =       1.4549216650, abs(fxr) =     4.52430e-08
Iter =    66, xr =       1.4549216451, abs(fxr) =     3.70808e-08
Iter =    67, xr =       1.4549216289, abs(fxr) =     3.03911e-08
Iter =    68, xr =       1.4549216155, abs(fxr) =     2.49083e-08
Iter =    69, xr =       1.4549216046, abs(fxr) =     2.04146e-08
Iter =    70, xr =       1.4549215956, abs(fxr) =     1.67317e-08
Iter =    71, xr =       1.4549215883, abs(fxr) =     1.37131e-08
Iter =    72, xr =       1.4549215823, abs(fxr) =     1.12392e-08
Iter =    73, xr =       1.4549215773, abs(fxr) =     9.21150e-09
Iter =    74, xr =       1.4549215733, abs(fxr) =     7.54968e-09
Iter =    75, xr =       1.4549215700, abs(fxr) =     6.18765e-09
Iter =    76, xr =       1.4549215672, abs(fxr) =     5.07135e-09
Iter =    77, xr =       1.4549215650, abs(fxr) =     4.15644e-09
Iter =    78, xr =       1.4549215632, abs(fxr) =     3.40657e-09
Iter =    79, xr =       1.4549215617, abs(fxr) =     2.79201e-09
Iter =    80, xr =       1.4549215605, abs(fxr) =     2.28830e-09
Iter =    81, xr =       1.4549215595, abs(fxr) =     1.87547e-09
Iter =    82, xr =       1.4549215586, abs(fxr) =     1.53711e-09
Iter =    83, xr =       1.4549215580, abs(fxr) =     1.25982e-09
Iter =    84, xr =       1.4549215574, abs(fxr) =     1.03253e-09
Iter =    85, xr =       1.4549215570, abs(fxr) =     8.46250e-10
Iter =    86, xr =       1.4549215566, abs(fxr) =     6.93579e-10
Iter =    87, xr =       1.4549215563, abs(fxr) =     5.68444e-10
Iter =    88, xr =       1.4549215560, abs(fxr) =     4.65909e-10
Iter =    89, xr =       1.4549215558, abs(fxr) =     3.81854e-10
Iter =    90, xr =       1.4549215557, abs(fxr) =     3.12969e-10
Iter =    91, xr =       1.4549215555, abs(fxr) =     2.56494e-10
Iter =    92, xr =       1.4549215554, abs(fxr) =     2.10207e-10
Iter =    93, xr =       1.4549215553, abs(fxr) =     1.72311e-10
Iter =    94, xr =       1.4549215553, abs(fxr) =     1.41208e-10
Iter =    95, xr =       1.4549215552, abs(fxr) =     1.15751e-10
Iterasi konvergen: akar ditemukan

    \end{Verbatim}

    \begin{Verbatim}[commandchars=\\\{\}]
{\color{incolor}In [{\color{incolor}21}]:} \PY{n}{xroot} \PY{o}{=} \PY{n}{secant}\PY{p}{(}\PY{n}{func\PYZus{}soal\PYZus{}06}\PY{p}{,} \PY{l+m+mf}{1.0}\PY{p}{)}
\end{Verbatim}


    \begin{Verbatim}[commandchars=\\\{\}]
Iter =     1, x =       1.0000000000, abs(f(x))     3.32548e-01
Iter =     2, x =       1.2652254171, abs(f(x))     9.62516e-02
Iter =     3, x =       1.3732608814, abs(f(x))     3.65943e-02
Iter =     4, x =       1.4395306564, abs(f(x))     6.42602e-03
Iter =     5, x =       1.4536465137, abs(f(x))     5.24650e-04
Iter =     6, x =       1.4549014568, abs(f(x))     8.25927e-06
Iter =     7, x =       1.4549215286, abs(f(x))     1.08043e-08
Iter =     8, x =       1.4549215549, abs(f(x))     2.27263e-13
Iterasi konvergen: akar ditemukan

    \end{Verbatim}

    \hypertarget{spherical-bessel}{%
\subsection{Spherical Bessel}\label{spherical-bessel}}

    \begin{Verbatim}[commandchars=\\\{\}]
{\color{incolor}In [{\color{incolor}26}]:} \PY{k+kn}{import} \PY{n+nn}{sympy}
\end{Verbatim}


    \begin{Verbatim}[commandchars=\\\{\}]
{\color{incolor}In [{\color{incolor}40}]:} \PY{n}{sympy}\PY{o}{.}\PY{n}{init\PYZus{}printing}\PY{p}{(}\PY{n}{use\PYZus{}latex}\PY{o}{=}\PY{k+kc}{True}\PY{p}{)}
\end{Verbatim}


    \begin{Verbatim}[commandchars=\\\{\}]
{\color{incolor}In [{\color{incolor}41}]:} \PY{n}{x} \PY{o}{=} \PY{n}{sympy}\PY{o}{.}\PY{n}{symbols}\PY{p}{(}\PY{l+s+s2}{\PYZdq{}}\PY{l+s+s2}{x}\PY{l+s+s2}{\PYZdq{}}\PY{p}{)}
\end{Verbatim}


    \begin{Verbatim}[commandchars=\\\{\}]
{\color{incolor}In [{\color{incolor}42}]:} \PY{n}{j0} \PY{o}{=} \PY{n}{sympy}\PY{o}{.}\PY{n}{sin}\PY{p}{(}\PY{n}{x}\PY{p}{)}\PY{o}{/}\PY{n}{x}
         \PY{n}{j0}
\end{Verbatim}

\texttt{\color{outcolor}Out[{\color{outcolor}42}]:}
    
    $$\frac{1}{x} \sin{\left (x \right )}$$

    

    \begin{Verbatim}[commandchars=\\\{\}]
{\color{incolor}In [{\color{incolor}49}]:} \PY{n}{j1} \PY{o}{=} \PY{p}{(}\PY{l+m+mi}{1}\PY{o}{/}\PY{n}{x}\PY{p}{)}\PY{o}{*}\PY{n}{sympy}\PY{o}{.}\PY{n}{diff}\PY{p}{(}\PY{n}{j0}\PY{p}{,}\PY{n}{x}\PY{p}{)}\PY{o}{*}\PY{p}{(}\PY{o}{\PYZhy{}}\PY{n}{x}\PY{p}{)}
         \PY{n}{j1}
\end{Verbatim}

\texttt{\color{outcolor}Out[{\color{outcolor}49}]:}
    
    $$- \frac{1}{x} \cos{\left (x \right )} + \frac{1}{x^{2}} \sin{\left (x \right )}$$

    

    \begin{Verbatim}[commandchars=\\\{\}]
{\color{incolor}In [{\color{incolor}51}]:} \PY{n}{j2} \PY{o}{=} \PY{p}{(}\PY{l+m+mi}{1}\PY{o}{/}\PY{n}{x}\PY{p}{)}\PY{o}{*}\PY{n}{sympy}\PY{o}{.}\PY{n}{diff}\PY{p}{(}\PY{n}{j1}\PY{p}{,}\PY{n}{x}\PY{p}{)}\PY{o}{*}\PY{p}{(}\PY{o}{\PYZhy{}}\PY{n}{x}\PY{o}{*}\PY{o}{*}\PY{l+m+mi}{2}\PY{p}{)}
         \PY{n}{sympy}\PY{o}{.}\PY{n}{simplify}\PY{p}{(}\PY{n}{j2}\PY{p}{)}
\end{Verbatim}

\texttt{\color{outcolor}Out[{\color{outcolor}51}]:}
    
    $$- \sin{\left (x \right )} - \frac{2}{x} \cos{\left (x \right )} + \frac{2}{x^{2}} \sin{\left (x \right )}$$

    

    \begin{Verbatim}[commandchars=\\\{\}]
{\color{incolor}In [{\color{incolor} }]:} \PY{n}{j1\PYZus{}v2} \PY{o}{=} 
\end{Verbatim}


    \begin{Verbatim}[commandchars=\\\{\}]
{\color{incolor}In [{\color{incolor}58}]:} \PY{n}{j3\PYZus{}v2} \PY{o}{=} \PY{p}{(}\PY{l+m+mi}{15}\PY{o}{/}\PY{n}{x}\PY{o}{*}\PY{o}{*}\PY{l+m+mi}{3} \PY{o}{\PYZhy{}} \PY{l+m+mi}{6}\PY{o}{/}\PY{n}{x}\PY{p}{)}\PY{o}{*}\PY{n}{sympy}\PY{o}{.}\PY{n}{sin}\PY{p}{(}\PY{n}{x}\PY{p}{)}\PY{o}{/}\PY{n}{x} \PY{o}{\PYZhy{}} \PY{p}{(}\PY{l+m+mi}{15}\PY{o}{/}\PY{n}{x}\PY{o}{*}\PY{o}{*}\PY{l+m+mi}{2} \PY{o}{\PYZhy{}} \PY{l+m+mi}{1}\PY{p}{)}\PY{o}{*}\PY{n}{sympy}\PY{o}{.}\PY{n}{cos}\PY{p}{(}\PY{n}{x}\PY{p}{)}\PY{o}{/}\PY{n}{x}
         \PY{n}{j3\PYZus{}v2}
\end{Verbatim}

\texttt{\color{outcolor}Out[{\color{outcolor}58}]:}
    
    $$- \frac{1}{x} \left(-1 + \frac{15}{x^{2}}\right) \cos{\left (x \right )} + \frac{1}{x} \left(- \frac{6}{x} + \frac{15}{x^{3}}\right) \sin{\left (x \right )}$$

    

    \begin{Verbatim}[commandchars=\\\{\}]
{\color{incolor}In [{\color{incolor}39}]:} \PY{n}{j3} \PY{o}{\PYZhy{}} \PY{n}{j3\PYZus{}v2}
\end{Verbatim}


\begin{Verbatim}[commandchars=\\\{\}]
{\color{outcolor}Out[{\color{outcolor}39}]:} (-1 + 15/x**2)*cos(x)/x - (-6/x + 15/x**3)*sin(x)/x + (-x**3*cos(x) + 3*x**2*sin(x) + 6*x*cos(x) - 6*sin(x))/x**10
\end{Verbatim}
            

    % Add a bibliography block to the postdoc
    
    
    
    \end{document}
