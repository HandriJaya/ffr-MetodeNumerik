
% Default to the notebook output style

    


% Inherit from the specified cell style.




    
\documentclass[11pt]{article}

    
    
    \usepackage[T1]{fontenc}
    % Nicer default font (+ math font) than Computer Modern for most use cases
    \usepackage{mathpazo}

    % Basic figure setup, for now with no caption control since it's done
    % automatically by Pandoc (which extracts ![](path) syntax from Markdown).
    \usepackage{graphicx}
    % We will generate all images so they have a width \maxwidth. This means
    % that they will get their normal width if they fit onto the page, but
    % are scaled down if they would overflow the margins.
    \makeatletter
    \def\maxwidth{\ifdim\Gin@nat@width>\linewidth\linewidth
    \else\Gin@nat@width\fi}
    \makeatother
    \let\Oldincludegraphics\includegraphics
    % Set max figure width to be 80% of text width, for now hardcoded.
    \renewcommand{\includegraphics}[1]{\Oldincludegraphics[width=.8\maxwidth]{#1}}
    % Ensure that by default, figures have no caption (until we provide a
    % proper Figure object with a Caption API and a way to capture that
    % in the conversion process - todo).
    \usepackage{caption}
    \DeclareCaptionLabelFormat{nolabel}{}
    \captionsetup{labelformat=nolabel}

    \usepackage{adjustbox} % Used to constrain images to a maximum size 
    \usepackage{xcolor} % Allow colors to be defined
    \usepackage{enumerate} % Needed for markdown enumerations to work
    \usepackage{geometry} % Used to adjust the document margins
    \usepackage{amsmath} % Equations
    \usepackage{amssymb} % Equations
    \usepackage{textcomp} % defines textquotesingle
    % Hack from http://tex.stackexchange.com/a/47451/13684:
    \AtBeginDocument{%
        \def\PYZsq{\textquotesingle}% Upright quotes in Pygmentized code
    }
    \usepackage{upquote} % Upright quotes for verbatim code
    \usepackage{eurosym} % defines \euro
    \usepackage[mathletters]{ucs} % Extended unicode (utf-8) support
    \usepackage[utf8x]{inputenc} % Allow utf-8 characters in the tex document
    \usepackage{fancyvrb} % verbatim replacement that allows latex
    \usepackage{grffile} % extends the file name processing of package graphics 
                         % to support a larger range 
    % The hyperref package gives us a pdf with properly built
    % internal navigation ('pdf bookmarks' for the table of contents,
    % internal cross-reference links, web links for URLs, etc.)
    \usepackage{hyperref}
    \usepackage{longtable} % longtable support required by pandoc >1.10
    \usepackage{booktabs}  % table support for pandoc > 1.12.2
    \usepackage[inline]{enumitem} % IRkernel/repr support (it uses the enumerate* environment)
    \usepackage[normalem]{ulem} % ulem is needed to support strikethroughs (\sout)
                                % normalem makes italics be italics, not underlines
    

    
    
    % Colors for the hyperref package
    \definecolor{urlcolor}{rgb}{0,.145,.698}
    \definecolor{linkcolor}{rgb}{.71,0.21,0.01}
    \definecolor{citecolor}{rgb}{.12,.54,.11}

    % ANSI colors
    \definecolor{ansi-black}{HTML}{3E424D}
    \definecolor{ansi-black-intense}{HTML}{282C36}
    \definecolor{ansi-red}{HTML}{E75C58}
    \definecolor{ansi-red-intense}{HTML}{B22B31}
    \definecolor{ansi-green}{HTML}{00A250}
    \definecolor{ansi-green-intense}{HTML}{007427}
    \definecolor{ansi-yellow}{HTML}{DDB62B}
    \definecolor{ansi-yellow-intense}{HTML}{B27D12}
    \definecolor{ansi-blue}{HTML}{208FFB}
    \definecolor{ansi-blue-intense}{HTML}{0065CA}
    \definecolor{ansi-magenta}{HTML}{D160C4}
    \definecolor{ansi-magenta-intense}{HTML}{A03196}
    \definecolor{ansi-cyan}{HTML}{60C6C8}
    \definecolor{ansi-cyan-intense}{HTML}{258F8F}
    \definecolor{ansi-white}{HTML}{C5C1B4}
    \definecolor{ansi-white-intense}{HTML}{A1A6B2}

    % commands and environments needed by pandoc snippets
    % extracted from the output of `pandoc -s`
    \providecommand{\tightlist}{%
      \setlength{\itemsep}{0pt}\setlength{\parskip}{0pt}}
    \DefineVerbatimEnvironment{Highlighting}{Verbatim}{commandchars=\\\{\}}
    % Add ',fontsize=\small' for more characters per line
    \newenvironment{Shaded}{}{}
    \newcommand{\KeywordTok}[1]{\textcolor[rgb]{0.00,0.44,0.13}{\textbf{{#1}}}}
    \newcommand{\DataTypeTok}[1]{\textcolor[rgb]{0.56,0.13,0.00}{{#1}}}
    \newcommand{\DecValTok}[1]{\textcolor[rgb]{0.25,0.63,0.44}{{#1}}}
    \newcommand{\BaseNTok}[1]{\textcolor[rgb]{0.25,0.63,0.44}{{#1}}}
    \newcommand{\FloatTok}[1]{\textcolor[rgb]{0.25,0.63,0.44}{{#1}}}
    \newcommand{\CharTok}[1]{\textcolor[rgb]{0.25,0.44,0.63}{{#1}}}
    \newcommand{\StringTok}[1]{\textcolor[rgb]{0.25,0.44,0.63}{{#1}}}
    \newcommand{\CommentTok}[1]{\textcolor[rgb]{0.38,0.63,0.69}{\textit{{#1}}}}
    \newcommand{\OtherTok}[1]{\textcolor[rgb]{0.00,0.44,0.13}{{#1}}}
    \newcommand{\AlertTok}[1]{\textcolor[rgb]{1.00,0.00,0.00}{\textbf{{#1}}}}
    \newcommand{\FunctionTok}[1]{\textcolor[rgb]{0.02,0.16,0.49}{{#1}}}
    \newcommand{\RegionMarkerTok}[1]{{#1}}
    \newcommand{\ErrorTok}[1]{\textcolor[rgb]{1.00,0.00,0.00}{\textbf{{#1}}}}
    \newcommand{\NormalTok}[1]{{#1}}
    
    % Additional commands for more recent versions of Pandoc
    \newcommand{\ConstantTok}[1]{\textcolor[rgb]{0.53,0.00,0.00}{{#1}}}
    \newcommand{\SpecialCharTok}[1]{\textcolor[rgb]{0.25,0.44,0.63}{{#1}}}
    \newcommand{\VerbatimStringTok}[1]{\textcolor[rgb]{0.25,0.44,0.63}{{#1}}}
    \newcommand{\SpecialStringTok}[1]{\textcolor[rgb]{0.73,0.40,0.53}{{#1}}}
    \newcommand{\ImportTok}[1]{{#1}}
    \newcommand{\DocumentationTok}[1]{\textcolor[rgb]{0.73,0.13,0.13}{\textit{{#1}}}}
    \newcommand{\AnnotationTok}[1]{\textcolor[rgb]{0.38,0.63,0.69}{\textbf{\textit{{#1}}}}}
    \newcommand{\CommentVarTok}[1]{\textcolor[rgb]{0.38,0.63,0.69}{\textbf{\textit{{#1}}}}}
    \newcommand{\VariableTok}[1]{\textcolor[rgb]{0.10,0.09,0.49}{{#1}}}
    \newcommand{\ControlFlowTok}[1]{\textcolor[rgb]{0.00,0.44,0.13}{\textbf{{#1}}}}
    \newcommand{\OperatorTok}[1]{\textcolor[rgb]{0.40,0.40,0.40}{{#1}}}
    \newcommand{\BuiltInTok}[1]{{#1}}
    \newcommand{\ExtensionTok}[1]{{#1}}
    \newcommand{\PreprocessorTok}[1]{\textcolor[rgb]{0.74,0.48,0.00}{{#1}}}
    \newcommand{\AttributeTok}[1]{\textcolor[rgb]{0.49,0.56,0.16}{{#1}}}
    \newcommand{\InformationTok}[1]{\textcolor[rgb]{0.38,0.63,0.69}{\textbf{\textit{{#1}}}}}
    \newcommand{\WarningTok}[1]{\textcolor[rgb]{0.38,0.63,0.69}{\textbf{\textit{{#1}}}}}
    
    
    % Define a nice break command that doesn't care if a line doesn't already
    % exist.
    \def\br{\hspace*{\fill} \\* }
    % Math Jax compatability definitions
    \def\gt{>}
    \def\lt{<}
    % Document parameters
    \title{NOTES\_Interpolasi}
    
    
    

    % Pygments definitions
    
\makeatletter
\def\PY@reset{\let\PY@it=\relax \let\PY@bf=\relax%
    \let\PY@ul=\relax \let\PY@tc=\relax%
    \let\PY@bc=\relax \let\PY@ff=\relax}
\def\PY@tok#1{\csname PY@tok@#1\endcsname}
\def\PY@toks#1+{\ifx\relax#1\empty\else%
    \PY@tok{#1}\expandafter\PY@toks\fi}
\def\PY@do#1{\PY@bc{\PY@tc{\PY@ul{%
    \PY@it{\PY@bf{\PY@ff{#1}}}}}}}
\def\PY#1#2{\PY@reset\PY@toks#1+\relax+\PY@do{#2}}

\expandafter\def\csname PY@tok@w\endcsname{\def\PY@tc##1{\textcolor[rgb]{0.73,0.73,0.73}{##1}}}
\expandafter\def\csname PY@tok@c\endcsname{\let\PY@it=\textit\def\PY@tc##1{\textcolor[rgb]{0.25,0.50,0.50}{##1}}}
\expandafter\def\csname PY@tok@cp\endcsname{\def\PY@tc##1{\textcolor[rgb]{0.74,0.48,0.00}{##1}}}
\expandafter\def\csname PY@tok@k\endcsname{\let\PY@bf=\textbf\def\PY@tc##1{\textcolor[rgb]{0.00,0.50,0.00}{##1}}}
\expandafter\def\csname PY@tok@kp\endcsname{\def\PY@tc##1{\textcolor[rgb]{0.00,0.50,0.00}{##1}}}
\expandafter\def\csname PY@tok@kt\endcsname{\def\PY@tc##1{\textcolor[rgb]{0.69,0.00,0.25}{##1}}}
\expandafter\def\csname PY@tok@o\endcsname{\def\PY@tc##1{\textcolor[rgb]{0.40,0.40,0.40}{##1}}}
\expandafter\def\csname PY@tok@ow\endcsname{\let\PY@bf=\textbf\def\PY@tc##1{\textcolor[rgb]{0.67,0.13,1.00}{##1}}}
\expandafter\def\csname PY@tok@nb\endcsname{\def\PY@tc##1{\textcolor[rgb]{0.00,0.50,0.00}{##1}}}
\expandafter\def\csname PY@tok@nf\endcsname{\def\PY@tc##1{\textcolor[rgb]{0.00,0.00,1.00}{##1}}}
\expandafter\def\csname PY@tok@nc\endcsname{\let\PY@bf=\textbf\def\PY@tc##1{\textcolor[rgb]{0.00,0.00,1.00}{##1}}}
\expandafter\def\csname PY@tok@nn\endcsname{\let\PY@bf=\textbf\def\PY@tc##1{\textcolor[rgb]{0.00,0.00,1.00}{##1}}}
\expandafter\def\csname PY@tok@ne\endcsname{\let\PY@bf=\textbf\def\PY@tc##1{\textcolor[rgb]{0.82,0.25,0.23}{##1}}}
\expandafter\def\csname PY@tok@nv\endcsname{\def\PY@tc##1{\textcolor[rgb]{0.10,0.09,0.49}{##1}}}
\expandafter\def\csname PY@tok@no\endcsname{\def\PY@tc##1{\textcolor[rgb]{0.53,0.00,0.00}{##1}}}
\expandafter\def\csname PY@tok@nl\endcsname{\def\PY@tc##1{\textcolor[rgb]{0.63,0.63,0.00}{##1}}}
\expandafter\def\csname PY@tok@ni\endcsname{\let\PY@bf=\textbf\def\PY@tc##1{\textcolor[rgb]{0.60,0.60,0.60}{##1}}}
\expandafter\def\csname PY@tok@na\endcsname{\def\PY@tc##1{\textcolor[rgb]{0.49,0.56,0.16}{##1}}}
\expandafter\def\csname PY@tok@nt\endcsname{\let\PY@bf=\textbf\def\PY@tc##1{\textcolor[rgb]{0.00,0.50,0.00}{##1}}}
\expandafter\def\csname PY@tok@nd\endcsname{\def\PY@tc##1{\textcolor[rgb]{0.67,0.13,1.00}{##1}}}
\expandafter\def\csname PY@tok@s\endcsname{\def\PY@tc##1{\textcolor[rgb]{0.73,0.13,0.13}{##1}}}
\expandafter\def\csname PY@tok@sd\endcsname{\let\PY@it=\textit\def\PY@tc##1{\textcolor[rgb]{0.73,0.13,0.13}{##1}}}
\expandafter\def\csname PY@tok@si\endcsname{\let\PY@bf=\textbf\def\PY@tc##1{\textcolor[rgb]{0.73,0.40,0.53}{##1}}}
\expandafter\def\csname PY@tok@se\endcsname{\let\PY@bf=\textbf\def\PY@tc##1{\textcolor[rgb]{0.73,0.40,0.13}{##1}}}
\expandafter\def\csname PY@tok@sr\endcsname{\def\PY@tc##1{\textcolor[rgb]{0.73,0.40,0.53}{##1}}}
\expandafter\def\csname PY@tok@ss\endcsname{\def\PY@tc##1{\textcolor[rgb]{0.10,0.09,0.49}{##1}}}
\expandafter\def\csname PY@tok@sx\endcsname{\def\PY@tc##1{\textcolor[rgb]{0.00,0.50,0.00}{##1}}}
\expandafter\def\csname PY@tok@m\endcsname{\def\PY@tc##1{\textcolor[rgb]{0.40,0.40,0.40}{##1}}}
\expandafter\def\csname PY@tok@gh\endcsname{\let\PY@bf=\textbf\def\PY@tc##1{\textcolor[rgb]{0.00,0.00,0.50}{##1}}}
\expandafter\def\csname PY@tok@gu\endcsname{\let\PY@bf=\textbf\def\PY@tc##1{\textcolor[rgb]{0.50,0.00,0.50}{##1}}}
\expandafter\def\csname PY@tok@gd\endcsname{\def\PY@tc##1{\textcolor[rgb]{0.63,0.00,0.00}{##1}}}
\expandafter\def\csname PY@tok@gi\endcsname{\def\PY@tc##1{\textcolor[rgb]{0.00,0.63,0.00}{##1}}}
\expandafter\def\csname PY@tok@gr\endcsname{\def\PY@tc##1{\textcolor[rgb]{1.00,0.00,0.00}{##1}}}
\expandafter\def\csname PY@tok@ge\endcsname{\let\PY@it=\textit}
\expandafter\def\csname PY@tok@gs\endcsname{\let\PY@bf=\textbf}
\expandafter\def\csname PY@tok@gp\endcsname{\let\PY@bf=\textbf\def\PY@tc##1{\textcolor[rgb]{0.00,0.00,0.50}{##1}}}
\expandafter\def\csname PY@tok@go\endcsname{\def\PY@tc##1{\textcolor[rgb]{0.53,0.53,0.53}{##1}}}
\expandafter\def\csname PY@tok@gt\endcsname{\def\PY@tc##1{\textcolor[rgb]{0.00,0.27,0.87}{##1}}}
\expandafter\def\csname PY@tok@err\endcsname{\def\PY@bc##1{\setlength{\fboxsep}{0pt}\fcolorbox[rgb]{1.00,0.00,0.00}{1,1,1}{\strut ##1}}}
\expandafter\def\csname PY@tok@kc\endcsname{\let\PY@bf=\textbf\def\PY@tc##1{\textcolor[rgb]{0.00,0.50,0.00}{##1}}}
\expandafter\def\csname PY@tok@kd\endcsname{\let\PY@bf=\textbf\def\PY@tc##1{\textcolor[rgb]{0.00,0.50,0.00}{##1}}}
\expandafter\def\csname PY@tok@kn\endcsname{\let\PY@bf=\textbf\def\PY@tc##1{\textcolor[rgb]{0.00,0.50,0.00}{##1}}}
\expandafter\def\csname PY@tok@kr\endcsname{\let\PY@bf=\textbf\def\PY@tc##1{\textcolor[rgb]{0.00,0.50,0.00}{##1}}}
\expandafter\def\csname PY@tok@bp\endcsname{\def\PY@tc##1{\textcolor[rgb]{0.00,0.50,0.00}{##1}}}
\expandafter\def\csname PY@tok@fm\endcsname{\def\PY@tc##1{\textcolor[rgb]{0.00,0.00,1.00}{##1}}}
\expandafter\def\csname PY@tok@vc\endcsname{\def\PY@tc##1{\textcolor[rgb]{0.10,0.09,0.49}{##1}}}
\expandafter\def\csname PY@tok@vg\endcsname{\def\PY@tc##1{\textcolor[rgb]{0.10,0.09,0.49}{##1}}}
\expandafter\def\csname PY@tok@vi\endcsname{\def\PY@tc##1{\textcolor[rgb]{0.10,0.09,0.49}{##1}}}
\expandafter\def\csname PY@tok@vm\endcsname{\def\PY@tc##1{\textcolor[rgb]{0.10,0.09,0.49}{##1}}}
\expandafter\def\csname PY@tok@sa\endcsname{\def\PY@tc##1{\textcolor[rgb]{0.73,0.13,0.13}{##1}}}
\expandafter\def\csname PY@tok@sb\endcsname{\def\PY@tc##1{\textcolor[rgb]{0.73,0.13,0.13}{##1}}}
\expandafter\def\csname PY@tok@sc\endcsname{\def\PY@tc##1{\textcolor[rgb]{0.73,0.13,0.13}{##1}}}
\expandafter\def\csname PY@tok@dl\endcsname{\def\PY@tc##1{\textcolor[rgb]{0.73,0.13,0.13}{##1}}}
\expandafter\def\csname PY@tok@s2\endcsname{\def\PY@tc##1{\textcolor[rgb]{0.73,0.13,0.13}{##1}}}
\expandafter\def\csname PY@tok@sh\endcsname{\def\PY@tc##1{\textcolor[rgb]{0.73,0.13,0.13}{##1}}}
\expandafter\def\csname PY@tok@s1\endcsname{\def\PY@tc##1{\textcolor[rgb]{0.73,0.13,0.13}{##1}}}
\expandafter\def\csname PY@tok@mb\endcsname{\def\PY@tc##1{\textcolor[rgb]{0.40,0.40,0.40}{##1}}}
\expandafter\def\csname PY@tok@mf\endcsname{\def\PY@tc##1{\textcolor[rgb]{0.40,0.40,0.40}{##1}}}
\expandafter\def\csname PY@tok@mh\endcsname{\def\PY@tc##1{\textcolor[rgb]{0.40,0.40,0.40}{##1}}}
\expandafter\def\csname PY@tok@mi\endcsname{\def\PY@tc##1{\textcolor[rgb]{0.40,0.40,0.40}{##1}}}
\expandafter\def\csname PY@tok@il\endcsname{\def\PY@tc##1{\textcolor[rgb]{0.40,0.40,0.40}{##1}}}
\expandafter\def\csname PY@tok@mo\endcsname{\def\PY@tc##1{\textcolor[rgb]{0.40,0.40,0.40}{##1}}}
\expandafter\def\csname PY@tok@ch\endcsname{\let\PY@it=\textit\def\PY@tc##1{\textcolor[rgb]{0.25,0.50,0.50}{##1}}}
\expandafter\def\csname PY@tok@cm\endcsname{\let\PY@it=\textit\def\PY@tc##1{\textcolor[rgb]{0.25,0.50,0.50}{##1}}}
\expandafter\def\csname PY@tok@cpf\endcsname{\let\PY@it=\textit\def\PY@tc##1{\textcolor[rgb]{0.25,0.50,0.50}{##1}}}
\expandafter\def\csname PY@tok@c1\endcsname{\let\PY@it=\textit\def\PY@tc##1{\textcolor[rgb]{0.25,0.50,0.50}{##1}}}
\expandafter\def\csname PY@tok@cs\endcsname{\let\PY@it=\textit\def\PY@tc##1{\textcolor[rgb]{0.25,0.50,0.50}{##1}}}

\def\PYZbs{\char`\\}
\def\PYZus{\char`\_}
\def\PYZob{\char`\{}
\def\PYZcb{\char`\}}
\def\PYZca{\char`\^}
\def\PYZam{\char`\&}
\def\PYZlt{\char`\<}
\def\PYZgt{\char`\>}
\def\PYZsh{\char`\#}
\def\PYZpc{\char`\%}
\def\PYZdl{\char`\$}
\def\PYZhy{\char`\-}
\def\PYZsq{\char`\'}
\def\PYZdq{\char`\"}
\def\PYZti{\char`\~}
% for compatibility with earlier versions
\def\PYZat{@}
\def\PYZlb{[}
\def\PYZrb{]}
\makeatother


    % Exact colors from NB
    \definecolor{incolor}{rgb}{0.0, 0.0, 0.5}
    \definecolor{outcolor}{rgb}{0.545, 0.0, 0.0}



    
    % Prevent overflowing lines due to hard-to-break entities
    \sloppy 
    % Setup hyperref package
    \hypersetup{
      breaklinks=true,  % so long urls are correctly broken across lines
      colorlinks=true,
      urlcolor=urlcolor,
      linkcolor=linkcolor,
      citecolor=citecolor,
      }
    % Slightly bigger margins than the latex defaults
    
    \geometry{verbose,tmargin=1in,bmargin=1in,lmargin=1in,rmargin=1in}
    
    

    \begin{document}
    
    
    \maketitle
    
    

    
    TF2202 Teknik Komputasi - Interpolasi

Fadjar Fathurrahman

    \begin{Verbatim}[commandchars=\\\{\}]
{\color{incolor}In [{\color{incolor}1}]:} \PY{k+kn}{import} \PY{n+nn}{numpy} \PY{k}{as} \PY{n+nn}{np}
        \PY{k+kn}{import} \PY{n+nn}{matplotlib}\PY{n+nn}{.}\PY{n+nn}{pyplot} \PY{k}{as} \PY{n+nn}{plt}
\end{Verbatim}


    \begin{Verbatim}[commandchars=\\\{\}]
{\color{incolor}In [{\color{incolor}2}]:} \PY{k+kn}{from} \PY{n+nn}{IPython}\PY{n+nn}{.}\PY{n+nn}{display} \PY{k}{import} \PY{n}{set\PYZus{}matplotlib\PYZus{}formats}
        \PY{n}{set\PYZus{}matplotlib\PYZus{}formats}\PY{p}{(}\PY{l+s+s2}{\PYZdq{}}\PY{l+s+s2}{svg}\PY{l+s+s2}{\PYZdq{}}\PY{p}{)}
        \PY{o}{\PYZpc{}}\PY{k}{matplotlib} inline
\end{Verbatim}


    \begin{Verbatim}[commandchars=\\\{\}]
{\color{incolor}In [{\color{incolor}3}]:} \PY{k+kn}{import} \PY{n+nn}{matplotlib}
        \PY{c+c1}{\PYZsh{}matplotlib.style.use(\PYZdq{}dark\PYZus{}background\PYZdq{})}
        \PY{n}{matplotlib}\PY{o}{.}\PY{n}{style}\PY{o}{.}\PY{n}{use}\PY{p}{(}\PY{l+s+s2}{\PYZdq{}}\PY{l+s+s2}{default}\PY{l+s+s2}{\PYZdq{}}\PY{p}{)}
\end{Verbatim}


    \hypertarget{interpolasi-dengan-polinomial-lagrange}{%
\section{Interpolasi dengan polinomial
Lagrange}\label{interpolasi-dengan-polinomial-lagrange}}

    \hypertarget{teori-polinomial-lagrange}{%
\subsection{Teori polinomial Lagrange}\label{teori-polinomial-lagrange}}

    Polinomial Lagrange didefinisikan sebagai: \[
P_n(x) = \sum_{i=0}^{n} y_{i} L_{i}(x)
\] Polinomial ini adalah interpolant yang memiliki derajat \(n\) dan
melewati \((n+1)\) titik data atau pasangan \((x_{i},y_{i})\) dan
\(L_{i}(x)\) adalah fungsi polinomial dengan bentuk: \[
L_{i}(x) = \prod_{\substack{j=0 \\ j \neq i}}^{n}
\frac{x - x_{j}}{x_{i} - x_{j}}
\]

    Catatan: pada buku Kiusalaas persamaan ini typo. Pada buku Chapra,
syarat hasil kali juga memiliki typo.

    \hypertarget{implementasi-interpolasi-lagrange}{%
\subsection{Implementasi interpolasi
Lagrange}\label{implementasi-interpolasi-lagrange}}

    \begin{Verbatim}[commandchars=\\\{\}]
{\color{incolor}In [{\color{incolor}4}]:} \PY{k}{def} \PY{n+nf}{lagrange\PYZus{}interp}\PY{p}{(}\PY{n}{x}\PY{p}{,} \PY{n}{y}\PY{p}{,} \PY{n}{xx}\PY{p}{)}\PY{p}{:}
            
            \PY{k}{assert} \PY{n+nb}{len}\PY{p}{(}\PY{n}{x}\PY{p}{)} \PY{o}{==} \PY{n+nb}{len}\PY{p}{(}\PY{n}{y}\PY{p}{)}
            
            \PY{c+c1}{\PYZsh{} Jumlah data adalah N + 1 dan derajat polynomial adalah N}
            \PY{c+c1}{\PYZsh{} atau:}
            \PY{c+c1}{\PYZsh{} Jumlah data adalah N dan derajat polynomial adalah N \PYZhy{} 1}
            \PY{n}{N} \PY{o}{=} \PY{n+nb}{len}\PY{p}{(}\PY{n}{x}\PY{p}{)} \PY{o}{\PYZhy{}} \PY{l+m+mi}{1}
        
            \PY{n}{yy} \PY{o}{=} \PY{l+m+mf}{0.0}
            \PY{k}{for} \PY{n}{i} \PY{o+ow}{in} \PY{n+nb}{range}\PY{p}{(}\PY{n}{N}\PY{o}{+}\PY{l+m+mi}{1}\PY{p}{)}\PY{p}{:}
                \PY{c+c1}{\PYZsh{} Evaluasi fungsi kardinal}
                \PY{n}{Li} \PY{o}{=} \PY{l+m+mf}{1.0} \PY{c+c1}{\PYZsh{} inisialisasi ke ke 1.0}
                \PY{k}{for} \PY{n}{j} \PY{o+ow}{in} \PY{n+nb}{range}\PY{p}{(}\PY{n}{N}\PY{o}{+}\PY{l+m+mi}{1}\PY{p}{)}\PY{p}{:}
                    \PY{k}{if} \PY{n}{i} \PY{o}{!=} \PY{n}{j}\PY{p}{:}
                        \PY{n}{Li} \PY{o}{=} \PY{n}{Li} \PY{o}{*} \PY{p}{(}\PY{n}{xx} \PY{o}{\PYZhy{}} \PY{n}{x}\PY{p}{[}\PY{n}{j}\PY{p}{]}\PY{p}{)}\PY{o}{/}\PY{p}{(}\PY{n}{x}\PY{p}{[}\PY{n}{i}\PY{p}{]} \PY{o}{\PYZhy{}} \PY{n}{x}\PY{p}{[}\PY{n}{j}\PY{p}{]}\PY{p}{)}
                \PY{n}{yy} \PY{o}{=} \PY{n}{yy} \PY{o}{+} \PY{n}{y}\PY{p}{[}\PY{n}{i}\PY{p}{]}\PY{o}{*}\PY{n}{Li}
            \PY{k}{return} \PY{n}{yy}
\end{Verbatim}


    \hypertarget{contoh}{%
\subsubsection{Contoh}\label{contoh}}

    Sebagai contoh, diberikan data sebagai berikut:

\begin{longtable}[]{@{}ll@{}}
\toprule
\(x_{i}\) & \(y_{i}\)\tabularnewline
\midrule
\endhead
1 & 0\tabularnewline
4 & 1.386294\tabularnewline
6 & 1.791760\tabularnewline
\bottomrule
\end{longtable}

    \begin{Verbatim}[commandchars=\\\{\}]
{\color{incolor}In [{\color{incolor}5}]:} \PY{n}{x} \PY{o}{=} \PY{n}{np}\PY{o}{.}\PY{n}{array}\PY{p}{(}\PY{p}{[}\PY{l+m+mf}{1.0}\PY{p}{,} \PY{l+m+mf}{4.0}\PY{p}{,} \PY{l+m+mf}{6.0}\PY{p}{]}\PY{p}{)}
        \PY{n}{y} \PY{o}{=} \PY{n}{np}\PY{o}{.}\PY{n}{array}\PY{p}{(}\PY{p}{[}\PY{l+m+mi}{0}\PY{p}{,} \PY{l+m+mf}{1.386294}\PY{p}{,} \PY{l+m+mf}{1.791760}\PY{p}{]}\PY{p}{)}
\end{Verbatim}


    \begin{Verbatim}[commandchars=\\\{\}]
{\color{incolor}In [{\color{incolor}6}]:} \PY{n}{lagrange\PYZus{}interp}\PY{p}{(}\PY{n}{x}\PY{p}{,} \PY{n}{y}\PY{p}{,} \PY{l+m+mf}{2.0}\PY{p}{)}
\end{Verbatim}


\begin{Verbatim}[commandchars=\\\{\}]
{\color{outcolor}Out[{\color{outcolor}6}]:} 0.5658439999999999
\end{Verbatim}
            
    \begin{Verbatim}[commandchars=\\\{\}]
{\color{incolor}In [{\color{incolor}7}]:} \PY{n}{NptsPlot} \PY{o}{=} \PY{l+m+mi}{1000}
        \PY{n}{xx} \PY{o}{=} \PY{n}{np}\PY{o}{.}\PY{n}{linspace}\PY{p}{(}\PY{n}{x}\PY{p}{[}\PY{l+m+mi}{0}\PY{p}{]}\PY{p}{,} \PY{n}{x}\PY{p}{[}\PY{o}{\PYZhy{}}\PY{l+m+mi}{1}\PY{p}{]}\PY{p}{,} \PY{n}{NptsPlot}\PY{p}{)}
        \PY{n}{yy} \PY{o}{=} \PY{n}{np}\PY{o}{.}\PY{n}{zeros}\PY{p}{(}\PY{n}{NptsPlot}\PY{p}{)}
        \PY{k}{for} \PY{n}{i} \PY{o+ow}{in} \PY{n+nb}{range}\PY{p}{(}\PY{n}{NptsPlot}\PY{p}{)}\PY{p}{:}
            \PY{n}{yy}\PY{p}{[}\PY{n}{i}\PY{p}{]} \PY{o}{=} \PY{n}{lagrange\PYZus{}interp}\PY{p}{(}\PY{n}{x}\PY{p}{,} \PY{n}{y}\PY{p}{,} \PY{n}{xx}\PY{p}{[}\PY{n}{i}\PY{p}{]}\PY{p}{)}
\end{Verbatim}


    \begin{Verbatim}[commandchars=\\\{\}]
{\color{incolor}In [{\color{incolor}8}]:} \PY{n}{plt}\PY{o}{.}\PY{n}{clf}\PY{p}{(}\PY{p}{)}
        \PY{n}{plt}\PY{o}{.}\PY{n}{plot}\PY{p}{(}\PY{n}{x}\PY{p}{,} \PY{n}{y}\PY{p}{,} \PY{n}{marker}\PY{o}{=}\PY{l+s+s2}{\PYZdq{}}\PY{l+s+s2}{o}\PY{l+s+s2}{\PYZdq{}}\PY{p}{,} \PY{n}{label}\PY{o}{=}\PY{l+s+s2}{\PYZdq{}}\PY{l+s+s2}{data}\PY{l+s+s2}{\PYZdq{}}\PY{p}{)}
        \PY{n}{plt}\PY{o}{.}\PY{n}{plot}\PY{p}{(}\PY{n}{xx}\PY{p}{,} \PY{n}{yy}\PY{p}{,} \PY{n}{label}\PY{o}{=}\PY{l+s+s2}{\PYZdq{}}\PY{l+s+s2}{Interpolasi Lagrange (orde\PYZhy{}2)}\PY{l+s+s2}{\PYZdq{}}\PY{p}{)}
        \PY{n}{plt}\PY{o}{.}\PY{n}{grid}\PY{p}{(}\PY{p}{)}
        \PY{n}{plt}\PY{o}{.}\PY{n}{legend}\PY{p}{(}\PY{p}{)}
\end{Verbatim}


\begin{Verbatim}[commandchars=\\\{\}]
{\color{outcolor}Out[{\color{outcolor}8}]:} <matplotlib.legend.Legend at 0x7fb4f20c5390>
\end{Verbatim}
            
    \begin{center}
    \adjustimage{max size={0.9\linewidth}{0.9\paperheight}}{NOTES_Interpolasi_files/NOTES_Interpolasi_15_1.pdf}
    \end{center}
    { \hspace*{\fill} \\}
    
    \hypertarget{interpolasi-dengan-polinomial-newton}{%
\subsection{Interpolasi dengan polinomial
Newton}\label{interpolasi-dengan-polinomial-newton}}

    \hypertarget{teori-polinomial-newton}{%
\subsubsection{Teori polinomial Newton}\label{teori-polinomial-newton}}

    Polinomial Newton memiliki bentuk sebagai berikut: \[
P_{n} = a_{0} + (x - x_{0}) a_{1} + (x - x_{0}) (x - x_{1}) a_{2} + \cdots + (x - x_{0})(x - x_{1}) \cdots (x - x_{n-1}) a_{n}
\]

    Koefisien \(a_{n}\) dapat dihitung dengan: \[
\begin{align*}
a_{0} & = f(x_0) \\
a_{1} & = f[x_1,x_0] \\
a_{2} & = f[x_2, x_1, x_0] \\
\vdots \\
a_{n} & = f[x_n, x_{n-1}, \ldots, x_1, x_0]
\end{align*}
\] di mana fungsi dengan tanda kurung siku merupakan beda terbagi hingga
(\emph{finite divided differences}).

Beda terbagi hingga pertama didefinisikan sebagai: \[
f[x_{i},x_{j}] = \frac{f(x_i) - f(x_j)}{x_i - x_j}
\] Beda terbagi hingga kedua didefinisikan sebagai: \[
f[x_{i},x_{j},x_{k}] = \frac{f[x_{i},x_{j}] - f[x_{j},f_{k}]}{x_i - x_k}
\] Secara umum, untuk beda terbagi hingga ke-\(n\) adalah: \[
f[x_{n},x_{n-1},\ldots,x_{1},x_{0}] =
\frac{f[x_{n},x_{n-1},\ldots,x_{1}]- f[x_{n-1},x_{n-2},\ldots,x_{0}] }{x_n - x_0}
\]

    \hypertarget{implementasi-interpolasi-newton}{%
\subsection{Implementasi interpolasi
Newton}\label{implementasi-interpolasi-newton}}

    \begin{Verbatim}[commandchars=\\\{\}]
{\color{incolor}In [{\color{incolor}9}]:} \PY{k}{def} \PY{n+nf}{create\PYZus{}newton\PYZus{}polynom}\PY{p}{(}\PY{n}{x}\PY{p}{,} \PY{n}{y}\PY{p}{)}\PY{p}{:}
            \PY{n}{Ndata} \PY{o}{=} \PY{n+nb}{len}\PY{p}{(}\PY{n}{x}\PY{p}{)} \PY{c+c1}{\PYZsh{} jumlah data}
            \PY{n}{coefs} \PY{o}{=} \PY{n}{np}\PY{o}{.}\PY{n}{copy}\PY{p}{(}\PY{n}{y}\PY{p}{)}
            \PY{k}{for} \PY{n}{k} \PY{o+ow}{in} \PY{n+nb}{range}\PY{p}{(}\PY{l+m+mi}{1}\PY{p}{,}\PY{n}{Ndata}\PY{p}{)}\PY{p}{:}
                \PY{n}{coefs}\PY{p}{[}\PY{n}{k}\PY{p}{:}\PY{n}{Ndata}\PY{p}{]} \PY{o}{=} \PY{p}{(}\PY{n}{coefs}\PY{p}{[}\PY{n}{k}\PY{p}{:}\PY{n}{Ndata}\PY{p}{]} \PY{o}{\PYZhy{}} \PY{n}{coefs}\PY{p}{[}\PY{n}{k}\PY{o}{\PYZhy{}}\PY{l+m+mi}{1}\PY{p}{]}\PY{p}{)}\PY{o}{/}\PY{p}{(}\PY{n}{x}\PY{p}{[}\PY{n}{k}\PY{p}{:}\PY{n}{Ndata}\PY{p}{]} \PY{o}{\PYZhy{}} \PY{n}{x}\PY{p}{[}\PY{n}{k}\PY{o}{\PYZhy{}}\PY{l+m+mi}{1}\PY{p}{]}\PY{p}{)}
            \PY{k}{return} \PY{n}{coefs}
        
        \PY{k}{def} \PY{n+nf}{eval\PYZus{}newton\PYZus{}polynom}\PY{p}{(}\PY{n}{coefs}\PY{p}{,} \PY{n}{x}\PY{p}{,} \PY{n}{xo}\PY{p}{)}\PY{p}{:}
            \PY{n}{N} \PY{o}{=} \PY{n+nb}{len}\PY{p}{(}\PY{n}{x}\PY{p}{)} \PY{o}{\PYZhy{}} \PY{l+m+mi}{1} \PY{c+c1}{\PYZsh{} derajat polinom}
            \PY{n}{p} \PY{o}{=} \PY{n}{coefs}\PY{p}{[}\PY{n}{N}\PY{p}{]}
            \PY{k}{for} \PY{n}{k} \PY{o+ow}{in} \PY{n+nb}{range}\PY{p}{(}\PY{l+m+mi}{1}\PY{p}{,}\PY{n}{N}\PY{o}{+}\PY{l+m+mi}{1}\PY{p}{)}\PY{p}{:}
                \PY{n}{p} \PY{o}{=} \PY{n}{coefs}\PY{p}{[}\PY{n}{N}\PY{o}{\PYZhy{}}\PY{n}{k}\PY{p}{]} \PY{o}{+} \PY{p}{(}\PY{n}{xo} \PY{o}{\PYZhy{}} \PY{n}{x}\PY{p}{[}\PY{n}{N}\PY{o}{\PYZhy{}}\PY{n}{k}\PY{p}{]}\PY{p}{)}\PY{o}{*}\PY{n}{p}
            \PY{k}{return} \PY{n}{p}
\end{Verbatim}


    \hypertarget{contoh-penggunaan-interpolasi-newton}{%
\subsubsection{Contoh penggunaan interpolasi
Newton}\label{contoh-penggunaan-interpolasi-newton}}

    \begin{Verbatim}[commandchars=\\\{\}]
{\color{incolor}In [{\color{incolor}10}]:} \PY{n}{x} \PY{o}{=} \PY{n}{np}\PY{o}{.}\PY{n}{array}\PY{p}{(}\PY{p}{[}\PY{l+m+mf}{1.0}\PY{p}{,} \PY{l+m+mf}{4.0}\PY{p}{,} \PY{l+m+mf}{6.0}\PY{p}{]}\PY{p}{)}
         \PY{n}{y} \PY{o}{=} \PY{n}{np}\PY{o}{.}\PY{n}{array}\PY{p}{(}\PY{p}{[}\PY{l+m+mi}{0}\PY{p}{,} \PY{l+m+mf}{1.386294}\PY{p}{,} \PY{l+m+mf}{1.791760}\PY{p}{]}\PY{p}{)}
\end{Verbatim}


    \begin{Verbatim}[commandchars=\\\{\}]
{\color{incolor}In [{\color{incolor}11}]:} \PY{n}{coefs} \PY{o}{=} \PY{n}{create\PYZus{}newton\PYZus{}polynom}\PY{p}{(}\PY{n}{x}\PY{p}{,} \PY{n}{y}\PY{p}{)}
         \PY{n}{coefs}
\end{Verbatim}


\begin{Verbatim}[commandchars=\\\{\}]
{\color{outcolor}Out[{\color{outcolor}11}]:} array([ 0.      ,  0.462098, -0.051873])
\end{Verbatim}
            
    \begin{Verbatim}[commandchars=\\\{\}]
{\color{incolor}In [{\color{incolor}12}]:} \PY{n}{eval\PYZus{}newton\PYZus{}polynom}\PY{p}{(}\PY{n}{coefs}\PY{p}{,} \PY{n}{x}\PY{p}{,} \PY{l+m+mf}{2.0}\PY{p}{)}
\end{Verbatim}


\begin{Verbatim}[commandchars=\\\{\}]
{\color{outcolor}Out[{\color{outcolor}12}]:} 0.5658439999999999
\end{Verbatim}
            
    \hypertarget{aplikasi-interpolasi-newton-pada-fungsi-cos}{%
\subsubsection{\texorpdfstring{Aplikasi interpolasi Newton pada fungsi
\(\cos\)}{Aplikasi interpolasi Newton pada fungsi \textbackslash{}cos}}\label{aplikasi-interpolasi-newton-pada-fungsi-cos}}

    \begin{Verbatim}[commandchars=\\\{\}]
{\color{incolor}In [{\color{incolor}13}]:} \PY{k}{def} \PY{n+nf}{func\PYZus{}01}\PY{p}{(}\PY{n}{x}\PY{p}{)}\PY{p}{:}
             \PY{k}{return} \PY{n}{np}\PY{o}{.}\PY{n}{cos}\PY{p}{(}\PY{l+m+mi}{2}\PY{o}{*}\PY{n}{x}\PY{p}{)}
\end{Verbatim}


    \begin{Verbatim}[commandchars=\\\{\}]
{\color{incolor}In [{\color{incolor}14}]:} \PY{n}{N} \PY{o}{=} \PY{l+m+mi}{5}
         \PY{n}{A} \PY{o}{=} \PY{l+m+mf}{0.0}
         \PY{n}{B} \PY{o}{=} \PY{l+m+mi}{2}\PY{o}{*}\PY{n}{np}\PY{o}{.}\PY{n}{pi}
         \PY{n}{x\PYZus{}sample} \PY{o}{=} \PY{n}{np}\PY{o}{.}\PY{n}{linspace}\PY{p}{(}\PY{n}{A}\PY{p}{,} \PY{n}{B}\PY{p}{,} \PY{n}{N}\PY{p}{)}
         \PY{n}{y\PYZus{}sample} \PY{o}{=} \PY{n}{func\PYZus{}01}\PY{p}{(}\PY{n}{x\PYZus{}sample}\PY{p}{)}
         
         \PY{n}{NptsPlot} \PY{o}{=} \PY{l+m+mi}{500}
         \PY{n}{x\PYZus{}dense} \PY{o}{=} \PY{n}{np}\PY{o}{.}\PY{n}{linspace}\PY{p}{(}\PY{n}{A}\PY{p}{,}\PY{n}{B}\PY{p}{,}\PY{n}{NptsPlot}\PY{p}{)}
         \PY{n}{y\PYZus{}dense} \PY{o}{=} \PY{n}{func\PYZus{}01}\PY{p}{(}\PY{n}{x\PYZus{}dense}\PY{p}{)}
         
         \PY{n}{Ninterp} \PY{o}{=} \PY{l+m+mi}{10}
         \PY{n}{x\PYZus{}interp} \PY{o}{=} \PY{n}{np}\PY{o}{.}\PY{n}{linspace}\PY{p}{(}\PY{n}{A}\PY{p}{,}\PY{n}{B}\PY{p}{,}\PY{n}{Ninterp}\PY{p}{)}
         \PY{n}{y\PYZus{}interp} \PY{o}{=} \PY{n}{func\PYZus{}01}\PY{p}{(}\PY{n}{x\PYZus{}interp}\PY{p}{)}
         \PY{n}{coefs} \PY{o}{=} \PY{n}{create\PYZus{}newton\PYZus{}polynom}\PY{p}{(}\PY{n}{x\PYZus{}interp}\PY{p}{,} \PY{n}{y\PYZus{}interp}\PY{p}{)}
         
         \PY{n}{x\PYZus{}interp\PYZus{}plt} \PY{o}{=} \PY{n}{np}\PY{o}{.}\PY{n}{linspace}\PY{p}{(}\PY{n}{A}\PY{p}{,}\PY{n}{B}\PY{p}{,}\PY{n}{NptsPlot}\PY{p}{)}
         \PY{n}{y\PYZus{}interp\PYZus{}plt} \PY{o}{=} \PY{n}{np}\PY{o}{.}\PY{n}{zeros}\PY{p}{(}\PY{n}{NptsPlot}\PY{p}{)}
         \PY{k}{for} \PY{n}{i} \PY{o+ow}{in} \PY{n+nb}{range}\PY{p}{(}\PY{n}{NptsPlot}\PY{p}{)}\PY{p}{:}
             \PY{n}{y\PYZus{}interp\PYZus{}plt}\PY{p}{[}\PY{n}{i}\PY{p}{]} \PY{o}{=} \PY{n}{eval\PYZus{}newton\PYZus{}polynom}\PY{p}{(}\PY{n}{coefs}\PY{p}{,} \PY{n}{x\PYZus{}interp}\PY{p}{,} \PY{n}{x\PYZus{}interp\PYZus{}plt}\PY{p}{[}\PY{n}{i}\PY{p}{]}\PY{p}{)}
\end{Verbatim}


    \begin{Verbatim}[commandchars=\\\{\}]
{\color{incolor}In [{\color{incolor}15}]:} \PY{n}{x\PYZus{}sample}\PY{p}{,} \PY{n}{y\PYZus{}sample}
\end{Verbatim}


\begin{Verbatim}[commandchars=\\\{\}]
{\color{outcolor}Out[{\color{outcolor}15}]:} (array([0.        , 1.57079633, 3.14159265, 4.71238898, 6.28318531]),
          array([ 1., -1.,  1., -1.,  1.]))
\end{Verbatim}
            
    \begin{Verbatim}[commandchars=\\\{\}]
{\color{incolor}In [{\color{incolor}16}]:} \PY{n}{plt}\PY{o}{.}\PY{n}{clf}\PY{p}{(}\PY{p}{)}
         \PY{n}{plt}\PY{o}{.}\PY{n}{plot}\PY{p}{(}\PY{n}{x\PYZus{}sample}\PY{p}{,} \PY{n}{y\PYZus{}sample}\PY{p}{,} \PY{n}{marker}\PY{o}{=}\PY{l+s+s2}{\PYZdq{}}\PY{l+s+s2}{o}\PY{l+s+s2}{\PYZdq{}}\PY{p}{,} \PY{n}{label}\PY{o}{=}\PY{l+s+s2}{\PYZdq{}}\PY{l+s+s2}{sampled}\PY{l+s+s2}{\PYZdq{}}\PY{p}{)}
         \PY{n}{plt}\PY{o}{.}\PY{n}{plot}\PY{p}{(}\PY{n}{x\PYZus{}dense}\PY{p}{,} \PY{n}{y\PYZus{}dense}\PY{p}{,} \PY{n}{label}\PY{o}{=}\PY{l+s+s2}{\PYZdq{}}\PY{l+s+s2}{exact}\PY{l+s+s2}{\PYZdq{}}\PY{p}{)}
         \PY{n}{plt}\PY{o}{.}\PY{n}{plot}\PY{p}{(}\PY{n}{x\PYZus{}interp\PYZus{}plt}\PY{p}{,} \PY{n}{y\PYZus{}interp\PYZus{}plt}\PY{p}{,} \PY{n}{label}\PY{o}{=}\PY{l+s+s2}{\PYZdq{}}\PY{l+s+s2}{interp}\PY{l+s+s2}{\PYZdq{}}\PY{p}{)}
         \PY{n}{plt}\PY{o}{.}\PY{n}{legend}\PY{p}{(}\PY{p}{)}
\end{Verbatim}


\begin{Verbatim}[commandchars=\\\{\}]
{\color{outcolor}Out[{\color{outcolor}16}]:} <matplotlib.legend.Legend at 0x7fb4f1e8c2e8>
\end{Verbatim}
            
    \begin{center}
    \adjustimage{max size={0.9\linewidth}{0.9\paperheight}}{NOTES_Interpolasi_files/NOTES_Interpolasi_30_1.pdf}
    \end{center}
    { \hspace*{\fill} \\}
    
    \hypertarget{metode-neville}{%
\section{Metode Neville}\label{metode-neville}}

    Metode ini pada dasarnya merupakan bentuk alternatif dari polinomial
Newton.

    \hypertarget{implementasi-metode-neville}{%
\subsection{Implementasi metode
Neville}\label{implementasi-metode-neville}}

    \begin{Verbatim}[commandchars=\\\{\}]
{\color{incolor}In [{\color{incolor}17}]:} \PY{k}{def} \PY{n+nf}{neville\PYZus{}interp}\PY{p}{(}\PY{n}{x}\PY{p}{,} \PY{n}{y\PYZus{}}\PY{p}{,} \PY{n}{xx}\PY{p}{)}\PY{p}{:}
             \PY{n}{m} \PY{o}{=} \PY{n+nb}{len}\PY{p}{(}\PY{n}{x}\PY{p}{)}
             \PY{n}{y} \PY{o}{=} \PY{n}{np}\PY{o}{.}\PY{n}{copy}\PY{p}{(}\PY{n}{y\PYZus{}}\PY{p}{)}
             \PY{k}{for} \PY{n}{k} \PY{o+ow}{in} \PY{n+nb}{range}\PY{p}{(}\PY{l+m+mi}{1}\PY{p}{,}\PY{n}{m}\PY{p}{)}\PY{p}{:}
                 \PY{n}{y}\PY{p}{[}\PY{l+m+mi}{0}\PY{p}{:}\PY{n}{m}\PY{o}{\PYZhy{}}\PY{n}{k}\PY{p}{]} \PY{o}{=} \PY{p}{(}\PY{p}{(}\PY{n}{xx}\PY{o}{\PYZhy{}} \PY{n}{x}\PY{p}{[}\PY{n}{k}\PY{p}{:}\PY{n}{m}\PY{p}{]}\PY{p}{)}\PY{o}{*}\PY{n}{y}\PY{p}{[}\PY{l+m+mi}{0}\PY{p}{:}\PY{n}{m}\PY{o}{\PYZhy{}}\PY{n}{k}\PY{p}{]} \PY{o}{+} \PY{p}{(}\PY{n}{x}\PY{p}{[}\PY{l+m+mi}{0}\PY{p}{:}\PY{n}{m}\PY{o}{\PYZhy{}}\PY{n}{k}\PY{p}{]} \PY{o}{\PYZhy{}} \PY{n}{xx}\PY{p}{)}\PY{o}{*}\PY{n}{y}\PY{p}{[}\PY{l+m+mi}{1}\PY{p}{:}\PY{n}{m}\PY{o}{\PYZhy{}}\PY{n}{k}\PY{o}{+}\PY{l+m+mi}{1}\PY{p}{]}\PY{p}{)}\PY{o}{/}\PY{p}{(}\PY{n}{x}\PY{p}{[}\PY{l+m+mi}{0}\PY{p}{:}\PY{n}{m}\PY{o}{\PYZhy{}}\PY{n}{k}\PY{p}{]} \PY{o}{\PYZhy{}} \PY{n}{x}\PY{p}{[}\PY{n}{k}\PY{p}{:}\PY{n}{m}\PY{p}{]}\PY{p}{)}
             \PY{k}{return} \PY{n}{y}\PY{p}{[}\PY{l+m+mi}{0}\PY{p}{]}
\end{Verbatim}


    \hypertarget{contoh-penggunaan-metode-neville}{%
\subsubsection{Contoh penggunaan metode
Neville}\label{contoh-penggunaan-metode-neville}}

    \begin{Verbatim}[commandchars=\\\{\}]
{\color{incolor}In [{\color{incolor}18}]:} \PY{n}{x} \PY{o}{=} \PY{n}{np}\PY{o}{.}\PY{n}{array}\PY{p}{(}\PY{p}{[}\PY{l+m+mf}{1.0}\PY{p}{,} \PY{l+m+mf}{4.0}\PY{p}{,} \PY{l+m+mf}{6.0}\PY{p}{]}\PY{p}{)}
         \PY{n}{y} \PY{o}{=} \PY{n}{np}\PY{o}{.}\PY{n}{array}\PY{p}{(}\PY{p}{[}\PY{l+m+mi}{0}\PY{p}{,} \PY{l+m+mf}{1.386294}\PY{p}{,} \PY{l+m+mf}{1.791760}\PY{p}{]}\PY{p}{)}
         \PY{n}{neville\PYZus{}interp}\PY{p}{(}\PY{n}{x}\PY{p}{,} \PY{n}{y}\PY{p}{,} \PY{l+m+mf}{2.0}\PY{p}{)}
\end{Verbatim}


\begin{Verbatim}[commandchars=\\\{\}]
{\color{outcolor}Out[{\color{outcolor}18}]:} 0.5658439999999999
\end{Verbatim}
            
    \hypertarget{interpolasi-dengan-fungsi-rasional}{%
\section{Interpolasi dengan fungsi
rasional}\label{interpolasi-dengan-fungsi-rasional}}

    Aproksimasi ini baik digunakan untuk fungsi yang memiliki pole, akan
tetapi pada beberapa jenis data atau fungsi interpolasi ini tidak
stabil.

    \hypertarget{implementasi-dalam-python}{%
\subsubsection{Implementasi dalam
Python}\label{implementasi-dalam-python}}

    \begin{Verbatim}[commandchars=\\\{\}]
{\color{incolor}In [{\color{incolor}19}]:} \PY{k}{def} \PY{n+nf}{rational\PYZus{}interp}\PY{p}{(}\PY{n}{x}\PY{p}{,} \PY{n}{y}\PY{p}{,} \PY{n}{xx}\PY{p}{)}\PY{p}{:}
             \PY{n}{m} \PY{o}{=} \PY{n+nb}{len}\PY{p}{(}\PY{n}{x}\PY{p}{)}
             \PY{n}{r} \PY{o}{=} \PY{n}{np}\PY{o}{.}\PY{n}{copy}\PY{p}{(}\PY{n}{y}\PY{p}{)}
             \PY{n}{rOld} \PY{o}{=} \PY{n}{np}\PY{o}{.}\PY{n}{zeros}\PY{p}{(}\PY{n}{m}\PY{p}{)}
             \PY{c+c1}{\PYZsh{}SMALL = np.finfo(\PYZdq{}float64\PYZdq{}).eps}
             \PY{n}{SMALL} \PY{o}{=} \PY{l+m+mf}{1e\PYZhy{}8}
             \PY{k}{for} \PY{n}{k} \PY{o+ow}{in} \PY{n+nb}{range}\PY{p}{(}\PY{n}{m}\PY{o}{\PYZhy{}}\PY{l+m+mi}{1}\PY{p}{)}\PY{p}{:}
                 \PY{k}{for} \PY{n}{i} \PY{o+ow}{in} \PY{n+nb}{range}\PY{p}{(}\PY{n}{m}\PY{o}{\PYZhy{}}\PY{n}{k}\PY{o}{\PYZhy{}}\PY{l+m+mi}{1}\PY{p}{)}\PY{p}{:}
                     \PY{k}{if} \PY{p}{(}\PY{n+nb}{abs}\PY{p}{(}\PY{n}{xx} \PY{o}{\PYZhy{}} \PY{n}{x}\PY{p}{[}\PY{n}{i}\PY{o}{+}\PY{n}{k}\PY{o}{+}\PY{l+m+mi}{1}\PY{p}{]}\PY{p}{)} \PY{o}{\PYZlt{}} \PY{n}{SMALL}\PY{p}{)}\PY{p}{:}
                         \PY{k}{return} \PY{n}{y}\PY{p}{[}\PY{n}{i}\PY{o}{+}\PY{n}{k}\PY{o}{+}\PY{l+m+mi}{1}\PY{p}{]}
                     \PY{k}{else}\PY{p}{:}
                         \PY{n}{c1} \PY{o}{=} \PY{n}{r}\PY{p}{[}\PY{n}{i}\PY{o}{+}\PY{l+m+mi}{1}\PY{p}{]} \PY{o}{\PYZhy{}} \PY{n}{r}\PY{p}{[}\PY{n}{i}\PY{p}{]}
                         \PY{n}{c2} \PY{o}{=} \PY{n}{r}\PY{p}{[}\PY{n}{i}\PY{o}{+}\PY{l+m+mi}{1}\PY{p}{]} \PY{o}{\PYZhy{}} \PY{n}{rOld}\PY{p}{[}\PY{n}{i}\PY{o}{+}\PY{l+m+mi}{1}\PY{p}{]}
                         \PY{n}{c3} \PY{o}{=} \PY{p}{(}\PY{n}{xx} \PY{o}{\PYZhy{}} \PY{n}{x}\PY{p}{[}\PY{n}{i}\PY{p}{]}\PY{p}{)}\PY{o}{/}\PY{p}{(}\PY{n}{xx} \PY{o}{\PYZhy{}} \PY{n}{x}\PY{p}{[}\PY{n}{i}\PY{o}{+}\PY{n}{k}\PY{o}{+}\PY{l+m+mi}{1}\PY{p}{]}\PY{p}{)}
                         \PY{n}{r}\PY{p}{[}\PY{n}{i}\PY{p}{]} \PY{o}{=} \PY{n}{r}\PY{p}{[}\PY{n}{i}\PY{o}{+}\PY{l+m+mi}{1}\PY{p}{]} \PY{o}{+} \PY{n}{c1}\PY{o}{/}\PY{p}{(}\PY{n}{c3}\PY{o}{*}\PY{p}{(}\PY{l+m+mf}{1.0} \PY{o}{\PYZhy{}} \PY{n}{c1}\PY{o}{/}\PY{n}{c2}\PY{p}{)} \PY{o}{\PYZhy{}} \PY{l+m+mf}{1.0}\PY{p}{)}
                         \PY{n}{rOld}\PY{p}{[}\PY{n}{i}\PY{o}{+}\PY{l+m+mi}{1}\PY{p}{]} \PY{o}{=} \PY{n}{r}\PY{p}{[}\PY{n}{i}\PY{o}{+}\PY{l+m+mi}{1}\PY{p}{]}
             \PY{k}{return} \PY{n}{r}\PY{p}{[}\PY{l+m+mi}{0}\PY{p}{]}
\end{Verbatim}


    \hypertarget{contoh-penggunaan-interpolasi-fungsi-rasional}{%
\subsubsection{Contoh penggunaan interpolasi fungsi
rasional}\label{contoh-penggunaan-interpolasi-fungsi-rasional}}

    \begin{Verbatim}[commandchars=\\\{\}]
{\color{incolor}In [{\color{incolor}20}]:} \PY{n}{x} \PY{o}{=} \PY{n}{np}\PY{o}{.}\PY{n}{array}\PY{p}{(}\PY{p}{[}\PY{l+m+mf}{0.0}\PY{p}{,} \PY{l+m+mf}{0.6}\PY{p}{,} \PY{l+m+mf}{0.8}\PY{p}{,} \PY{l+m+mf}{0.95}\PY{p}{]}\PY{p}{)}
         \PY{n}{y} \PY{o}{=} \PY{n}{np}\PY{o}{.}\PY{n}{array}\PY{p}{(}\PY{p}{[}\PY{l+m+mf}{0.0}\PY{p}{,} \PY{l+m+mf}{1.3764}\PY{p}{,} \PY{l+m+mf}{3.0777}\PY{p}{,} \PY{l+m+mf}{12.7062}\PY{p}{]}\PY{p}{)}
\end{Verbatim}


    \begin{Verbatim}[commandchars=\\\{\}]
{\color{incolor}In [{\color{incolor}21}]:} \PY{n}{rational\PYZus{}interp}\PY{p}{(}\PY{n}{x}\PY{p}{,} \PY{n}{y}\PY{p}{,} \PY{l+m+mf}{0.5}\PY{p}{)}
\end{Verbatim}


\begin{Verbatim}[commandchars=\\\{\}]
{\color{outcolor}Out[{\color{outcolor}21}]:} 1.0131205116558464
\end{Verbatim}
            
    \begin{Verbatim}[commandchars=\\\{\}]
{\color{incolor}In [{\color{incolor}22}]:} \PY{n}{x} \PY{o}{=} \PY{n}{np}\PY{o}{.}\PY{n}{array}\PY{p}{(}\PY{p}{[}\PY{l+m+mf}{0.1}\PY{p}{,} \PY{l+m+mf}{0.2}\PY{p}{,} \PY{l+m+mf}{0.5}\PY{p}{,} \PY{l+m+mf}{0.6}\PY{p}{,} \PY{l+m+mf}{0.8}\PY{p}{,} \PY{l+m+mf}{1.2}\PY{p}{,} \PY{l+m+mf}{1.5}\PY{p}{]}\PY{p}{)}
         \PY{n}{y} \PY{o}{=} \PY{n}{np}\PY{o}{.}\PY{n}{array}\PY{p}{(}\PY{p}{[}\PY{o}{\PYZhy{}}\PY{l+m+mf}{1.5342}\PY{p}{,} \PY{o}{\PYZhy{}}\PY{l+m+mf}{1.0811}\PY{p}{,} \PY{o}{\PYZhy{}}\PY{l+m+mf}{0.4445}\PY{p}{,} \PY{o}{\PYZhy{}}\PY{l+m+mf}{0.3085}\PY{p}{,} \PY{o}{\PYZhy{}}\PY{l+m+mf}{0.0868}\PY{p}{,} \PY{l+m+mf}{0.2281}\PY{p}{,} \PY{l+m+mf}{0.3824}\PY{p}{]}\PY{p}{)}
\end{Verbatim}


    \begin{Verbatim}[commandchars=\\\{\}]
{\color{incolor}In [{\color{incolor}29}]:} \PY{n}{NptsPlot} \PY{o}{=} \PY{l+m+mi}{500}
         \PY{n}{A} \PY{o}{=} \PY{l+m+mf}{0.1}
         \PY{n}{B} \PY{o}{=} \PY{l+m+mf}{1.5}
         \PY{n}{x\PYZus{}plot} \PY{o}{=} \PY{n}{np}\PY{o}{.}\PY{n}{linspace}\PY{p}{(}\PY{l+m+mf}{0.1}\PY{p}{,} \PY{l+m+mf}{1.5}\PY{p}{,} \PY{n}{NptsPlot}\PY{p}{)}
         \PY{n}{y\PYZus{}neville} \PY{o}{=} \PY{n}{np}\PY{o}{.}\PY{n}{zeros}\PY{p}{(}\PY{n}{NptsPlot}\PY{p}{)}
         \PY{n}{y\PYZus{}rational} \PY{o}{=} \PY{n}{np}\PY{o}{.}\PY{n}{zeros}\PY{p}{(}\PY{n}{NptsPlot}\PY{p}{)}
         \PY{k}{for} \PY{n}{i} \PY{o+ow}{in} \PY{n+nb}{range}\PY{p}{(}\PY{n}{NptsPlot}\PY{p}{)}\PY{p}{:}
             \PY{n}{y\PYZus{}neville}\PY{p}{[}\PY{n}{i}\PY{p}{]} \PY{o}{=} \PY{n}{neville\PYZus{}interp}\PY{p}{(}\PY{n}{x}\PY{p}{,} \PY{n}{y}\PY{p}{,} \PY{n}{x\PYZus{}plot}\PY{p}{[}\PY{n}{i}\PY{p}{]}\PY{p}{)}
             \PY{n}{y\PYZus{}rational}\PY{p}{[}\PY{n}{i}\PY{p}{]} \PY{o}{=} \PY{n}{rational\PYZus{}interp}\PY{p}{(}\PY{n}{x}\PY{p}{,} \PY{n}{y}\PY{p}{,} \PY{n}{x\PYZus{}plot}\PY{p}{[}\PY{n}{i}\PY{p}{]}\PY{p}{)}
         \PY{n}{plt}\PY{o}{.}\PY{n}{clf}\PY{p}{(}\PY{p}{)}
         \PY{n}{plt}\PY{o}{.}\PY{n}{plot}\PY{p}{(}\PY{n}{x\PYZus{}plot}\PY{p}{,} \PY{n}{y\PYZus{}neville}\PY{p}{,} \PY{n}{label}\PY{o}{=}\PY{l+s+s2}{\PYZdq{}}\PY{l+s+s2}{Menggunakan interpolasi Neville}\PY{l+s+s2}{\PYZdq{}}\PY{p}{)}
         \PY{n}{plt}\PY{o}{.}\PY{n}{plot}\PY{p}{(}\PY{n}{x\PYZus{}plot}\PY{p}{,} \PY{n}{y\PYZus{}rational}\PY{p}{,} \PY{n}{label}\PY{o}{=}\PY{l+s+s2}{\PYZdq{}}\PY{l+s+s2}{Menggunakan interpolasi fungsi rasional}\PY{l+s+s2}{\PYZdq{}}\PY{p}{)}
         \PY{n}{plt}\PY{o}{.}\PY{n}{legend}\PY{p}{(}\PY{p}{)}\PY{p}{;}
\end{Verbatim}


    \begin{center}
    \adjustimage{max size={0.9\linewidth}{0.9\paperheight}}{NOTES_Interpolasi_files/NOTES_Interpolasi_45_0.pdf}
    \end{center}
    { \hspace*{\fill} \\}
    
    \hypertarget{tes-fungsi-x2}{%
\subsubsection{\texorpdfstring{Tes fungsi
\(x^2\)}{Tes fungsi x\^{}2}}\label{tes-fungsi-x2}}

    \begin{Verbatim}[commandchars=\\\{\}]
{\color{incolor}In [{\color{incolor}24}]:} \PY{k}{def} \PY{n+nf}{func\PYZus{}02}\PY{p}{(}\PY{n}{x}\PY{p}{)}\PY{p}{:}
             \PY{k}{return} \PY{n}{x}\PY{o}{*}\PY{o}{*}\PY{l+m+mi}{2} \PY{o}{\PYZhy{}} \PY{l+m+mf}{3.0}
\end{Verbatim}


    \begin{Verbatim}[commandchars=\\\{\}]
{\color{incolor}In [{\color{incolor}30}]:} \PY{n}{A} \PY{o}{=} \PY{o}{\PYZhy{}}\PY{l+m+mf}{1.5}
         \PY{n}{B} \PY{o}{=} \PY{l+m+mf}{1.5}
         
         \PY{n}{Nsample} \PY{o}{=} \PY{l+m+mi}{5}
         \PY{n}{x\PYZus{}sample} \PY{o}{=} \PY{n}{np}\PY{o}{.}\PY{n}{linspace}\PY{p}{(}\PY{n}{A}\PY{p}{,} \PY{n}{B}\PY{p}{,} \PY{n}{Nsample}\PY{p}{)}
         \PY{n}{y\PYZus{}sample} \PY{o}{=} \PY{n}{func\PYZus{}02}\PY{p}{(}\PY{n}{x\PYZus{}sample}\PY{p}{)}
             
         \PY{n}{NptsPlot} \PY{o}{=} \PY{l+m+mi}{100}
         \PY{n}{x\PYZus{}plot} \PY{o}{=} \PY{n}{np}\PY{o}{.}\PY{n}{linspace}\PY{p}{(}\PY{n}{A}\PY{p}{,} \PY{n}{B}\PY{p}{,} \PY{n}{NptsPlot}\PY{p}{)}
         \PY{n}{y\PYZus{}exact} \PY{o}{=} \PY{n}{func\PYZus{}02}\PY{p}{(}\PY{n}{x\PYZus{}plot}\PY{p}{)}
         \PY{n}{y\PYZus{}neville} \PY{o}{=} \PY{n}{np}\PY{o}{.}\PY{n}{zeros}\PY{p}{(}\PY{n}{NptsPlot}\PY{p}{)}
         \PY{n}{y\PYZus{}rational} \PY{o}{=} \PY{n}{np}\PY{o}{.}\PY{n}{zeros}\PY{p}{(}\PY{n}{NptsPlot}\PY{p}{)}
         \PY{k}{for} \PY{n}{i} \PY{o+ow}{in} \PY{n+nb}{range}\PY{p}{(}\PY{n}{NptsPlot}\PY{p}{)}\PY{p}{:}
             \PY{n}{y\PYZus{}neville}\PY{p}{[}\PY{n}{i}\PY{p}{]} \PY{o}{=} \PY{n}{neville\PYZus{}interp}\PY{p}{(}\PY{n}{x\PYZus{}sample}\PY{p}{,} \PY{n}{y\PYZus{}sample}\PY{p}{,} \PY{n}{x\PYZus{}plot}\PY{p}{[}\PY{n}{i}\PY{p}{]}\PY{p}{)}
             \PY{n}{y\PYZus{}rational}\PY{p}{[}\PY{n}{i}\PY{p}{]} \PY{o}{=} \PY{n}{rational\PYZus{}interp}\PY{p}{(}\PY{n}{x\PYZus{}sample}\PY{p}{,} \PY{n}{y\PYZus{}sample}\PY{p}{,} \PY{n}{x\PYZus{}plot}\PY{p}{[}\PY{n}{i}\PY{p}{]}\PY{p}{)}
         \PY{n}{plt}\PY{o}{.}\PY{n}{clf}\PY{p}{(}\PY{p}{)}
         \PY{n}{plt}\PY{o}{.}\PY{n}{plot}\PY{p}{(}\PY{n}{x\PYZus{}plot}\PY{p}{,} \PY{n}{y\PYZus{}exact}\PY{p}{,} \PY{n}{label}\PY{o}{=}\PY{l+s+s2}{\PYZdq{}}\PY{l+s+s2}{eksak}\PY{l+s+s2}{\PYZdq{}}\PY{p}{)}
         \PY{n}{plt}\PY{o}{.}\PY{n}{plot}\PY{p}{(}\PY{n}{x\PYZus{}plot}\PY{p}{,} \PY{n}{y\PYZus{}neville}\PY{p}{,} \PY{n}{label}\PY{o}{=}\PY{l+s+s2}{\PYZdq{}}\PY{l+s+s2}{Interpolasi Neville}\PY{l+s+s2}{\PYZdq{}}\PY{p}{)}
         \PY{n}{plt}\PY{o}{.}\PY{n}{plot}\PY{p}{(}\PY{n}{x\PYZus{}plot}\PY{p}{,} \PY{n}{y\PYZus{}rational}\PY{p}{,} \PY{n}{label}\PY{o}{=}\PY{l+s+s2}{\PYZdq{}}\PY{l+s+s2}{Interpolasi fungsi rasional}\PY{l+s+s2}{\PYZdq{}}\PY{p}{)}
         \PY{n}{plt}\PY{o}{.}\PY{n}{legend}\PY{p}{(}\PY{p}{)}\PY{p}{;}
\end{Verbatim}


    \begin{center}
    \adjustimage{max size={0.9\linewidth}{0.9\paperheight}}{NOTES_Interpolasi_files/NOTES_Interpolasi_48_0.pdf}
    \end{center}
    { \hspace*{\fill} \\}
    
    \hypertarget{tes-fungsi-x3}{%
\subsubsection{\texorpdfstring{Tes fungsi
\(x^3\)}{Tes fungsi x\^{}3}}\label{tes-fungsi-x3}}

    \begin{Verbatim}[commandchars=\\\{\}]
{\color{incolor}In [{\color{incolor}26}]:} \PY{k}{def} \PY{n+nf}{func\PYZus{}03}\PY{p}{(}\PY{n}{x}\PY{p}{)}\PY{p}{:}
             \PY{k}{return} \PY{n}{x}\PY{o}{*}\PY{o}{*}\PY{l+m+mi}{3} \PY{o}{\PYZhy{}} \PY{l+m+mf}{3.0}
\end{Verbatim}


    \begin{Verbatim}[commandchars=\\\{\}]
{\color{incolor}In [{\color{incolor}31}]:} \PY{n}{A} \PY{o}{=} \PY{o}{\PYZhy{}}\PY{l+m+mf}{1.5}
         \PY{n}{B} \PY{o}{=} \PY{l+m+mf}{1.5}
         
         \PY{n}{Nsample} \PY{o}{=} \PY{l+m+mi}{10}
         \PY{n}{x\PYZus{}sample} \PY{o}{=} \PY{n}{np}\PY{o}{.}\PY{n}{linspace}\PY{p}{(}\PY{n}{A}\PY{p}{,} \PY{n}{B}\PY{p}{,} \PY{n}{Nsample}\PY{p}{)}
         \PY{n}{y\PYZus{}sample} \PY{o}{=} \PY{n}{func\PYZus{}03}\PY{p}{(}\PY{n}{x\PYZus{}sample}\PY{p}{)}
             
         \PY{n}{NptsPlot} \PY{o}{=} \PY{l+m+mi}{100}
         \PY{n}{x\PYZus{}plot} \PY{o}{=} \PY{n}{np}\PY{o}{.}\PY{n}{linspace}\PY{p}{(}\PY{n}{A}\PY{p}{,} \PY{n}{B}\PY{p}{,} \PY{n}{NptsPlot}\PY{p}{)}
         \PY{n}{y\PYZus{}exact} \PY{o}{=} \PY{n}{func\PYZus{}03}\PY{p}{(}\PY{n}{x\PYZus{}plot}\PY{p}{)}
         \PY{n}{y\PYZus{}neville} \PY{o}{=} \PY{n}{np}\PY{o}{.}\PY{n}{zeros}\PY{p}{(}\PY{n}{NptsPlot}\PY{p}{)}
         \PY{n}{y\PYZus{}rational} \PY{o}{=} \PY{n}{np}\PY{o}{.}\PY{n}{zeros}\PY{p}{(}\PY{n}{NptsPlot}\PY{p}{)}
         \PY{k}{for} \PY{n}{i} \PY{o+ow}{in} \PY{n+nb}{range}\PY{p}{(}\PY{n}{NptsPlot}\PY{p}{)}\PY{p}{:}
             \PY{n}{y\PYZus{}neville}\PY{p}{[}\PY{n}{i}\PY{p}{]} \PY{o}{=} \PY{n}{neville\PYZus{}interp}\PY{p}{(}\PY{n}{x\PYZus{}sample}\PY{p}{,} \PY{n}{y\PYZus{}sample}\PY{p}{,} \PY{n}{x\PYZus{}plot}\PY{p}{[}\PY{n}{i}\PY{p}{]}\PY{p}{)}
             \PY{n}{y\PYZus{}rational}\PY{p}{[}\PY{n}{i}\PY{p}{]} \PY{o}{=} \PY{n}{rational\PYZus{}interp}\PY{p}{(}\PY{n}{x\PYZus{}sample}\PY{p}{,} \PY{n}{y\PYZus{}sample}\PY{p}{,} \PY{n}{x\PYZus{}plot}\PY{p}{[}\PY{n}{i}\PY{p}{]}\PY{p}{)}
         \PY{n}{plt}\PY{o}{.}\PY{n}{clf}\PY{p}{(}\PY{p}{)}
         \PY{n}{plt}\PY{o}{.}\PY{n}{plot}\PY{p}{(}\PY{n}{x\PYZus{}plot}\PY{p}{,} \PY{n}{y\PYZus{}exact}\PY{p}{,} \PY{n}{label}\PY{o}{=}\PY{l+s+s2}{\PYZdq{}}\PY{l+s+s2}{eksak}\PY{l+s+s2}{\PYZdq{}}\PY{p}{)}
         \PY{n}{plt}\PY{o}{.}\PY{n}{plot}\PY{p}{(}\PY{n}{x\PYZus{}plot}\PY{p}{,} \PY{n}{y\PYZus{}neville}\PY{p}{,} \PY{n}{label}\PY{o}{=}\PY{l+s+s2}{\PYZdq{}}\PY{l+s+s2}{Interpolasi Neville}\PY{l+s+s2}{\PYZdq{}}\PY{p}{)}
         \PY{n}{plt}\PY{o}{.}\PY{n}{plot}\PY{p}{(}\PY{n}{x\PYZus{}plot}\PY{p}{,} \PY{n}{y\PYZus{}rational}\PY{p}{,} \PY{n}{label}\PY{o}{=}\PY{l+s+s2}{\PYZdq{}}\PY{l+s+s2}{Interpolasi rational}\PY{l+s+s2}{\PYZdq{}}\PY{p}{)}
         \PY{n}{plt}\PY{o}{.}\PY{n}{legend}\PY{p}{(}\PY{p}{)}\PY{p}{;}
\end{Verbatim}


    \begin{Verbatim}[commandchars=\\\{\}]
/home/efefer/miniconda3/lib/python3.7/site-packages/ipykernel\_launcher.py:15: RuntimeWarning: divide by zero encountered in double\_scalars
  from ipykernel import kernelapp as app
/home/efefer/miniconda3/lib/python3.7/site-packages/ipykernel\_launcher.py:15: RuntimeWarning: invalid value encountered in double\_scalars
  from ipykernel import kernelapp as app

    \end{Verbatim}

    \begin{center}
    \adjustimage{max size={0.9\linewidth}{0.9\paperheight}}{NOTES_Interpolasi_files/NOTES_Interpolasi_51_1.pdf}
    \end{center}
    { \hspace*{\fill} \\}
    
    \hypertarget{interpolasi-dengan-spline}{%
\section{Interpolasi dengan spline}\label{interpolasi-dengan-spline}}

    TODO.

Merupakan metode yang dianggap paling robus dan banyak digunakan dalam
aplikasi. Implementasinya cukup rumit untuk dilakukan.

    \hypertarget{menggunakan-pustaka-python}{%
\section{Menggunakan pustaka Python}\label{menggunakan-pustaka-python}}

    Module \texttt{scipy.interpolate} menyediakan banyak fungsi untuk
melakukan interpolasi. Informasi lebih lengkap dapat dilihat pada
\href{https://docs.scipy.org/doc/scipy/reference/interpolate.html}{dokumentasinya}.

Beberapa fungsi yang sering dipakai adalah: - fungsi \texttt{interp1d} -
\texttt{splrep} dan \texttt{splev} (menggunakan metode B-spline)

    \begin{Verbatim}[commandchars=\\\{\}]
{\color{incolor}In [{\color{incolor}41}]:} \PY{k+kn}{import} \PY{n+nn}{scipy}\PY{n+nn}{.}\PY{n+nn}{interpolate}
\end{Verbatim}


    Buat data dengan menggunakan fungsi yang sudah kita ketahui bentuknya.

    \begin{Verbatim}[commandchars=\\\{\}]
{\color{incolor}In [{\color{incolor}39}]:} \PY{n}{A} \PY{o}{=} \PY{l+m+mf}{0.0}
         \PY{n}{B} \PY{o}{=} \PY{n}{np}\PY{o}{.}\PY{n}{pi}
         \PY{n}{Nsample} \PY{o}{=} \PY{l+m+mi}{7}
         \PY{n}{x\PYZus{}sample} \PY{o}{=} \PY{n}{np}\PY{o}{.}\PY{n}{linspace}\PY{p}{(}\PY{n}{A}\PY{p}{,}\PY{n}{B}\PY{p}{,}\PY{n}{Nsample}\PY{p}{)}
         \PY{n}{y\PYZus{}sample} \PY{o}{=} \PY{n}{func\PYZus{}01}\PY{p}{(}\PY{n}{x\PYZus{}sample}\PY{p}{)}
\end{Verbatim}


    Buat interpolant, dengan menggunakan spline linear, kuadratic, dan
kubik.

    \begin{Verbatim}[commandchars=\\\{\}]
{\color{incolor}In [{\color{incolor}48}]:} \PY{n}{f\PYZus{}slinear} \PY{o}{=} \PY{n}{scipy}\PY{o}{.}\PY{n}{interpolate}\PY{o}{.}\PY{n}{interp1d}\PY{p}{(}\PY{n}{x\PYZus{}sample}\PY{p}{,} \PY{n}{y\PYZus{}sample}\PY{p}{,} \PY{n}{kind}\PY{o}{=}\PY{l+s+s2}{\PYZdq{}}\PY{l+s+s2}{slinear}\PY{l+s+s2}{\PYZdq{}}\PY{p}{)}
         \PY{n}{f\PYZus{}quadratic} \PY{o}{=} \PY{n}{scipy}\PY{o}{.}\PY{n}{interpolate}\PY{o}{.}\PY{n}{interp1d}\PY{p}{(}\PY{n}{x\PYZus{}sample}\PY{p}{,} \PY{n}{y\PYZus{}sample}\PY{p}{,} \PY{n}{kind}\PY{o}{=}\PY{l+s+s2}{\PYZdq{}}\PY{l+s+s2}{quadratic}\PY{l+s+s2}{\PYZdq{}}\PY{p}{)}
         \PY{n}{f\PYZus{}cubic} \PY{o}{=} \PY{n}{scipy}\PY{o}{.}\PY{n}{interpolate}\PY{o}{.}\PY{n}{interp1d}\PY{p}{(}\PY{n}{x\PYZus{}sample}\PY{p}{,} \PY{n}{y\PYZus{}sample}\PY{p}{,} \PY{n}{kind}\PY{o}{=}\PY{l+s+s2}{\PYZdq{}}\PY{l+s+s2}{cubic}\PY{l+s+s2}{\PYZdq{}}\PY{p}{)}
\end{Verbatim}


    \begin{Verbatim}[commandchars=\\\{\}]
{\color{incolor}In [{\color{incolor}49}]:} \PY{n}{NptsPlot} \PY{o}{=} \PY{l+m+mi}{500}
         \PY{n}{x\PYZus{}plot} \PY{o}{=} \PY{n}{np}\PY{o}{.}\PY{n}{linspace}\PY{p}{(}\PY{n}{A}\PY{p}{,}\PY{n}{B}\PY{p}{,}\PY{n}{NptsPlot}\PY{p}{)}
         \PY{n}{y\PYZus{}slinear} \PY{o}{=} \PY{n}{f\PYZus{}slinear}\PY{p}{(}\PY{n}{x\PYZus{}plot}\PY{p}{)}
         \PY{n}{y\PYZus{}quadratic} \PY{o}{=} \PY{n}{f\PYZus{}quadratic}\PY{p}{(}\PY{n}{x\PYZus{}plot}\PY{p}{)}
         \PY{n}{y\PYZus{}cubic} \PY{o}{=} \PY{n}{f\PYZus{}cubic}\PY{p}{(}\PY{n}{x\PYZus{}plot}\PY{p}{)}
\end{Verbatim}


    \begin{Verbatim}[commandchars=\\\{\}]
{\color{incolor}In [{\color{incolor}54}]:} \PY{n}{plt}\PY{o}{.}\PY{n}{clf}\PY{p}{(}\PY{p}{)}
         \PY{n}{plt}\PY{o}{.}\PY{n}{plot}\PY{p}{(}\PY{n}{x\PYZus{}sample}\PY{p}{,} \PY{n}{y\PYZus{}sample}\PY{p}{,} \PY{n}{marker}\PY{o}{=}\PY{l+s+s2}{\PYZdq{}}\PY{l+s+s2}{o}\PY{l+s+s2}{\PYZdq{}}\PY{p}{,} \PY{n}{label}\PY{o}{=}\PY{l+s+s2}{\PYZdq{}}\PY{l+s+s2}{sample}\PY{l+s+s2}{\PYZdq{}}\PY{p}{)}
         \PY{n}{plt}\PY{o}{.}\PY{n}{plot}\PY{p}{(}\PY{n}{x\PYZus{}plot}\PY{p}{,} \PY{n}{y\PYZus{}slinear}\PY{p}{,} \PY{n}{label}\PY{o}{=}\PY{l+s+s2}{\PYZdq{}}\PY{l+s+s2}{slinear}\PY{l+s+s2}{\PYZdq{}}\PY{p}{)}
         \PY{n}{plt}\PY{o}{.}\PY{n}{plot}\PY{p}{(}\PY{n}{x\PYZus{}plot}\PY{p}{,} \PY{n}{y\PYZus{}quadratic}\PY{p}{,} \PY{n}{label}\PY{o}{=}\PY{l+s+s2}{\PYZdq{}}\PY{l+s+s2}{squadratic}\PY{l+s+s2}{\PYZdq{}}\PY{p}{)}
         \PY{n}{plt}\PY{o}{.}\PY{n}{plot}\PY{p}{(}\PY{n}{x\PYZus{}plot}\PY{p}{,} \PY{n}{y\PYZus{}cubic}\PY{p}{,} \PY{n}{label}\PY{o}{=}\PY{l+s+s2}{\PYZdq{}}\PY{l+s+s2}{cubic}\PY{l+s+s2}{\PYZdq{}}\PY{p}{)}
         \PY{n}{plt}\PY{o}{.}\PY{n}{legend}\PY{p}{(}\PY{p}{)}\PY{p}{;}
\end{Verbatim}


    \begin{center}
    \adjustimage{max size={0.9\linewidth}{0.9\paperheight}}{NOTES_Interpolasi_files/NOTES_Interpolasi_62_0.pdf}
    \end{center}
    { \hspace*{\fill} \\}
    

    % Add a bibliography block to the postdoc
    
    
    
    \end{document}
