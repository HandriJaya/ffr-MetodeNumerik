% Default to the notebook output style
% Inherit from the specified cell style.
\documentclass[10pt]{article}

\usepackage[a4paper]{geometry}    
\geometry{verbose,tmargin=1.5cm,bmargin=1.5cm,lmargin=1.5cm,rmargin=1.5cm}


    \usepackage[T1]{fontenc}
    % Nicer default font (+ math font) than Computer Modern for most use cases
    \usepackage{mathpazo}

    % Basic figure setup, for now with no caption control since it's done
    % automatically by Pandoc (which extracts ![](path) syntax from Markdown).
    \usepackage{graphicx}
    % We will generate all images so they have a width \maxwidth. This means
    % that they will get their normal width if they fit onto the page, but
    % are scaled down if they would overflow the margins.
    \makeatletter
    \def\maxwidth{\ifdim\Gin@nat@width>\linewidth\linewidth
    \else\Gin@nat@width\fi}
    \makeatother
    \let\Oldincludegraphics\includegraphics
    % Set max figure width to be 80% of text width, for now hardcoded.
    \renewcommand{\includegraphics}[1]{\Oldincludegraphics[width=.8\maxwidth]{#1}}
    % Ensure that by default, figures have no caption (until we provide a
    % proper Figure object with a Caption API and a way to capture that
    % in the conversion process - todo).
    \usepackage{caption}
    \DeclareCaptionLabelFormat{nolabel}{}
    \captionsetup{labelformat=nolabel}

    \usepackage{adjustbox} % Used to constrain images to a maximum size 
    \usepackage{xcolor} % Allow colors to be defined
    \usepackage{enumerate} % Needed for markdown enumerations to work
    \usepackage{geometry} % Used to adjust the document margins
    \usepackage{amsmath} % Equations
    \usepackage{amssymb} % Equations
    \usepackage{textcomp} % defines textquotesingle
    % Hack from http://tex.stackexchange.com/a/47451/13684:
    \AtBeginDocument{%
        \def\PYZsq{\textquotesingle}% Upright quotes in Pygmentized code
    }
    \usepackage{upquote} % Upright quotes for verbatim code
    \usepackage{eurosym} % defines \euro
    \usepackage[mathletters]{ucs} % Extended unicode (utf-8) support
    \usepackage[utf8x]{inputenc} % Allow utf-8 characters in the tex document
    \usepackage{fancyvrb} % verbatim replacement that allows latex
    \usepackage{grffile} % extends the file name processing of package graphics 
                         % to support a larger range 
    % The hyperref package gives us a pdf with properly built
    % internal navigation ('pdf bookmarks' for the table of contents,
    % internal cross-reference links, web links for URLs, etc.)
    \usepackage{hyperref}
    \usepackage{longtable} % longtable support required by pandoc >1.10
    \usepackage{booktabs}  % table support for pandoc > 1.12.2
    \usepackage[inline]{enumitem} % IRkernel/repr support (it uses the enumerate* environment)
    \usepackage[normalem]{ulem} % ulem is needed to support strikethroughs (\sout)
                                % normalem makes italics be italics, not underlines
    

    
    
    % Colors for the hyperref package
    \definecolor{urlcolor}{rgb}{0,.145,.698}
    \definecolor{linkcolor}{rgb}{.71,0.21,0.01}
    \definecolor{citecolor}{rgb}{.12,.54,.11}

    % ANSI colors
    \definecolor{ansi-black}{HTML}{3E424D}
    \definecolor{ansi-black-intense}{HTML}{282C36}
    \definecolor{ansi-red}{HTML}{E75C58}
    \definecolor{ansi-red-intense}{HTML}{B22B31}
    \definecolor{ansi-green}{HTML}{00A250}
    \definecolor{ansi-green-intense}{HTML}{007427}
    \definecolor{ansi-yellow}{HTML}{DDB62B}
    \definecolor{ansi-yellow-intense}{HTML}{B27D12}
    \definecolor{ansi-blue}{HTML}{208FFB}
    \definecolor{ansi-blue-intense}{HTML}{0065CA}
    \definecolor{ansi-magenta}{HTML}{D160C4}
    \definecolor{ansi-magenta-intense}{HTML}{A03196}
    \definecolor{ansi-cyan}{HTML}{60C6C8}
    \definecolor{ansi-cyan-intense}{HTML}{258F8F}
    \definecolor{ansi-white}{HTML}{C5C1B4}
    \definecolor{ansi-white-intense}{HTML}{A1A6B2}

    % commands and environments needed by pandoc snippets
    % extracted from the output of `pandoc -s`
    \providecommand{\tightlist}{%
      \setlength{\itemsep}{0pt}\setlength{\parskip}{0pt}}
    \DefineVerbatimEnvironment{Highlighting}{Verbatim}{commandchars=\\\{\}}
    % Add ',fontsize=\small' for more characters per line
    \newenvironment{Shaded}{}{}
    \newcommand{\KeywordTok}[1]{\textcolor[rgb]{0.00,0.44,0.13}{\textbf{{#1}}}}
    \newcommand{\DataTypeTok}[1]{\textcolor[rgb]{0.56,0.13,0.00}{{#1}}}
    \newcommand{\DecValTok}[1]{\textcolor[rgb]{0.25,0.63,0.44}{{#1}}}
    \newcommand{\BaseNTok}[1]{\textcolor[rgb]{0.25,0.63,0.44}{{#1}}}
    \newcommand{\FloatTok}[1]{\textcolor[rgb]{0.25,0.63,0.44}{{#1}}}
    \newcommand{\CharTok}[1]{\textcolor[rgb]{0.25,0.44,0.63}{{#1}}}
    \newcommand{\StringTok}[1]{\textcolor[rgb]{0.25,0.44,0.63}{{#1}}}
    \newcommand{\CommentTok}[1]{\textcolor[rgb]{0.38,0.63,0.69}{\textit{{#1}}}}
    \newcommand{\OtherTok}[1]{\textcolor[rgb]{0.00,0.44,0.13}{{#1}}}
    \newcommand{\AlertTok}[1]{\textcolor[rgb]{1.00,0.00,0.00}{\textbf{{#1}}}}
    \newcommand{\FunctionTok}[1]{\textcolor[rgb]{0.02,0.16,0.49}{{#1}}}
    \newcommand{\RegionMarkerTok}[1]{{#1}}
    \newcommand{\ErrorTok}[1]{\textcolor[rgb]{1.00,0.00,0.00}{\textbf{{#1}}}}
    \newcommand{\NormalTok}[1]{{#1}}
    
    % Additional commands for more recent versions of Pandoc
    \newcommand{\ConstantTok}[1]{\textcolor[rgb]{0.53,0.00,0.00}{{#1}}}
    \newcommand{\SpecialCharTok}[1]{\textcolor[rgb]{0.25,0.44,0.63}{{#1}}}
    \newcommand{\VerbatimStringTok}[1]{\textcolor[rgb]{0.25,0.44,0.63}{{#1}}}
    \newcommand{\SpecialStringTok}[1]{\textcolor[rgb]{0.73,0.40,0.53}{{#1}}}
    \newcommand{\ImportTok}[1]{{#1}}
    \newcommand{\DocumentationTok}[1]{\textcolor[rgb]{0.73,0.13,0.13}{\textit{{#1}}}}
    \newcommand{\AnnotationTok}[1]{\textcolor[rgb]{0.38,0.63,0.69}{\textbf{\textit{{#1}}}}}
    \newcommand{\CommentVarTok}[1]{\textcolor[rgb]{0.38,0.63,0.69}{\textbf{\textit{{#1}}}}}
    \newcommand{\VariableTok}[1]{\textcolor[rgb]{0.10,0.09,0.49}{{#1}}}
    \newcommand{\ControlFlowTok}[1]{\textcolor[rgb]{0.00,0.44,0.13}{\textbf{{#1}}}}
    \newcommand{\OperatorTok}[1]{\textcolor[rgb]{0.40,0.40,0.40}{{#1}}}
    \newcommand{\BuiltInTok}[1]{{#1}}
    \newcommand{\ExtensionTok}[1]{{#1}}
    \newcommand{\PreprocessorTok}[1]{\textcolor[rgb]{0.74,0.48,0.00}{{#1}}}
    \newcommand{\AttributeTok}[1]{\textcolor[rgb]{0.49,0.56,0.16}{{#1}}}
    \newcommand{\InformationTok}[1]{\textcolor[rgb]{0.38,0.63,0.69}{\textbf{\textit{{#1}}}}}
    \newcommand{\WarningTok}[1]{\textcolor[rgb]{0.38,0.63,0.69}{\textbf{\textit{{#1}}}}}
    
    
    % Define a nice break command that doesn't care if a line doesn't already
    % exist.
    \def\br{\hspace*{\fill} \\* }
    % Math Jax compatability definitions
    \def\gt{>}
    \def\lt{<}
    % Document parameters
    \title{NOTES\_SistemPersLinear}
    
    
    

    % Pygments definitions
    
\makeatletter
\def\PY@reset{\let\PY@it=\relax \let\PY@bf=\relax%
    \let\PY@ul=\relax \let\PY@tc=\relax%
    \let\PY@bc=\relax \let\PY@ff=\relax}
\def\PY@tok#1{\csname PY@tok@#1\endcsname}
\def\PY@toks#1+{\ifx\relax#1\empty\else%
    \PY@tok{#1}\expandafter\PY@toks\fi}
\def\PY@do#1{\PY@bc{\PY@tc{\PY@ul{%
    \PY@it{\PY@bf{\PY@ff{#1}}}}}}}
\def\PY#1#2{\PY@reset\PY@toks#1+\relax+\PY@do{#2}}

\expandafter\def\csname PY@tok@w\endcsname{\def\PY@tc##1{\textcolor[rgb]{0.73,0.73,0.73}{##1}}}
\expandafter\def\csname PY@tok@c\endcsname{\let\PY@it=\textit\def\PY@tc##1{\textcolor[rgb]{0.25,0.50,0.50}{##1}}}
\expandafter\def\csname PY@tok@cp\endcsname{\def\PY@tc##1{\textcolor[rgb]{0.74,0.48,0.00}{##1}}}
\expandafter\def\csname PY@tok@k\endcsname{\let\PY@bf=\textbf\def\PY@tc##1{\textcolor[rgb]{0.00,0.50,0.00}{##1}}}
\expandafter\def\csname PY@tok@kp\endcsname{\def\PY@tc##1{\textcolor[rgb]{0.00,0.50,0.00}{##1}}}
\expandafter\def\csname PY@tok@kt\endcsname{\def\PY@tc##1{\textcolor[rgb]{0.69,0.00,0.25}{##1}}}
\expandafter\def\csname PY@tok@o\endcsname{\def\PY@tc##1{\textcolor[rgb]{0.40,0.40,0.40}{##1}}}
\expandafter\def\csname PY@tok@ow\endcsname{\let\PY@bf=\textbf\def\PY@tc##1{\textcolor[rgb]{0.67,0.13,1.00}{##1}}}
\expandafter\def\csname PY@tok@nb\endcsname{\def\PY@tc##1{\textcolor[rgb]{0.00,0.50,0.00}{##1}}}
\expandafter\def\csname PY@tok@nf\endcsname{\def\PY@tc##1{\textcolor[rgb]{0.00,0.00,1.00}{##1}}}
\expandafter\def\csname PY@tok@nc\endcsname{\let\PY@bf=\textbf\def\PY@tc##1{\textcolor[rgb]{0.00,0.00,1.00}{##1}}}
\expandafter\def\csname PY@tok@nn\endcsname{\let\PY@bf=\textbf\def\PY@tc##1{\textcolor[rgb]{0.00,0.00,1.00}{##1}}}
\expandafter\def\csname PY@tok@ne\endcsname{\let\PY@bf=\textbf\def\PY@tc##1{\textcolor[rgb]{0.82,0.25,0.23}{##1}}}
\expandafter\def\csname PY@tok@nv\endcsname{\def\PY@tc##1{\textcolor[rgb]{0.10,0.09,0.49}{##1}}}
\expandafter\def\csname PY@tok@no\endcsname{\def\PY@tc##1{\textcolor[rgb]{0.53,0.00,0.00}{##1}}}
\expandafter\def\csname PY@tok@nl\endcsname{\def\PY@tc##1{\textcolor[rgb]{0.63,0.63,0.00}{##1}}}
\expandafter\def\csname PY@tok@ni\endcsname{\let\PY@bf=\textbf\def\PY@tc##1{\textcolor[rgb]{0.60,0.60,0.60}{##1}}}
\expandafter\def\csname PY@tok@na\endcsname{\def\PY@tc##1{\textcolor[rgb]{0.49,0.56,0.16}{##1}}}
\expandafter\def\csname PY@tok@nt\endcsname{\let\PY@bf=\textbf\def\PY@tc##1{\textcolor[rgb]{0.00,0.50,0.00}{##1}}}
\expandafter\def\csname PY@tok@nd\endcsname{\def\PY@tc##1{\textcolor[rgb]{0.67,0.13,1.00}{##1}}}
\expandafter\def\csname PY@tok@s\endcsname{\def\PY@tc##1{\textcolor[rgb]{0.73,0.13,0.13}{##1}}}
\expandafter\def\csname PY@tok@sd\endcsname{\let\PY@it=\textit\def\PY@tc##1{\textcolor[rgb]{0.73,0.13,0.13}{##1}}}
\expandafter\def\csname PY@tok@si\endcsname{\let\PY@bf=\textbf\def\PY@tc##1{\textcolor[rgb]{0.73,0.40,0.53}{##1}}}
\expandafter\def\csname PY@tok@se\endcsname{\let\PY@bf=\textbf\def\PY@tc##1{\textcolor[rgb]{0.73,0.40,0.13}{##1}}}
\expandafter\def\csname PY@tok@sr\endcsname{\def\PY@tc##1{\textcolor[rgb]{0.73,0.40,0.53}{##1}}}
\expandafter\def\csname PY@tok@ss\endcsname{\def\PY@tc##1{\textcolor[rgb]{0.10,0.09,0.49}{##1}}}
\expandafter\def\csname PY@tok@sx\endcsname{\def\PY@tc##1{\textcolor[rgb]{0.00,0.50,0.00}{##1}}}
\expandafter\def\csname PY@tok@m\endcsname{\def\PY@tc##1{\textcolor[rgb]{0.40,0.40,0.40}{##1}}}
\expandafter\def\csname PY@tok@gh\endcsname{\let\PY@bf=\textbf\def\PY@tc##1{\textcolor[rgb]{0.00,0.00,0.50}{##1}}}
\expandafter\def\csname PY@tok@gu\endcsname{\let\PY@bf=\textbf\def\PY@tc##1{\textcolor[rgb]{0.50,0.00,0.50}{##1}}}
\expandafter\def\csname PY@tok@gd\endcsname{\def\PY@tc##1{\textcolor[rgb]{0.63,0.00,0.00}{##1}}}
\expandafter\def\csname PY@tok@gi\endcsname{\def\PY@tc##1{\textcolor[rgb]{0.00,0.63,0.00}{##1}}}
\expandafter\def\csname PY@tok@gr\endcsname{\def\PY@tc##1{\textcolor[rgb]{1.00,0.00,0.00}{##1}}}
\expandafter\def\csname PY@tok@ge\endcsname{\let\PY@it=\textit}
\expandafter\def\csname PY@tok@gs\endcsname{\let\PY@bf=\textbf}
\expandafter\def\csname PY@tok@gp\endcsname{\let\PY@bf=\textbf\def\PY@tc##1{\textcolor[rgb]{0.00,0.00,0.50}{##1}}}
\expandafter\def\csname PY@tok@go\endcsname{\def\PY@tc##1{\textcolor[rgb]{0.53,0.53,0.53}{##1}}}
\expandafter\def\csname PY@tok@gt\endcsname{\def\PY@tc##1{\textcolor[rgb]{0.00,0.27,0.87}{##1}}}
\expandafter\def\csname PY@tok@err\endcsname{\def\PY@bc##1{\setlength{\fboxsep}{0pt}\fcolorbox[rgb]{1.00,0.00,0.00}{1,1,1}{\strut ##1}}}
\expandafter\def\csname PY@tok@kc\endcsname{\let\PY@bf=\textbf\def\PY@tc##1{\textcolor[rgb]{0.00,0.50,0.00}{##1}}}
\expandafter\def\csname PY@tok@kd\endcsname{\let\PY@bf=\textbf\def\PY@tc##1{\textcolor[rgb]{0.00,0.50,0.00}{##1}}}
\expandafter\def\csname PY@tok@kn\endcsname{\let\PY@bf=\textbf\def\PY@tc##1{\textcolor[rgb]{0.00,0.50,0.00}{##1}}}
\expandafter\def\csname PY@tok@kr\endcsname{\let\PY@bf=\textbf\def\PY@tc##1{\textcolor[rgb]{0.00,0.50,0.00}{##1}}}
\expandafter\def\csname PY@tok@bp\endcsname{\def\PY@tc##1{\textcolor[rgb]{0.00,0.50,0.00}{##1}}}
\expandafter\def\csname PY@tok@fm\endcsname{\def\PY@tc##1{\textcolor[rgb]{0.00,0.00,1.00}{##1}}}
\expandafter\def\csname PY@tok@vc\endcsname{\def\PY@tc##1{\textcolor[rgb]{0.10,0.09,0.49}{##1}}}
\expandafter\def\csname PY@tok@vg\endcsname{\def\PY@tc##1{\textcolor[rgb]{0.10,0.09,0.49}{##1}}}
\expandafter\def\csname PY@tok@vi\endcsname{\def\PY@tc##1{\textcolor[rgb]{0.10,0.09,0.49}{##1}}}
\expandafter\def\csname PY@tok@vm\endcsname{\def\PY@tc##1{\textcolor[rgb]{0.10,0.09,0.49}{##1}}}
\expandafter\def\csname PY@tok@sa\endcsname{\def\PY@tc##1{\textcolor[rgb]{0.73,0.13,0.13}{##1}}}
\expandafter\def\csname PY@tok@sb\endcsname{\def\PY@tc##1{\textcolor[rgb]{0.73,0.13,0.13}{##1}}}
\expandafter\def\csname PY@tok@sc\endcsname{\def\PY@tc##1{\textcolor[rgb]{0.73,0.13,0.13}{##1}}}
\expandafter\def\csname PY@tok@dl\endcsname{\def\PY@tc##1{\textcolor[rgb]{0.73,0.13,0.13}{##1}}}
\expandafter\def\csname PY@tok@s2\endcsname{\def\PY@tc##1{\textcolor[rgb]{0.73,0.13,0.13}{##1}}}
\expandafter\def\csname PY@tok@sh\endcsname{\def\PY@tc##1{\textcolor[rgb]{0.73,0.13,0.13}{##1}}}
\expandafter\def\csname PY@tok@s1\endcsname{\def\PY@tc##1{\textcolor[rgb]{0.73,0.13,0.13}{##1}}}
\expandafter\def\csname PY@tok@mb\endcsname{\def\PY@tc##1{\textcolor[rgb]{0.40,0.40,0.40}{##1}}}
\expandafter\def\csname PY@tok@mf\endcsname{\def\PY@tc##1{\textcolor[rgb]{0.40,0.40,0.40}{##1}}}
\expandafter\def\csname PY@tok@mh\endcsname{\def\PY@tc##1{\textcolor[rgb]{0.40,0.40,0.40}{##1}}}
\expandafter\def\csname PY@tok@mi\endcsname{\def\PY@tc##1{\textcolor[rgb]{0.40,0.40,0.40}{##1}}}
\expandafter\def\csname PY@tok@il\endcsname{\def\PY@tc##1{\textcolor[rgb]{0.40,0.40,0.40}{##1}}}
\expandafter\def\csname PY@tok@mo\endcsname{\def\PY@tc##1{\textcolor[rgb]{0.40,0.40,0.40}{##1}}}
\expandafter\def\csname PY@tok@ch\endcsname{\let\PY@it=\textit\def\PY@tc##1{\textcolor[rgb]{0.25,0.50,0.50}{##1}}}
\expandafter\def\csname PY@tok@cm\endcsname{\let\PY@it=\textit\def\PY@tc##1{\textcolor[rgb]{0.25,0.50,0.50}{##1}}}
\expandafter\def\csname PY@tok@cpf\endcsname{\let\PY@it=\textit\def\PY@tc##1{\textcolor[rgb]{0.25,0.50,0.50}{##1}}}
\expandafter\def\csname PY@tok@c1\endcsname{\let\PY@it=\textit\def\PY@tc##1{\textcolor[rgb]{0.25,0.50,0.50}{##1}}}
\expandafter\def\csname PY@tok@cs\endcsname{\let\PY@it=\textit\def\PY@tc##1{\textcolor[rgb]{0.25,0.50,0.50}{##1}}}

\def\PYZbs{\char`\\}
\def\PYZus{\char`\_}
\def\PYZob{\char`\{}
\def\PYZcb{\char`\}}
\def\PYZca{\char`\^}
\def\PYZam{\char`\&}
\def\PYZlt{\char`\<}
\def\PYZgt{\char`\>}
\def\PYZsh{\char`\#}
\def\PYZpc{\char`\%}
\def\PYZdl{\char`\$}
\def\PYZhy{\char`\-}
\def\PYZsq{\char`\'}
\def\PYZdq{\char`\"}
\def\PYZti{\char`\~}
% for compatibility with earlier versions
\def\PYZat{@}
\def\PYZlb{[}
\def\PYZrb{]}
\makeatother


    % Exact colors from NB
    \definecolor{incolor}{rgb}{0.0, 0.0, 0.5}
    \definecolor{outcolor}{rgb}{0.545, 0.0, 0.0}



    
    % Prevent overflowing lines due to hard-to-break entities
    \sloppy 
    % Setup hyperref package
    \hypersetup{
      breaklinks=true,  % so long urls are correctly broken across lines
      colorlinks=true,
      urlcolor=urlcolor,
      linkcolor=linkcolor,
      citecolor=citecolor,
      }
    % Slightly bigger margins than the latex defaults
    
    \geometry{verbose,tmargin=1in,bmargin=1in,lmargin=1in,rmargin=1in}
    
    

    \begin{document}
    
    
    \maketitle
    
    

    
    \hypertarget{tf2202-teknik-komputasi---sistem-persamaan-linear}{%
\section{TF2202 Teknik Komputasi - Sistem Persamaan
Linear}\label{tf2202-teknik-komputasi---sistem-persamaan-linear}}

Fadjar Fathurrahman

    \begin{Verbatim}[commandchars=\\\{\}]
{\color{incolor}In [{\color{incolor}1}]:} \PY{k+kn}{import} \PY{n+nn}{numpy} \PY{k}{as} \PY{n+nn}{np}
\end{Verbatim}


    Pada catatan ini kita kan fokus pada metode untuk mencari solusi dari
sistem persamaan linear yang dapat dituliskan dalam bentuk matriks
sebagai berikut: \[
\mathbf{A}\mathbf{x} = \mathbf{b}
\] di mana \(A\) dan \(b\) masing-masing diberikan dan tugas kita adalah
mencari \(\mathbf{x}\).

    Sebelum membahas mengenai metode numerik untuk menyelesaikan sistem
persamaan linear, kita akan mulai dengan pembahasan mengenai operasi
matriks dan vektor dengan dalam Numpy.

    \hypertarget{matrix-vs-ndarray}{%
\subsection{Matrix vs ndarray}\label{matrix-vs-ndarray}}

    \texttt{ndarray} adalah tipe array yang paling penting pada Numpy. Untuk
merepresentasikan matriks kita dapat menggunakan \texttt{ndarray} dengan
menggunakan \texttt{ndarray} dengan dua dimensi. Untuk merepresentasikan
vektor (baris atau kolom) kita dapat menggunakan \texttt{ndarray} dengan
satu dimensi.

    Contoh untuk matriks: \[
A = \begin{bmatrix}
4 & 1 & 2 & 3 \\
3 & 8 & 1 & 9 \\
3 & 4 & 10 & 4
\end{bmatrix}
\]

    Dengan menggunakan \texttt{ndarray}, kita dapa mendefinisikan matriks
\(A\) dengan:

    \begin{Verbatim}[commandchars=\\\{\}]
{\color{incolor}In [{\color{incolor}2}]:} \PY{n}{A} \PY{o}{=} \PY{n}{np}\PY{o}{.}\PY{n}{array}\PY{p}{(}\PY{p}{[}
            \PY{p}{[}\PY{l+m+mi}{4}\PY{p}{,}\PY{l+m+mi}{1}\PY{p}{,}\PY{l+m+mi}{2}\PY{p}{,}\PY{l+m+mi}{3}\PY{p}{]}\PY{p}{,}
            \PY{p}{[}\PY{l+m+mi}{3}\PY{p}{,}\PY{l+m+mi}{8}\PY{p}{,}\PY{l+m+mi}{1}\PY{p}{,}\PY{l+m+mi}{9}\PY{p}{]}\PY{p}{,}
            \PY{p}{[}\PY{l+m+mi}{3}\PY{p}{,}\PY{l+m+mi}{4}\PY{p}{,}\PY{l+m+mi}{10}\PY{p}{,}\PY{l+m+mi}{4}\PY{p}{]}
        \PY{p}{]}\PY{p}{)}
        \PY{n}{A}
\end{Verbatim}


\begin{Verbatim}[commandchars=\\\{\}]
{\color{outcolor}Out[{\color{outcolor}2}]:} array([[ 4,  1,  2,  3],
               [ 3,  8,  1,  9],
               [ 3,  4, 10,  4]])
\end{Verbatim}
            
    \begin{Verbatim}[commandchars=\\\{\}]
{\color{incolor}In [{\color{incolor}3}]:} \PY{n+nb}{type}\PY{p}{(}\PY{n}{A}\PY{p}{)}
\end{Verbatim}


\begin{Verbatim}[commandchars=\\\{\}]
{\color{outcolor}Out[{\color{outcolor}3}]:} numpy.ndarray
\end{Verbatim}
            
    Properti \texttt{shape} dapat digunakan untuk mengetahu ukuran dari
\texttt{ndarray}. Dalam hal ini kita akan mendapatkan tupel berisi dua
integer, yang masing-masing integer merupakan jumlah baris dan kolom
dari matriks \(A\)

    \begin{Verbatim}[commandchars=\\\{\}]
{\color{incolor}In [{\color{incolor}4}]:} \PY{n}{A}\PY{o}{.}\PY{n}{shape}
\end{Verbatim}


\begin{Verbatim}[commandchars=\\\{\}]
{\color{outcolor}Out[{\color{outcolor}4}]:} (3, 4)
\end{Verbatim}
            
    Anda dapat menuliskan sebagai berikut:

    \begin{Verbatim}[commandchars=\\\{\}]
{\color{incolor}In [{\color{incolor}5}]:} \PY{n}{Nrow} \PY{o}{=} \PY{n}{A}\PY{o}{.}\PY{n}{shape}\PY{p}{[}\PY{l+m+mi}{0}\PY{p}{]}
        \PY{n}{Ncol} \PY{o}{=} \PY{n}{A}\PY{o}{.}\PY{n}{shape}\PY{p}{[}\PY{l+m+mi}{1}\PY{p}{]}
        \PY{n+nb}{print}\PY{p}{(}\PY{l+s+s2}{\PYZdq{}}\PY{l+s+s2}{Nrow = }\PY{l+s+si}{\PYZpc{}d}\PY{l+s+s2}{, Ncol = }\PY{l+s+si}{\PYZpc{}d}\PY{l+s+s2}{\PYZdq{}} \PY{o}{\PYZpc{}} \PY{p}{(}\PY{n}{Nrow}\PY{p}{,} \PY{n}{Ncol}\PY{p}{)}\PY{p}{)}
\end{Verbatim}


    \begin{Verbatim}[commandchars=\\\{\}]
Nrow = 3, Ncol = 4

    \end{Verbatim}

    \begin{Verbatim}[commandchars=\\\{\}]
{\color{incolor}In [{\color{incolor}6}]:} \PY{n}{Nrow}\PY{p}{,} \PY{n}{Ncol} \PY{o}{=} \PY{n}{A}\PY{o}{.}\PY{n}{shape}
        \PY{n+nb}{print}\PY{p}{(}\PY{l+s+s2}{\PYZdq{}}\PY{l+s+s2}{Nrow = }\PY{l+s+si}{\PYZpc{}d}\PY{l+s+s2}{, Ncol = }\PY{l+s+si}{\PYZpc{}d}\PY{l+s+s2}{\PYZdq{}} \PY{o}{\PYZpc{}} \PY{p}{(}\PY{n}{Nrow}\PY{p}{,} \PY{n}{Ncol}\PY{p}{)}\PY{p}{)}
\end{Verbatim}


    \begin{Verbatim}[commandchars=\\\{\}]
Nrow = 3, Ncol = 4

    \end{Verbatim}

    Untuk merepresentasikan vektor, kita dapat menggunakan \texttt{ndarray}
1d, misalnya:

    \begin{Verbatim}[commandchars=\\\{\}]
{\color{incolor}In [{\color{incolor}7}]:} \PY{n}{x} \PY{o}{=} \PY{n}{np}\PY{o}{.}\PY{n}{array}\PY{p}{(}\PY{p}{[}\PY{l+m+mi}{3}\PY{p}{,}\PY{l+m+mi}{1}\PY{p}{,}\PY{l+m+mi}{6}\PY{p}{,}\PY{l+m+mi}{7}\PY{p}{]}\PY{p}{)}
        \PY{n}{x}
\end{Verbatim}


\begin{Verbatim}[commandchars=\\\{\}]
{\color{outcolor}Out[{\color{outcolor}7}]:} array([3, 1, 6, 7])
\end{Verbatim}
            
    \begin{Verbatim}[commandchars=\\\{\}]
{\color{incolor}In [{\color{incolor}8}]:} \PY{n+nb}{type}\PY{p}{(}\PY{n}{x}\PY{p}{)}
\end{Verbatim}


\begin{Verbatim}[commandchars=\\\{\}]
{\color{outcolor}Out[{\color{outcolor}8}]:} numpy.ndarray
\end{Verbatim}
            
    Properti \texttt{shape} dapat digunakan seperti pada \texttt{ndarray}
dua dimensi. Pada kasus ini akan dikembalikan tupel dengan satu bilangan
integer.

    \begin{Verbatim}[commandchars=\\\{\}]
{\color{incolor}In [{\color{incolor}9}]:} \PY{n}{x}\PY{o}{.}\PY{n}{shape}
\end{Verbatim}


\begin{Verbatim}[commandchars=\\\{\}]
{\color{outcolor}Out[{\color{outcolor}9}]:} (4,)
\end{Verbatim}
            
    Fungsi \texttt{len} juga dapat digunakan dalam kasus \texttt{ndarray} 1d
untuk mengetahui jumlah elemen pada suatu vektor:

    \begin{Verbatim}[commandchars=\\\{\}]
{\color{incolor}In [{\color{incolor}10}]:} \PY{n+nb}{len}\PY{p}{(}\PY{n}{x}\PY{p}{)}
\end{Verbatim}


\begin{Verbatim}[commandchars=\\\{\}]
{\color{outcolor}Out[{\color{outcolor}10}]:} 4
\end{Verbatim}
            
    Fungsi \texttt{len} juga dapat diaplikasikan pada \texttt{ndarray} dua
dimensi, namun fungsi ini akan mengembalikan banyak elemen pada dimensi
pertama.

    \begin{Verbatim}[commandchars=\\\{\}]
{\color{incolor}In [{\color{incolor}11}]:} \PY{n+nb}{len}\PY{p}{(}\PY{n}{A}\PY{p}{)}
\end{Verbatim}


\begin{Verbatim}[commandchars=\\\{\}]
{\color{outcolor}Out[{\color{outcolor}11}]:} 3
\end{Verbatim}
            
    Untuk mengetahui jumlah kolom, kita dapat mencari panjang dari
\texttt{A{[}0{]}} (misalnya):

    \begin{Verbatim}[commandchars=\\\{\}]
{\color{incolor}In [{\color{incolor}12}]:} \PY{n+nb}{len}\PY{p}{(}\PY{n}{A}\PY{p}{[}\PY{l+m+mi}{0}\PY{p}{]}\PY{p}{)}
\end{Verbatim}


\begin{Verbatim}[commandchars=\\\{\}]
{\color{outcolor}Out[{\color{outcolor}12}]:} 4
\end{Verbatim}
            
    \hypertarget{operasi-perkalian-matriks-dan-vektor}{%
\subsubsection{Operasi perkalian matriks dan
vektor}\label{operasi-perkalian-matriks-dan-vektor}}

    Untuk menghitung operasi perkalian, misalnya
\(\mathbf{b} = \mathbf{A}\mathbf{x}\). Untuk tipe \texttt{numpy.ndarray}
kita dapat menggunakan fungsi \texttt{np.matmul}:

    \begin{Verbatim}[commandchars=\\\{\}]
{\color{incolor}In [{\color{incolor}13}]:} \PY{n}{b} \PY{o}{=} \PY{n}{np}\PY{o}{.}\PY{n}{matmul}\PY{p}{(}\PY{n}{A}\PY{p}{,}\PY{n}{x}\PY{p}{)}
         \PY{n}{b}
\end{Verbatim}


\begin{Verbatim}[commandchars=\\\{\}]
{\color{outcolor}Out[{\color{outcolor}13}]:} array([ 46,  86, 101])
\end{Verbatim}
            
    Operator \texttt{*} memiliki arti yang berbeda untuk operasi antara dua
\texttt{ndarray}:

    \begin{Verbatim}[commandchars=\\\{\}]
{\color{incolor}In [{\color{incolor}14}]:} \PY{n}{A}
\end{Verbatim}


\begin{Verbatim}[commandchars=\\\{\}]
{\color{outcolor}Out[{\color{outcolor}14}]:} array([[ 4,  1,  2,  3],
                [ 3,  8,  1,  9],
                [ 3,  4, 10,  4]])
\end{Verbatim}
            
    \begin{Verbatim}[commandchars=\\\{\}]
{\color{incolor}In [{\color{incolor}15}]:} \PY{n}{Ax} \PY{o}{=} \PY{n}{A}\PY{o}{*}\PY{n}{x}
         \PY{n}{Ax}
\end{Verbatim}


\begin{Verbatim}[commandchars=\\\{\}]
{\color{outcolor}Out[{\color{outcolor}15}]:} array([[12,  1, 12, 21],
                [ 9,  8,  6, 63],
                [ 9,  4, 60, 28]])
\end{Verbatim}
            
    \begin{Verbatim}[commandchars=\\\{\}]
{\color{incolor}In [{\color{incolor}16}]:} \PY{n+nb}{print}\PY{p}{(}\PY{n}{Ax}\PY{p}{[}\PY{p}{:}\PY{p}{,}\PY{l+m+mi}{0}\PY{p}{]}\PY{o}{/}\PY{n}{A}\PY{p}{[}\PY{p}{:}\PY{p}{,}\PY{l+m+mi}{0}\PY{p}{]}\PY{p}{)}
         \PY{n+nb}{print}\PY{p}{(}\PY{n}{Ax}\PY{p}{[}\PY{p}{:}\PY{p}{,}\PY{l+m+mi}{1}\PY{p}{]}\PY{o}{/}\PY{n}{A}\PY{p}{[}\PY{p}{:}\PY{p}{,}\PY{l+m+mi}{1}\PY{p}{]}\PY{p}{)}
         \PY{n+nb}{print}\PY{p}{(}\PY{n}{Ax}\PY{p}{[}\PY{p}{:}\PY{p}{,}\PY{l+m+mi}{2}\PY{p}{]}\PY{o}{/}\PY{n}{A}\PY{p}{[}\PY{p}{:}\PY{p}{,}\PY{l+m+mi}{2}\PY{p}{]}\PY{p}{)}
\end{Verbatim}


    \begin{Verbatim}[commandchars=\\\{\}]
[3. 3. 3.]
[1. 1. 1.]
[6. 6. 6.]

    \end{Verbatim}

    \begin{Verbatim}[commandchars=\\\{\}]
{\color{incolor}In [{\color{incolor}17}]:} \PY{n}{B} \PY{o}{=} \PY{n}{np}\PY{o}{.}\PY{n}{matrix}\PY{p}{(}\PY{p}{[}
             \PY{p}{[}\PY{l+m+mi}{1}\PY{p}{,} \PY{l+m+mi}{2}\PY{p}{]}\PY{p}{,}
             \PY{p}{[}\PY{l+m+mi}{3}\PY{p}{,} \PY{l+m+mi}{4}\PY{p}{]}\PY{p}{,}
             \PY{p}{[}\PY{l+m+mi}{5}\PY{p}{,} \PY{l+m+mi}{6}\PY{p}{]}\PY{p}{,}
             \PY{p}{[}\PY{l+m+mi}{7}\PY{p}{,} \PY{l+m+mi}{8}\PY{p}{]}
         \PY{p}{]}\PY{p}{)}
         \PY{n}{np}\PY{o}{.}\PY{n}{matmul}\PY{p}{(}\PY{n}{A}\PY{p}{,}\PY{n}{B}\PY{p}{)}
\end{Verbatim}


\begin{Verbatim}[commandchars=\\\{\}]
{\color{outcolor}Out[{\color{outcolor}17}]:} matrix([[ 38,  48],
                 [ 95, 116],
                 [ 93, 114]])
\end{Verbatim}
            
    Jika ingin menggunakan operator \texttt{*} kita dapat menggunakan
konstruktor \texttt{np.matrix}.

    \begin{Verbatim}[commandchars=\\\{\}]
{\color{incolor}In [{\color{incolor}18}]:} \PY{n}{AA} \PY{o}{=} \PY{n}{np}\PY{o}{.}\PY{n}{matrix}\PY{p}{(}\PY{n}{A}\PY{p}{)}
         \PY{n}{BB} \PY{o}{=} \PY{n}{np}\PY{o}{.}\PY{n}{matrix}\PY{p}{(}\PY{n}{B}\PY{p}{)}
\end{Verbatim}


    \begin{Verbatim}[commandchars=\\\{\}]
{\color{incolor}In [{\color{incolor}19}]:} \PY{n}{AA}
\end{Verbatim}


\begin{Verbatim}[commandchars=\\\{\}]
{\color{outcolor}Out[{\color{outcolor}19}]:} matrix([[ 4,  1,  2,  3],
                 [ 3,  8,  1,  9],
                 [ 3,  4, 10,  4]])
\end{Verbatim}
            
    \begin{Verbatim}[commandchars=\\\{\}]
{\color{incolor}In [{\color{incolor}20}]:} \PY{n}{BB}
\end{Verbatim}


\begin{Verbatim}[commandchars=\\\{\}]
{\color{outcolor}Out[{\color{outcolor}20}]:} matrix([[1, 2],
                 [3, 4],
                 [5, 6],
                 [7, 8]])
\end{Verbatim}
            
    \begin{Verbatim}[commandchars=\\\{\}]
{\color{incolor}In [{\color{incolor}21}]:} \PY{n+nb}{type}\PY{p}{(}\PY{n}{AA}\PY{p}{)}
\end{Verbatim}


\begin{Verbatim}[commandchars=\\\{\}]
{\color{outcolor}Out[{\color{outcolor}21}]:} numpy.matrixlib.defmatrix.matrix
\end{Verbatim}
            
    \begin{Verbatim}[commandchars=\\\{\}]
{\color{incolor}In [{\color{incolor}22}]:} \PY{n}{AA}\PY{o}{.}\PY{n}{shape}\PY{p}{,} \PY{n}{BB}\PY{o}{.}\PY{n}{shape}
\end{Verbatim}


\begin{Verbatim}[commandchars=\\\{\}]
{\color{outcolor}Out[{\color{outcolor}22}]:} ((3, 4), (4, 2))
\end{Verbatim}
            
    \begin{Verbatim}[commandchars=\\\{\}]
{\color{incolor}In [{\color{incolor}23}]:} \PY{n}{AA}\PY{o}{*}\PY{n}{BB}
\end{Verbatim}


\begin{Verbatim}[commandchars=\\\{\}]
{\color{outcolor}Out[{\color{outcolor}23}]:} matrix([[ 38,  48],
                 [ 95, 116],
                 [ 93, 114]])
\end{Verbatim}
            
    \hypertarget{perkalian-dot-skalar}{%
\subsubsection{Perkalian dot (skalar)}\label{perkalian-dot-skalar}}

    Untuk operasi skalar antara dua vektor kita dapat menggunakan fungsi
\texttt{np.dot}

    \begin{Verbatim}[commandchars=\\\{\}]
{\color{incolor}In [{\color{incolor}24}]:} \PY{n}{y} \PY{o}{=} \PY{n}{np}\PY{o}{.}\PY{n}{array}\PY{p}{(}\PY{p}{[}\PY{l+m+mi}{2}\PY{p}{,}\PY{l+m+mi}{1}\PY{p}{,}\PY{l+m+mi}{3}\PY{p}{,}\PY{l+m+mi}{6}\PY{p}{]}\PY{p}{)}
         \PY{n}{y}
\end{Verbatim}


\begin{Verbatim}[commandchars=\\\{\}]
{\color{outcolor}Out[{\color{outcolor}24}]:} array([2, 1, 3, 6])
\end{Verbatim}
            
    \begin{Verbatim}[commandchars=\\\{\}]
{\color{incolor}In [{\color{incolor}25}]:} \PY{n}{np}\PY{o}{.}\PY{n}{dot}\PY{p}{(}\PY{n}{y}\PY{p}{,}\PY{n}{y}\PY{p}{)}
\end{Verbatim}


\begin{Verbatim}[commandchars=\\\{\}]
{\color{outcolor}Out[{\color{outcolor}25}]:} 50
\end{Verbatim}
            
    \begin{Verbatim}[commandchars=\\\{\}]
{\color{incolor}In [{\color{incolor}26}]:} \PY{n}{np}\PY{o}{.}\PY{n}{dot}\PY{p}{(}\PY{n}{x}\PY{p}{,}\PY{n}{y}\PY{p}{)}
\end{Verbatim}


\begin{Verbatim}[commandchars=\\\{\}]
{\color{outcolor}Out[{\color{outcolor}26}]:} 67
\end{Verbatim}
            
    \begin{Verbatim}[commandchars=\\\{\}]
{\color{incolor}In [{\color{incolor}27}]:} \PY{n}{np}\PY{o}{.}\PY{n}{dot}\PY{p}{(}\PY{n}{x}\PY{p}{,}\PY{n}{x}\PY{p}{)}
\end{Verbatim}


\begin{Verbatim}[commandchars=\\\{\}]
{\color{outcolor}Out[{\color{outcolor}27}]:} 95
\end{Verbatim}
            
    Metode \texttt{dot} juga dapat digunakan untuk operasi dot product:

    \begin{Verbatim}[commandchars=\\\{\}]
{\color{incolor}In [{\color{incolor}28}]:} \PY{n}{x}\PY{o}{.}\PY{n}{dot}\PY{p}{(}\PY{n}{x}\PY{p}{)}
\end{Verbatim}


\begin{Verbatim}[commandchars=\\\{\}]
{\color{outcolor}Out[{\color{outcolor}28}]:} 95
\end{Verbatim}
            
    Operasi \texttt{dot} juga dapat digunakan untuk melakukan perkalian
antara matriks dengan vektor:

    \begin{Verbatim}[commandchars=\\\{\}]
{\color{incolor}In [{\color{incolor}29}]:} \PY{n}{A}\PY{o}{.}\PY{n}{dot}\PY{p}{(}\PY{n}{x}\PY{p}{)}
\end{Verbatim}


\begin{Verbatim}[commandchars=\\\{\}]
{\color{outcolor}Out[{\color{outcolor}29}]:} array([ 46,  86, 101])
\end{Verbatim}
            
    \begin{Verbatim}[commandchars=\\\{\}]
{\color{incolor}In [{\color{incolor}30}]:} \PY{n}{np}\PY{o}{.}\PY{n}{matmul}\PY{p}{(}\PY{n}{A}\PY{p}{,}\PY{n}{x}\PY{p}{)}
\end{Verbatim}


\begin{Verbatim}[commandchars=\\\{\}]
{\color{outcolor}Out[{\color{outcolor}30}]:} array([ 46,  86, 101])
\end{Verbatim}
            
    Untuk tipe \texttt{matrix}:

    \begin{Verbatim}[commandchars=\\\{\}]
{\color{incolor}In [{\color{incolor}31}]:} \PY{n}{xx} \PY{o}{=} \PY{n}{np}\PY{o}{.}\PY{n}{matrix}\PY{p}{(}\PY{n}{x}\PY{p}{)}\PY{o}{.}\PY{n}{transpose}\PY{p}{(}\PY{p}{)}
         \PY{n}{xx}
\end{Verbatim}


\begin{Verbatim}[commandchars=\\\{\}]
{\color{outcolor}Out[{\color{outcolor}31}]:} matrix([[3],
                 [1],
                 [6],
                 [7]])
\end{Verbatim}
            
    \begin{Verbatim}[commandchars=\\\{\}]
{\color{incolor}In [{\color{incolor}32}]:} \PY{n}{xx}\PY{o}{.}\PY{n}{transpose}\PY{p}{(}\PY{p}{)}\PY{o}{.}\PY{n}{dot}\PY{p}{(}\PY{n}{xx}\PY{p}{)}\PY{p}{[}\PY{l+m+mi}{0}\PY{p}{,}\PY{l+m+mi}{0}\PY{p}{]}
\end{Verbatim}


\begin{Verbatim}[commandchars=\\\{\}]
{\color{outcolor}Out[{\color{outcolor}32}]:} 95
\end{Verbatim}
            
    \begin{Verbatim}[commandchars=\\\{\}]
{\color{incolor}In [{\color{incolor}33}]:} \PY{n}{AA}\PY{o}{.}\PY{n}{dot}\PY{p}{(}\PY{n}{xx}\PY{p}{)}
\end{Verbatim}


\begin{Verbatim}[commandchars=\\\{\}]
{\color{outcolor}Out[{\color{outcolor}33}]:} matrix([[ 46],
                 [ 86],
                 [101]])
\end{Verbatim}
            
    \begin{Verbatim}[commandchars=\\\{\}]
{\color{incolor}In [{\color{incolor}34}]:} \PY{n}{AA}\PY{o}{*}\PY{n}{xx}
\end{Verbatim}


\begin{Verbatim}[commandchars=\\\{\}]
{\color{outcolor}Out[{\color{outcolor}34}]:} matrix([[ 46],
                 [ 86],
                 [101]])
\end{Verbatim}
            
    \begin{Verbatim}[commandchars=\\\{\}]
{\color{incolor}In [{\color{incolor}35}]:} \PY{n}{AA}\PY{o}{.}\PY{n}{dot}\PY{p}{(}\PY{n}{BB}\PY{p}{)}
\end{Verbatim}


\begin{Verbatim}[commandchars=\\\{\}]
{\color{outcolor}Out[{\color{outcolor}35}]:} matrix([[ 38,  48],
                 [ 95, 116],
                 [ 93, 114]])
\end{Verbatim}
            
    \begin{Verbatim}[commandchars=\\\{\}]
{\color{incolor}In [{\color{incolor}36}]:} \PY{n}{AA}\PY{o}{*}\PY{n}{BB}
\end{Verbatim}


\begin{Verbatim}[commandchars=\\\{\}]
{\color{outcolor}Out[{\color{outcolor}36}]:} matrix([[ 38,  48],
                 [ 95, 116],
                 [ 93, 114]])
\end{Verbatim}
            
    Pada catatan ini, penulis akan menggunakan \texttt{matrix}. Jika Anda
lebih suka menggunakan \texttt{ndarray} langsung, Anda dapat
menggunakannya dengan dengan sedikit modifikasi.

    Operasi vektorisasi sebisa mungkin akan dihindari.

    \hypertarget{metode-eliminasi-gauss}{%
\subsection{Metode Eliminasi Gauss}\label{metode-eliminasi-gauss}}

    Dalam metode eliminasi Gauss, matriks \(\mathbf{A}\) dan vektor kolom
\(\mathbf{b}\) akan ditransformasi sedemikian rupa sehingga diperoleh
matriks dalam bentuk segitiga atas atau segitiga bawah.

Transformasi yang dilakukan adalah sebagai berikut.
\begin{align*}
A_{ij} \leftarrow A_{ij} - \alpha A_{kj} \\
b_{i} \leftarrow b_{i} - \alpha b_{k}
\end{align*}

Setelah matriks \(\mathbf{A}\) direduksi menjadi bentuk segitiga
atas, solusi persamaan linear yang dihasilkan dapat dicari dengan
menggunakan substitusi mundur (\emph{backward substitution}). \[
x_{k} = \left(
b_{k} - \sum_{j=k+1}^{N} A_{kj} x_{j}
\right)\frac{1}{A_{kk}}\,\,\,, k = N-1, N-2, \ldots, 1
\]

    \hypertarget{contoh-penggunaan-metode-eliminasi-gauss}{%
\subsubsection{Contoh penggunaan metode eliminasi
Gauss}\label{contoh-penggunaan-metode-eliminasi-gauss}}

    Perhatikan sistem persamaan linear berikut ini:

\begin{align*}
x_{1} + x_{2} + x_{3} & = 4 \\
2x_{1} + 3x_{2} + x_{3} & = 9 \\
x_{1} - x_{2} - x_{3} & = -2
\end{align*}

Persamaan di atas dapat diubah dalam bentuk matriks sebagai

\[
\begin{bmatrix}
1 & 1 & 1 \\
2 & 3 & 1 \\
1 & -1 & -1
\end{bmatrix}
\begin{bmatrix}
x_{1} \\
x_{2} \\
x_{3}
\end{bmatrix} =
\begin{bmatrix}
4 \\
9 \\
-2
\end{bmatrix}
\]

    \begin{Verbatim}[commandchars=\\\{\}]
{\color{incolor}In [{\color{incolor}37}]:} \PY{n}{A} \PY{o}{=} \PY{n}{np}\PY{o}{.}\PY{n}{matrix}\PY{p}{(}\PY{p}{[}
             \PY{p}{[}\PY{l+m+mi}{1}\PY{p}{,} \PY{l+m+mi}{1}\PY{p}{,} \PY{l+m+mi}{1}\PY{p}{]}\PY{p}{,}
             \PY{p}{[}\PY{l+m+mi}{2}\PY{p}{,} \PY{l+m+mi}{3}\PY{p}{,} \PY{l+m+mi}{1}\PY{p}{]}\PY{p}{,}
             \PY{p}{[}\PY{l+m+mi}{1}\PY{p}{,} \PY{o}{\PYZhy{}}\PY{l+m+mi}{1}\PY{p}{,} \PY{o}{\PYZhy{}}\PY{l+m+mi}{1}\PY{p}{]}
         \PY{p}{]}\PY{p}{)}
         \PY{n}{A}
\end{Verbatim}


\begin{Verbatim}[commandchars=\\\{\}]
{\color{outcolor}Out[{\color{outcolor}37}]:} matrix([[ 1,  1,  1],
                 [ 2,  3,  1],
                 [ 1, -1, -1]])
\end{Verbatim}
            
    \begin{Verbatim}[commandchars=\\\{\}]
{\color{incolor}In [{\color{incolor}38}]:} \PY{n}{b} \PY{o}{=} \PY{n}{np}\PY{o}{.}\PY{n}{matrix}\PY{p}{(}\PY{p}{[}\PY{l+m+mi}{4}\PY{p}{,} \PY{l+m+mi}{9}\PY{p}{,} \PY{o}{\PYZhy{}}\PY{l+m+mi}{2}\PY{p}{]}\PY{p}{)}\PY{o}{.}\PY{n}{transpose}\PY{p}{(}\PY{p}{)}
         \PY{n}{b}
\end{Verbatim}


\begin{Verbatim}[commandchars=\\\{\}]
{\color{outcolor}Out[{\color{outcolor}38}]:} matrix([[ 4],
                 [ 9],
                 [-2]])
\end{Verbatim}
            
    Karena kita akan memodifikasi matrix \texttt{A} dan \texttt{b}, maka
kita harus membuat backup (copy) dari nilai asli mereka.

    \begin{Verbatim}[commandchars=\\\{\}]
{\color{incolor}In [{\color{incolor}39}]:} \PY{n}{A\PYZus{}orig} \PY{o}{=} \PY{n}{np}\PY{o}{.}\PY{n}{matrix}\PY{o}{.}\PY{n}{copy}\PY{p}{(}\PY{n}{A}\PY{p}{)}
         \PY{n}{b\PYZus{}orig} \PY{o}{=} \PY{n}{np}\PY{o}{.}\PY{n}{matrix}\PY{o}{.}\PY{n}{copy}\PY{p}{(}\PY{n}{b}\PY{p}{)}
\end{Verbatim}


    Dimulai dengan menggunakan elemen matriks \(A_{11}\) (atau
\texttt{A{[}0,0{]}} dalam notasi Numpy) sebagai pivot, kita akan
melakukan reduksi pada baris kedua dan ketiga.

Kita akan mulai dengan reduksi baris kedua.

    \begin{Verbatim}[commandchars=\\\{\}]
{\color{incolor}In [{\color{incolor}40}]:} \PY{n}{alpha} \PY{o}{=} \PY{n}{A}\PY{p}{[}\PY{l+m+mi}{1}\PY{p}{,}\PY{l+m+mi}{0}\PY{p}{]}\PY{o}{/}\PY{n}{A}\PY{p}{[}\PY{l+m+mi}{0}\PY{p}{,}\PY{l+m+mi}{0}\PY{p}{]}
         \PY{n}{A}\PY{p}{[}\PY{l+m+mi}{1}\PY{p}{,}\PY{p}{:}\PY{p}{]} \PY{o}{=} \PY{n}{A}\PY{p}{[}\PY{l+m+mi}{1}\PY{p}{,}\PY{p}{:}\PY{p}{]} \PY{o}{\PYZhy{}} \PY{n}{alpha}\PY{o}{*}\PY{n}{A}\PY{p}{[}\PY{l+m+mi}{0}\PY{p}{,}\PY{p}{:}\PY{p}{]}
         \PY{n}{b}\PY{p}{[}\PY{l+m+mi}{1}\PY{p}{]} \PY{o}{=} \PY{n}{b}\PY{p}{[}\PY{l+m+mi}{1}\PY{p}{]} \PY{o}{\PYZhy{}} \PY{n}{alpha}\PY{o}{*}\PY{n}{b}\PY{p}{[}\PY{l+m+mi}{0}\PY{p}{]}
         
         \PY{n+nb}{print}\PY{p}{(}\PY{l+s+s2}{\PYZdq{}}\PY{l+s+s2}{A = }\PY{l+s+se}{\PYZbs{}n}\PY{l+s+s2}{\PYZdq{}}\PY{p}{,} \PY{n}{A}\PY{p}{)}
         \PY{n+nb}{print}\PY{p}{(}\PY{l+s+s2}{\PYZdq{}}\PY{l+s+s2}{b = }\PY{l+s+se}{\PYZbs{}n}\PY{l+s+s2}{\PYZdq{}}\PY{p}{,} \PY{n}{b}\PY{p}{)}
\end{Verbatim}


    \begin{Verbatim}[commandchars=\\\{\}]
A = 
 [[ 1  1  1]
 [ 0  1 -1]
 [ 1 -1 -1]]
b = 
 [[ 4]
 [ 1]
 [-2]]

    \end{Verbatim}

    Masih menggunakan \texttt{A{[}0,0{]}} sebagai pivot, kita akan reduksi
baris ketiga:

    \begin{Verbatim}[commandchars=\\\{\}]
{\color{incolor}In [{\color{incolor}41}]:} \PY{n}{alpha} \PY{o}{=} \PY{n}{A}\PY{p}{[}\PY{l+m+mi}{2}\PY{p}{,}\PY{l+m+mi}{0}\PY{p}{]}\PY{o}{/}\PY{n}{A}\PY{p}{[}\PY{l+m+mi}{0}\PY{p}{,}\PY{l+m+mi}{0}\PY{p}{]}
         \PY{n}{A}\PY{p}{[}\PY{l+m+mi}{2}\PY{p}{,}\PY{p}{:}\PY{p}{]} \PY{o}{=} \PY{n}{A}\PY{p}{[}\PY{l+m+mi}{2}\PY{p}{,}\PY{p}{:}\PY{p}{]} \PY{o}{\PYZhy{}} \PY{n}{alpha}\PY{o}{*}\PY{n}{A}\PY{p}{[}\PY{l+m+mi}{0}\PY{p}{,}\PY{p}{:}\PY{p}{]}
         \PY{n}{b}\PY{p}{[}\PY{l+m+mi}{2}\PY{p}{]} \PY{o}{=} \PY{n}{b}\PY{p}{[}\PY{l+m+mi}{2}\PY{p}{]} \PY{o}{\PYZhy{}} \PY{n}{alpha}\PY{o}{*}\PY{n}{b}\PY{p}{[}\PY{l+m+mi}{0}\PY{p}{]}
         
         \PY{n+nb}{print}\PY{p}{(}\PY{l+s+s2}{\PYZdq{}}\PY{l+s+s2}{A = }\PY{l+s+se}{\PYZbs{}n}\PY{l+s+s2}{\PYZdq{}}\PY{p}{,} \PY{n}{A}\PY{p}{)}
         \PY{n+nb}{print}\PY{p}{(}\PY{l+s+s2}{\PYZdq{}}\PY{l+s+s2}{b = }\PY{l+s+se}{\PYZbs{}n}\PY{l+s+s2}{\PYZdq{}}\PY{p}{,} \PY{n}{b}\PY{p}{)}
\end{Verbatim}


    \begin{Verbatim}[commandchars=\\\{\}]
A = 
 [[ 1  1  1]
 [ 0  1 -1]
 [ 0 -2 -2]]
b = 
 [[ 4]
 [ 1]
 [-6]]

    \end{Verbatim}

    Setelah menjadikan \texttt{A{[}0,0{]}} sebagai pivot dan mereduksi baris
kedua dan ketiga, kita akan menggunakan \texttt{A{[}1,1{]}} sebagai
pivot dan mereduksi baris ketiga:

    \begin{Verbatim}[commandchars=\\\{\}]
{\color{incolor}In [{\color{incolor}42}]:} \PY{n}{alpha} \PY{o}{=} \PY{n}{A}\PY{p}{[}\PY{l+m+mi}{2}\PY{p}{,}\PY{l+m+mi}{1}\PY{p}{]}\PY{o}{/}\PY{n}{A}\PY{p}{[}\PY{l+m+mi}{1}\PY{p}{,}\PY{l+m+mi}{1}\PY{p}{]}
         \PY{n}{A}\PY{p}{[}\PY{l+m+mi}{2}\PY{p}{,}\PY{p}{:}\PY{p}{]} \PY{o}{=} \PY{n}{A}\PY{p}{[}\PY{l+m+mi}{2}\PY{p}{,}\PY{p}{:}\PY{p}{]} \PY{o}{\PYZhy{}} \PY{n}{alpha}\PY{o}{*}\PY{n}{A}\PY{p}{[}\PY{l+m+mi}{1}\PY{p}{,}\PY{p}{:}\PY{p}{]}
         \PY{n}{b}\PY{p}{[}\PY{l+m+mi}{2}\PY{p}{]} \PY{o}{=} \PY{n}{b}\PY{p}{[}\PY{l+m+mi}{2}\PY{p}{]} \PY{o}{\PYZhy{}} \PY{n}{alpha}\PY{o}{*}\PY{n}{b}\PY{p}{[}\PY{l+m+mi}{1}\PY{p}{]}
         
         \PY{n+nb}{print}\PY{p}{(}\PY{l+s+s2}{\PYZdq{}}\PY{l+s+s2}{A = }\PY{l+s+se}{\PYZbs{}n}\PY{l+s+s2}{\PYZdq{}}\PY{p}{,} \PY{n}{A}\PY{p}{)}
         \PY{n+nb}{print}\PY{p}{(}\PY{l+s+s2}{\PYZdq{}}\PY{l+s+s2}{b = }\PY{l+s+se}{\PYZbs{}n}\PY{l+s+s2}{\PYZdq{}}\PY{p}{,} \PY{n}{b}\PY{p}{)}
\end{Verbatim}


    \begin{Verbatim}[commandchars=\\\{\}]
A = 
 [[ 1  1  1]
 [ 0  1 -1]
 [ 0  0 -4]]
b = 
 [[ 4]
 [ 1]
 [-4]]

    \end{Verbatim}

    Sekarang \texttt{A} telah tereduksi menjadi bentuk matriks segitiga
atas. Persamaan yang kita miliki sekarang adalah:

\[
\begin{bmatrix}
1 & 1 & 1 \\
0 & 1 & -3 \\
0 & 0 & -8
\end{bmatrix}
\begin{bmatrix}
x_{1} \\
x_{2} \\
x_{3}
\end{bmatrix} =
\begin{bmatrix}
4 \\
1 \\
-4
\end{bmatrix}
\]

    Sistem persamaan linear ini dapat dengan mudah diselesaikan dengan
menggunakan substitusi balik: mulai dari mencari \(x_3\), kemudian
\(x_2\), dan akhirnya \(x_1\).

    \begin{Verbatim}[commandchars=\\\{\}]
{\color{incolor}In [{\color{incolor}43}]:} \PY{n}{N} \PY{o}{=} \PY{n+nb}{len}\PY{p}{(}\PY{n}{b}\PY{p}{)}
         \PY{n}{x} \PY{o}{=} \PY{n}{np}\PY{o}{.}\PY{n}{matrix}\PY{p}{(}\PY{n}{np}\PY{o}{.}\PY{n}{zeros}\PY{p}{(}\PY{p}{(}\PY{n}{N}\PY{p}{,}\PY{l+m+mi}{1}\PY{p}{)}\PY{p}{)}\PY{p}{)}
         
         \PY{n}{x}\PY{p}{[}\PY{n}{N}\PY{o}{\PYZhy{}}\PY{l+m+mi}{1}\PY{p}{]} \PY{o}{=} \PY{n}{b}\PY{p}{[}\PY{n}{N}\PY{o}{\PYZhy{}}\PY{l+m+mi}{1}\PY{p}{]}\PY{o}{/}\PY{n}{A}\PY{p}{[}\PY{n}{N}\PY{o}{\PYZhy{}}\PY{l+m+mi}{1}\PY{p}{,}\PY{n}{N}\PY{o}{\PYZhy{}}\PY{l+m+mi}{1}\PY{p}{]}
         \PY{k}{for} \PY{n}{k} \PY{o+ow}{in} \PY{n+nb}{range}\PY{p}{(}\PY{n}{N}\PY{o}{\PYZhy{}}\PY{l+m+mi}{2}\PY{p}{,}\PY{o}{\PYZhy{}}\PY{l+m+mi}{1}\PY{p}{,}\PY{o}{\PYZhy{}}\PY{l+m+mi}{1}\PY{p}{)}\PY{p}{:}
             \PY{n}{ss} \PY{o}{=} \PY{l+m+mf}{0.0}
             \PY{k}{for} \PY{n}{j} \PY{o+ow}{in} \PY{n+nb}{range}\PY{p}{(}\PY{n}{k}\PY{o}{+}\PY{l+m+mi}{1}\PY{p}{,}\PY{n}{N}\PY{p}{)}\PY{p}{:}
                 \PY{n}{ss} \PY{o}{=} \PY{n}{ss} \PY{o}{+} \PY{n}{A}\PY{p}{[}\PY{n}{k}\PY{p}{,}\PY{n}{j}\PY{p}{]}\PY{o}{*}\PY{n}{x}\PY{p}{[}\PY{n}{j}\PY{p}{]}
             \PY{n}{x}\PY{p}{[}\PY{n}{k}\PY{p}{]} \PY{o}{=} \PY{p}{(}\PY{n}{b}\PY{p}{[}\PY{n}{k}\PY{p}{]} \PY{o}{\PYZhy{}} \PY{n}{ss}\PY{p}{)}\PY{o}{/}\PY{n}{A}\PY{p}{[}\PY{n}{k}\PY{p}{,}\PY{n}{k}\PY{p}{]}
         
         \PY{k}{for} \PY{n}{i} \PY{o+ow}{in} \PY{n+nb}{range}\PY{p}{(}\PY{n}{N}\PY{p}{)}\PY{p}{:}
             \PY{n+nb}{print}\PY{p}{(}\PY{n}{x}\PY{p}{[}\PY{n}{i}\PY{p}{]}\PY{p}{)}
\end{Verbatim}


    \begin{Verbatim}[commandchars=\\\{\}]
[[1.]]
[[2.]]
[[1.]]

    \end{Verbatim}

    Sekarang kita dapat mengecek apakah solusi yang kita dapatkan sudah
benar.

    \begin{Verbatim}[commandchars=\\\{\}]
{\color{incolor}In [{\color{incolor}44}]:} \PY{c+c1}{\PYZsh{} Gunakan matrix dan vektor original}
         \PY{c+c1}{\PYZsh{} Hasil seharunya berupa vektor kolom dengan elemen\PYZhy{}element 0}
         \PY{n}{A\PYZus{}orig}\PY{o}{*}\PY{n}{x} \PY{o}{\PYZhy{}} \PY{n}{b\PYZus{}orig}
\end{Verbatim}


\begin{Verbatim}[commandchars=\\\{\}]
{\color{outcolor}Out[{\color{outcolor}44}]:} matrix([[0.],
                 [0.],
                 [0.]])
\end{Verbatim}
            
    \hypertarget{kode-python-untuk-eliminasi-gauss}{%
\subsubsection{Kode Python untuk eliminasi
Gauss}\label{kode-python-untuk-eliminasi-gauss}}

    Berikut ini adalah implementasi metode eliminasi Gauss pada Python.

Kode ini menerima masukan matriks \texttt{A\_} dan vektor kolom
\texttt{b\_}. dan mengembalikan solusi \texttt{x} dari sistem persamaan
linear \texttt{A\_*x\ =\ b\_}. Tanda \texttt{\_} digunakan untuk
menunjukkan input asli. Pada kode berikut kita bekerja dengan matriks
\texttt{A} dan vektor \texttt{x} yang merupakan kopi dari input-input
awal. Kode ini terbatas pada vektor input \texttt{b\_} yang hanya
terdiri dari satu kolom. Kode dapat dikembangkan untuk kasus kolom lebih
dari satu.

    \begin{Verbatim}[commandchars=\\\{\}]
{\color{incolor}In [{\color{incolor}45}]:} \PY{k}{def} \PY{n+nf}{gauss\PYZus{}elim}\PY{p}{(}\PY{n}{A\PYZus{}}\PY{p}{,} \PY{n}{b\PYZus{}}\PY{p}{)}\PY{p}{:}
             
             \PY{n}{N}\PY{p}{,} \PY{n}{Nrhs} \PY{o}{=} \PY{n}{b\PYZus{}}\PY{o}{.}\PY{n}{shape}
             
             \PY{k}{assert} \PY{n}{Nrhs} \PY{o}{==} \PY{l+m+mi}{1}
         
             \PY{n}{A} \PY{o}{=} \PY{n}{np}\PY{o}{.}\PY{n}{matrix}\PY{o}{.}\PY{n}{copy}\PY{p}{(}\PY{n}{A\PYZus{}}\PY{p}{)}
             \PY{n}{b} \PY{o}{=} \PY{n}{np}\PY{o}{.}\PY{n}{matrix}\PY{o}{.}\PY{n}{copy}\PY{p}{(}\PY{n}{b\PYZus{}}\PY{p}{)}
         
             \PY{c+c1}{\PYZsh{} Eliminasi maju}
             \PY{k}{for} \PY{n}{k} \PY{o+ow}{in} \PY{n+nb}{range}\PY{p}{(}\PY{l+m+mi}{0}\PY{p}{,}\PY{n}{N}\PY{o}{\PYZhy{}}\PY{l+m+mi}{1}\PY{p}{)}\PY{p}{:}
                 \PY{k}{for} \PY{n}{i} \PY{o+ow}{in} \PY{n+nb}{range}\PY{p}{(}\PY{n}{k}\PY{o}{+}\PY{l+m+mi}{1}\PY{p}{,}\PY{n}{N}\PY{p}{)}\PY{p}{:}
                     \PY{k}{if} \PY{n}{A}\PY{p}{[}\PY{n}{i}\PY{p}{,}\PY{n}{k}\PY{p}{]} \PY{o}{!=} \PY{l+m+mf}{0.0}\PY{p}{:}
                         \PY{n}{alpha} \PY{o}{=} \PY{n}{A}\PY{p}{[}\PY{n}{i}\PY{p}{,}\PY{n}{k}\PY{p}{]}\PY{o}{/}\PY{n}{A}\PY{p}{[}\PY{n}{k}\PY{p}{,}\PY{n}{k}\PY{p}{]}
                         \PY{n}{A}\PY{p}{[}\PY{n}{i}\PY{p}{,}\PY{p}{:}\PY{p}{]} \PY{o}{=} \PY{n}{A}\PY{p}{[}\PY{n}{i}\PY{p}{,}\PY{p}{:}\PY{p}{]} \PY{o}{\PYZhy{}} \PY{n}{alpha}\PY{o}{*}\PY{n}{A}\PY{p}{[}\PY{n}{k}\PY{p}{,}\PY{p}{:}\PY{p}{]}
                         \PY{n}{b}\PY{p}{[}\PY{n}{i}\PY{p}{]} \PY{o}{=} \PY{n}{b}\PY{p}{[}\PY{n}{i}\PY{p}{]} \PY{o}{\PYZhy{}} \PY{n}{alpha}\PY{o}{*}\PY{n}{b}\PY{p}{[}\PY{n}{k}\PY{p}{]}
             
             \PY{c+c1}{\PYZsh{} Alokasi mememori untuk solusi}
             \PY{n}{x} \PY{o}{=} \PY{n}{np}\PY{o}{.}\PY{n}{matrix}\PY{p}{(}\PY{n}{np}\PY{o}{.}\PY{n}{zeros}\PY{p}{(}\PY{p}{(}\PY{n}{N}\PY{p}{,}\PY{l+m+mi}{1}\PY{p}{)}\PY{p}{)}\PY{p}{)}
             
             \PY{c+c1}{\PYZsh{} Substitusi balik}
             \PY{n}{x}\PY{p}{[}\PY{n}{N}\PY{o}{\PYZhy{}}\PY{l+m+mi}{1}\PY{p}{]} \PY{o}{=} \PY{n}{b}\PY{p}{[}\PY{n}{N}\PY{o}{\PYZhy{}}\PY{l+m+mi}{1}\PY{p}{]}\PY{o}{/}\PY{n}{A}\PY{p}{[}\PY{n}{N}\PY{o}{\PYZhy{}}\PY{l+m+mi}{1}\PY{p}{,}\PY{n}{N}\PY{o}{\PYZhy{}}\PY{l+m+mi}{1}\PY{p}{]}
             \PY{k}{for} \PY{n}{k} \PY{o+ow}{in} \PY{n+nb}{range}\PY{p}{(}\PY{n}{N}\PY{o}{\PYZhy{}}\PY{l+m+mi}{2}\PY{p}{,}\PY{o}{\PYZhy{}}\PY{l+m+mi}{1}\PY{p}{,}\PY{o}{\PYZhy{}}\PY{l+m+mi}{1}\PY{p}{)}\PY{p}{:}
                 \PY{n}{ss} \PY{o}{=} \PY{l+m+mf}{0.0}
                 \PY{k}{for} \PY{n}{j} \PY{o+ow}{in} \PY{n+nb}{range}\PY{p}{(}\PY{n}{k}\PY{o}{+}\PY{l+m+mi}{1}\PY{p}{,}\PY{n}{N}\PY{p}{)}\PY{p}{:}
                     \PY{n}{ss} \PY{o}{=} \PY{n}{ss} \PY{o}{+} \PY{n}{A}\PY{p}{[}\PY{n}{k}\PY{p}{,}\PY{n}{j}\PY{p}{]}\PY{o}{*}\PY{n}{x}\PY{p}{[}\PY{n}{j}\PY{p}{]}
                 \PY{n}{x}\PY{p}{[}\PY{n}{k}\PY{p}{]} \PY{o}{=} \PY{p}{(}\PY{n}{b}\PY{p}{[}\PY{n}{k}\PY{p}{]} \PY{o}{\PYZhy{}} \PY{n}{ss}\PY{p}{)}\PY{o}{/}\PY{n}{A}\PY{p}{[}\PY{n}{k}\PY{p}{,}\PY{n}{k}\PY{p}{]}
             
             \PY{k}{return} \PY{n}{x}
\end{Verbatim}


    \begin{Verbatim}[commandchars=\\\{\}]
{\color{incolor}In [{\color{incolor}46}]:} \PY{n}{gauss\PYZus{}elim}\PY{p}{(}\PY{n}{A\PYZus{}orig}\PY{p}{,} \PY{n}{b\PYZus{}orig}\PY{p}{)}
\end{Verbatim}


\begin{Verbatim}[commandchars=\\\{\}]
{\color{outcolor}Out[{\color{outcolor}46}]:} matrix([[1.],
                 [2.],
                 [1.]])
\end{Verbatim}
            
    \hypertarget{dekomposisi-lu}{%
\subsection{Dekomposisi LU}\label{dekomposisi-lu}}

    Pada dekomposisi LU, matriks persegi \(\mathbf{A}\) dapat dinyatakan
sebagai hasil kali dari matriks segitiga bawah \(\mathbf{L}\) dan
matriks segitiga atas \(\mathbf{U}\): \[
\mathbf{A} = \mathbf{LU}
\] Proses untuk mendapatkan matriks \(\mathbf{L}\) dan \(\mathbf{U}\)
dari matriks \(\mathbf{A}\) disebut dengan dekomposisi LU atau
faktorisasi LU. Dekomposisi LU tidak unik, artinya bisa terdapat banyak
kombinasi \(\mathbf{L}\) dan \(\mathbf{U}\) yang mungkin untuk suatu
matriks \(\mathbf{A}\) yang diberikan. Untuk mendapatkan pasangan
\(\mathbf{L}\) dan \(\mathbf{U}\) yang unik, kita perlu memberikan
batasan atau konstrain terhadap proses dekomposisi LU. Terdapat beberapa
varian dekomposisi LU:

\begin{longtable}[]{@{}ll@{}}
\toprule
\begin{minipage}[b]{0.29\columnwidth}\raggedright
Nama\strut
\end{minipage} & \begin{minipage}[b]{0.65\columnwidth}\raggedright
Konstrain\strut
\end{minipage}\tabularnewline
\midrule
\endhead
\begin{minipage}[t]{0.29\columnwidth}\raggedright
Dekomposisi Doolittle\strut
\end{minipage} & \begin{minipage}[t]{0.65\columnwidth}\raggedright
\(L_{ii} = 1\)\strut
\end{minipage}\tabularnewline
\begin{minipage}[t]{0.29\columnwidth}\raggedright
Dekomposisi Crout\strut
\end{minipage} & \begin{minipage}[t]{0.65\columnwidth}\raggedright
\(U_{ii} = 1\)\strut
\end{minipage}\tabularnewline
\begin{minipage}[t]{0.29\columnwidth}\raggedright
Dekomposisi Cholesky\strut
\end{minipage} & \begin{minipage}[t]{0.65\columnwidth}\raggedright
\(\mathbf{L} = \mathbf{U}^{T}\) (untuk matriks \(\mathbf{A}\) simetrik
dan definit positif\strut
\end{minipage}\tabularnewline
\bottomrule
\end{longtable}

    Dengan dekomposisi LU kita dapat menuliskan sistem persamaan linear \[
\mathbf{Ax} = \mathbf{b}
\] menjadi: \[
\mathbf{LUx} = \mathbf{b}
\] Dengan menggunakan \(\mathbf{y} = \mathbf{Ux}\) kita dapat
menuliskan: \[
\mathbf{Ly} = \mathbf{b}
\] Persamaan ini dapat diselesaikan dengan menggunakan substitusi maju.
Solusi \(\mathbf{x}\) dapat dicari dengan menggunakan substitusi balik
(mundur).

    Varian Doolittle untuk dekomposisi LU memiliki bentuk berikut untuk
matriks \(\mathbf{L}\) dan \(\mathbf{U}\), misalnya untuk ukuran
\(3\times3\)

\[
\mathbf{L} =
\begin{bmatrix}
1 & 0 & 0 \\
L_{21} & 1 & 0 \\
L_{31} & L_{32} & 1
\end{bmatrix},\,\,\,
\mathbf{U} =
\begin{bmatrix}
U_{11} & U_{12} & U_{13} \\
0 & U_{22} & U_{23} \\
0 & 0 & U_{33}
\end{bmatrix}
\]

Dengan melakukan perkalian \(\mathbf{A} = \mathbf{LU}\)

\[
\mathbf{A} = \begin{bmatrix}
U_{11} & U_{12} & U_{13} \\
U_{11}L_{21} & U_{12}L_{21} + U_{22} & U_{13}L_{21} + U_{23} \\
U_{11}L_{31} & U_{12}L_{31} + U_{22}L_{32} & U_{13}L_{31} + U_{23}L_{32} + U_{33}
\end{bmatrix}
\]

    Dapat ditunjukkan bahwa matriks \(\mathbf{U}\) adalah matriks segitiga
atas hasil dari eliminasi Gauss pada matriks \(\mathbf{A}\). Elemen
\emph{off-diagonal} dari matriks \(\mathbf{L}\) adalah pengali
\(\alpha\), atau \(L_{ij}\) adalah pengali yang mengelimisasi
\(A_{ij}\).

    \hypertarget{kode-python-untuk-dekomposisi-lu-varian-doolittle}{%
\subsubsection{Kode Python untuk dekomposisi LU (varian
Doolittle)}\label{kode-python-untuk-dekomposisi-lu-varian-doolittle}}

    \begin{Verbatim}[commandchars=\\\{\}]
{\color{incolor}In [{\color{incolor}47}]:} \PY{k}{def} \PY{n+nf}{LU\PYZus{}decomp}\PY{p}{(}\PY{n}{A\PYZus{}}\PY{p}{)}\PY{p}{:}
             
             \PY{n}{Nrow}\PY{p}{,} \PY{n}{Ncol} \PY{o}{=} \PY{n}{A\PYZus{}}\PY{o}{.}\PY{n}{shape}
             
             \PY{k}{assert} \PY{n}{Nrow} \PY{o}{==} \PY{n}{Ncol}
         
             \PY{n}{A} \PY{o}{=} \PY{n}{np}\PY{o}{.}\PY{n}{matrix}\PY{o}{.}\PY{n}{copy}\PY{p}{(}\PY{n}{A\PYZus{}}\PY{p}{)}
         
             \PY{c+c1}{\PYZsh{} Eliminasi Gauss (maju)}
             \PY{k}{for} \PY{n}{k} \PY{o+ow}{in} \PY{n+nb}{range}\PY{p}{(}\PY{l+m+mi}{0}\PY{p}{,}\PY{n}{N}\PY{o}{\PYZhy{}}\PY{l+m+mi}{1}\PY{p}{)}\PY{p}{:}
                 \PY{k}{for} \PY{n}{i} \PY{o+ow}{in} \PY{n+nb}{range}\PY{p}{(}\PY{n}{k}\PY{o}{+}\PY{l+m+mi}{1}\PY{p}{,}\PY{n}{N}\PY{p}{)}\PY{p}{:}
                     \PY{k}{if} \PY{n}{A}\PY{p}{[}\PY{n}{i}\PY{p}{,}\PY{n}{k}\PY{p}{]} \PY{o}{!=} \PY{l+m+mf}{0.0}\PY{p}{:}
                         \PY{n}{alpha} \PY{o}{=} \PY{n}{A}\PY{p}{[}\PY{n}{i}\PY{p}{,}\PY{n}{k}\PY{p}{]}\PY{o}{/}\PY{n}{A}\PY{p}{[}\PY{n}{k}\PY{p}{,}\PY{n}{k}\PY{p}{]}
                         \PY{n}{A}\PY{p}{[}\PY{n}{i}\PY{p}{,}\PY{n}{k}\PY{o}{+}\PY{l+m+mi}{1}\PY{p}{:}\PY{n}{N}\PY{p}{]} \PY{o}{=} \PY{n}{A}\PY{p}{[}\PY{n}{i}\PY{p}{,}\PY{n}{k}\PY{o}{+}\PY{l+m+mi}{1}\PY{p}{:}\PY{n}{N}\PY{p}{]} \PY{o}{\PYZhy{}} \PY{n}{alpha}\PY{o}{*}\PY{n}{A}\PY{p}{[}\PY{n}{k}\PY{p}{,}\PY{n}{k}\PY{o}{+}\PY{l+m+mi}{1}\PY{p}{:}\PY{n}{N}\PY{p}{]}
                         \PY{n}{A}\PY{p}{[}\PY{n}{i}\PY{p}{,}\PY{n}{k}\PY{p}{]} \PY{o}{=} \PY{n}{alpha}
             
             \PY{n}{L} \PY{o}{=} \PY{n}{np}\PY{o}{.}\PY{n}{matrix}\PY{p}{(} \PY{n}{np}\PY{o}{.}\PY{n}{tril}\PY{p}{(}\PY{n}{A}\PY{p}{,}\PY{o}{\PYZhy{}}\PY{l+m+mi}{1}\PY{p}{)} \PY{p}{)}
             \PY{k}{for} \PY{n}{i} \PY{o+ow}{in} \PY{n+nb}{range}\PY{p}{(}\PY{n}{N}\PY{p}{)}\PY{p}{:}
                 \PY{n}{L}\PY{p}{[}\PY{n}{i}\PY{p}{,}\PY{n}{i}\PY{p}{]} \PY{o}{=} \PY{l+m+mf}{1.0} \PY{c+c1}{\PYZsh{} konstrain Doolittle}
             \PY{n}{U} \PY{o}{=} \PY{n}{np}\PY{o}{.}\PY{n}{matrix}\PY{p}{(} \PY{n}{np}\PY{o}{.}\PY{n}{triu}\PY{p}{(}\PY{n}{A}\PY{p}{)} \PY{p}{)}
             
             \PY{k}{return} \PY{n}{L}\PY{p}{,} \PY{n}{U} \PY{c+c1}{\PYZsh{} kembalikan matriks L dan U}
\end{Verbatim}


    Definisikan lagi matriks dan vektor yang ada pada contoh sebelumnya.

    \begin{Verbatim}[commandchars=\\\{\}]
{\color{incolor}In [{\color{incolor}48}]:} \PY{n}{A} \PY{o}{=} \PY{n}{np}\PY{o}{.}\PY{n}{matrix}\PY{p}{(}\PY{p}{[}
             \PY{p}{[}\PY{l+m+mi}{1}\PY{p}{,} \PY{l+m+mi}{1}\PY{p}{,} \PY{l+m+mi}{1}\PY{p}{]}\PY{p}{,}
             \PY{p}{[}\PY{l+m+mi}{2}\PY{p}{,} \PY{l+m+mi}{3}\PY{p}{,} \PY{l+m+mi}{1}\PY{p}{]}\PY{p}{,}
             \PY{p}{[}\PY{l+m+mi}{1}\PY{p}{,} \PY{o}{\PYZhy{}}\PY{l+m+mi}{1}\PY{p}{,} \PY{o}{\PYZhy{}}\PY{l+m+mi}{1}\PY{p}{]}
         \PY{p}{]}\PY{p}{)}
         \PY{n}{b} \PY{o}{=} \PY{n}{np}\PY{o}{.}\PY{n}{matrix}\PY{p}{(}\PY{p}{[}\PY{l+m+mi}{4}\PY{p}{,} \PY{l+m+mi}{9}\PY{p}{,} \PY{o}{\PYZhy{}}\PY{l+m+mi}{2}\PY{p}{]}\PY{p}{)}\PY{o}{.}\PY{n}{transpose}\PY{p}{(}\PY{p}{)}
\end{Verbatim}


    Lakukan dekomposisi LU:

    \begin{Verbatim}[commandchars=\\\{\}]
{\color{incolor}In [{\color{incolor}49}]:} \PY{n}{L}\PY{p}{,} \PY{n}{U} \PY{o}{=} \PY{n}{LU\PYZus{}decomp}\PY{p}{(}\PY{n}{A}\PY{p}{)}
         \PY{n+nb}{print}\PY{p}{(}\PY{l+s+s2}{\PYZdq{}}\PY{l+s+s2}{L = }\PY{l+s+se}{\PYZbs{}n}\PY{l+s+s2}{\PYZdq{}}\PY{p}{,} \PY{n}{L}\PY{p}{)}
         \PY{n+nb}{print}\PY{p}{(}\PY{l+s+s2}{\PYZdq{}}\PY{l+s+s2}{U = }\PY{l+s+se}{\PYZbs{}n}\PY{l+s+s2}{\PYZdq{}}\PY{p}{,} \PY{n}{U}\PY{p}{)}
\end{Verbatim}


    \begin{Verbatim}[commandchars=\\\{\}]
L = 
 [[ 1  0  0]
 [ 2  1  0]
 [ 1 -2  1]]
U = 
 [[ 1  1  1]
 [ 0  1 -1]
 [ 0  0 -4]]

    \end{Verbatim}

    Periksa bahwa \(\mathbf{A} = \mathbf{LU}\):

    \begin{Verbatim}[commandchars=\\\{\}]
{\color{incolor}In [{\color{incolor}50}]:} \PY{n}{L}\PY{o}{*}\PY{n}{U} \PY{o}{\PYZhy{}} \PY{n}{A}
\end{Verbatim}


\begin{Verbatim}[commandchars=\\\{\}]
{\color{outcolor}Out[{\color{outcolor}50}]:} matrix([[0, 0, 0],
                 [0, 0, 0],
                 [0, 0, 0]])
\end{Verbatim}
            
    \hypertarget{implementasi-solusi-dengan-matrix-l-dan-u-yang-sudah-dihitung}{%
\subsubsection{Implementasi solusi dengan matrix L dan U yang sudah
dihitung}\label{implementasi-solusi-dengan-matrix-l-dan-u-yang-sudah-dihitung}}

    Substitusi maju \[
y_{k} = b_{k} - \sum_{j=1}^{k-1} L_{kj} y_{j},\,\,\,k = 2,3,\ldots,N
\]

    \begin{Verbatim}[commandchars=\\\{\}]
{\color{incolor}In [{\color{incolor}51}]:} \PY{k}{def} \PY{n+nf}{LU\PYZus{}solve}\PY{p}{(}\PY{n}{L}\PY{p}{,} \PY{n}{U}\PY{p}{,} \PY{n}{b}\PY{p}{)}\PY{p}{:}
             
             \PY{n}{N} \PY{o}{=} \PY{n}{L}\PY{o}{.}\PY{n}{shape}\PY{p}{[}\PY{l+m+mi}{0}\PY{p}{]}
             
             \PY{n}{x} \PY{o}{=} \PY{n}{np}\PY{o}{.}\PY{n}{matrix}\PY{p}{(}\PY{n}{np}\PY{o}{.}\PY{n}{zeros}\PY{p}{(}\PY{p}{(}\PY{n}{N}\PY{p}{,}\PY{p}{)}\PY{p}{)}\PY{p}{)}\PY{o}{.}\PY{n}{transpose}\PY{p}{(}\PY{p}{)}
             \PY{n}{y} \PY{o}{=} \PY{n}{np}\PY{o}{.}\PY{n}{matrix}\PY{p}{(}\PY{n}{np}\PY{o}{.}\PY{n}{zeros}\PY{p}{(}\PY{p}{(}\PY{n}{N}\PY{p}{,}\PY{p}{)}\PY{p}{)}\PY{p}{)}\PY{o}{.}\PY{n}{transpose}\PY{p}{(}\PY{p}{)}
             
             \PY{c+c1}{\PYZsh{} Ly = b}
             \PY{c+c1}{\PYZsh{} Substitusi maju}
             \PY{n}{y}\PY{p}{[}\PY{l+m+mi}{0}\PY{p}{]} \PY{o}{=} \PY{n}{b}\PY{p}{[}\PY{l+m+mi}{0}\PY{p}{]}\PY{o}{/}\PY{n}{L}\PY{p}{[}\PY{l+m+mi}{0}\PY{p}{,}\PY{l+m+mi}{0}\PY{p}{]}
             \PY{k}{for} \PY{n}{k} \PY{o+ow}{in} \PY{n+nb}{range}\PY{p}{(}\PY{l+m+mi}{1}\PY{p}{,}\PY{n}{N}\PY{p}{)}\PY{p}{:}
                 \PY{n}{ss} \PY{o}{=} \PY{l+m+mf}{0.0}
                 \PY{k}{for} \PY{n}{j} \PY{o+ow}{in} \PY{n+nb}{range}\PY{p}{(}\PY{n}{k}\PY{p}{)}\PY{p}{:}
                     \PY{n}{ss} \PY{o}{=} \PY{n}{ss} \PY{o}{+} \PY{n}{L}\PY{p}{[}\PY{n}{k}\PY{p}{,}\PY{n}{j}\PY{p}{]}\PY{o}{*}\PY{n}{y}\PY{p}{[}\PY{n}{j}\PY{p}{]}
                 \PY{n}{y}\PY{p}{[}\PY{n}{k}\PY{p}{]} \PY{o}{=} \PY{p}{(}\PY{n}{b}\PY{p}{[}\PY{n}{k}\PY{p}{]} \PY{o}{\PYZhy{}} \PY{n}{ss}\PY{p}{)}\PY{o}{/}\PY{n}{L}\PY{p}{[}\PY{n}{k}\PY{p}{,}\PY{n}{k}\PY{p}{]}
             
             \PY{c+c1}{\PYZsh{} Ux = y}
             \PY{c+c1}{\PYZsh{} Substitusi balik}
             \PY{n}{x}\PY{p}{[}\PY{n}{N}\PY{o}{\PYZhy{}}\PY{l+m+mi}{1}\PY{p}{]} \PY{o}{=} \PY{n}{y}\PY{p}{[}\PY{n}{N}\PY{o}{\PYZhy{}}\PY{l+m+mi}{1}\PY{p}{]}\PY{o}{/}\PY{n}{U}\PY{p}{[}\PY{n}{N}\PY{o}{\PYZhy{}}\PY{l+m+mi}{1}\PY{p}{,}\PY{n}{N}\PY{o}{\PYZhy{}}\PY{l+m+mi}{1}\PY{p}{]}
             \PY{k}{for} \PY{n}{k} \PY{o+ow}{in} \PY{n+nb}{range}\PY{p}{(}\PY{n}{N}\PY{o}{\PYZhy{}}\PY{l+m+mi}{2}\PY{p}{,}\PY{o}{\PYZhy{}}\PY{l+m+mi}{1}\PY{p}{,}\PY{o}{\PYZhy{}}\PY{l+m+mi}{1}\PY{p}{)}\PY{p}{:}
                 \PY{n}{ss} \PY{o}{=} \PY{l+m+mf}{0.0}
                 \PY{k}{for} \PY{n}{j} \PY{o+ow}{in} \PY{n+nb}{range}\PY{p}{(}\PY{n}{k}\PY{o}{+}\PY{l+m+mi}{1}\PY{p}{,}\PY{n}{N}\PY{p}{)}\PY{p}{:}
                     \PY{n}{ss} \PY{o}{=} \PY{n}{ss} \PY{o}{+} \PY{n}{U}\PY{p}{[}\PY{n}{k}\PY{p}{,}\PY{n}{j}\PY{p}{]}\PY{o}{*}\PY{n}{x}\PY{p}{[}\PY{n}{j}\PY{p}{]}
                 \PY{n}{x}\PY{p}{[}\PY{n}{k}\PY{p}{]} \PY{o}{=} \PY{p}{(}\PY{n}{y}\PY{p}{[}\PY{n}{k}\PY{p}{]} \PY{o}{\PYZhy{}} \PY{n}{ss}\PY{p}{)}\PY{o}{/}\PY{n}{U}\PY{p}{[}\PY{n}{k}\PY{p}{,}\PY{n}{k}\PY{p}{]}
             
             \PY{k}{return} \PY{n}{x}
\end{Verbatim}


    \begin{Verbatim}[commandchars=\\\{\}]
{\color{incolor}In [{\color{incolor}52}]:} \PY{n}{x} \PY{o}{=} \PY{n}{LU\PYZus{}solve}\PY{p}{(}\PY{n}{L}\PY{p}{,} \PY{n}{U}\PY{p}{,} \PY{n}{b}\PY{p}{)}
\end{Verbatim}


    Cek hasil:

    \begin{Verbatim}[commandchars=\\\{\}]
{\color{incolor}In [{\color{incolor}53}]:} \PY{n}{A}\PY{o}{*}\PY{n}{x} \PY{o}{\PYZhy{}} \PY{n}{b}
\end{Verbatim}


\begin{Verbatim}[commandchars=\\\{\}]
{\color{outcolor}Out[{\color{outcolor}53}]:} matrix([[0.],
                 [0.],
                 [0.]])
\end{Verbatim}
            
    \hypertarget{latihan-1}{%
\subsection{Latihan 1}\label{latihan-1}}

    Selasaikan persamaan:

\[
\begin{bmatrix}
6 & -4 & 1 \\
4 & 6 & -4 \\
1 & -4 & 6
\end{bmatrix}
\begin{bmatrix}
x_1 \\
x_2 \\
x_3
\end{bmatrix} = 
\begin{bmatrix}
-14 \\
36 \\
6
\end{bmatrix}
\]

    \hypertarget{latihan-1-solusi}{%
\subsubsection{Latihan 1 (Solusi)}\label{latihan-1-solusi}}

    Menggunakan eliminasi Gauss:

    \begin{Verbatim}[commandchars=\\\{\}]
{\color{incolor}In [{\color{incolor}54}]:} \PY{n}{A} \PY{o}{=} \PY{n}{np}\PY{o}{.}\PY{n}{matrix}\PY{p}{(}\PY{p}{[}
             \PY{p}{[}\PY{l+m+mi}{6}\PY{p}{,} \PY{o}{\PYZhy{}}\PY{l+m+mi}{4}\PY{p}{,} \PY{l+m+mi}{1}\PY{p}{]}\PY{p}{,}
             \PY{p}{[}\PY{o}{\PYZhy{}}\PY{l+m+mi}{4}\PY{p}{,} \PY{l+m+mi}{6}\PY{p}{,} \PY{o}{\PYZhy{}}\PY{l+m+mi}{4}\PY{p}{]}\PY{p}{,}
             \PY{p}{[}\PY{l+m+mi}{1}\PY{p}{,} \PY{o}{\PYZhy{}}\PY{l+m+mi}{4}\PY{p}{,} \PY{l+m+mi}{6}\PY{p}{]}\PY{p}{]}\PY{p}{,} \PY{n}{dtype}\PY{o}{=}\PY{n}{np}\PY{o}{.}\PY{n}{float64}\PY{p}{)}
         \PY{n}{b} \PY{o}{=} \PY{n}{np}\PY{o}{.}\PY{n}{matrix}\PY{p}{(}\PY{p}{[}\PY{o}{\PYZhy{}}\PY{l+m+mi}{14}\PY{p}{,} \PY{l+m+mi}{36}\PY{p}{,} \PY{l+m+mi}{6}\PY{p}{]}\PY{p}{,} \PY{n}{dtype}\PY{o}{=}\PY{n}{np}\PY{o}{.}\PY{n}{float64}\PY{p}{)}\PY{o}{.}\PY{n}{transpose}\PY{p}{(}\PY{p}{)}
         \PY{n}{x} \PY{o}{=} \PY{n}{gauss\PYZus{}elim}\PY{p}{(}\PY{n}{A}\PY{p}{,} \PY{n}{b}\PY{p}{)}
         \PY{n+nb}{print}\PY{p}{(}\PY{l+s+s2}{\PYZdq{}}\PY{l+s+s2}{Solusi x=}\PY{l+s+se}{\PYZbs{}n}\PY{l+s+s2}{\PYZdq{}}\PY{p}{,} \PY{n}{x}\PY{p}{)}
         \PY{n+nb}{print}\PY{p}{(}\PY{l+s+s2}{\PYZdq{}}\PY{l+s+s2}{Cek solusi: Ax \PYZhy{} b}\PY{l+s+se}{\PYZbs{}n}\PY{l+s+s2}{\PYZdq{}}\PY{p}{,} \PY{n}{A}\PY{o}{*}\PY{n}{x} \PY{o}{\PYZhy{}} \PY{n}{b}\PY{p}{)}
\end{Verbatim}


    \begin{Verbatim}[commandchars=\\\{\}]
Solusi x=
 [[10.]
 [22.]
 [14.]]
Cek solusi: Ax - b
 [[1.77635684e-15]
 [2.13162821e-14]
 [0.00000000e+00]]

    \end{Verbatim}

    Menggunakan dekomposisi LU:

    \begin{Verbatim}[commandchars=\\\{\}]
{\color{incolor}In [{\color{incolor}55}]:} \PY{n}{A} \PY{o}{=} \PY{n}{np}\PY{o}{.}\PY{n}{matrix}\PY{p}{(}\PY{p}{[}
             \PY{p}{[}\PY{l+m+mi}{6}\PY{p}{,} \PY{o}{\PYZhy{}}\PY{l+m+mi}{4}\PY{p}{,} \PY{l+m+mi}{1}\PY{p}{]}\PY{p}{,}
             \PY{p}{[}\PY{o}{\PYZhy{}}\PY{l+m+mi}{4}\PY{p}{,} \PY{l+m+mi}{6}\PY{p}{,} \PY{o}{\PYZhy{}}\PY{l+m+mi}{4}\PY{p}{]}\PY{p}{,}
             \PY{p}{[}\PY{l+m+mi}{1}\PY{p}{,} \PY{o}{\PYZhy{}}\PY{l+m+mi}{4}\PY{p}{,} \PY{l+m+mi}{6}\PY{p}{]}\PY{p}{]}\PY{p}{,} \PY{n}{dtype}\PY{o}{=}\PY{n}{np}\PY{o}{.}\PY{n}{float64}\PY{p}{)}
         \PY{n}{b} \PY{o}{=} \PY{n}{np}\PY{o}{.}\PY{n}{matrix}\PY{p}{(}\PY{p}{[}\PY{o}{\PYZhy{}}\PY{l+m+mi}{14}\PY{p}{,} \PY{l+m+mi}{36}\PY{p}{,} \PY{l+m+mi}{6}\PY{p}{]}\PY{p}{,} \PY{n}{dtype}\PY{o}{=}\PY{n}{np}\PY{o}{.}\PY{n}{float64}\PY{p}{)}\PY{o}{.}\PY{n}{transpose}\PY{p}{(}\PY{p}{)}
         \PY{n}{L}\PY{p}{,} \PY{n}{U} \PY{o}{=} \PY{n}{LU\PYZus{}decomp}\PY{p}{(}\PY{n}{A}\PY{p}{)}
         \PY{n}{x} \PY{o}{=} \PY{n}{LU\PYZus{}solve}\PY{p}{(}\PY{n}{L}\PY{p}{,} \PY{n}{U}\PY{p}{,} \PY{n}{b}\PY{p}{)}
         \PY{n+nb}{print}\PY{p}{(}\PY{l+s+s2}{\PYZdq{}}\PY{l+s+s2}{Solusi x=}\PY{l+s+se}{\PYZbs{}n}\PY{l+s+s2}{\PYZdq{}}\PY{p}{,} \PY{n}{x}\PY{p}{)}
         \PY{n+nb}{print}\PY{p}{(}\PY{l+s+s2}{\PYZdq{}}\PY{l+s+s2}{Cek solusi: Ax \PYZhy{} b}\PY{l+s+se}{\PYZbs{}n}\PY{l+s+s2}{\PYZdq{}}\PY{p}{,} \PY{n}{A}\PY{o}{*}\PY{n}{x} \PY{o}{\PYZhy{}} \PY{n}{b}\PY{p}{)}
\end{Verbatim}


    \begin{Verbatim}[commandchars=\\\{\}]
Solusi x=
 [[10.]
 [22.]
 [14.]]
Cek solusi: Ax - b
 [[1.77635684e-15]
 [2.13162821e-14]
 [0.00000000e+00]]

    \end{Verbatim}

    \hypertarget{latihan-2}{%
\subsection{Latihan 2}\label{latihan-2}}

    Selasaikan persamaan:

\[
\begin{bmatrix}
3 & 1 & -1 \\
5 & 8 & 2 \\
3 & 1 & 10
\end{bmatrix}
\begin{bmatrix}
x_1 \\
x_2 \\
x_3
\end{bmatrix} =
\begin{bmatrix}
1 \\
-4 \\
5
\end{bmatrix}
\]

    Menggunakan eliminasi Gauss:

    \begin{Verbatim}[commandchars=\\\{\}]
{\color{incolor}In [{\color{incolor}56}]:} \PY{n}{A} \PY{o}{=} \PY{n}{np}\PY{o}{.}\PY{n}{matrix}\PY{p}{(} \PY{p}{[}
             \PY{p}{[}\PY{l+m+mi}{3}\PY{p}{,} \PY{l+m+mi}{1}\PY{p}{,} \PY{o}{\PYZhy{}}\PY{l+m+mi}{1}\PY{p}{]}\PY{p}{,}
             \PY{p}{[}\PY{l+m+mi}{5}\PY{p}{,} \PY{l+m+mi}{8}\PY{p}{,} \PY{l+m+mi}{2}\PY{p}{]}\PY{p}{,}
             \PY{p}{[}\PY{l+m+mi}{3}\PY{p}{,} \PY{l+m+mi}{1}\PY{p}{,} \PY{l+m+mi}{10}\PY{p}{]}
         \PY{p}{]}\PY{p}{,} \PY{n}{dtype}\PY{o}{=}\PY{n}{np}\PY{o}{.}\PY{n}{float64}\PY{p}{)}
         \PY{n}{b} \PY{o}{=} \PY{n}{np}\PY{o}{.}\PY{n}{matrix}\PY{p}{(}\PY{p}{[}\PY{l+m+mi}{1}\PY{p}{,} \PY{o}{\PYZhy{}}\PY{l+m+mi}{4}\PY{p}{,} \PY{l+m+mi}{5}\PY{p}{]}\PY{p}{,} \PY{n}{np}\PY{o}{.}\PY{n}{float64}\PY{p}{)}\PY{o}{.}\PY{n}{transpose}\PY{p}{(}\PY{p}{)}
         \PY{n}{x} \PY{o}{=} \PY{n}{gauss\PYZus{}elim}\PY{p}{(}\PY{n}{A}\PY{p}{,} \PY{n}{b}\PY{p}{)}
         \PY{n+nb}{print}\PY{p}{(}\PY{l+s+s2}{\PYZdq{}}\PY{l+s+s2}{Solusi x=}\PY{l+s+se}{\PYZbs{}n}\PY{l+s+s2}{\PYZdq{}}\PY{p}{,} \PY{n}{x}\PY{p}{)}
         \PY{n+nb}{print}\PY{p}{(}\PY{l+s+s2}{\PYZdq{}}\PY{l+s+s2}{Cek solusi: Ax \PYZhy{} b}\PY{l+s+se}{\PYZbs{}n}\PY{l+s+s2}{\PYZdq{}}\PY{p}{,} \PY{n}{A}\PY{o}{*}\PY{n}{x} \PY{o}{\PYZhy{}} \PY{n}{b}\PY{p}{)}
\end{Verbatim}


    \begin{Verbatim}[commandchars=\\\{\}]
Solusi x=
 [[ 0.82296651]
 [-1.10526316]
 [ 0.36363636]]
Cek solusi: Ax - b
 [[-1.11022302e-16]
 [-8.88178420e-16]
 [ 0.00000000e+00]]

    \end{Verbatim}

    Menggunakan dekomposisi LU:

    \begin{Verbatim}[commandchars=\\\{\}]
{\color{incolor}In [{\color{incolor}57}]:} \PY{n}{A} \PY{o}{=} \PY{n}{np}\PY{o}{.}\PY{n}{matrix}\PY{p}{(} \PY{p}{[}
             \PY{p}{[}\PY{l+m+mi}{3}\PY{p}{,} \PY{l+m+mi}{1}\PY{p}{,} \PY{o}{\PYZhy{}}\PY{l+m+mi}{1}\PY{p}{]}\PY{p}{,}
             \PY{p}{[}\PY{l+m+mi}{5}\PY{p}{,} \PY{l+m+mi}{8}\PY{p}{,} \PY{l+m+mi}{2}\PY{p}{]}\PY{p}{,}
             \PY{p}{[}\PY{l+m+mi}{3}\PY{p}{,} \PY{l+m+mi}{1}\PY{p}{,} \PY{l+m+mi}{10}\PY{p}{]}
         \PY{p}{]}\PY{p}{,} \PY{n}{dtype}\PY{o}{=}\PY{n}{np}\PY{o}{.}\PY{n}{float64}\PY{p}{)}
         \PY{n}{b} \PY{o}{=} \PY{n}{np}\PY{o}{.}\PY{n}{matrix}\PY{p}{(}\PY{p}{[}\PY{l+m+mi}{1}\PY{p}{,} \PY{o}{\PYZhy{}}\PY{l+m+mi}{4}\PY{p}{,} \PY{l+m+mi}{5}\PY{p}{]}\PY{p}{,} \PY{n}{np}\PY{o}{.}\PY{n}{float64}\PY{p}{)}\PY{o}{.}\PY{n}{transpose}\PY{p}{(}\PY{p}{)}
         \PY{n}{L}\PY{p}{,} \PY{n}{U} \PY{o}{=} \PY{n}{LU\PYZus{}decomp}\PY{p}{(}\PY{n}{A}\PY{p}{)}
         \PY{n}{x} \PY{o}{=} \PY{n}{LU\PYZus{}solve}\PY{p}{(}\PY{n}{L}\PY{p}{,} \PY{n}{U}\PY{p}{,} \PY{n}{b}\PY{p}{)}
         \PY{n+nb}{print}\PY{p}{(}\PY{l+s+s2}{\PYZdq{}}\PY{l+s+s2}{Solusi x=}\PY{l+s+se}{\PYZbs{}n}\PY{l+s+s2}{\PYZdq{}}\PY{p}{,} \PY{n}{x}\PY{p}{)}
         \PY{n+nb}{print}\PY{p}{(}\PY{l+s+s2}{\PYZdq{}}\PY{l+s+s2}{Cek solusi: Ax \PYZhy{} b}\PY{l+s+se}{\PYZbs{}n}\PY{l+s+s2}{\PYZdq{}}\PY{p}{,} \PY{n}{A}\PY{o}{*}\PY{n}{x} \PY{o}{\PYZhy{}} \PY{n}{b}\PY{p}{)}
\end{Verbatim}


    \begin{Verbatim}[commandchars=\\\{\}]
Solusi x=
 [[ 0.82296651]
 [-1.10526316]
 [ 0.36363636]]
Cek solusi: Ax - b
 [[-1.11022302e-16]
 [-8.88178420e-16]
 [ 0.00000000e+00]]

    \end{Verbatim}

    \hypertarget{pivoting}{%
\subsection{Pivoting}\label{pivoting}}

    Metode eliminasi Gauss dan LU memiliki beberapa kekurangan. Salah satu
yang sering ditemui adalah ketika elemen pivot yang ditemukan adalah 0.
Misalkan pada persamaan berikut ini:

\[
\begin{bmatrix}
0 & -3 & 7 \\
1 & 2 & -1 \\
5 & -2 & 0
\end{bmatrix}
\begin{bmatrix}
x_1 \\
x_2 \\
x_3
\end{bmatrix} = 
\begin{bmatrix}
2 \\
3 \\
4
\end{bmatrix}
\]

    Jika kita langsung menggunakan fungsi \texttt{gauss\_elim} kita akan
mendapatkan pesan peringatan sebagai berikut:

\begin{Shaded}
\begin{Highlighting}[]
\NormalTok{A }\OperatorTok{=}\NormalTok{ np.matrix( [}
\NormalTok{    [}\DecValTok{0}\NormalTok{, }\DecValTok{-3}\NormalTok{, }\DecValTok{7}\NormalTok{],}
\NormalTok{    [}\DecValTok{1}\NormalTok{, }\DecValTok{2}\NormalTok{, }\DecValTok{-1}\NormalTok{],}
\NormalTok{    [}\DecValTok{5}\NormalTok{, }\DecValTok{-2}\NormalTok{, }\DecValTok{0}\NormalTok{]}
\NormalTok{], dtype}\OperatorTok{=}\NormalTok{np.float64)}
\NormalTok{b }\OperatorTok{=}\NormalTok{ np.matrix([}\DecValTok{2}\NormalTok{, }\DecValTok{3}\NormalTok{, }\DecValTok{4}\NormalTok{], np.float64).transpose()}
\NormalTok{x }\OperatorTok{=}\NormalTok{ gauss_elim(A, b)}
\end{Highlighting}
\end{Shaded}

\begin{verbatim}
/home/efefer/miniconda3/lib/python3.7/site-packages/ipykernel_launcher.py:14: RuntimeWarning: divide by zero encountered in double_scalars
  
/home/efefer/miniconda3/lib/python3.7/site-packages/ipykernel_launcher.py:15: RuntimeWarning: invalid value encountered in multiply
  from ipykernel import kernelapp as app
/home/efefer/miniconda3/lib/python3.7/site-packages/ipykernel_launcher.py:14: RuntimeWarning: invalid value encountered in double_scalars
\end{verbatim}

    Solusi yang dapat digunakan adalah dengan cara menukar baris atau
pivoting sedemikian rupa sehingga elemen pivot yang diperoleh tidak
menjadi nol. Dalam kasus ini, kita dapat menukar baris pertama dengan
baris ketiga:

\[
\begin{bmatrix}
5 & -2 & 0 \\
1 & 2 & -1 \\
0 & -3 & 7
\end{bmatrix}
\begin{bmatrix}
x_3 \\
x_2 \\
x_1
\end{bmatrix} =
\begin{bmatrix}
4 \\
3 \\
2
\end{bmatrix}
\]

    \begin{Verbatim}[commandchars=\\\{\}]
{\color{incolor}In [{\color{incolor}58}]:} \PY{n}{A} \PY{o}{=} \PY{n}{np}\PY{o}{.}\PY{n}{matrix}\PY{p}{(} \PY{p}{[}
             \PY{p}{[}\PY{l+m+mi}{5}\PY{p}{,} \PY{o}{\PYZhy{}}\PY{l+m+mi}{2}\PY{p}{,} \PY{l+m+mi}{0}\PY{p}{]}\PY{p}{,}
             \PY{p}{[}\PY{l+m+mi}{1}\PY{p}{,} \PY{l+m+mi}{2}\PY{p}{,} \PY{o}{\PYZhy{}}\PY{l+m+mi}{1}\PY{p}{]}\PY{p}{,}
             \PY{p}{[}\PY{l+m+mi}{0}\PY{p}{,} \PY{o}{\PYZhy{}}\PY{l+m+mi}{3}\PY{p}{,} \PY{l+m+mi}{7}\PY{p}{]}
         \PY{p}{]}\PY{p}{,} \PY{n}{dtype}\PY{o}{=}\PY{n}{np}\PY{o}{.}\PY{n}{float64}\PY{p}{)}
         \PY{n}{b} \PY{o}{=} \PY{n}{np}\PY{o}{.}\PY{n}{matrix}\PY{p}{(}\PY{p}{[}\PY{l+m+mi}{4}\PY{p}{,} \PY{l+m+mi}{3}\PY{p}{,} \PY{l+m+mi}{2}\PY{p}{]}\PY{p}{,} \PY{n}{np}\PY{o}{.}\PY{n}{float64}\PY{p}{)}\PY{o}{.}\PY{n}{transpose}\PY{p}{(}\PY{p}{)}
         \PY{n}{x} \PY{o}{=} \PY{n}{gauss\PYZus{}elim}\PY{p}{(}\PY{n}{A}\PY{p}{,} \PY{n}{b}\PY{p}{)}
         \PY{n+nb}{print}\PY{p}{(}\PY{l+s+s2}{\PYZdq{}}\PY{l+s+s2}{Solusi x=}\PY{l+s+se}{\PYZbs{}n}\PY{l+s+s2}{\PYZdq{}}\PY{p}{,} \PY{n}{x}\PY{p}{)}
         \PY{n+nb}{print}\PY{p}{(}\PY{l+s+s2}{\PYZdq{}}\PY{l+s+s2}{Cek solusi: Ax \PYZhy{} b}\PY{l+s+se}{\PYZbs{}n}\PY{l+s+s2}{\PYZdq{}}\PY{p}{,} \PY{n}{A}\PY{o}{*}\PY{n}{x} \PY{o}{\PYZhy{}} \PY{n}{b}\PY{p}{)}
\end{Verbatim}


    \begin{Verbatim}[commandchars=\\\{\}]
Solusi x=
 [[1.30434783]
 [1.26086957]
 [0.82608696]]
Cek solusi: Ax - b
 [[ 0.0000000e+00]
 [ 4.4408921e-16]
 [-8.8817842e-16]]

    \end{Verbatim}

    Dapat dilihat bahwa pivoting pada dasarnya bertujuan untuk memperbaik
sifat dominan diagonal dari matriks. Suatu matriks dikatakan dominan
diagonal apabila nilai absolut dari elemen diagoanal matriks memiliki
nilai yang terbesar bila dibandingkan dengan nilai-nilai elemen lainnya
dalam satu baris.

    Ada beberapa strategi yang dapat digunakan untuk pivoting, salah satu
yang paling sederhana adalah dengan menggunakan pivoting terskala.
Dengan metode ini, pertama kali kita mencari array \(\mathbf{s}\) dengan
elemen sebagai berikut:

\[
s_{i} = \max\left|A_{ij}\right|,\,\,\,i=1,2,\ldots,N
\]

\(s_{i}\) akan disebut sebagai faktor skala dari baris ke-\(i\) yang
merupakan elemen dengan nilai absolut terbesar dari baris ke-\(i\).
Ukuran relatif dari elemen \(A_{ij}\) relatif terhadap elemen dengan
nilai terbesar adalah: \[
r_{ij} = \frac{\left|A_{ij}\right|}{s_{i}}
\] Elemen pivot dari matriks \(\mathbf{A}\) akan ditentukan berdasarkan
\(r_{ij}\). Elemen \(A_{kk}\) tidak secara otomatis menjadi elemen
pivot, namun kita akan mencari elemen lain di bawah \(A_{kk}\) pada
kolom ke-\(k\) untuk elemen pivot yang terbaik. Misalkan elemen terbaik
ini ada pada baris ke-\(p\), yaitu \(A_{pk}\) yang memiliki ukuran
relatif terbesar, yakni: \[
r_{pk} = \max_{j} r_{jk},\,\,\, j \geq k
\] Jika elemen tersebut ditemukan maka kita melakukan pertukaran baris
antara baris ke-\(k\) dan ke-\(p\).

    \hypertarget{kode-python-untuk-eliminasi-gauss-dengan-pivoting}{%
\subsubsection{Kode Python untuk eliminasi Gauss dengan
pivoting}\label{kode-python-untuk-eliminasi-gauss-dengan-pivoting}}

    \begin{Verbatim}[commandchars=\\\{\}]
{\color{incolor}In [{\color{incolor}59}]:} \PY{k}{def} \PY{n+nf}{tukar\PYZus{}baris}\PY{p}{(}\PY{n}{v}\PY{p}{,} \PY{n}{i}\PY{p}{,} \PY{n}{j}\PY{p}{)}\PY{p}{:}
             \PY{k}{if} \PY{n+nb}{len}\PY{p}{(}\PY{n}{v}\PY{o}{.}\PY{n}{shape}\PY{p}{)} \PY{o}{==} \PY{l+m+mi}{1}\PY{p}{:} \PY{c+c1}{\PYZsh{} array satu dimensi atau vektor kolom}
                 \PY{n}{v}\PY{p}{[}\PY{n}{i}\PY{p}{]}\PY{p}{,} \PY{n}{v}\PY{p}{[}\PY{n}{j}\PY{p}{]} \PY{o}{=} \PY{n}{v}\PY{p}{[}\PY{n}{j}\PY{p}{]}\PY{p}{,} \PY{n}{v}\PY{p}{[}\PY{n}{i}\PY{p}{]}
             \PY{k}{else}\PY{p}{:}
                 \PY{n}{v}\PY{p}{[}\PY{p}{[}\PY{n}{i}\PY{p}{,}\PY{n}{j}\PY{p}{]}\PY{p}{,}\PY{p}{:}\PY{p}{]} \PY{o}{=} \PY{n}{v}\PY{p}{[}\PY{p}{[}\PY{n}{j}\PY{p}{,}\PY{n}{i}\PY{p}{]}\PY{p}{,}\PY{p}{:}\PY{p}{]}
\end{Verbatim}


    \begin{Verbatim}[commandchars=\\\{\}]
{\color{incolor}In [{\color{incolor}60}]:} \PY{k}{def} \PY{n+nf}{gauss\PYZus{}elim\PYZus{}pivot}\PY{p}{(}\PY{n}{A\PYZus{}}\PY{p}{,} \PY{n}{b\PYZus{}}\PY{p}{)}\PY{p}{:}
             \PY{n}{N}\PY{p}{,} \PY{n}{Nrhs} \PY{o}{=} \PY{n}{b\PYZus{}}\PY{o}{.}\PY{n}{shape}
             
             \PY{k}{assert} \PY{n}{Nrhs} \PY{o}{==} \PY{l+m+mi}{1}
         
             \PY{n}{A} \PY{o}{=} \PY{n}{np}\PY{o}{.}\PY{n}{matrix}\PY{o}{.}\PY{n}{copy}\PY{p}{(}\PY{n}{A\PYZus{}}\PY{p}{)}
             \PY{n}{b} \PY{o}{=} \PY{n}{np}\PY{o}{.}\PY{n}{matrix}\PY{o}{.}\PY{n}{copy}\PY{p}{(}\PY{n}{b\PYZus{}}\PY{p}{)}
             
             \PY{c+c1}{\PYZsh{} Faktor skala}
             \PY{n}{s} \PY{o}{=} \PY{n}{np}\PY{o}{.}\PY{n}{matrix}\PY{p}{(}\PY{n}{np}\PY{o}{.}\PY{n}{zeros}\PY{p}{(}\PY{p}{(}\PY{n}{N}\PY{p}{,}\PY{l+m+mi}{1}\PY{p}{)}\PY{p}{)}\PY{p}{)}
             \PY{k}{for} \PY{n}{i} \PY{o+ow}{in} \PY{n+nb}{range}\PY{p}{(}\PY{n}{N}\PY{p}{)}\PY{p}{:}
                 \PY{n}{s}\PY{p}{[}\PY{n}{i}\PY{p}{]} \PY{o}{=} \PY{n}{np}\PY{o}{.}\PY{n}{max}\PY{p}{(}\PY{n}{np}\PY{o}{.}\PY{n}{abs}\PY{p}{(}\PY{n}{A}\PY{p}{[}\PY{n}{i}\PY{p}{,}\PY{p}{:}\PY{p}{]}\PY{p}{)}\PY{p}{)}
         
             \PY{n}{SMALL} \PY{o}{=} \PY{n}{np}\PY{o}{.}\PY{n}{finfo}\PY{p}{(}\PY{n}{np}\PY{o}{.}\PY{n}{float64}\PY{p}{)}\PY{o}{.}\PY{n}{eps}
             
             \PY{c+c1}{\PYZsh{} Eliminasi maju}
             \PY{k}{for} \PY{n}{k} \PY{o+ow}{in} \PY{n+nb}{range}\PY{p}{(}\PY{l+m+mi}{0}\PY{p}{,}\PY{n}{N}\PY{o}{\PYZhy{}}\PY{l+m+mi}{1}\PY{p}{)}\PY{p}{:}
                 
                 \PY{n}{r} \PY{o}{=} \PY{n}{np}\PY{o}{.}\PY{n}{abs}\PY{p}{(}\PY{n}{A}\PY{p}{[}\PY{n}{k}\PY{p}{:}\PY{n}{N}\PY{p}{,}\PY{n}{k}\PY{p}{]}\PY{p}{)}\PY{o}{/}\PY{n}{s}\PY{p}{[}\PY{n}{k}\PY{p}{:}\PY{n}{N}\PY{p}{]}
                 \PY{n}{p} \PY{o}{=} \PY{n}{np}\PY{o}{.}\PY{n}{argmax}\PY{p}{(}\PY{n}{r}\PY{p}{)} \PY{o}{+} \PY{n}{k}
                 \PY{k}{if} \PY{n+nb}{abs}\PY{p}{(}\PY{n}{A}\PY{p}{[}\PY{n}{p}\PY{p}{,}\PY{n}{k}\PY{p}{]}\PY{p}{)} \PY{o}{\PYZlt{}} \PY{n}{SMALL}\PY{p}{:}
                     \PY{k}{raise} \PY{n+ne}{RuntimeError}\PY{p}{(}\PY{l+s+s2}{\PYZdq{}}\PY{l+s+s2}{Matriks A singular}\PY{l+s+s2}{\PYZdq{}}\PY{p}{)}
                 \PY{c+c1}{\PYZsh{} Tukar baris jika diperlukan}
                 \PY{k}{if} \PY{n}{p} \PY{o}{!=} \PY{n}{k}\PY{p}{:}
                     \PY{n+nb}{print}\PY{p}{(}\PY{l+s+s2}{\PYZdq{}}\PY{l+s+s2}{INFO: tukar baris }\PY{l+s+si}{\PYZpc{}d}\PY{l+s+s2}{ dengan }\PY{l+s+si}{\PYZpc{}d}\PY{l+s+s2}{\PYZdq{}} \PY{o}{\PYZpc{}} \PY{p}{(}\PY{n}{p}\PY{p}{,}\PY{n}{k}\PY{p}{)}\PY{p}{)}
                     \PY{n}{tukar\PYZus{}baris}\PY{p}{(}\PY{n}{b}\PY{p}{,} \PY{n}{k}\PY{p}{,} \PY{n}{p}\PY{p}{)}
                     \PY{n}{tukar\PYZus{}baris}\PY{p}{(}\PY{n}{s}\PY{p}{,} \PY{n}{k}\PY{p}{,} \PY{n}{p}\PY{p}{)}
                     \PY{n}{tukar\PYZus{}baris}\PY{p}{(}\PY{n}{A}\PY{p}{,} \PY{n}{k}\PY{p}{,} \PY{n}{p}\PY{p}{)}
                 
                 \PY{k}{for} \PY{n}{i} \PY{o+ow}{in} \PY{n+nb}{range}\PY{p}{(}\PY{n}{k}\PY{o}{+}\PY{l+m+mi}{1}\PY{p}{,}\PY{n}{N}\PY{p}{)}\PY{p}{:}
                     \PY{k}{if} \PY{n}{A}\PY{p}{[}\PY{n}{i}\PY{p}{,}\PY{n}{k}\PY{p}{]} \PY{o}{!=} \PY{l+m+mf}{0.0}\PY{p}{:}
                         \PY{n}{alpha} \PY{o}{=} \PY{n}{A}\PY{p}{[}\PY{n}{i}\PY{p}{,}\PY{n}{k}\PY{p}{]}\PY{o}{/}\PY{n}{A}\PY{p}{[}\PY{n}{k}\PY{p}{,}\PY{n}{k}\PY{p}{]}
                         \PY{n}{A}\PY{p}{[}\PY{n}{i}\PY{p}{,}\PY{p}{:}\PY{p}{]} \PY{o}{=} \PY{n}{A}\PY{p}{[}\PY{n}{i}\PY{p}{,}\PY{p}{:}\PY{p}{]} \PY{o}{\PYZhy{}} \PY{n}{alpha}\PY{o}{*}\PY{n}{A}\PY{p}{[}\PY{n}{k}\PY{p}{,}\PY{p}{:}\PY{p}{]}
                         \PY{n}{b}\PY{p}{[}\PY{n}{i}\PY{p}{]} \PY{o}{=} \PY{n}{b}\PY{p}{[}\PY{n}{i}\PY{p}{]} \PY{o}{\PYZhy{}} \PY{n}{alpha}\PY{o}{*}\PY{n}{b}\PY{p}{[}\PY{n}{k}\PY{p}{]}
             
             \PY{c+c1}{\PYZsh{} Alokasi mememori untuk solusi}
             \PY{n}{x} \PY{o}{=} \PY{n}{np}\PY{o}{.}\PY{n}{matrix}\PY{p}{(}\PY{n}{np}\PY{o}{.}\PY{n}{zeros}\PY{p}{(}\PY{p}{(}\PY{n}{N}\PY{p}{,}\PY{l+m+mi}{1}\PY{p}{)}\PY{p}{)}\PY{p}{)}
             
             \PY{c+c1}{\PYZsh{} Substitusi balik}
             \PY{n}{x}\PY{p}{[}\PY{n}{N}\PY{o}{\PYZhy{}}\PY{l+m+mi}{1}\PY{p}{]} \PY{o}{=} \PY{n}{b}\PY{p}{[}\PY{n}{N}\PY{o}{\PYZhy{}}\PY{l+m+mi}{1}\PY{p}{]}\PY{o}{/}\PY{n}{A}\PY{p}{[}\PY{n}{N}\PY{o}{\PYZhy{}}\PY{l+m+mi}{1}\PY{p}{,}\PY{n}{N}\PY{o}{\PYZhy{}}\PY{l+m+mi}{1}\PY{p}{]}
             \PY{k}{for} \PY{n}{k} \PY{o+ow}{in} \PY{n+nb}{range}\PY{p}{(}\PY{n}{N}\PY{o}{\PYZhy{}}\PY{l+m+mi}{2}\PY{p}{,}\PY{o}{\PYZhy{}}\PY{l+m+mi}{1}\PY{p}{,}\PY{o}{\PYZhy{}}\PY{l+m+mi}{1}\PY{p}{)}\PY{p}{:}
                 \PY{n}{ss} \PY{o}{=} \PY{l+m+mf}{0.0}
                 \PY{k}{for} \PY{n}{j} \PY{o+ow}{in} \PY{n+nb}{range}\PY{p}{(}\PY{n}{k}\PY{o}{+}\PY{l+m+mi}{1}\PY{p}{,}\PY{n}{N}\PY{p}{)}\PY{p}{:}
                     \PY{n}{ss} \PY{o}{=} \PY{n}{ss} \PY{o}{+} \PY{n}{A}\PY{p}{[}\PY{n}{k}\PY{p}{,}\PY{n}{j}\PY{p}{]}\PY{o}{*}\PY{n}{x}\PY{p}{[}\PY{n}{j}\PY{p}{]}
                 \PY{n}{x}\PY{p}{[}\PY{n}{k}\PY{p}{]} \PY{o}{=} \PY{p}{(}\PY{n}{b}\PY{p}{[}\PY{n}{k}\PY{p}{]} \PY{o}{\PYZhy{}} \PY{n}{ss}\PY{p}{)}\PY{o}{/}\PY{n}{A}\PY{p}{[}\PY{n}{k}\PY{p}{,}\PY{n}{k}\PY{p}{]}
             
             \PY{k}{return} \PY{n}{x}
\end{Verbatim}


    Contoh penggunaan:

    \begin{Verbatim}[commandchars=\\\{\}]
{\color{incolor}In [{\color{incolor}61}]:} \PY{n}{A} \PY{o}{=} \PY{n}{np}\PY{o}{.}\PY{n}{matrix}\PY{p}{(} \PY{p}{[}
             \PY{p}{[}\PY{l+m+mi}{0}\PY{p}{,} \PY{o}{\PYZhy{}}\PY{l+m+mi}{3}\PY{p}{,} \PY{l+m+mi}{7}\PY{p}{]}\PY{p}{,}
             \PY{p}{[}\PY{l+m+mi}{1}\PY{p}{,} \PY{l+m+mi}{2}\PY{p}{,} \PY{o}{\PYZhy{}}\PY{l+m+mi}{1}\PY{p}{]}\PY{p}{,}
             \PY{p}{[}\PY{l+m+mi}{5}\PY{p}{,} \PY{o}{\PYZhy{}}\PY{l+m+mi}{2}\PY{p}{,} \PY{l+m+mi}{0}\PY{p}{]}
         \PY{p}{]}\PY{p}{,} \PY{n}{dtype}\PY{o}{=}\PY{n}{np}\PY{o}{.}\PY{n}{float64}\PY{p}{)}
         \PY{n}{b} \PY{o}{=} \PY{n}{np}\PY{o}{.}\PY{n}{matrix}\PY{p}{(}\PY{p}{[}\PY{l+m+mi}{2}\PY{p}{,} \PY{l+m+mi}{3}\PY{p}{,} \PY{l+m+mi}{4}\PY{p}{]}\PY{p}{,} \PY{n}{np}\PY{o}{.}\PY{n}{float64}\PY{p}{)}\PY{o}{.}\PY{n}{transpose}\PY{p}{(}\PY{p}{)}
         
         \PY{n}{x} \PY{o}{=} \PY{n}{gauss\PYZus{}elim\PYZus{}pivot}\PY{p}{(}\PY{n}{A}\PY{p}{,} \PY{n}{b}\PY{p}{)}
         \PY{n+nb}{print}\PY{p}{(}\PY{l+s+s2}{\PYZdq{}}\PY{l+s+s2}{Solusi x=}\PY{l+s+se}{\PYZbs{}n}\PY{l+s+s2}{\PYZdq{}}\PY{p}{,} \PY{n}{x}\PY{p}{)}
         \PY{n+nb}{print}\PY{p}{(}\PY{l+s+s2}{\PYZdq{}}\PY{l+s+s2}{Cek solusi: Ax \PYZhy{} b}\PY{l+s+se}{\PYZbs{}n}\PY{l+s+s2}{\PYZdq{}}\PY{p}{,} \PY{n}{A}\PY{o}{*}\PY{n}{x} \PY{o}{\PYZhy{}} \PY{n}{b}\PY{p}{)}
\end{Verbatim}


    \begin{Verbatim}[commandchars=\\\{\}]
INFO: tukar baris 2 dengan 0
Solusi x=
 [[1.30434783]
 [1.26086957]
 [0.82608696]]
Cek solusi: Ax - b
 [[-8.8817842e-16]
 [ 4.4408921e-16]
 [ 0.0000000e+00]]

    \end{Verbatim}

    \hypertarget{kode-python-untuk-dekomposisi-lu-dengan-pivoting}{%
\subsubsection{Kode Python untuk dekomposisi LU dengan
pivoting}\label{kode-python-untuk-dekomposisi-lu-dengan-pivoting}}

    \begin{Verbatim}[commandchars=\\\{\}]
{\color{incolor}In [{\color{incolor}62}]:} \PY{k}{def} \PY{n+nf}{LU\PYZus{}decomp\PYZus{}pivot}\PY{p}{(}\PY{n}{A\PYZus{}}\PY{p}{)}\PY{p}{:}
             
             \PY{n}{Nrow}\PY{p}{,} \PY{n}{Ncol} \PY{o}{=} \PY{n}{A\PYZus{}}\PY{o}{.}\PY{n}{shape}
             
             \PY{k}{assert} \PY{n}{Nrow} \PY{o}{==} \PY{n}{Ncol}
             
             \PY{n}{N} \PY{o}{=} \PY{n}{Nrow}
         
             \PY{n}{A} \PY{o}{=} \PY{n}{np}\PY{o}{.}\PY{n}{matrix}\PY{o}{.}\PY{n}{copy}\PY{p}{(}\PY{n}{A\PYZus{}}\PY{p}{)}
             
             \PY{c+c1}{\PYZsh{} Faktor skala}
             \PY{n}{s} \PY{o}{=} \PY{n}{np}\PY{o}{.}\PY{n}{matrix}\PY{p}{(}\PY{n}{np}\PY{o}{.}\PY{n}{zeros}\PY{p}{(}\PY{p}{(}\PY{n}{N}\PY{p}{,}\PY{l+m+mi}{1}\PY{p}{)}\PY{p}{)}\PY{p}{)}
             \PY{k}{for} \PY{n}{i} \PY{o+ow}{in} \PY{n+nb}{range}\PY{p}{(}\PY{n}{N}\PY{p}{)}\PY{p}{:}
                 \PY{n}{s}\PY{p}{[}\PY{n}{i}\PY{p}{]} \PY{o}{=} \PY{n}{np}\PY{o}{.}\PY{n}{max}\PY{p}{(}\PY{n}{np}\PY{o}{.}\PY{n}{abs}\PY{p}{(}\PY{n}{A}\PY{p}{[}\PY{n}{i}\PY{p}{,}\PY{p}{:}\PY{p}{]}\PY{p}{)}\PY{p}{)}
                 
             \PY{n}{iperm} \PY{o}{=} \PY{n}{np}\PY{o}{.}\PY{n}{arange}\PY{p}{(}\PY{n}{N}\PY{p}{)}
         
             \PY{n}{SMALL} \PY{o}{=} \PY{n}{np}\PY{o}{.}\PY{n}{finfo}\PY{p}{(}\PY{n}{np}\PY{o}{.}\PY{n}{float64}\PY{p}{)}\PY{o}{.}\PY{n}{eps}
         
             \PY{c+c1}{\PYZsh{} Eliminasi Gauss (maju)}
             \PY{k}{for} \PY{n}{k} \PY{o+ow}{in} \PY{n+nb}{range}\PY{p}{(}\PY{l+m+mi}{0}\PY{p}{,}\PY{n}{N}\PY{o}{\PYZhy{}}\PY{l+m+mi}{1}\PY{p}{)}\PY{p}{:}
                 
                 \PY{n}{r} \PY{o}{=} \PY{n}{np}\PY{o}{.}\PY{n}{abs}\PY{p}{(}\PY{n}{A}\PY{p}{[}\PY{n}{k}\PY{p}{:}\PY{n}{N}\PY{p}{,}\PY{n}{k}\PY{p}{]}\PY{p}{)}\PY{o}{/}\PY{n}{s}\PY{p}{[}\PY{n}{k}\PY{p}{:}\PY{n}{N}\PY{p}{]}
                 \PY{n}{p} \PY{o}{=} \PY{n}{np}\PY{o}{.}\PY{n}{argmax}\PY{p}{(}\PY{n}{r}\PY{p}{)} \PY{o}{+} \PY{n}{k}
                 \PY{k}{if} \PY{n+nb}{abs}\PY{p}{(}\PY{n}{A}\PY{p}{[}\PY{n}{p}\PY{p}{,}\PY{n}{k}\PY{p}{]}\PY{p}{)} \PY{o}{\PYZlt{}} \PY{n}{SMALL}\PY{p}{:}
                     \PY{k}{raise} \PY{n+ne}{RuntimeError}\PY{p}{(}\PY{l+s+s2}{\PYZdq{}}\PY{l+s+s2}{Matriks A singular}\PY{l+s+s2}{\PYZdq{}}\PY{p}{)}
                 \PY{c+c1}{\PYZsh{} Tukar baris jika diperlukan}
                 \PY{k}{if} \PY{n}{p} \PY{o}{!=} \PY{n}{k}\PY{p}{:}
                     \PY{n+nb}{print}\PY{p}{(}\PY{l+s+s2}{\PYZdq{}}\PY{l+s+s2}{INFO: tukar baris }\PY{l+s+si}{\PYZpc{}d}\PY{l+s+s2}{ dengan }\PY{l+s+si}{\PYZpc{}d}\PY{l+s+s2}{\PYZdq{}} \PY{o}{\PYZpc{}} \PY{p}{(}\PY{n}{p}\PY{p}{,}\PY{n}{k}\PY{p}{)}\PY{p}{)}
                     \PY{n}{tukar\PYZus{}baris}\PY{p}{(}\PY{n}{A}\PY{p}{,} \PY{n}{k}\PY{p}{,} \PY{n}{p}\PY{p}{)}
                     \PY{n}{tukar\PYZus{}baris}\PY{p}{(}\PY{n}{s}\PY{p}{,} \PY{n}{k}\PY{p}{,} \PY{n}{p}\PY{p}{)}
                     \PY{n}{tukar\PYZus{}baris}\PY{p}{(}\PY{n}{iperm}\PY{p}{,} \PY{n}{k}\PY{p}{,} \PY{n}{p}\PY{p}{)}
                 
                 \PY{k}{for} \PY{n}{i} \PY{o+ow}{in} \PY{n+nb}{range}\PY{p}{(}\PY{n}{k}\PY{o}{+}\PY{l+m+mi}{1}\PY{p}{,}\PY{n}{N}\PY{p}{)}\PY{p}{:}
                     \PY{k}{if} \PY{n}{A}\PY{p}{[}\PY{n}{i}\PY{p}{,}\PY{n}{k}\PY{p}{]} \PY{o}{!=} \PY{l+m+mf}{0.0}\PY{p}{:}
                         \PY{n}{alpha} \PY{o}{=} \PY{n}{A}\PY{p}{[}\PY{n}{i}\PY{p}{,}\PY{n}{k}\PY{p}{]}\PY{o}{/}\PY{n}{A}\PY{p}{[}\PY{n}{k}\PY{p}{,}\PY{n}{k}\PY{p}{]}
                         \PY{n}{A}\PY{p}{[}\PY{n}{i}\PY{p}{,}\PY{n}{k}\PY{o}{+}\PY{l+m+mi}{1}\PY{p}{:}\PY{n}{N}\PY{p}{]} \PY{o}{=} \PY{n}{A}\PY{p}{[}\PY{n}{i}\PY{p}{,}\PY{n}{k}\PY{o}{+}\PY{l+m+mi}{1}\PY{p}{:}\PY{n}{N}\PY{p}{]} \PY{o}{\PYZhy{}} \PY{n}{alpha}\PY{o}{*}\PY{n}{A}\PY{p}{[}\PY{n}{k}\PY{p}{,}\PY{n}{k}\PY{o}{+}\PY{l+m+mi}{1}\PY{p}{:}\PY{n}{N}\PY{p}{]}
                         \PY{n}{A}\PY{p}{[}\PY{n}{i}\PY{p}{,}\PY{n}{k}\PY{p}{]} \PY{o}{=} \PY{n}{alpha}
             
             \PY{n}{L} \PY{o}{=} \PY{n}{np}\PY{o}{.}\PY{n}{matrix}\PY{p}{(} \PY{n}{np}\PY{o}{.}\PY{n}{tril}\PY{p}{(}\PY{n}{A}\PY{p}{,}\PY{o}{\PYZhy{}}\PY{l+m+mi}{1}\PY{p}{)} \PY{p}{)}
             \PY{k}{for} \PY{n}{i} \PY{o+ow}{in} \PY{n+nb}{range}\PY{p}{(}\PY{n}{N}\PY{p}{)}\PY{p}{:}
                 \PY{n}{L}\PY{p}{[}\PY{n}{i}\PY{p}{,}\PY{n}{i}\PY{p}{]} \PY{o}{=} \PY{l+m+mf}{1.0} \PY{c+c1}{\PYZsh{} konstrain Doolittle}
             \PY{n}{U} \PY{o}{=} \PY{n}{np}\PY{o}{.}\PY{n}{matrix}\PY{p}{(} \PY{n}{np}\PY{o}{.}\PY{n}{triu}\PY{p}{(}\PY{n}{A}\PY{p}{)} \PY{p}{)}
             
             \PY{k}{return} \PY{n}{L}\PY{p}{,} \PY{n}{U}\PY{p}{,} \PY{n}{iperm} \PY{c+c1}{\PYZsh{} kembalikan matriks L dan U serta vektor permutasi}
\end{Verbatim}


    \begin{Verbatim}[commandchars=\\\{\}]
{\color{incolor}In [{\color{incolor}63}]:} \PY{k}{def} \PY{n+nf}{LU\PYZus{}solve\PYZus{}pivot}\PY{p}{(}\PY{n}{L}\PY{p}{,} \PY{n}{U}\PY{p}{,} \PY{n}{iperm}\PY{p}{,} \PY{n}{b\PYZus{}}\PY{p}{)}\PY{p}{:}
             
             \PY{n}{N} \PY{o}{=} \PY{n}{L}\PY{o}{.}\PY{n}{shape}\PY{p}{[}\PY{l+m+mi}{0}\PY{p}{]}
             
             \PY{n}{x} \PY{o}{=} \PY{n}{np}\PY{o}{.}\PY{n}{matrix}\PY{p}{(}\PY{n}{np}\PY{o}{.}\PY{n}{zeros}\PY{p}{(}\PY{p}{(}\PY{n}{N}\PY{p}{,}\PY{p}{)}\PY{p}{)}\PY{p}{)}\PY{o}{.}\PY{n}{transpose}\PY{p}{(}\PY{p}{)}
             \PY{n}{y} \PY{o}{=} \PY{n}{np}\PY{o}{.}\PY{n}{matrix}\PY{p}{(}\PY{n}{np}\PY{o}{.}\PY{n}{zeros}\PY{p}{(}\PY{p}{(}\PY{n}{N}\PY{p}{,}\PY{p}{)}\PY{p}{)}\PY{p}{)}\PY{o}{.}\PY{n}{transpose}\PY{p}{(}\PY{p}{)}
             
             \PY{n}{b} \PY{o}{=} \PY{n}{np}\PY{o}{.}\PY{n}{matrix}\PY{o}{.}\PY{n}{copy}\PY{p}{(}\PY{n}{b\PYZus{}}\PY{p}{)}
             \PY{k}{for} \PY{n}{i} \PY{o+ow}{in} \PY{n+nb}{range}\PY{p}{(}\PY{n}{N}\PY{p}{)}\PY{p}{:}
                 \PY{n}{b}\PY{p}{[}\PY{n}{i}\PY{p}{]} \PY{o}{=} \PY{n}{b\PYZus{}}\PY{p}{[}\PY{n}{iperm}\PY{p}{[}\PY{n}{i}\PY{p}{]}\PY{p}{]}
             
             \PY{c+c1}{\PYZsh{} Ly = b}
             \PY{c+c1}{\PYZsh{} Substitusi maju}
             \PY{n}{y}\PY{p}{[}\PY{l+m+mi}{0}\PY{p}{]} \PY{o}{=} \PY{n}{b}\PY{p}{[}\PY{l+m+mi}{0}\PY{p}{]}\PY{o}{/}\PY{n}{L}\PY{p}{[}\PY{l+m+mi}{0}\PY{p}{,}\PY{l+m+mi}{0}\PY{p}{]}
             \PY{k}{for} \PY{n}{k} \PY{o+ow}{in} \PY{n+nb}{range}\PY{p}{(}\PY{l+m+mi}{1}\PY{p}{,}\PY{n}{N}\PY{p}{)}\PY{p}{:}
                 \PY{n}{ss} \PY{o}{=} \PY{l+m+mf}{0.0}
                 \PY{k}{for} \PY{n}{j} \PY{o+ow}{in} \PY{n+nb}{range}\PY{p}{(}\PY{n}{k}\PY{p}{)}\PY{p}{:}
                     \PY{n}{ss} \PY{o}{=} \PY{n}{ss} \PY{o}{+} \PY{n}{L}\PY{p}{[}\PY{n}{k}\PY{p}{,}\PY{n}{j}\PY{p}{]}\PY{o}{*}\PY{n}{y}\PY{p}{[}\PY{n}{j}\PY{p}{]}
                 \PY{n}{y}\PY{p}{[}\PY{n}{k}\PY{p}{]} \PY{o}{=} \PY{p}{(}\PY{n}{b}\PY{p}{[}\PY{n}{k}\PY{p}{]} \PY{o}{\PYZhy{}} \PY{n}{ss}\PY{p}{)}\PY{o}{/}\PY{n}{L}\PY{p}{[}\PY{n}{k}\PY{p}{,}\PY{n}{k}\PY{p}{]}
             
             \PY{c+c1}{\PYZsh{} Ux = y}
             \PY{c+c1}{\PYZsh{} Substitusi balik}
             \PY{n}{x}\PY{p}{[}\PY{n}{N}\PY{o}{\PYZhy{}}\PY{l+m+mi}{1}\PY{p}{]} \PY{o}{=} \PY{n}{y}\PY{p}{[}\PY{n}{N}\PY{o}{\PYZhy{}}\PY{l+m+mi}{1}\PY{p}{]}\PY{o}{/}\PY{n}{U}\PY{p}{[}\PY{n}{N}\PY{o}{\PYZhy{}}\PY{l+m+mi}{1}\PY{p}{,}\PY{n}{N}\PY{o}{\PYZhy{}}\PY{l+m+mi}{1}\PY{p}{]}
             \PY{k}{for} \PY{n}{k} \PY{o+ow}{in} \PY{n+nb}{range}\PY{p}{(}\PY{n}{N}\PY{o}{\PYZhy{}}\PY{l+m+mi}{2}\PY{p}{,}\PY{o}{\PYZhy{}}\PY{l+m+mi}{1}\PY{p}{,}\PY{o}{\PYZhy{}}\PY{l+m+mi}{1}\PY{p}{)}\PY{p}{:}
                 \PY{n}{ss} \PY{o}{=} \PY{l+m+mf}{0.0}
                 \PY{k}{for} \PY{n}{j} \PY{o+ow}{in} \PY{n+nb}{range}\PY{p}{(}\PY{n}{k}\PY{o}{+}\PY{l+m+mi}{1}\PY{p}{,}\PY{n}{N}\PY{p}{)}\PY{p}{:}
                     \PY{n}{ss} \PY{o}{=} \PY{n}{ss} \PY{o}{+} \PY{n}{U}\PY{p}{[}\PY{n}{k}\PY{p}{,}\PY{n}{j}\PY{p}{]}\PY{o}{*}\PY{n}{x}\PY{p}{[}\PY{n}{j}\PY{p}{]}
                 \PY{n}{x}\PY{p}{[}\PY{n}{k}\PY{p}{]} \PY{o}{=} \PY{p}{(}\PY{n}{y}\PY{p}{[}\PY{n}{k}\PY{p}{]} \PY{o}{\PYZhy{}} \PY{n}{ss}\PY{p}{)}\PY{o}{/}\PY{n}{U}\PY{p}{[}\PY{n}{k}\PY{p}{,}\PY{n}{k}\PY{p}{]}
             
             \PY{k}{return} \PY{n}{x}
\end{Verbatim}


    Contoh penggunaan:

    \begin{Verbatim}[commandchars=\\\{\}]
{\color{incolor}In [{\color{incolor}64}]:} \PY{n}{A} \PY{o}{=} \PY{n}{np}\PY{o}{.}\PY{n}{matrix}\PY{p}{(}\PY{p}{[}
             \PY{p}{[}\PY{l+m+mi}{2}\PY{p}{,} \PY{o}{\PYZhy{}}\PY{l+m+mi}{2}\PY{p}{,} \PY{l+m+mi}{6}\PY{p}{]}\PY{p}{,}
             \PY{p}{[}\PY{o}{\PYZhy{}}\PY{l+m+mi}{2}\PY{p}{,} \PY{l+m+mi}{4}\PY{p}{,} \PY{l+m+mi}{3}\PY{p}{]}\PY{p}{,}
             \PY{p}{[}\PY{o}{\PYZhy{}}\PY{l+m+mi}{1}\PY{p}{,} \PY{l+m+mi}{8}\PY{p}{,} \PY{l+m+mi}{4}\PY{p}{]}
         \PY{p}{]}\PY{p}{,} \PY{n}{dtype}\PY{o}{=}\PY{n}{np}\PY{o}{.}\PY{n}{float64}\PY{p}{)}
         \PY{n}{b} \PY{o}{=} \PY{n}{np}\PY{o}{.}\PY{n}{matrix}\PY{p}{(}\PY{p}{[}\PY{l+m+mi}{16}\PY{p}{,} \PY{l+m+mi}{0}\PY{p}{,} \PY{o}{\PYZhy{}}\PY{l+m+mi}{1}\PY{p}{]}\PY{p}{)}\PY{o}{.}\PY{n}{transpose}\PY{p}{(}\PY{p}{)}
         \PY{n}{L}\PY{p}{,} \PY{n}{U}\PY{p}{,} \PY{n}{iperm} \PY{o}{=} \PY{n}{LU\PYZus{}decomp\PYZus{}pivot}\PY{p}{(}\PY{n}{A}\PY{p}{)}
         \PY{n+nb}{print}\PY{p}{(}\PY{l+s+s2}{\PYZdq{}}\PY{l+s+s2}{L = }\PY{l+s+se}{\PYZbs{}n}\PY{l+s+s2}{\PYZdq{}}\PY{p}{,} \PY{n}{L}\PY{p}{)}
         \PY{n+nb}{print}\PY{p}{(}\PY{l+s+s2}{\PYZdq{}}\PY{l+s+s2}{U = }\PY{l+s+se}{\PYZbs{}n}\PY{l+s+s2}{\PYZdq{}}\PY{p}{,} \PY{n}{U}\PY{p}{)}
         \PY{n}{x} \PY{o}{=} \PY{n}{LU\PYZus{}solve\PYZus{}pivot}\PY{p}{(}\PY{n}{L}\PY{p}{,} \PY{n}{U}\PY{p}{,} \PY{n}{iperm}\PY{p}{,} \PY{n}{b}\PY{p}{)}
         \PY{n+nb}{print}\PY{p}{(}\PY{l+s+s2}{\PYZdq{}}\PY{l+s+s2}{Solusi x = }\PY{l+s+se}{\PYZbs{}n}\PY{l+s+s2}{\PYZdq{}}\PY{p}{,} \PY{n}{x}\PY{p}{)}
         \PY{n+nb}{print}\PY{p}{(}\PY{l+s+s2}{\PYZdq{}}\PY{l+s+s2}{Cek solusi A*x \PYZhy{} b =}\PY{l+s+se}{\PYZbs{}n}\PY{l+s+s2}{\PYZdq{}}\PY{p}{,} \PY{n}{A}\PY{o}{*}\PY{n}{x} \PY{o}{\PYZhy{}} \PY{n}{b}\PY{p}{)}
\end{Verbatim}


    \begin{Verbatim}[commandchars=\\\{\}]
INFO: tukar baris 1 dengan 0
INFO: tukar baris 2 dengan 1
L = 
 [[ 1.          0.          0.        ]
 [ 0.5         1.          0.        ]
 [-1.          0.33333333  1.        ]]
U = 
 [[-2.          4.          3.        ]
 [ 0.          6.          2.5       ]
 [ 0.          0.          8.16666667]]
Solusi x = 
 [[ 1.]
 [-1.]
 [ 2.]]
Cek solusi A*x - b =
 [[0.]
 [0.]
 [0.]]

    \end{Verbatim}

    Contoh lain:

    \begin{Verbatim}[commandchars=\\\{\}]
{\color{incolor}In [{\color{incolor}65}]:} \PY{n}{A} \PY{o}{=} \PY{n}{np}\PY{o}{.}\PY{n}{matrix}\PY{p}{(}\PY{p}{[}
             \PY{p}{[}\PY{l+m+mi}{0}\PY{p}{,} \PY{l+m+mi}{2}\PY{p}{,} \PY{l+m+mi}{5}\PY{p}{,} \PY{o}{\PYZhy{}}\PY{l+m+mi}{1}\PY{p}{]}\PY{p}{,}
             \PY{p}{[}\PY{l+m+mi}{2}\PY{p}{,} \PY{l+m+mi}{1}\PY{p}{,} \PY{l+m+mi}{3}\PY{p}{,} \PY{l+m+mi}{0}\PY{p}{]}\PY{p}{,}
             \PY{p}{[}\PY{o}{\PYZhy{}}\PY{l+m+mi}{2}\PY{p}{,} \PY{o}{\PYZhy{}}\PY{l+m+mi}{1}\PY{p}{,} \PY{l+m+mi}{3}\PY{p}{,} \PY{l+m+mi}{1}\PY{p}{]}\PY{p}{,}
             \PY{p}{[}\PY{l+m+mi}{3}\PY{p}{,} \PY{l+m+mi}{3}\PY{p}{,} \PY{o}{\PYZhy{}}\PY{l+m+mi}{1}\PY{p}{,} \PY{l+m+mi}{2}\PY{p}{]}
         \PY{p}{]}\PY{p}{,} \PY{n}{dtype}\PY{o}{=}\PY{n}{np}\PY{o}{.}\PY{n}{float64}\PY{p}{)}
         \PY{n}{b} \PY{o}{=} \PY{n}{np}\PY{o}{.}\PY{n}{matrix}\PY{p}{(}\PY{p}{[}\PY{o}{\PYZhy{}}\PY{l+m+mi}{3}\PY{p}{,} \PY{l+m+mi}{3}\PY{p}{,} \PY{o}{\PYZhy{}}\PY{l+m+mi}{2}\PY{p}{,} \PY{l+m+mi}{5}\PY{p}{]}\PY{p}{)}\PY{o}{.}\PY{n}{transpose}\PY{p}{(}\PY{p}{)}
         \PY{n}{L}\PY{p}{,} \PY{n}{U}\PY{p}{,} \PY{n}{iperm} \PY{o}{=} \PY{n}{LU\PYZus{}decomp\PYZus{}pivot}\PY{p}{(}\PY{n}{A}\PY{p}{)}
         \PY{n+nb}{print}\PY{p}{(}\PY{l+s+s2}{\PYZdq{}}\PY{l+s+s2}{L = }\PY{l+s+se}{\PYZbs{}n}\PY{l+s+s2}{\PYZdq{}}\PY{p}{,} \PY{n}{L}\PY{p}{)}
         \PY{n+nb}{print}\PY{p}{(}\PY{l+s+s2}{\PYZdq{}}\PY{l+s+s2}{U = }\PY{l+s+se}{\PYZbs{}n}\PY{l+s+s2}{\PYZdq{}}\PY{p}{,} \PY{n}{U}\PY{p}{)}
         \PY{n}{x} \PY{o}{=} \PY{n}{LU\PYZus{}solve\PYZus{}pivot}\PY{p}{(}\PY{n}{L}\PY{p}{,} \PY{n}{U}\PY{p}{,} \PY{n}{iperm}\PY{p}{,} \PY{n}{b}\PY{p}{)}
         \PY{n+nb}{print}\PY{p}{(}\PY{l+s+s2}{\PYZdq{}}\PY{l+s+s2}{Solusi x = }\PY{l+s+se}{\PYZbs{}n}\PY{l+s+s2}{\PYZdq{}}\PY{p}{,} \PY{n}{x}\PY{p}{)}
         \PY{n+nb}{print}\PY{p}{(}\PY{l+s+s2}{\PYZdq{}}\PY{l+s+s2}{Cek solusi A*x \PYZhy{} b =}\PY{l+s+se}{\PYZbs{}n}\PY{l+s+s2}{\PYZdq{}}\PY{p}{,} \PY{n}{A}\PY{o}{*}\PY{n}{x} \PY{o}{\PYZhy{}} \PY{n}{b}\PY{p}{)}
\end{Verbatim}


    \begin{Verbatim}[commandchars=\\\{\}]
INFO: tukar baris 3 dengan 0
INFO: tukar baris 3 dengan 1
INFO: tukar baris 3 dengan 2
L = 
 [[ 1.          0.          0.          0.        ]
 [ 0.          1.          0.          0.        ]
 [ 0.66666667 -0.5         1.          0.        ]
 [-0.66666667  0.5        -0.02702703  1.        ]]
U = 
 [[ 3.          3.         -1.          2.        ]
 [ 0.          2.          5.         -1.        ]
 [ 0.          0.          6.16666667 -1.83333333]
 [ 0.          0.          0.          2.78378378]]
Solusi x = 
 [[ 2.00000000e+00]
 [-1.00000000e+00]
 [ 3.60072332e-17]
 [ 1.00000000e+00]]
Cek solusi A*x - b =
 [[0.]
 [0.]
 [0.]
 [0.]]

    \end{Verbatim}

    \hypertarget{menggunakan-pustaka-pada-python}{%
\subsection{Menggunakan pustaka pada
Python}\label{menggunakan-pustaka-pada-python}}

    Untuk berbagai aplikasi pada sains dan teknik, kita biasanya
menyelesaikan sistem persamaan linear yang ditemui dengan menggunakan
berbagai macam pustaka yang sudah tersedia.

Python sudah memiliki beberapa fungsi yang terkait dengan sistem
persamaan linear dan operasi terkait seperti menghitung determinan dan
invers matriks.

    Fungsi \texttt{np.linalg.solve} dapat digunakan untuk menyelesaikan
sistem persamaan linear:

    \begin{Verbatim}[commandchars=\\\{\}]
{\color{incolor}In [{\color{incolor}66}]:} \PY{n}{A} \PY{o}{=} \PY{n}{np}\PY{o}{.}\PY{n}{matrix}\PY{p}{(}\PY{p}{[}
             \PY{p}{[}\PY{l+m+mi}{1}\PY{p}{,} \PY{l+m+mi}{1}\PY{p}{,} \PY{l+m+mi}{1}\PY{p}{]}\PY{p}{,}
             \PY{p}{[}\PY{l+m+mi}{2}\PY{p}{,} \PY{l+m+mi}{3}\PY{p}{,} \PY{o}{\PYZhy{}}\PY{l+m+mi}{1}\PY{p}{]}\PY{p}{,}
             \PY{p}{[}\PY{l+m+mi}{1}\PY{p}{,} \PY{o}{\PYZhy{}}\PY{l+m+mi}{1}\PY{p}{,} \PY{o}{\PYZhy{}}\PY{l+m+mi}{1}\PY{p}{]}
         \PY{p}{]}\PY{p}{)}
         \PY{n}{B} \PY{o}{=} \PY{n}{np}\PY{o}{.}\PY{n}{matrix}\PY{p}{(}\PY{p}{[}\PY{l+m+mi}{4}\PY{p}{,} \PY{l+m+mi}{9}\PY{p}{,} \PY{l+m+mi}{2}\PY{p}{]}\PY{p}{)}\PY{o}{.}\PY{n}{transpose}\PY{p}{(}\PY{p}{)}
\end{Verbatim}


    \begin{Verbatim}[commandchars=\\\{\}]
{\color{incolor}In [{\color{incolor}67}]:} \PY{n}{x} \PY{o}{=} \PY{n}{np}\PY{o}{.}\PY{n}{linalg}\PY{o}{.}\PY{n}{solve}\PY{p}{(}\PY{n}{A}\PY{p}{,}\PY{n}{B}\PY{p}{)}
         \PY{n}{x}
\end{Verbatim}


\begin{Verbatim}[commandchars=\\\{\}]
{\color{outcolor}Out[{\color{outcolor}67}]:} matrix([[3.00000000e+00],
                 [1.00000000e+00],
                 [1.73472348e-17]])
\end{Verbatim}
            
    \begin{Verbatim}[commandchars=\\\{\}]
{\color{incolor}In [{\color{incolor}68}]:} \PY{n}{A}\PY{o}{*}\PY{n}{x} \PY{o}{\PYZhy{}} \PY{n}{B}
\end{Verbatim}


\begin{Verbatim}[commandchars=\\\{\}]
{\color{outcolor}Out[{\color{outcolor}68}]:} matrix([[0.],
                 [0.],
                 [0.]])
\end{Verbatim}
            
    Fungsi \texttt{np.linalg.det} dapat digunakan untuk menghitung
determinan dari suatu matriks

    \begin{Verbatim}[commandchars=\\\{\}]
{\color{incolor}In [{\color{incolor}69}]:} \PY{n}{np}\PY{o}{.}\PY{n}{linalg}\PY{o}{.}\PY{n}{det}\PY{p}{(}\PY{n}{A}\PY{p}{)}
\end{Verbatim}


\begin{Verbatim}[commandchars=\\\{\}]
{\color{outcolor}Out[{\color{outcolor}69}]:} -8.000000000000002
\end{Verbatim}
            
    Fungsi \texttt{np.linalg.inv} dapat digunakan untuk menghitung invers
dari suatu matriks

    \begin{Verbatim}[commandchars=\\\{\}]
{\color{incolor}In [{\color{incolor}70}]:} \PY{n}{np}\PY{o}{.}\PY{n}{linalg}\PY{o}{.}\PY{n}{inv}\PY{p}{(}\PY{n}{A}\PY{p}{)}
\end{Verbatim}


\begin{Verbatim}[commandchars=\\\{\}]
{\color{outcolor}Out[{\color{outcolor}70}]:} matrix([[ 0.5  ,  0.   ,  0.5  ],
                 [-0.125,  0.25 , -0.375],
                 [ 0.625, -0.25 , -0.125]])
\end{Verbatim}
            

    % Add a bibliography block to the postdoc
    
    
    
    \end{document}
