\section{Chapra Contoh 5.3}

Kita ingin mencari nilai dari $c$ dari persamaan
\begin{equation*}
v(t) = \frac{gm}{c}(1 - e^{-(c/m)t})
\end{equation*}
sehingga untuk $v(t=10) = 40$, dengan $m=9.81$ dan $m=68.1$.
Nilai $c$ dapat dicari sebagai akar dari persamaan:
\begin{equation*}
f(c) \equiv \frac{gm}{c}(1 - e^{-(c/m)t}) - v(t)
\end{equation*}

Kita akan menggunakan metode \textit{bisection} untuk mengaproksimasi akar atau solusi dari
persamaan $f(c) = 0$. Diberikan dua nilai $x_{l}$ (lower) dan $x_{u}$ (upper), di mana
$x_{u} > x_{l}$ dan $f(c=x_{l})f(c=x_{u}) < 0$, metode \textit{bisection} memberikan
aproksimasi akar $x_{r}$ sebagai berikut:
\begin{equation}
x_{r} = \frac{x_{l} + x_{u}}{2}
\end{equation}

Kode Python berikut ini mengilustrasikan penggunaan 
\begin{pythoncode}
import numpy as np

m = 68.1 # mass, kg
v = 40.0 # velocity, m/s
t = 10.0 # time, s
g = 9.81
    
def f(c):
    return ... # lengkapi
    
x_true = 14.8011 # from the text
    
# Initial guess
xl = 12.0
xu = 16.0
    
# First iteration
print("\n1st iteration: ")
print("xl = %f, xu = %f" % (xl, xu))
print("f(xl) = %f, f(xu) = %f" % (f(xl), f(xu)))
xr = # ... lengkapi
print("xr = ", xr)
ε_t = abs(xr - x_true)/x_true*100 # error in percent
print("ε_t = %.1f %%" % ε_t)

# Determine new xr should replace xu or xl (make new interval)
if f(xl)*f(xr) < 0:
    xu = xr
else:
    xl = xr
    
# Second iteration
print("\n2nd iteration: ")
print("xl = %f, xu = %f" % (xl, xu))
print("f(xl) = %f, f(xu) = %f" % (f(xl), f(xu)))
xr = (xl + xu)/2
print("xr = ", xr)
ε_t = abs(xr - x_true)/x_true*100 # error in percent
print("ε_t = %.1f %%" % ε_t)

if f(xl)*f(xr) < 0:
    xu = xr
else:
    xl = xr
    
# Third iteration
# ... teruskan jika diperlukan

# Jika xr merupakan akar dari f, maka f(xr) harus mendekati 0
# Tampilkan hasil dari f(xr) di sini
\end{pythoncode}

\begin{soal}
Lengkapi kode untuk ilustrasi penggunakan bisection tersebut. Lakukan iterasi sampai
suatu kriteria tertentu yang Anda tentukan. Anda boleh menggunakan loop
untuk menghindari pengulangan kode.
\end{soal}

\section{Chapra Contoh 5.5}
Dengan menggunakan metode \textit{regula falsi} aproksimasi akar diberikan oleh:
\begin{equation}
x_{r} = x_{u} - \frac{f(x_u)(x_{l} - x_{u})}{f(x_l) - f(x_u)}
\end{equation}

\begin{soal}
Lakukan modifikasi pada kode yang diberikan di soal sebelumnya sehingga dapat
memberikan ilustrasi penggunaan metode \textit{regula falsi}. Bandingkan hasil yang
Anda dapatkan dengan metode \textit{bisection}.
\end{soal}
