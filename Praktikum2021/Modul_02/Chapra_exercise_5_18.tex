\subsection{Chapra Latihan 5.18}

Konsentrasi jenuh dari oksigen yang terlarut dalam air dapat
dihitung dengan menggunakan persamaan:
\begin{equation*}
\mathrm{ln}o_{\mathrm{sf}} = C_{0} + \frac{C_{1}}{T_{a}} +
\frac{C_{2}}{T_{a}^{2}} + \frac{C_{3}}{T_{a}^{3}} +
\frac{C_{4}}{T_{a}^{4}}
\end{equation*}
di mana $o_{\mathrm{sf}}$ adalah konsentrasi oksigen dalam mg/L pada tekanan
1 atm, $T_{a}$ adalah temperatur absolut, $T_{a} = T + 273.15$, $T$ dalam
derajat Celcius, dan parameter:
\begin{align*}
C_{0} & = -139.34411 \\
C_{1} & = 1.575701 \times 10^5 \\
C_{2} & = -6.642308 \times 10^7 \\
C_{3} & = 1.243800 \times 10^{10} \\
C_{4} & = -8.621949 \times 10^{11} \\
\end{align*}