\documentclass[a4paper,11pt,bahasa]{article} % screen setting

\usepackage[a4paper]{geometry}
\geometry{verbose,tmargin=1.5cm,bmargin=1.5cm,lmargin=1.5cm,rmargin=1.5cm}

\setlength{\parskip}{\smallskipamount}
\setlength{\parindent}{0pt}

%\usepackage{cmbright}
%\renewcommand{\familydefault}{\sfdefault}

%\usepackage{fontspec}
\usepackage[libertine]{newtxmath}
\usepackage[no-math]{fontspec}
\setmainfont{Linux Libertine O}
\setmonofont{DejaVu Sans Mono}
%\setmonofont{JuliaMono-Regular}


\usepackage{hyperref}
\usepackage{url}
\usepackage{xcolor}

\usepackage{amsmath}
\usepackage{amssymb}

\usepackage{graphicx}
\usepackage{float}

\usepackage{minted}

\newminted{julia}{breaklines,fontsize=\footnotesize}
\newminted{python}{breaklines,fontsize=\footnotesize}

\newminted{bash}{breaklines,fontsize=\footnotesize}
\newminted{text}{breaklines,fontsize=\footnotesize}

\newcommand{\txtinline}[1]{\mintinline[breaklines,fontsize=\footnotesize]{text}{#1}}
\newcommand{\jlinline}[1]{\mintinline[breaklines,fontsize=\footnotesize]{julia}{#1}}
\newcommand{\pyinline}[1]{\mintinline[breaklines,fontsize=\footnotesize]{python}{#1}}

\newmintedfile[juliafile]{julia}{breaklines,fontsize=\footnotesize}
\newmintedfile[pythonfile]{python}{breaklines,fontsize=\footnotesize}

\definecolor{mintedbg}{rgb}{0.90,0.90,0.90}
\usepackage{mdframed}
\BeforeBeginEnvironment{minted}{
    \begin{mdframed}[backgroundcolor=mintedbg,%
        topline=false,bottomline=false,%
        leftline=false,rightline=false]
}
\AfterEndEnvironment{minted}{\end{mdframed}}


\usepackage{setspace}

\onehalfspacing

\usepackage{appendix}


\newcommand{\highlighteq}[1]{\colorbox{blue!25}{$\displaystyle#1$}}
\newcommand{\highlight}[1]{\colorbox{red!25}{#1}}



\begin{document}

\title{%
{\small TF2202 Komputasi Rekayasa}\\
Pengenalan Python untuk Komputasi Rekayasa
}
\author{Tim Praktikum Komputasi Rekayasa 2021\\
Teknik Fisika\\
Institut Teknologi Bandung}
\date{}
\maketitle

Kerjakan pada Jupyter Notebook (mengunakan JupyterLab atau Google Colab).

\section{Review Python}

\subsection{Interaksi dengan Python}

\begin{itemize}
\item Bagaimana cara Anda menjalankan program berikut ini di komputer/laptop Anda
(pilih salah satu jika tidak ada)
  \begin{itemize}
  \item Python interpreter (console) Python
  \item Jupyter notebook atau JupyterLab dan 
  \item Python IDE (integrated development environment): Spyder
  \end{itemize}
\item Bagaimana cara membuat script Python (ekstensi \txtinline{.py}) pada editor
atau IDE dan menjalankannya pada komputer Anda.
\end{itemize}


\section{Dasar pemrograman Python}
\begin{itemize}
\item Jelaskan mengenai beberapa tipe data dasar pada Python:
  \begin{itemize}
  \item numerik (\txtinline{int} dan \txtinline{float})
  \item string
  \item bool
  \end{itemize}
\item Jelaskan juga mengenai beberapa operator yang bekerja pada tipe data dasar tersebut.
\end{itemize}
Tipe data kumpulan pada Python:

Beberapa sumber Internet yang dapat Anda pelajari.
\begin{itemize}
\item {\footnotesize \url{https://docs.python.org/3/tutorial/}}
\item {\footnotesize \url{https://learnxinyminutes.com/docs/python/}}
\end{itemize}


\section{Numpy}

\begin{itemize}
\item Tipe data numerik pada Numpy: float64, float32, float16, float128
\item f
\end{itemize}


Beberapa sumber Internet yang dapat Anda pelajari:
\begin{itemize}
\item {\footnotesize \url{https://cs231n.github.io/python-numpy-tutorial/}}
\item {\footnotesize \url{https://github.com/rougier/numpy-100/blob/master/100_Numpy_exercises_with_hints_with_solutions.md}}
\end{itemize}

\section{Matplotlib}


\subsection{Plot fungsi sinusoidal}

Jalankan script Python berikut:

\subsection{Ekpansi dengan deret Fourier}

Buatlah plot dari fungsi-fungsi berikut dengan memvariasikan nilai $N$, misalnya
$N = 2, 5, 10, 100$. Plot Anda harus menunjukkan fungsi yang sama dengan nilai $N$
yang berbeda.
Silakan gunakan parameter $L$ yang Anda tentukan sendiri. Coba plot pada rentang
$x=0$ dan $x=L$
Contoh (gelombang kotak):
{\center
\includegraphics[scale=0.75]{codes/IMG_gel_kotak_v01.pdf}
\par}

\begin{itemize}
%
\item Gelombang kotak:
\begin{equation*}
f(x) = \frac{4}{\pi} \sum_{n=1,3,5,\ldots}^{N} \frac{1}{n}\sin\left(
\frac{2\pi n x}{L}
\right)
\end{equation*}
%
\item Gelombang gergaji:
\begin{equation*}
f(x) = \frac{1}{2} - \frac{1}{\pi} \sum_{n=1}^{N} \frac{1}{n}\sin\left(
\frac{2 \pi n  x}{L} \right)
\end{equation*}
%
\item Gelombang segitiga:
\begin{equation*}
f(x) = \sum_{n=1,3,5,\ldots}^{N} \frac{(-1)^{(n-1)/2}}{n^2}
\sin\left( \frac{2\pi n x}{L} \right)
\end{equation*}
%
\end{itemize}


\end{document}
