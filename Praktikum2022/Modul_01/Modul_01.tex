\documentclass[a4paper,12pt,bahasa]{extarticle} % screen setting
\usepackage[a4paper]{geometry}

%\documentclass[b5paper,11pt,bahasa]{article} % screen setting
%\usepackage[b5paper]{geometry}

\geometry{verbose,tmargin=1.5cm,bmargin=1.5cm,lmargin=1.5cm,rmargin=1.5cm}

\setlength{\parskip}{\smallskipamount}
\setlength{\parindent}{0pt}

%\usepackage{cmbright}
%\renewcommand{\familydefault}{\sfdefault}

\usepackage{amsmath}
\usepackage{amssymb}

\usepackage[libertine]{newtxmath}
\usepackage[no-math]{fontspec}
\setmainfont{Linux Libertine O}
\setmonofont{JuliaMono-Regular}


\usepackage{hyperref}
\usepackage{url}
\usepackage{xcolor}

\usepackage{graphicx}
\usepackage{float}

\usepackage{minted}

\newminted{julia}{breaklines,fontsize=\footnotesize,linenos}
\newminted{python}{breaklines,fontsize=\footnotesize,linenos}

\newminted{bash}{breaklines,fontsize=\footnotesize}
\newminted{text}{breaklines,fontsize=\footnotesize}

\newcommand{\txtinline}[1]{\mintinline[breaklines,fontsize=\footnotesize]{text}{#1}}
\newcommand{\jlinline}[1]{\mintinline[breaklines,fontsize=\footnotesize]{julia}{#1}}
\newcommand{\pyinline}[1]{\mintinline[breaklines,fontsize=\footnotesize]{python}{#1}}

\newmintedfile[juliafile]{julia}{breaklines,fontsize=\footnotesize}
\newmintedfile[pythonfile]{python}{breaklines,fontsize=\footnotesize}
% f-o-otnotesize

\usepackage{mdframed}
\usepackage{setspace}
\onehalfspacing

\usepackage{appendix}

\newcommand{\highlighteq}[1]{\colorbox{blue!25}{$\displaystyle#1$}}
\newcommand{\highlight}[1]{\colorbox{red!25}{#1}}

\newcounter{soal}%[section]
\newenvironment{soal}[1][]{\refstepcounter{soal}\par\medskip
   \noindent \textbf{Soal~\thesoal. #1} \sffamily}{\medskip}


\definecolor{mintedbg}{rgb}{0.95,0.95,0.95}
\BeforeBeginEnvironment{minted}{
    \begin{mdframed}[backgroundcolor=mintedbg,%
        topline=false,bottomline=false,%
        leftline=false,rightline=false]
}
\AfterEndEnvironment{minted}{\end{mdframed}}


\BeforeBeginEnvironment{soal}{
    \begin{mdframed}[%
        topline=true,bottomline=true,%
        leftline=true,rightline=true]
}
\AfterEndEnvironment{soal}{\end{mdframed}}

% -------------------------
\begin{document}

\title{%
{\small TF2202 Komputasi Rekayasa}\\
Aproksimasi, Kesalahan, dan Deret Taylor
}
\author{Tim Praktikum Komputasi Rekayasa 2022\\
Teknik Fisika\\
Institut Teknologi Bandung}
\date{}
\maketitle

\tableofcontents

\section{Contoh}
\subsection{Aproksimasi dengan deret Taylor}

\textbf{Chapra Contoh 3.2}

Hubungan antara nilai eksak atau benar dan aproksimasi dapat dinyatakan
sebagai:
\begin{equation*}
\text{nilai benar} = \text{aproksimasi} + \text{error atau galat}
\end{equation*}
Dari persamaan di atas, diberikan nilai benar dan aproksimasi, kita dapat menghitung
kesalahan sebenarnya, dilambangkan dengan $E_t$ (subskrip $t$: \textit{true}):
\begin{equation}
E_t = \text{nilai benar} - \text{aproksimasi}
\end{equation}
Terkadang, kita juga dapat menggunakan nilai relatif, dengan cara membandingkannya
dengan nilai benar:
\begin{equation}
\epsilon_{t} = \frac{\text{error sebenarnya}}{\text{nilai sebenarnya}}\times 100\%
\end{equation}
yang dapat dinyatakan dalam persentase.

Dalam banyak kasus, kita tidak memiliki nilai sebenarnya sehingga $\epsilon_t$
tidak dapat dihitung. Sebagai alternatif, kita dapat menggunakan error relatif:
\begin{equation}
\epsilon_{a} = \frac{\text{error aproksimasi}}{\text{nilai aproksimasi}}
\end{equation}
Ada beberapa cara untuk mengestimasi error aproksimasi. Dalam kasus proses iteratif,
kita dapat menggunakan persamaan berikut.
\begin{equation}
\epsilon_{a} = \frac{\text{aproksimasi sekarang} - \text{aproksimasi sebelumnya}}%
{\text{aproksimasi sekarang}} \times 100\%
\end{equation}
Dalam konteks perhitungan iteratif, biasanya perhitungan dilakukan sampai:
\begin{equation}
|\epsilon_{a}| < \epsilon_{s}
\end{equation}
di mana $\epsilon_{s}$ adalah suatu nilai toleransi yang sudah ditentukan.

Scarborough memberikan suatu kriteria yang menghubungkan antara jumlah digit signifikan
dengan dan nilai aproksimasi. Menurut kriteria ini, jika menggunakan:
\begin{equation}
\epsilon_{s} = (0.5 \times 10^{2-n})\%
\end{equation}
maka hasil yang diperoleh akan benar untuk setidaknya $n$ digit signifikan.

Fungsi eksponensial dapat dihitung dengan menggunakan deret sebagai berikut:
\begin{equation}
e^{x} = 1 + x + \frac{x^2}{2} + \frac{x^3}{3!} + \cdots + \frac{x^n}{n!}
\label{eq:exp-deret}
\end{equation}
Kita ingin menggunakan Persamaan \eqref{eq:exp-deret} untuk menghitung estimasi
dari $e^{0.5}$ untuk setidaknya tiga digit signifikan.
Dengan menggunakan kriteria dari Scarborough, nilai $n=3$ diperoleh:
\begin{equation*}
\epsilon_{s} = (0.5 \times 10^{2-3}) \% = 0.05 \%
\end{equation*}
Kita akan menambahkan suku-suku pada Persamaan \eqref{eq:exp-deret} sampai $\epsilon_{a}$
lebih kecil dari $\epsilon_{s}$.

Program Python berikut ini dapat digunakan untuk melakukan perhitungan yang ada pada buku teks.
\begin{pythoncode}
from math import factorial, exp

def approx_exp(x, N):
    assert(N >= 0)
    if N == 0:
        return 1
    s = 0.0
    for i in range(N+1):
        s = s + x**i/factorial(i)
    return s
  
x = 0.5
true_val = exp(x) # from math module
  
n_digit = 3
# Equation 3.7
ε_s_percent = 0.5*10**(2-n_digit)
  
prev_approx = 0.0
for N in range(50):
    approx_val = approx_exp(x, N)
    ε_t_percent = abs(approx_val - true_val)/true_val * 100
    if N > 0:
        ε_a_percent = abs(approx_val - prev_approx)/approx_val * 100
    else:
        ε_a_percent = float('nan')
    prev_approx = approx_val
    print("%3d %18.10f %10.5f%% %10.5f%%" % (N+1, approx_val, ε_t_percent, ε_a_percent))
    if ε_a_percent < ε_s_percent:
        print("Converged within %d significant digits" % n_digit)
        break

print("true_val   is %18.10f" % true_val)
print("approx_val is %18.10f" % approx_val)
\end{pythoncode}

Catatan: Pada program di atas, \pyinline{for}-loop digunakan dengan jumlah iterasi yang relatif
besar. Anda dapat menggunakan \pyinline{while}-loop sebagai gantinya.

Contoh output:
\begin{textcode}
  1       1.0000000000   39.34693%        nan%
  2       1.5000000000    9.02040%   33.33333%
  3       1.6250000000    1.43877%    7.69231%
  4       1.6458333333    0.17516%    1.26582%
  5       1.6484375000    0.01721%    0.15798%
  6       1.6486979167    0.00142%    0.01580%
Converged within 3 significant digits
true_val   is       1.6487212707
approx_val is       1.6486979167
\end{textcode}


\begin{soal}
Ulangi perhitungan ini untuk jumlah digit signifikan yang berbeda, misalnya 5, 8, dan 10
digit signifikan. Silakan melakukan modifikasi terhadap program yang diberikan.
\end{soal}
\subsection{Chapra Contoh 3.2, \textit{single precision}}
Secara default, perhitungan dengan \textit{floating number} pada Python (dan NumPy)
dilakukan dengan menggunakan \textit{double precision}. Pada bagian ini, kita akan
mengulangi Chapra Contoh 3.2 dengan menggunakan \textit{single precision}.
Pada C dan C++, tipe yang relevan adalah \txtinline{float}
untuk \textit{single precision} dan \txtinline{double} untuk \textit{double
precision}.
Pada Fortran kita dapat menggunakan \txtinline{REAL(4)} untuk \textit{single precision}
dan \txtinline{REAL(8)} untuk \textit{double precision}.

Karena Python merupakan bahasa pemrograman dinamik yang \textit{type-loose} kita tidak dapat
dengan mudah memberikan spesifikasi pada variabel yang kita gunakan. Meskipun demikian, 
kita dapat menggunakan \textit{single precision} pada Python melalui
\txtinline{np.float32}
\footnote{\txtinline{np} digunakan sebagai
singkatan dari modul \txtinline{numpy}},
meskipun program yang dihasilkan kurang elegan. Selain itu, kita juga harus mengecek
apakah hasil akhir yang diberikan tetap berupa \textit{single precision} (tidak terjadi
\textit{type promotion} ke \textit{double precision}).

\begin{pythoncode}
from math import factorial
import numpy as np
  
def approx_exp(x, N):
    assert(N >= 0)
    if N == 0:
        return 1
    s = np.float32(0.0)
    for i in range(N+1):
        s = s + np.float32(x**i)/np.float32(factorial(i))
    return s
  
x = np.float32(0.5)
true_val = np.exp(x) # from np module
  
n_digit = 3
# Equation 3.7
ε_s_percent = np.float32(0.5)*np.float32(10**(2-n_digit))
  
prev_approx = np.float32(0.0)
for N in range(50):
    approx_val = approx_exp(x, N)
    ε_t_percent = abs(approx_val - true_val)/true_val * 100
    if N > 0:
        ε_a_percent = abs(approx_val - prev_approx)/approx_val * 100
    else:
        ε_a_percent = float('nan')
    prev_approx = approx_val
    print("%3d %18.10f %10.5f%% %10.5f%%" % (N+1, approx_val, ε_t_percent, ε_a_percent))
    if ε_a_percent < ε_s_percent:
        print("Converged within %d significant digits" % n_digit)
        break
  
print("true_val   is %18.10f" % true_val)
print("approx_val is %18.10f" % approx_val)
  
# Make sure that float32 is used
print()
print("type(true_val)   = ", type(true_val))
print("type(approx_val) = ", type(approx_val))
\end{pythoncode}

Berikut ini adalah keluaran dari program.
\begin{textcode}
  1       1.0000000000   39.34693%        nan%
  2       1.5000000000    9.02040%   33.33333%
  3       1.6250000000    1.43876%    7.69231%
  4       1.6458333731    0.17516%    1.26583%
  5       1.6484375000    0.01721%    0.15798%
  6       1.6486979723    0.00141%    0.01580%
Converged within 3 significant digits
true_val   is       1.6487212181
approx_val is       1.6486979723

type(true_val)   =  <class 'numpy.float32'>
type(approx_val) =  <class 'numpy.float32'>
\end{textcode}

\begin{soal}
Ulangi perhitungan pada Chapra Contoh 3.2 dengan menggunakan \textit{single precision}
dengan jumlah digit signifikan yang berbeda, misalnya 5, 8, dan 10 digit signifikan
(berdasarkan kriteria Scarborough). Bandingkan hasil yang Anda dapatkan jika
\textit{double precision}. Apa yang dapat Anda simpulkan?
\end{soal}

\subsection{Galat pembulatan (rounding error)}

\textbf{Chapra Contoh 3.8}

Akar-akar suatu polinomial kuadrat:
\begin{equation*}
ax^2 + bx + c = 0
\end{equation*}
diberikan oleh formula berikut
\begin{equation}
x_{1,2} = \frac{-b \pm \sqrt{b^2 - 4ac}}{2a}
\label{eq:quadeq1}
\end{equation}
Untuk kasus di mana $b^2 \gg 4ac$ perbedaan antara pembilang dapat menjadi sangat kecil.
Pada kasus tersebut, kita dapat menggunakan \textit{double precision} untuk mengurangi
kesalahan pembulatan. Selain itu, kita juga dapat menggunakan formula:
\begin{equation}
x_{1,2} = \frac{-2c}{b \pm \sqrt{b^2 - 4ac}}
\label{eq:quadeq2}
\end{equation}

Mengikuti contoh yang diberikan pada buku, kita akan menggunakan
$a = 1$, $b = 3000.001$, dan $c = 3$. Akar-akar eksaknya adalah
$x_{1} = -0.001$ dan $x_2 = -3000$.

\begin{soal}
Buat program Python dengan menggunakan \textit{single precision} dan
\textit{double precision} untuk melihat perbedaan hasil yang diberikan 
dari Persamaan \eqref{eq:quadeq1} dan Persamaan \eqref{eq:quadeq2}.

Program berikut ini adalah dalam \textit{single precision}
yang dapat Anda lengkapi. Anda juga dapat menggunakan program yang Anda tulis sendiri
dari awal atau modifikasi dari program ini.
\begin{pythoncode}
import numpy as np

def calc_quad_root_v1(a, b, c):
    D = np.float32(b**2) - np.float32(4)*a*c
    x1 = (-b + np.sqrt(D))/(np.float32(2)*a)
    x2 = # ... lengkapi
    return x1, x2
  
def calc_quad_root_v2(a, b, c):
    D = # ... lengkapi
    x1 = # ... lengkapi
    x2 = # ... lengkapi
    return x1, x2
  
a = np.float32(1.0)
b = np.float32(3000.001)
c = np.float32(3.0)
  
x1_true = np.float32(-0.001)
x2_true = np.float32(-3000.0)
  
x1, x2 = calc_quad_root_v1(a, b, c)
print("Using 1st formula: approx roots: ", x1, " ", x2)
print(type(x1), type(x2)) # pastikan x1 dan x2 merupakan np.float32
  
x1, x2 = calc_quad_root_v2(a, b, c)
print(type(x1), type(x2))
print("Using 2nd formula: approx roots: ", x1, " ", x2)  
print("True roots: ", x1_true, " ", x2_true)
\end{pythoncode}

Bandingkan akar-akar yang Anda peroleh dengan akar-akar eksak.
Untuk masing-masing akar, formula mana yang memberikan hasil yang paling dekat dengan
hasil eksak?
\end{soal}

\begin{soal}
Program berikut ini, kita akan menggunakan CAS atau \textit{computer algebra system}
untuk memastikan bahwa formula \eqref{eq:quadeq1} dan \eqref{eq:quadeq2} memberikan
hasil yang identik. Lengkapi kode berikut ini dan cek apakah hasil yang diberikan
dari kedua formula tersebut adalah sama.
\begin{pythoncode}
from sympy import *

def calc_quad_root_v1(a, b, c):
    D = b**2 - 4*a*c
    x1 = (-b + sqrt(D))/(2*a)
    x2 = (-b - sqrt(D))/(2*a)
    return x1, x2
  
def calc_quad_root_v2(a, b, c):
    D = # lengkapi ...
    x1 = # lengkapi ...
    x2 = # lengkapi ... 
    return x1, x2
  
a = Rational(1)
b = Rational(3000001, 1000)
c = Rational(3)
  
x1_true = -Rational(1, 1000)
x2_true = -3000
  
x1, x2 = calc_quad_root_v1(a, b, c)
print("Using 1st formula: appprox roots: ", x1, " ", x2)

x1, x2 = calc_quad_root_v2(a, b, c)
print("Using 2nd formula: appprox roots: ", x1, " ", x2)

print("True roots: ", x1_true, " ", x2_true)
\end{pythoncode}
Perhatikan bahwa kode di atas juga mencetak tipe dari variabel \txtinline{x1} dan
\txtinline{x2} adalah bilangan integer atau rasional.
Pada SymPy, tipe untuk integer dan rasional adalah:
\begin{textcode}
<class 'sympy.core.numbers.Integer'> <class 'sympy.core.numbers.Rational'>
\end{textcode}
Coba turunkan formula \eqref{eq:quadeq2} dari \eqref{eq:quadeq1}.
\end{soal}
\subsection{Chapra Contoh 4.2}

Deret Taylor dapat dituliskan sebagai berikut.
\begin{equation}
f(x_{i+1}) = f(x_{i}) + f'(x_{i}) h + \frac{f''(x_{i})}{2!}h^2 + 
\frac{f^{(3)}(x_{i})}{3!}h^3 + \cdots +
\frac{f^{(n)}(x_{i})}{n!}h^n + R_{n}
\end{equation}
dengan $h = x_{i+1} - x_{i}$ dan $R_{n}$ adalah suku sisa (\textit{remainder}).

Kita akan menggunakat deret Taylor dari $n=0$ sampai $n=6$ untuk mengaproksimasi
$f(x) = \cos(x)$ pada $x_{i+1} = \pi/3$ dengan nilai $f(x)$ dari turunan-turunannya
pada $x_{i} = \pi/4$, atau $h = x_{i+1} - x_{i} = \pi/12$.

Program berikut ini menggunakan kombinasi perhitungan simbolik dan numerik
dengan SymPy.
\begin{pythoncode}
from sympy import * # be very careful when using this!

init_printing(use_unicode=True)
# if you are using Jupyter Lab or Notebook, use the following line instead:
#init_printing(use_latex=True)

x = symbols("x")

f = cos(x)

xi = pi/4
xip1 = pi/3
h = xip1 - xi

# zeroth order
f_approx = diff(f, x, 0).subs({x: xi}) # or simply call cos(xi)

for n in range(1,7): # from 1 to 6
    new_term = diff(f, x, n) * h**n / factorial(n)
    f_approx = f_approx + new_term
    pprint(f_approx)
    print(N(f_approx.subs({x: xi}))) # use N to force numerical expression

f_true = N(f.subs({x: xip1}))
print("f_true = ", f_true)
print(type(f_true))
\end{pythoncode}


\begin{soal}
Lakukan modifikasi pada program di atas sehingga dapat menampilkan error atau
perbedaan antara nilai aproksimasi dan nilai benar. Program di atas juga menampilkan
deret Taylor yang digunakan secara simbolik. Anda dapat menonaktifkan baris
kode yang sesuai dengan cara menghapusnya atau menjadikannya komentar.
\end{soal}

\subsection{Chapra Contoh 4.4}
Diketahui sebuah fungsi:
\begin{equation}
f(x) = -0.1x^4 - 0.15x^3 - 0.5x^2 - 0.25x + 1.25
\end{equation}
Kita ingin menghitung pendekatan nilai turunan fungsi ini pada $x=0.5$ dengan menggunakan tiga formula.
Formula pertama adalah beda hingga maju (\textit{forward finite difference}):
\begin{equation}
f'(x_{i}) \approx \frac{f(x_{i+1}) - f(x_{i})}{x_{i+1} - x_{i}} =
\frac{f(x_{i}+h) - f(x_{i})}{h}
\end{equation}
beda hingga mundur:
\begin{equation}
f'(x_{i}) \approx \frac{f(x_{i}) - f(x_{i-1})}{x_{i} - x_{i-1}} =
\frac{f(x_{i}) - f(x_{i}-h)}{h}
\end{equation}
dan beda hingga tengah:
\begin{equation}
f'(x_{i}) \approx \frac{f(x_{i+1}) - f(x_{i-1})}{x_{i+1} - x_{i-1}} =
\frac{f(x_{i}+h) - f(x_{i}-h)}{2h}
\end{equation}

Nilai pendekatan akan dibandingkan dengan hasil evaluasi langsung dari
turunan $f(x)$:
\begin{equation}
f'(x) = -0.4x^3 - 0.45x^2 - x - 0.25
\end{equation}

\begin{soal}
Buat program Python untuk menghitung pendekatan nilai $f'(x)$ pada $x=0.5$ dengan
menggunakan $h=0.5$ dan $h=0.25$. Bandingkan hasilnya dengan nilai eksak. Formula mana yang
memberikan kesalahan paling kecil?
\end{soal}
\subsection{Chapra Contoh 4.6}

Defleksi $y$ dari bagian atas tiang kapal dapat dinyatakan
dengan
\begin{equation}
y = \frac{FL^{4}}{8EI}
\end{equation}
dengan $F$ menyatakan \textit{loading} seragam yang bekerja pada sisi
tiang (N/m), $L$ menyatakan tinggi tiang (m), dan $E$ menyatakan
modulus elastisitas tiang $\textrm{N/m}^{2}$, dan $I$ menyatakan
momen inersia ($\textrm{m}^{4}$). Kita ingin menghitung estimasi kesalahan
pada $y$ dengan menggunakan data-data berikut.
\begin{itemize}
\item $\tilde{F} = 750\,\textrm{N/m}$ dan $\Delta\tilde{F} = 30\,\textrm{N/m}$
\item $\tilde{L} = 9\,\textrm{m}$ dan $\Delta\tilde{L} = 0.03\,\textrm{m}$
\item $\tilde{E} = 7.5\times10^{9}\,\textrm{N/m}^{2}$ dan $\Delta\tilde{E} = 5\times10^{7}\,\textrm{N/m}^{2}$
\item $\tilde{I} = 0.0005\,\textrm{m}^{4}$ dan $\Delta\tilde{I} = 0.000005\,\textrm{m}^{4}$
\end{itemize}

Kita akan menggunakan hasil dari analisis error orde-1 (Persamaan 4.27 pada Chapra), kita
mendapatkan
\begin{equation*}
\Delta y(\tilde{F},\tilde{L},\tilde{E},\tilde{I}) =
\left| \frac{\partial y}{\partial F} \right| \Delta\tilde{F} +
\left| \frac{\partial y}{\partial L} \right| \Delta\tilde{L} +
\left| \frac{\partial y}{\partial E} \right| \Delta\tilde{E} +
\left| \frac{\partial y}{\partial I} \right| \Delta\tilde{I}
\end{equation*}

Kita akan menggunakan SymPy untuk melakukan perhitungan ini.

\begin{pythoncode}
from sympy import *

F, L, E, I = symbols("F L E I")
y = # ... lengkapi, tulis formula untuk menghitung y di sini
    
F_num = 750; ΔF = 30 # N/m
L_num = 9; ΔL = 0.03 # m
E_num = 7.5e9; ΔE = 5e7 # N/m^2
I_num = 0.0005; ΔI = 0.000005 # m^4

# Hitung nilai y dengan nilai numerik
dict_subs = {F: F_num, L: L_num, E: E_num, I: I_num}
y_num = y.subs(dict_subs)
print("y = ", y_num)

# Hitung nilai Δy dengan Pers. 4.27
Δy = abs(diff(y,F))*ΔF + ... # lengkapi
pprint(Δy) # bentuk simbolik
print("Δy = ", Δy.subs(dict_subs)) # evaluasi/substitusi nilai numerik

# Hitung nilai ekstrim

dict_subs = {
  F: F_num - ΔF, L: L_num - ΔL,
  E: E_num + ΔE, I: I_num + ΔI  # mengapa seperti ini?
} 
ymin = y.subs(dict_subs)
print("ymin = ", ymin)
    
dict_subs = .... # lengkapi 
ymax = y.subs(dict_subs)
print("ymax = ", ymax)    
\end{pythoncode}

\begin{soal}
Lengkapi dan/atau modifikasi kode di atas sehingga dapat menampilkan $\Delta y$
seperti yang dijelaskan pada Chapra Contoh 4.6.
\end{soal}
\subsection{Chapra Contoh 4.8}
Diketahui sebuah fungsi:
\begin{equation}
f(x) = -0.1x^4 - 0.15x^3 - 0.5x^2 - 0.25x + 1.2
\end{equation}
Turunan pertama dari $f(x)$ adalah:
\begin{equation}
f'(x) = -0.4x^3 - 0.45x^2 - x - 0.25
\end{equation}
Kita akan menghitung pendekatan fungsi ini pada $x=0.5$ dengan menggunakan
formula beda hingga tengah. Kita akan mulai dari $h=1$, kemudian secara bertahap
memperkecil nilai $h$ dengan faktor 10 untuk mempelajari bagaimana
pengaruh nilai $h$ terhadap kesalahan.

Berikut ini adalah program yang belum lengkap (versi \textit{single precision}).
\begin{pythoncode}
import numpy as np
  
def f(x):
    return -np.float32(0.1)*x**np.float32(4) - \
      np.float32(0.15)*x**np.float32(3) - \
      np.float32(0.5)*x**np.float32(2) - \
      np.float32(0.25)*x + np.float32(1.2)
    
def df(x):
    return -np.float32(0.4)*x**np.float32(3) - \
      np.float32(0.45)*x**np.float32(2) - \
      x - np.float32(0.25)
    
def centered_diff(f, x, h):
    return ( .... )/(np.float32(2)*h) # isi titik-titik
  
    x = np.float32(0.5)
    h = np.float32(1.0)
    true_val = # ... lengkapi
  
    print("--------------------------------------------------------")
    print("           h             approx_val             error")
    print("--------------------------------------------------------")
    
for i in range(11):
    approx_val = # ... lengkapi
    εt = abs(approx_val - true_val)
    print("%18.10f %18.14f %18.13f" % (h, approx_val, εt))
    h = h/np.float32(10)

print(type(h))
print(type(centered_diff(f,x,h)))  
\end{pythoncode}

\begin{soal}
Lengkapi program di atas.
Coba juga untuk versi \textit{double precision} (default pada NumPy
atau \txtinline{np.float64}).
Apakah error yang Anda peroleh semakin mendekati nol apabila nilai $h$ semakin
diperkecil?
NumPy juga menyediakan tipe bilangan \textit{quadruple precision}, yang lebih
\textit{precise} dari \textit{double precision}, yaitu
\txtinline{np.float128}. Coba ulangi perhitungan dan analisis Anda dengan
menggunakan \txtinline{np.float128}.
Berikan penjelasan mengenai hasil yang Anda dapatkan.
\end{soal}

\section{Soal tambahan}

\begin{soal}
\textbf{Chapra Latihan 3.5}
Deret tak-hingga:
\begin{equation*}
f(n) = \sum_{i=1}^{n} \frac{1}{i^4}
\end{equation*}
konvergen ke nilai $\pi^{4}/90$ untuk $n$ mendekati tak-hingga
Untuk kasus $n \rightarrow \infty$ nilai ini merupakan nilai dari fungsi
zeta Riemannn $\zeta(4)$
\footnote{{\scriptsize\url{https://en.wikipedia.org/wiki/Riemann_zeta_function\#Specific_values}}}.
Tulis program Python menggunakan \txtinline{np.float32} atau
\textit{single precision} untuk menghitung
nilai $f(n)$ dengan $n=10000$.
Pastikan bahwa tipe numerik yang Anda gunakan pada program Python
adalah \textit{single precision} dengan cara menampilkan tipe dari hasil akhir yang didapatkan.
Lakukan penjumlahan dengan menggunakan perulangan
dari $i=1$ ke $10000$. Ulangi perhitungan dengan menggunakan perulangan
dengan urutan kebalikan yaitu dari $i=10000$ ke $i=1$. Hitung kesalahan untuk
kedua kasus tersebut.
Jelaskan hasil yang Anda dapatkan.
Ulangi perhitungan Anda dengan menggunakan \textit{double precision}.
\end{soal}

\begin{soal}
\textbf{Chapra Latihan 3.6}
Evaluasi aproksimasi nilai dari $e^{-5}$ dengan menggunakan dua formula:
\begin{equation}
e^{-x} = 1 - x + \frac{x^2}{2!} - \frac{x^3}{3!} + \cdots
\end{equation}
dan
\begin{equation}
e^{-x} = \frac{1}{e^{x}} = \frac{1}{1 + x + \dfrac{x^2}{2!} + \dfrac{x^3}{3!} + \cdots}
\end{equation}
Bandingkan nilai yang diperoleh dengan mengunakan fungsi \textit{built-in} dari modul
\txtinline{math} atau \txtinline{numpy}. Gunakan 20 suku untuk masing-masing formula.
Hitung juga kesalahan sebenarnya dan kesalahan relatif dari aproksimasi yang dibuat.
\end{soal}


\begin{soal}
\textbf{Chapra Latihan 3.11}
Gunakan formula \eqref{eq:quadeq1} dan \eqref{eq:quadeq2} untuk mencari akar-akar
pada persamaan kuadrat:
\begin{equation*}
x^2 - 5000.002x + 10
\end{equation*}
Lakukan dalam \textit{single precision} dan \textit{double precision}.
\end{soal}


\begin{soal}
\textbf{Chapra Latihan 3.13}
Metode "bagi dan rata-rata"
\footnote{{\scriptsize\url{https://en.wikipedia.org/wiki/Methods_of_computing_square_roots\#Babylonian_method}}}
adalah metode yang dapat digunakan untuk
mengaproksimasi akar kuadrat dari suatu bilangan positif $a$ dengan input
tebakan awal $x^{0}$. 
Dalam bentuk iterasi, metode ini dapat dituliskan sebagai:
\begin{equation}
x^{i+1} = \dfrac{x^{i} + \dfrac{a}{x^{i}}}{2}
\end{equation}
di mana $x^{i}$ menyatakan nilai aproksimasi akar pada iterasi ke-$i$.
Buatlah program Python untuk mengimplementasikan metode ini.
Hitung kesalahan relatif dan sebenarnya (dengan referensi hasil fungsi
\textit{built-in} \txtinline{sqrt} pada Python).
\end{soal}

\begin{soal}
\textbf{Chapra Latihan 4.2}
Deret Maclaurin untuk $\cos(x)$ adalah
\begin{equation*}
\cos(x) = 1 - \frac{x^2}{2!} + \frac{x^4}{4!} - \frac{x^6}{6!} + \frac{x^8}{8!} - \cdots
\end{equation*}
Buatlah program Python untuk menghitung aproksimasi dari $\cos(\pi/3)$ menggunakan deret
Maclaurin tersebut. Hitung juga kesalahan untuk tiap penambahan suku baru dengan referensi
nilai yang diberikan oleh fungsi \textit{built-in} pada modul \txtinline{math} atau
\txtinline{numpy}.
\end{soal}


\begin{soal}
\textbf{Chapra Latihan 4.3}
Lakukan hal yang sama dengan soal sebelumnya untuk mengestimasi nilai $\sin(\pi/3)$
dengan menggunakan deret Maclaurin:
\begin{equation*}
\sin(x) = x - \frac{x^3}{3!} + \frac{x^5}{5!} - \frac{x^7}{7!} + \cdots
\end{equation*}
\end{soal}

\begin{soal}
\textbf{Chapra Latihan 4.5}
Gunakan ekspansi deret Taylor orde-0 sampai dengan orde-3 untuk
memprediksi $f(3)$ untuk
\begin{equation*}
f(x) = 25x^3 - 6x^2 + 7x - 88
\end{equation*}
dengan titik basis pada $x=1$. Hitung kesalahan pada masing-masing aproksimasi.
\end{soal}

\begin{soal}
\textbf{Chapra Latihan 4.6}
Gunakan deret Taylor dengan orde-0 sampai dengan orde-10 untuk memprediksi
$f(2.5)$ untuk $f(x) = ln(x)$ dengan menggunakan nilai fungsi pada $x=1$.
Hitung juga kesalahan pada masing-masing aproksimasi. Berikan penjelasan mengenai hasil
yang Anda dapatkan.
\end{soal}

\begin{soal}
\textbf{Chapra Latihan 4.11}
Diketahui kecepatan jatuh seorang penerjun dapat dihitung dengan:
\begin{equation*}
v(t) = \frac{gm}{c}(1 - e^{-(c/m)t})
\end{equation*}
Gunakan analisis kesalahan orde-1 untuk mengestimasi kesalahan pada $v$ pada $t=6$ s,
jika $g = 9.81\,\mathrm{m/s}^{2}$ dan $m=50$ kg, namun
$c = 12.5 \pm 1.5 \, \mathrm{kg/s}$.
\end{soal}

\begin{soal}
\textbf{Chapra Latihan 4.12}
Ulangi soal sebelumnya dengan $g = 9.81$, $t = 6$, $c = 12.5 \pm 1.5$,
dan $m = 50 \pm 2$.
\end{soal}

\end{document}
