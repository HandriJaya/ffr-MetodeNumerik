\section{Modifikasi metode Newton-Raphson}

\textbf{Chapra Contoh 6.10}
Untuk kasus di mana persamaan nonlinear yang memiliki akar
lebih dari satu, metode Newton-Raphson dapat mengalami kesulitan
untuk konvergen. Metode Newton-Raphson perlu untuk dimodifikasi
sebagai berikut:
\begin{equation}
x_{i+1} = x_{i} - m\frac{f(x_{i})}{f'(x_{i})}
\label{eq:newton_raphson_2}
\end{equation}
di mana $m$ adalah multiplisitas dari akar.
Alternatif lain adalah dengan mendefinisikan fungsi:
\begin{equation*}
u(x) = \frac{f(x)}{f'(x)}
\end{equation*}
yang memiliki lokasi akar yang sama dengan fungsi awal $f(x)$.
Dengan definisi tersebut, aplikasi metode Newton-Raphson memberikan skema
iterasi sebagai berikut.
\begin{equation}
x_{i+1} = x_{i} - \frac{f(x_{i}) f'(x_{i})}{[f'(x_{i})]^{2} - f(x_{i})f''(x_{i})}
\label{eq:newton_raphson_3}
\end{equation}

\begin{soal}
Buat implementasi dengan Python untuk mengimplementasikan modifikasi metode
Newton-Raphson untuk akar dengan multiplisitas lebih dari satu
(menggunakan persamaan \eqref{eq:newton_raphson_2} dan \eqref{eq:newton_raphson_3})
dan uji pada persamaan nonlinear $f(x) = x^3 - 5x^2 + 7x - 3$ dengan nilai tebakan akar
awal $x_{0} = 0$. Bandingkan hasil yang Anda dapatkan jika menggunakan metode
Newton-Raphson tanpa modifikasi (persaman \eqref{eq:newton_raphson_1}).
\end{soal}