\section{Sistem persamaan nonlinear}

\textbf{Chapra Contoh 6.12}
Tinjau suatu sistem persamaan nonlinear berikut:
\begin{align*}
u(x,y) = x^{2} + xy - 10 = 0 \\
v(x,y) = y + 3xy^{2} - 57 = 0
\end{align*}
Dengan menggunakan ekstensi dari metode Newton-Raphson untuk 2 variabel
diperoleh skema iterasi berikut:
\begin{align}
x_{i+1} = x_{i} - \frac{
u_{i}\dfrac{\partial v_{i}}{\partial y} - 
v_{i}\dfrac{\partial u_{i}}{\partial y}
}%
{\dfrac{\partial u_{i}}{\partial x}\dfrac{\partial v_{i}}{\partial y} -
 \dfrac{\partial u_{i}}{\partial y}\dfrac{\partial v_{i}}{\partial x} } \\
y_{i+1} = y_{i} - \frac{
v_{i}\dfrac{\partial u_{i}}{\partial x} - 
u_{i}\dfrac{\partial v_{i}}{\partial x}
}%
{\dfrac{\partial u_{i}}{\partial x}\dfrac{\partial v_{i}}{\partial y} -
 \dfrac{\partial u_{i}}{\partial y}\dfrac{\partial v_{i}}{\partial x} }
\end{align}

Kode berikut ini mengimplementasikan metode Newton-Raphson untuk dua variabel:
\begin{pythoncode}
def u(x,y):
    return x**2 + x*y - 10

def dudx(x,y):
    return 2*x + y

def dudy(x,y):
    return x

def v(x,y):
    return ... # lengkapi

def dvdx(x,y):
    return ... # lengkapi

def dvdy(x,y):
    return ... # lengkapi

# Guess solutions
x = 1.5
y = 3.5

for i in range(1,5):
    # Jacobian matrix elements
    J11 = dudx(x,y)
    J12 = dudy(x,y)
    J21 = dvdx(x,y)
    J22 = dvdy(x,y)
    detJ = J11*J22 - J12*J21
    #
    ui = u(x,y)
    vi = v(x,y)
    # Update x
    xnew = .... # lengkapi
    ynew = .... # lengkapi
    print("x, y = %18.10f %18.10f" % (xnew, ynew))
    # TODO: Check convergence
    x = xnew
    y = ynew
\end{pythoncode}

\begin{soal}
Lengkapi program Python untuk metode Newton-Raphson untuk dua variabel.
Implementasikan program Anda sehingga
dapat melakukan iterasi sampai nilai kesalahan menjadi lebih kecil dari
suatu nilai tertentu yang diberikan. Anda dapat menggunakan loop \txtinline{while}
atau loop \txtinline{for} dengan jumlah iterasi maksimum tertentu.
\end{soal}

Untuk variabel yang lebih dari dua, metode Newton-Raphson dapat dituliskan dengan
menggunakan notasi matriks-vektor.
Persamaan yang diimplementasikan adalah (lihat slide 27 oleh Pak Haris):
\begin{equation}
\mathbf{X}_{i+1} = \mathbf{X}_{i} - \mathbf{J}_{i}^{-1} \mathbf{F}(\mathbf{X}_{i})
\end{equation}
di mana $\mathbf{J}$ adalah matriks Jacobian:
\begin{equation}
\mathbf{J} = \begin{bmatrix}
\dfrac{\partial f_{1}}{\partial x_{1}} & \dfrac{\partial f_{1}}{\partial x_{2}} & \cdots &
\dfrac{\partial f_{1}}{\partial x_{n}} \\[0.4cm]
\dfrac{\partial f_{2}}{\partial x_{1}} & \dfrac{\partial f_{2}}{\partial x_{2}} & \cdots &
\dfrac{\partial f_{2}}{\partial x_{n}} \\[0.4cm]
\vdots & \vdots &  & \vdots \\[0.4cm]
\dfrac{\partial f_{n}}{\partial x_{1}} & \dfrac{\partial f_{n}}{\partial x_{2}} & \cdots &
\dfrac{\partial f_{n}}{\partial x_{n}}
\end{bmatrix}
\label{eq:jacobian}
\end{equation}
Turunan parsial pada persaman \eqref{eq:jacobian} dievaluasi pada titik $\mathbf{X}_{i}$.
Nilai fungsi $\mathbf{F}(\mathbf{X})$ dan $\mathbf{X}$ direpresentasikan sebagai vektor kolom:
\begin{equation}
\mathbf{X} = \begin{bmatrix}
x_{1} \\
x_{2} \\
\vdots \\
x_{n}
\end{bmatrix}
\end{equation}
dan
\begin{equation}
\mathbf{F}(\mathbf{X}) = \begin{bmatrix}
f_{1}(\mathbf{X}) \\
f_{2}(\mathbf{X}) \\
\vdots \\
f_{n}(\mathbf{X})
\end{bmatrix}
\end{equation}

Kode berikut ini adalah alternatif implementasi dari metode Newton-Raphson
dengan menggunakan notasi matriks-vektor.
\begin{pythoncode}
import numpy as np

def f(X):
    x = X[0]
    y = X[1]
    f1 = x**2 + x*y - 10
    f2 = ... # lengkapi 
    return np.array([f1,f2])

def calc_jac(X):
    x = X[0]
    y = X[1]
    dudx = ... # lengkapi
    dudy = ... # lengkapi
    dvdx = ... # lengkapi
    dvdy = ... # lengkapi
    return np.array([
        [dudx, dudy],
        [dvdx, dvdy]
    ])
    
X = np.array([1.5, 3.5]) # initial guess
for i in range(1,6): # change this if needed
    fX = f(X)
    nfX = np.linalg.norm(fX) # calculate norm of f(X)
    print("X = ", X, "nfX = ", nfX)
    # stop the iteration if norm of f(X) become smaller than certain value
    if nfX <= 1e-10:
        print("Converged")
        break
    J = ... # calculate Jacobian matrix here
    invJ = np.linalg.inv(J) # calculate inverse of Jacobian matrix
    Xnew = X - np.matmul(invJ, fX)  # Update X
    X = np.copy(Xnew)  # replace X with Xnew
\end{pythoncode}

\begin{soal}
Lengkapi program Newton-Raphson di atas dan lakukan modifikasi jika diperlukan.
Perhatikan bahwa pustaka Numpy digunakan untuk melakukan
operasi matriks-vektor. Misalnya fungsi \txtinline{np.linalg.inv} digunakan untuk
menhitung invers dari matriks Jacobian dan \txtinline{np.matmul} untuk melakukan operasi
perkalian matriks $\mathbf{J}^{-1}$ dengan vektor kolom $\mathbf{F}(\mathbf{X})$.
\end{soal}