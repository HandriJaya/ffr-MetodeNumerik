\begin{soal}[Chapra Latihan 5.19]
Menurut prinsip Archimedes, gaya apung sama dengan berat fluida yang dipindahkan
oleh bagian benda yang terendam pada fluida.
Perhatikan gambar di bawah ini.

{\centering
\includegraphics[scale=1.0]{images_priv/Chapra_Fig_P5_19.pdf}
\par}

Tentukan tinggi $h$ yang mewakili bagian bola yang berada di atas air.
Gunakan nilai-nilai berikut: $r=1$ m, $\rho_{s}$ = kerapatan bola =
200 $\mathrm{kg/m}^{3}$ dan $\rho_{w}$ = kerapatan air = 1000 $\mathrm{kg/m}^{3}$.
Volume bagian bola yang berada di atas permukaan air (tidak terendam) dapat dihitung dari:
\begin{equation*}
V = \frac{\pi h^2}{3} (3r - h)
\end{equation*}
\end{soal}