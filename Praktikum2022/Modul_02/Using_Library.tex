\section{Menggunakan Pustaka Python}

Pada bagian ini, kita akan menggunakan beberapa pustaka SciPy yang dapat digunakan
untuk mencari akar persamaan nonlinear. Beberapa fungsi tersebut dapat ditemukan
pada modul \txtinline{scipy.optimize}.
\begin{itemize}
\item \txtinline{scipy.optimize.root_scalar}: untuk mencari akar persamaan nonlinear
yang terdiri dari satu variabel.
\item \txtinline{scipy.optimize.fsolve}: untuk mencari akar dari sistem persamaan
nonlinear (lebih dari satu variabel)
\end{itemize}

Pada contoh berikut, kita akan mencari akar dari $f(x) = x - \cos(x)$.
Untuk metode terbuka, kita menggunakan tebakan akar awal $x_0 = 0$
sedangkan untuk 
\begin{pythoncode}
from scipy import optimize
import math
    
def f(x):
    return x - math.cos(x)
    
def fprime(x):
    return 1 + math.sin(x)
    
print("\nUsing Newton-Raphson method")
sol = optimize.root_scalar(f, x0=0.0, fprime=fprime, method='newton')
print(sol)
    
print("\nUsing secant method")
sol = optimize.root_scalar(f, x0=0.0, x1=1.0, method='secant')
print(sol)
    
# Bracketing methods
for method in ["brentq", "brenth", "ridder", "bisect"]:
    print("\nUsing %s method" % (method))
    sol = optimize.root_scalar(f, bracket=[0.0, 1.0], method=method)
    print(sol)

print("\nUsing bisect directly")
xroot, sol = optimize.bisect(f, a=0.0, b=1.0, full_output=True)
print(sol)    
\end{pythoncode}
Modul \txtinline{scipy.optimize} juga menyediakan beberapa fungsi lain seperti
\txtinline{optimize.bisect} yang dapat digunakan untuk mencari akar persamaan
nonlinear, yang berbeda hanyalah \txtinline{interface} atau \txtinline{function signature}-nya.

Pada contoh berikut ini, kita akan mencari akar dari sistem persamaan nonlinear
pada Chapra Contoh 6.12 dengan menggunakan \txtinline{fsolve}:
\begin{pythoncode}
from scipy.optimize import fsolve

def f(x_):
    # x = x_[0] and y = x_[1] 
    x = x_[0]
    y = x_[1]
    u = x**2 + x*y - 10
    v = y + 3*x*y**2 - 57
    return [u, v]
    
root = fsolve(f, [1.0, 3.5]) # using initial guess as in the book
print(root)
\end{pythoncode}

Numpy menyediakan modul khusus untuk merepresentasikan polinomial, yaitu
\txtinline{numpy.polynomial}. Modul ini menyediakan banyak fungsi untuk
melakukan berbagai operasi terkait polinomial.
Pada contoh berikut ini, kita akan mencari akar-akar (real dan kompleks)
dari polinomial:
\begin{equation*}
f(x) = x^5 - 3.5x^4 + 2.75x^3 + 2.125x^2 - 3.875x + 1.25
\end{equation*}

Kode Python:
\begin{pythoncode}
from numpy.polynomial import Polynomial

p = Polynomial([1.25, -3.875, 2.125, 2.75, -3.5, 1.0])
print(p.roots())
\end{pythoncode}

Silakan membaca dokumentasi berikut untuk informasi lebih lanjut.
\begin{itemize}
\item {\scriptsize\url{https://docs.scipy.org/doc/scipy/reference/optimize.html\#root-finding}}
\item {\scriptsize\url{https://docs.scipy.org/doc/scipy/reference/generated/scipy.optimize.fsolve.html}}
\item {\scriptsize\url{https://numpy.org/doc/stable/reference/routines.polynomials.package.html}}
\end{itemize}

