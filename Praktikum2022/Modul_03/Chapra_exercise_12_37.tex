\begin{soal}[Chapra Latihan 12.37]

Tinjau sistem yang terdiri dari 3 massa dan 4 pegas yang terhubung
seperti Gambar berikut.

{\centering
\includegraphics[scale=1.0]{images_priv/Chapra_Fig_12_37}
\par}

Dengan menggunakan Hukum Newton, dapat diturunkan persamaan diferensial
berikut.
\begin{align*}
\ddot{x_{1}} + \frac{k_1 + k_2}{m_1} x_{1} - \frac{k_2}{m_1} = 0 \\
\ddot{x_{2}} - \frac{k_2}{m_2}x_1 + \frac{k_2 + k_3}{m_2}x_2 - \frac{k_3}{m_2}x_3 = 0 \\
\ddot{x_{3}} - \frac{k_3}{m_3}x_2 + \frac{k_3 + k_4}{m_3}x_3 = 0\\
\end{align*}
di mana $k_1 = k_4 = 10\,\mathrm{N/m}$, $k_2 = k_3 = 30\,\mathrm{N/m}$,
dan $m_1 = m_2 = m_3 = 2\,\mathrm{kg}$. Tulis tiga persamaan dalam bentuk matriks:
\begin{equation*}
0 = \text{[vektor akselarasi]} + \text{[matriks }k/m\text{]}\text{vektor perpindahan}
\end{equation*}
pada perpindahan (dalam meter) $x_{1} = 0.05, x_{2} = 0.04, x_{3} = 0.03$.
Cari akselarasi untuk setiap massa.
\end{soal}