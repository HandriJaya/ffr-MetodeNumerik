\section{Catatan Tambahan}

\subsection{Contoh penggunaan SymPy}
Contoh penggunaan SymPy untuk perhitungan integral tentu.
\begin{pythoncode}
import sympy
x = sympy.symbols("x")
func_symb = 6 + 3*sympy.cos(x)
resExact = sympy.N(sympy.integrate(func_symb, (x, 0, sympy.pi/2)))
# fungsi sympy.N digunakan untuk memaksa hasil dalam bentuk numerik.
\end{pythoncode}

Untuk batas integral yang melibatkan tak-hingga, Anda dapat menggunakan
\pyinline{sympy.oo}. Contoh:
\begin{pythoncode}
import sympy
x = sympy.symbols("x")
func_symb = sympy.exp(-x)*sympy.sin(x)**2
resExact = sympy.N(sympy.integrate(func_symb, (x, 0, sympy.oo)))
\end{pythoncode}


\subsection{Metode Newton-Cotes untuk interval terbuka}

Metode Newton-Cotes, interval terbuka (lihat Tabel 21.4 pada Chapra)
\begin{pythoncode}
# hanya untuk satu interval, f diberikan dalam bentuk analitik (bukan tabel)
def integ_newtoncotes_open6seg(f, a, b):
    # menggunakan 6 segmen atau 5 titik
    # silakan gunakan formula lain jika diperlukan
    h = (b-a)/6
    x1 = a + h
    x2 = a + 2*h
    x3 = a + 3*h
    x4 = a + 4*h
    x5 = a + 5*h
    #
    s = 11*f(x1) - 14*f(x2) + 26*f(x3) - 14*f(x4) + 11*f(x5)
    return (b-a)*s/20.0
\end{pythoncode}


\subsection{Fungsi Python untuk integral multi-interval yang seragam}

Mengaplikasikan aturan integrasi pada beberapa interval homogen
dengan menggunakan fungsi aturan integrasi satu interval, diberi nama
\pyinline{quad_func} pada argumen fungsi berikut.
\begin{pythoncode}
def apply_quadrature_multi_interval(quad_func, f, a, b, Ninterval):
    s = 0.0
    Δ = (b-a)/Ninterval
    for i in range(Ninterval):
        aa = a + i*Δ
        bb = a + (i+1)*Δ
        s = s + quad_func(f, aa, bb)
    return s
\end{pythoncode}

Contoh penggunaan:
\begin{pythoncode}
from math import cos, pi

import sympy
x = sympy.symbols("x")
func_symb = 6 + 3*sympy.cos(x)
resExact = sympy.N(sympy.integrate(func_symb, (x, 0, sympy.pi/2)))

def my_func(x):
    return 6 + 3*cos(x)

# Pastikan fungsi-fungsi yang terkait sudah di-import atau didefinisikan

a = 0.0
b = pi/2
resN = apply_quadrature_multi_interval(
    integ_newtoncotes_open6seg, my_func, a, b, 10
)
print("resN = %18.10f" % resN)
print("res  = %18.10f" % resExact)
print("err  = %18.10e" % abs(resExact-resN))

resN = apply_quadrature_multi_interval(
    integ_simpson38, my_func, a, b, 10
)
print("\nUsing integ_simpson38 10 intervals")
print("resN = %18.10f" % resN)
print("res  = %18.10f" % resExact)
print("err  = %18.10e" % abs(resExact-resN))

resN = apply_quadrature_multi_interval(
    integ_boole, my_func, a, b, 10
)
print("\nUsing integ_boole 10 intervals")
print("resN = %18.10f" % resN)
print("res  = %18.10f" % resExact)
print("err  = %18.10e" % abs(resExact-resN))
\end{pythoncode}


\subsection{Integral yang melibatkan batas tak-hingga}
Tinjau integral:
\begin{equation}
\int_{a}^{b} f(x)\ \mathrm{d}x
\label{eq:orig_integ}
\end{equation}
Jika $a$ dan/atau $b$ bernilai tak-hingga maka, kita dapat menggunakan suatu bilangan
yang cukup besar atau cukup kecil, atau \textit{practical infinity},
untuk mengaproksimasi tak-hingga dan metode
standard seperti aturan trapesium, Simpson, dan lainnya dapat digunakan.

Cara ini efektif apabila integran $f(x)$ mendekati nol pada tak-hingga dan nilai fungsi
sudah sangat kecil pada \textit{practical infinity}.
Pada beberapa fungsi atau integran yang terlokalisasi pada suatu titik, seperti Gaussian,
hal ini cukup efektif.

Cara ini tidak efektif apabila $f(x)$ mendekati nol dengan sangat lambat sehingga
\textit{practical infinity} yang digunakan menjadi sangat besar sehingga diperlukan
banyak selang untuk mengevaluasi integral. Metode adaptif dapat digunakan untuk
mengatasi hal ini.

Metode lain yang dapat digunakan adalah dengan melakukan penggantian variabel.
Kita akan menggunakan penggantian variabel $x \rightarrow \dfrac{1}{t}$ dan melakukan
integrasi pada domain $t$.

Dengan menggunakan substitusi variabel $x \rightarrow \dfrac{1}{t}$, sehingga
$\mathrm{d}x \rightarrow -\dfrac{1}{t^2} \mathrm{d}t$. Batas integrasi menjadi
$x=a \rightarrow t=1/a$ dan $x=b \rightarrow t=1/b$.
Integral pada Pers. \eqref{eq:orig_integ} menjadi:
\begin{align*}
\int_{a}^{b} f(x)\ \mathrm{d}x & = \int_{1/a}^{1/b}
\left( -\frac{1}{t^2} \right) f\left(\frac{1}{t}\right)\ \mathrm{dt} \\
& = \int_{1/b}^{1/a}
\left( \frac{1}{t^2} \right) f\left(\frac{1}{t}\right)\ \mathrm{dt}
\end{align*}
Metode integrasi Newton-Cotes selang terbuka biasanya digunakan
untuk menghitung integral ini secara numerik (mengapa?).

Sebagai contoh, kita akan menghitung integral:
\begin{equation}
\int_{1}^{\infty} \frac{1}{x^3 + 1}\ \mathrm{d}x
\end{equation}
SymPy mengalami kesulitan untuk mengevaluasi integral ini secara analitik
jika kita menggunakan \pyinline{sympy.oo} sebagai batas atas, namun kita
dapat melakukan aproksimasi dengan menggunakan suatu bilangan yang besar
sebagai pengganti dari tak-hingga. Untuk integral yang ada pada
Latihan Chapra 22.9, kita dapat menggunakan \pyinline{sympy.oo} secara
langsung.

Sebagai alternatif lain, Anda juga dapat menggunakan Wolfram Alpha
atau Mathematica.

Akses laman
{\scriptsize\url{https://www.wolframalpha.com/}} dan ketikkan teks
berikut pada input teks yang tersedia.
\begin{textcode}
N[Integrate[1/(x^3 +1), {x,1,Infinity}], 20]
\end{textcode}
Hasil diberikan dalam 20 digit. Anda dapat membandingkannya dengan hasil SymPy.

Berikut ini adalah kode Python yang akan kita gunakan.
\begin{pythoncode}
import sympy
x = sympy.symbols("x")
func_symb = 1/(x**3 + 1)
a = 1.0
# SymPy cannot evaluate this integral when we used sympy.oo
# We use a large number instead
resExact = sympy.N(sympy.integrate(func_symb, (x, a, 1e10)))

#
# Don't forget to import the needed functions or add their
# definitions
#

def my_func(x):
    return 1/(x**3 + 1)

b = 1000.0 # practical, approximate infinity

resApproxInf = sympy.N(sympy.integrate(func_symb, (x, a, b)))

print()
print("resExact     = %18.12f" % resExact) # "exact" result from SymPy
print("resApproxInf = %18.12f" % resApproxInf) # result with approximate infinity
print("diff         = %18.10e" % abs(resExact-resApproxInf))
\end{pythoncode}

Hasil
\begin{textcode}
resExact     =     0.373550727891
resApproxInf =     0.373550227891
diff         =   4.9999999890e-07    
\end{textcode}

Program di atas menggunakan nilai 1000 sebagai \textit{practical infinity}.
Perbandingan antara integral eksak dengan batas atas tak hingga, \pyinline{resExact},
dengan integral eksak dengan batas atas \textit{practical infinity},
yaitu \pyinline{resApproxInf}, diberikan pada
bagian akhir. Perbedaannya berada pada orde $10^{-7}$, sehingga metode
yang menggunakan \textit{practical infinity} tidak akan memiliki ketelitian
yang lebih tingga dari ini.

Berikutnya, kita akan menggunakan aturan Boole
\begin{pythoncode}
print("\nUsing Boole's rule") # naive
for n in [1, 10, 50, 100, 200, 500, 1000, 2000, 3000, 5000, 10000]:
    resN = apply_quadrature_multi_interval(
        integ_boole, my_func, a, b, n
    )    
    print("%5d %18.10f %18.10e %18.10e" % (n,
        resN, abs(resN-resExact), abs(resN-resApproxInf)))
\end{pythoncode}

Hasil:
\begin{textcode}
Using Boole's rule
    1      38.8500245106   3.8476473783e+01   3.8476474283e+01
   10       3.8872647405   3.5137140126e+00   3.5137145126e+00
   50       0.8148676096   4.4131688175e-01   4.4131738175e-01
  100       0.4862953453   1.1274461740e-01   1.1274511740e-01
  200       0.3851574116   1.1606683700e-02   1.1607183700e-02
  500       0.3730326569   5.1807101260e-04   5.1757101260e-04
 1000       0.3735156871   3.5040805719e-05   3.4540805720e-05
 2000       0.3735497047   1.0231788408e-06   5.2317884186e-07
 3000       0.3735501911   5.3678128181e-07   3.6781282908e-08
 5000       0.3735502264   5.0144397934e-07   1.4439804352e-09
10000       0.3735502279   5.0002079310e-07   2.0794199695e-11
\end{textcode}
Pada hasil di atas, selain nilai aproksimasi integral, juga diberikan
perbedaan antara aproksimasi dengan \pyinline{resExact} dan
\pyinline{resApproxInf} (karena nilai batas atas integral yang digunakan
pada aturan Boole di atas
adalah \textit{approximate infinity}).
Dari hasil yang diberikan, kita melihat bahwa aturan Boole multi interval
kesulitan untuk mendapatkan hasil yang akurat. Dengan menggunakan 10000 subinterval,
aturan Boole mendapatkan akurasi sekitar $10^{-11}$ (dibandingkan dengan hasil eksak
yang diperoleh dengan menggunakan \textit{approximate infinity}).

Berikutnya akan dicoba menggunakan metode Romberg.
\begin{pythoncode}
print("\nUsing integ_romberg")
resN = integ_romberg(my_func, a, b, MAXIT=14)
print("resN = %18.10f %18.10e %18.10e" % (resN,
    abs(resN-resExact), abs(resN-resApproxInf)))
\end{pythoncode}

Hasil:
\begin{textcode}
Using integ_romberg
Converged
iterConv =  13
resN =       0.3735503753   3.5260186448e-07   1.4739813442e-07
\end{textcode}
Penggunaan aturan Romberg dapat mencapai orde kesalahan $10^{-7}$
dengan kurang lebih 13 level rekursi.

Berikutnya akan dicoba menggunakan aturan Newton-Cotes untuk interval terbuka
dan penggantian variabel.
\begin{pythoncode}
def my_func2(t):
    x = 1/t
    return (1/t**2)*my_func(x)

# x -> a, t = 1/x -> 1/a
# x -> oo, t = 1/x -> 0
t1 = 0
t2 = 1/a

print("\nUsing Newton-Cotes 6seg")
for n in [1, 2, 4, 10, 50, 100]:
    resN = apply_quadrature_multi_interval(
        integ_newtoncotes_open6seg, my_func2, t1, t2, n
    )
    print("%5d %18.10f %18.10e" % (n, resN, abs(resN-resExact)))
\end{pythoncode}

Hasil:
\begin{textcode}
Using Newton-Cotes 6seg
    1       0.3743445981   7.9387016285e-04
    2       0.3735469160   3.8118722618e-06
    4       0.3735507339   5.9745146053e-09
   10       0.3735507279   3.8686998050e-11
   50       0.3735507279   3.5527136788e-15
  100       0.3735507279   7.7715611724e-16
\end{textcode}

Aturan Newton-Cotes 6-segment multi-interval pada integral yang sudah
ditransformasi dapat mencapai orde kesalahan $10^{-11}$ hanya dengan 10 interval
(sekitar 50 evaluasi fungsi).

Perhatikan bahwa penggunaan metode transformasi variabel ini tidak selalu
memberikan hasil yang lebih baik. Hal ini sangat bergantung pada integran yang
terlibat.
Anda disarankan untuk membuat plot dari integran maupun
integran yang sudah ditransformasi untuk mendapatkan informasi lebih lanjut
sebelum mengaplikasikan metode yang sudah ada.


\subsection{Diferensiasi numerik}

Chapra Contoh 23.1
\begin{pythoncode}
def forward_diff_Oh(f, x, h):
    return ( f(x+h) - f(x) )/h

def forward_diff_Oh2(f, x, h):
    return ( -f(x+2*h) + 4*f(x+h) - 3*f(x) )/(2*h)

def backward_diff_Oh(f, x, h):
    return ( f(x) - f(x-h) )/h

def backward_diff_Oh2(f, x, h):
    return ( 3*f(x) - 4*f(x-h) + f(x-2*h) )/(2*h)

def centered_diff_Oh2(f, x, h):
    return ( f(x+h) - f(x-h) )/(2*h)

def centered_diff_Oh4(f, x, h):
    return ( -f(x+2*h) + 8*f(x+h) - 8*f(x-h) + f(x-2*h) )/(12*h)

def f(x):
    return -0.1*x**4 - 0.15*x**3 - 0.5*x**2 - 0.25*x + 1.2

x = 0.5
h = 0.25
true_val = -0.9125

print("Using h = ", h)

print()
print("Forward diff")
#
df = forward_diff_Oh(f, x, h)
print("Oh  = %18.10f %18.10e" % (df, abs(df-true_val)) )
#
df = forward_diff_Oh2(f, x, h)
print("Oh2 = %18.10f %18.10e" % (df, abs(df-true_val)) )


print()
print("Backward diff")
#
df = backward_diff_Oh(f, x, h)
print("Oh  = %18.10f %18.10e" % (df, abs(df-true_val)) )
#
df = backward_diff_Oh2(f, x, h)
print("Oh2 = %18.10f %18.10e" % (df, abs(df-true_val)) )

print()
print("Centered diff")
#
df = centered_diff_Oh2(f, x, h)
print("Oh2 = %18.10f %18.10e" % (df, abs(df-true_val)) )
#
df = centered_diff_Oh4(f, x, h)
print("Oh4 = %18.10f %18.10e" % (df, abs(df-true_val)) )
\end{pythoncode}

Hasil:
\begin{textcode}
Using h =  0.25

Forward diff
Oh  =      -1.1546875000   2.4218750000e-01
Oh2 =      -0.8593750000   5.3125000000e-02
    
Backward diff
Oh  =      -0.7140625000   1.9843750000e-01
Oh2 =      -0.8781250000   3.4375000000e-02
    
Centered diff
Oh2 =      -0.9343750000   2.1875000000e-02
Oh4 =      -0.9125000000   2.2204460493e-1
\end{textcode}

Chapra Contoh 23.2
\begin{pythoncode}
# Tambahkan kode-kode yang diperlukan di sini
# (sama dengan Chapra Contoh 23.1)

x = 0.5
h1 = 0.5
h2 = 0.25
true_val = -0.9125

Dh1 = centered_diff_Oh2(f, x, h1)
Dh2 = centered_diff_Oh2(f, x, h2)

# Richardson extrapolation
Dh12 = 4*Dh2/3 - Dh1/3

print("Dh1  = %18.10f %18.10e" % (Dh1, abs(Dh1-true_val)) )
print("Dh2  = %18.10f %18.10e" % (Dh2, abs(Dh2-true_val)) )
print("Dh12 = %18.10f %18.10e" % (Dh12, abs(Dh12-true_val)) )
\end{pythoncode}

Hasil:
\begin{textcode}
Dh1  =      -1.0000000000   8.7500000000e-02
Dh2  =      -0.9343750000   2.1875000000e-02
Dh12 =      -0.9125000000   3.3306690739e-16    
\end{textcode}