\section{Fungsi yang diberikan pada tabel}

Pada soal-soal sebelumnya, fungsi yang akan dihitung diberikan fungsi analitiknya.
Aturan integral yang telah digunakan sebelumnya dapat digunakan juga untuk
fungsi yang diberikan dalam bentuk tabulasi data. Dalam hal ini kita perlu
menyesuaikan kode yang telah kita buat agar menerima array (bukan fungsi).
Implementasi program cukup sederhana jika data yang diberikan didefinisikan
pada titik-titik dengan panjang segmen yang sama. Jika tidak maka kita perlu
memberikan perlakuan yang berbeda untuk setiap segmen.
Salah satu metode sederhana yang dapat kita lakukan adalah dengan cara mengaplikasikan
aturan trapesium untuk setiap segmen secara terpisah. Pada soal berikut kita akan
melakukan hal tersebut.

\begin{soal}[Chapra Contoh 21.7]
Gunakan data pada Tabel 21.3 untuk mencari integral fungsi yang didefinisikan pada
Contoh 21.1 pada Chapra. Bandingkan hasilnya dengan nilai eksak. Perhatikan
bahwa data pada Tabel 21.3 tidak memilik panjang segmen yang sama.
\end{soal}

Anda dapat melengkapi kode beriku.
\begin{pythoncode}
def integ_trapz_table( fa, fb, a, b ):
    I = .... # LENGKAPI
    return I
# fa adalah nilai fungsi pada x=a
# fb adalah nilai fungsi pada x=b

x = [0.0, 0.12, 0.22, 0.32, 0.36, 0.40,
     0.44, 0.54, 0.64, 0.70, 0.80]

fx = [0.200000, 1.309729, 1.305241, 1.743393, 2.074903, 2.456000, 
      2.842985, 3.507297, 3.181929, 2.363000, 0.232000]

I_exact = 1.640533

Ndata = len(x)
I = 0.0
for i in range(Ndata-1):
    I = I + integ_trapz_table( fx[i], fx[i+1], x[i], x[i+1] )

E_t = (I_exact - I)
ε_t = E_t/I_exact * 100
print("Integral result = %.6f" % I)
print("True error      = %.6f" % E_t)
print("ε_t             = %.1f%%" % ε_t)
\end{pythoncode}