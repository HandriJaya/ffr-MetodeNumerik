\documentclass[a4paper,10pt,bahasa]{extarticle} % screen setting
\usepackage[a4paper]{geometry}

%\documentclass[b5paper,11pt,bahasa]{article} % screen setting
%\usepackage[b5paper]{geometry}

%\geometry{verbose,tmargin=1.5cm,bmargin=1.5cm,lmargin=1.5cm,rmargin=1.5cm}

\geometry{verbose,tmargin=2.0cm,bmargin=2.0cm,lmargin=2.0cm,rmargin=2.0cm}

\setlength{\parskip}{\smallskipamount}
\setlength{\parindent}{0pt}

%\usepackage{cmbright}
%\renewcommand{\familydefault}{\sfdefault}

\usepackage{amsmath}
\usepackage{amssymb}

\usepackage[libertine]{newtxmath}

\usepackage[no-math]{fontspec}
\setmainfont{Linux Libertine O}

%\usepackage{fontspec}
%\usepackage{lmodern}

\setmonofont{JuliaMono-Regular}


\usepackage{hyperref}
\usepackage{url}
\usepackage{xcolor}
\usepackage{enumitem}
\usepackage{mhchem}
\usepackage{graphicx}
\usepackage{float}

\usepackage{minted}

\newminted{julia}{breaklines,fontsize=\footnotesize}
\newminted{python}{breaklines,fontsize=\footnotesize}

\newminted{bash}{breaklines,fontsize=\footnotesize}
\newminted{text}{breaklines,fontsize=\footnotesize}

\newcommand{\txtinline}[1]{\mintinline[breaklines,fontsize=\footnotesize]{text}{#1}}
\newcommand{\jlinline}[1]{\mintinline[breaklines,fontsize=\footnotesize]{julia}{#1}}
\newcommand{\pyinline}[1]{\mintinline[breaklines,fontsize=\footnotesize]{python}{#1}}

\newmintedfile[juliafile]{julia}{breaklines,fontsize=\footnotesize}
\newmintedfile[pythonfile]{python}{breaklines,fontsize=\footnotesize}
\newmintedfile[fortranfile]{fortran}{breaklines,fontsize=\footnotesize}

\usepackage{mdframed}
\usepackage{setspace}
\onehalfspacing

\usepackage{babel}
\usepackage{appendix}

\newcommand{\highlighteq}[1]{\colorbox{blue!25}{$\displaystyle#1$}}
\newcommand{\highlight}[1]{\colorbox{red!25}{#1}}

\newcounter{soal}%[section]
\newenvironment{soal}[1][]{\refstepcounter{soal}\par\medskip
   \noindent \textbf{Soal~\thesoal. #1} \sffamily}{\medskip}


\definecolor{mintedbg}{rgb}{0.95,0.95,0.95}
\BeforeBeginEnvironment{minted}{
    \begin{mdframed}[backgroundcolor=mintedbg,%
        topline=false,bottomline=false,%
        leftline=false,rightline=false]
}
\AfterEndEnvironment{minted}{\end{mdframed}}


\BeforeBeginEnvironment{soal}{
    \begin{mdframed}[%
        topline=true,bottomline=false,%
        leftline=true,rightline=false]
}
\AfterEndEnvironment{soal}{\end{mdframed}}

\begin{document}

\title{%
{\small TF2202 Komputasi Rekayasa}\\
Integral dan Diferensiasi Numerik
}
\author{Tim Praktikum Komputasi Rekayasa 2021\\
Teknik Fisika\\
Institut Teknologi Bandung}
\date{}
\maketitle

\begin{soal}[Chapra Latihan 21.1-21.7]
Untuk integral-integral berikut ini
\begin{align}
& \int_{0}^{\pi/2} (6 + 3\cos(x))\ \mathrm{d}x \\
& \int_{0}^{3} (1 - e^{-2x})\ \mathrm{d}x \\
& \int_{-2}^{4} (1 - x - 4x^3 + 2x^5)\ \mathrm{d}x \\
& \int_{1}^{2} (x - 2/x)^2\ \mathrm{d}x \\
& \int_{-3}^{5} (4x - 3)^3 \mathrm{d}x \\
& \int_{0}^{3} x^2 e^x \mathrm{d}x \\
& \int_{0}^{1} 14^{2x}\ \mathrm{d}x
\end{align}
dengan menggunakan metode:
\begin{enumerate}[label=(\alph*)]
\item analitik
\item aturan trapesium
\item aturan 1/3 Simpson
\item aturan 3/8 Simpson
\item aturan Boole
\end{enumerate}
Variasikan parameter numerik yang terkait, seperti jumlah titik yang dievaluasi,
untuk setiap metode yang digunakan.
Bandingkan hasil numerik yang diberikan dengan hasil analitik.
Anda dapat menggunakan SymPy untuk menghitung integral secara analitik.
\end{soal}

\begin{soal}[Chapra 21.10 dan 21.11]
Hitung integral dari data berikut ini dengan menggunakan aturan trapesium dan aturan
Simpson.

Data Chapra 21.10

{\centering
\begin{tabular}{|c|cccccc|}
\hline
$x$    & 0 & 0.1 & 0.2 & 0.3 & 0.4 & 0.5 \\
$f(x)$ & 1 & 8   & 4   & 3.5 & 5   & 1 \\
\hline
\end{tabular}
\par}

Data Chapra 21.11

{\centering
\begin{tabular}{|c|ccccccc|}
\hline
$x$    & -2 & 0 & 2 & 4 & 6 & 8 & 10 \\
$f(x)$ & 35 & 5 & -10 & 2 & 5 & 3 & 20 \\
\hline
\end{tabular}
\par}
\end{soal}

\begin{soal}[Chapra 21.13]
Diketahui data berikut ini dihasilkan dari fungsi $f(x)=2e^{-1.5x}$

{\centering
\begin{tabular}{|c|ccccccc|}
\hline
$x$    & 0 & 0.05   & 0.15 & 0.25 & 0.35 & 0.475 & 0.6 \\
$f(x)$ & 2 & 1.8555 & 1.5970 & 1.3746 & 1.1831 & 0.9808 & 0.8131 \\
\hline
\end{tabular}
\par}

Hitung integral dari data ini dari $a=0$ dan $b=0.6$ dengan menggunakan
aturan trapesium dan aturan Simpson. Bandingkan hasil yang Anda peroleh
dengan melakukan integrasi $f(x)$ secara analitik.
\end{soal}

\begin{soal}[Chapra Latihan 22.1, 22.2, dan 22.3]
Gunakan metode Romberg untuk menghitung integral berikut.
\begin{align}
& \int_{0}^{3} x e^{2x}\ \mathrm{d}x \\
& \int_{1}^{2} \left( x + \frac{1}{x} \right)^2 \ \mathrm{d}x \\
& \int_{0}^{2} \frac{e^{x}\sin(x)}{1 + x^2}\ \mathrm{d}x
\end{align}
Variasikan parameter numerik yang Anda gunakan, misalnya jumlah maksimal selang
yang digunakan.
Bandingkan hasil yang Anda dapatkan dengan hasil analitik atau dari SymPy.
\end{soal}


\begin{soal}[Chapra Latihan 22.8]
Gunakan formula Gauss-Legendre dengan dua sampai 6 titik untuk menghitung integral
\begin{equation}
\int_{-3}^{3} \frac{1}{1 + x^2}\ \mathrm{d}x
\end{equation}
Bandingkan hasil yang Anda dapatkan dengan hasil analitik atau dari SymPy.
\end{soal}

\begin{soal}[Chapra Latihan 22.9]
Gunakan metode integrasi numerik untuk menghitung integral:
\begin{align}
& \int_{2}^{\infty} \frac{1}{x(x + 2)}\ \mathrm{dx} \\
& \int_{0}^{\infty} e^{-x} \sin^{2} x\ \mathrm{dx}
\end{align}
Bandingkan hasil yang Anda peroleh dengan hasil analitik atau dari SymPy.
Gunakan metode standard (trapesium, Simpson, Romberg, dll) secara
langsung dan dengan melakukan penggantian variabel. Anda juga dapat menggunakan
kombinasi metode tersebut jika integral yang Anda lakukan dibagi menjadi dua
integral (Lihat Contoh 22.6 pada Chapra). Plot juga integran yang terlibat.
\end{soal}


\begin{soal}[Chapra Latihan 23.8]
Hitung turunan pertama dari fungsi-fungsi berikut dengan menggunakan metode beda hingga
tengah $\mathcal{O}(h^4)$:
\begin{itemize}
\item $y = x^3 + 4x - 15$ pada $x=0$ dengan $h=0.25$
\item $y = x^2 \cos(x)$ pada $x=0.4$ dengan $h=0.1$
\item $y = \tan(x/3)$ pada $x=3$ dengan $h=0.5$
\item $y = \sin(0.5\sqrt{x})/x$ pada $x=1$ dengan $h=0.2$
\item $y = e^{x} + x$ pada $x=2$ dengan $h=0.2$
\end{itemize}
Bandingkan hasil yang Anda peroleh dengan hasil analitik atau SymPy.
\end{soal}


\begin{soal}[Chapra Latihan 23.9]
Diketahui data jarak dan waktu yang ditempuh pada suatu roket.

{\centering
\begin{tabular}{|c|cccccc|}
\hline
t (s)  & 0 & 25 & 50 & 75 & 100 & 125 \\
y (km) & 0 & 32 & 58 & 78 &  92 & 100 \\
\hline
\end{tabular}
\par}

Gunakan metode numerik untuk untuk mengestimasi kecepatan dan percepatan roket
pada masing-masing waktu pada tabel (jika memungkinkan dengan metode yang Anda
gunakan).
\end{soal}


\section*{Kode Python}

\subsection*{Contoh penggunaan SymPy}
Contoh penggunaan SymPy untuk perhitungan integral tentu.
\begin{pythoncode}
import sympy
x = sympy.symbols("x")
func_symb = 6 + 3*sympy.cos(x)
resExact = sympy.N(sympy.integrate(func_symb, (x, 0, sympy.pi/2)))
# fungsi sympy.N digunakan untuk memaksa hasil dalam bentuk numerik.
\end{pythoncode}

Untuk batas integral yang melibatkan tak-hingga, Anda dapat menggunakan
\pyinline{sympy.oo}. Contoh:
\begin{pythoncode}
import sympy
x = sympy.symbols("x")
func_symb = sympy.exp(-x)*sympy.sin(x)**2
resExact = sympy.N(sympy.integrate(func_symb, (x, 0, sympy.oo)))
\end{pythoncode}

\subsection*{Beberapa metode integrasi dengan interval tertutup}

Metode trapesium:
\begin{pythoncode}
# Satu interval
def integ_trapz( f, a, b ):
    I = (b - a) * (f(a) + f(b)) / 2
    return I

# Jumlah interval N bisa lebih dari satu
def integ_trapz_multiple( f, a, b, N ):
    x0 = a
    xN = b
    h = (b - a)/N
    ss = 0.0
    for i in range(1,N):
        xi = x0 + i*h
        ss = ss + f(xi)
    I = h/2 * ( f(x0) + 2*ss + f(xN) )
    return I
\end{pythoncode}

Metode Boole (lihat Tabel 21.2 pada Chapra):
\begin{pythoncode}
# hanya untuk satu interval, f diberikan dalam bentuk analitik (bukan tabel)
def integ_boole(f, a, b):
    h = (b-a)/4
    x0 = a
    x1 = a + h
    x2 = a + 2*h
    x3 = a + 3*h
    x4 = b

    s = 7*f(x0) + 32*f(x1) + 12*f(x2) + 32*f(x3) + 7*f(x4)
    return (b-a)*s/90
\end{pythoncode}

\subsection*{Metode Newton-Cotes untuk interval terbuka}

Metode Newton-Cotes, interval terbuka (lihat Tabel 21.4 pada Chapra)
\begin{pythoncode}
# hanya untuk satu interval, f diberikan dalam bentuk analitik (bukan tabel)
def integ_newtoncotes_open6seg(f, a, b):
    # menggunakan 6 segmen atau 5 titik
    # silakan gunakan formula lain jika diperlukan
    h = (b-a)/6
    x1 = a + h
    x2 = a + 2*h
    x3 = a + 3*h
    x4 = a + 4*h
    x5 = a + 5*h
    #
    s = 11*f(x1) - 14*f(x2) + 26*f(x3) - 14*f(x4) + 11*f(x5)
    return (b-a)*s/20.0
\end{pythoncode}


\subsection*{Fungsi Python untuk integral multi-interval yang seragam}

Mengaplikasikan aturan integrasi pada beberapa interval homogen
dengan menggunakan fungsi aturan integrasi satu interval, diberi nama
\pyinline{quad_func} pada argumen fungsi berikut.
\begin{pythoncode}
def apply_quadrature_multi_interval(quad_func, f, a, b, Ninterval):
    s = 0.0
    Δ = (b-a)/Ninterval
    for i in range(Ninterval):
        aa = a + i*Δ
        bb = a + (i+1)*Δ
        s = s + quad_func(f, aa, bb)
    return s
\end{pythoncode}

Contoh penggunaan:
\begin{pythoncode}
from math import cos, pi

import sympy
x = sympy.symbols("x")
func_symb = 6 + 3*sympy.cos(x)
resExact = sympy.N(sympy.integrate(func_symb, (x, 0, sympy.pi/2)))

def my_func(x):
    return 6 + 3*cos(x)

# Pastikan fungsi-fungsi yang terkait sudah di-import atau didefinisikan

a = 0.0
b = pi/2
resN = apply_quadrature_multi_interval(
    integ_newtoncotes_open6seg, my_func, a, b, 10
)
print("resN = %18.10f" % resN)
print("res  = %18.10f" % resExact)
print("err  = %18.10e" % abs(resExact-resN))

resN = apply_quadrature_multi_interval(
    integ_simpson38, my_func, a, b, 10
)
print("\nUsing integ_simpson38 10 intervals")
print("resN = %18.10f" % resN)
print("res  = %18.10f" % resExact)
print("err  = %18.10e" % abs(resExact-resN))

resN = apply_quadrature_multi_interval(
    integ_boole, my_func, a, b, 10
)
print("\nUsing integ_boole 10 intervals")
print("resN = %18.10f" % resN)
print("res  = %18.10f" % resExact)
print("err  = %18.10e" % abs(resExact-resN))
\end{pythoncode}

\subsection*{Integral Romberg}
Metode Romberg untuk integral numerik.
\begin{pythoncode}
# Lihat Gambar 22.4 pada Chapra
def integ_romberg(f, a, b, es=1e-10, MAXIT=10):
    I = np.zeros( (MAXIT+2,MAXIT+2) )
    n = 1
    # We start from I[1,1], to follow the book's notation
    I[1,1] = integ_trapz_multiple(f, a, b, n)
    iterConv = 0
    for i in range(1,MAXIT+1):
        n = 2**i
        I[i+1,1] = integ_trapz_multiple(f, a, b, n)
        #
        for k in range(2,i+2):
            j = 2 + i - k
            I[j,k] = .... # LENGKAPI
        #
        ea = abs( (I[1,i+1] - I[2,i])/I[1,i+1] )*100 # in percent
        if ea <= es:
            iterConv = i
            #print("converged")
            break
        iterConv = i
    return I[1,iterConv+1]
\end{pythoncode}


\subsection*{Integral yang melibatkan batas tak-hingga}
Tinjau integral:
\begin{equation}
\int_{a}^{b} f(x)\ \mathrm{d}x
\label{eq:orig_integ}
\end{equation}
Jika $a$ dan/atau $b$ bernilai tak-hingga maka, kita dapat menggunakan suatu bilangan
yang cukup besar atau cukup kecil, atau \textit{practical infinity},
untuk mengaproksimasi tak-hingga dan metode
standard seperti aturan trapesium, Simpson, dan lainnya dapat digunakan.

Cara ini efektif apabila integran $f(x)$ mendekati nol pada tak-hingga dan nilai fungsi
sudah sangat kecil pada \textit{practical infinity}.
Pada beberapa fungsi atau integran yang terlokalisasi pada suatu titik, seperti Gaussian,
hal ini cukup efektif.

Cara ini tidak efektif apabila $f(x)$ mendekati nol dengan sangat lambat sehingga
\textit{practical infinity} yang digunakan menjadi sangat besar sehingga diperlukan
banyak selang untuk mengevaluasi integral. Metode adaptif dapat digunakan untuk
mengatasi hal ini.

Metode lain yang dapat digunakan adalah dengan melakukan penggantian variabel.
Kita akan menggunakan penggantian variabel $x \rightarrow \dfrac{1}{t}$ dan melakukan
integrasi pada domain $t$.

Dengan menggunakan substitusi variabel $x \rightarrow \dfrac{1}{t}$, sehingga
$\mathrm{d}x \rightarrow -\dfrac{1}{t^2} \mathrm{d}t$. Batas integrasi menjadi
$x=a \rightarrow t=1/a$ dan $x=b \rightarrow t=1/b$.
Integral pada Pers. \eqref{eq:orig_integ} menjadi:
\begin{align*}
\int_{a}^{b} f(x)\ \mathrm{d}x & = \int_{1/a}^{1/b}
\left( -\frac{1}{t^2} \right) f\left(\frac{1}{t}\right)\ \mathrm{dt} \\
& = \int_{1/b}^{1/a}
\left( \frac{1}{t^2} \right) f\left(\frac{1}{t}\right)\ \mathrm{dt}
\end{align*}
Metode integrasi Newton-Cotes selang terbuka biasanya digunakan
untuk menghitung integral ini secara numerik (mengapa?).

Sebagai contoh, kita akan menghitung integral:
\begin{equation}
\int_{1}^{\infty} \frac{1}{x^3 + 1}\ \mathrm{d}x
\end{equation}
SymPy mengalami kesulitan untuk mengevaluasi integral ini secara analitik
jika kita menggunakan \pyinline{sympy.oo} sebagai batas atas, namun kita
dapat melakukan aproksimasi dengan menggunakan suatu bilangan yang besar
sebagai pengganti dari tak-hingga. Untuk integral yang ada pada
Latihan Chapra 22.9, kita dapat menggunakan \pyinline{sympy.oo} secara
langsung.

Sebagai alternatif lain, Anda juga dapat menggunakan Wolfram Alpha
atau Mathematica.

Akses laman
{\scriptsize\url{https://www.wolframalpha.com/}} dan ketikkan teks
berikut pada input teks yang tersedia.
\begin{textcode}
N[Integrate[1/(x^3 +1), {x,1,Infinity}], 20]
\end{textcode}
Hasil diberikan dalam 20 digit. Anda dapat membandingkannya dengan hasil SymPy.

Berikut ini adalah kode Python yang akan kita gunakan.
\begin{pythoncode}
import sympy
x = sympy.symbols("x")
func_symb = 1/(x**3 + 1)
a = 1.0
# SymPy cannot evaluate this integral when we used sympy.oo
# We use a large number instead
resExact = sympy.N(sympy.integrate(func_symb, (x, a, 1e10)))

#
# Don't forget to import the needed functions or add their
# definitions
#

def my_func(x):
    return 1/(x**3 + 1)

b = 1000.0 # practical, approximate infinity

resApproxInf = sympy.N(sympy.integrate(func_symb, (x, a, b)))

print("\nUsing Boole's rule") # naive
for n in [1, 2, 4, 10, 50, 100, 200, 300, 500, 1000, 10000]:
    resN = apply_quadrature_multi_interval(
        integ_boole, my_func, a, b, n
    )    
    print("%5d %18.10f %18.10e %18.10e" % (n,
        resN, abs(resN-resExact), abs(resN-resApproxInf)))

print("\nUsing integ_romberg")
resN = integ_romberg(my_func, a, b, MAXIT=14)
print("resN = %18.10f %18.10e %18.10e" % (resN,
    abs(resN-resExact), abs(resN-resApproxInf)))

def my_func2(t):
    x = 1/t
    return (1/t**2)*my_func(x)

# x -> a, t = 1/x -> 1/a
# x -> oo, t = 1/x -> 0
t1 = 0
t2 = 1/a

print("\nUsing Newton-Cotes 6seg")
for n in [1, 2, 4, 10, 50, 100]:
    resN = apply_quadrature_multi_interval(
        integ_newtoncotes_open6seg, my_func2, t1, t2, n
    )
    print("%5d %18.10f %18.10e" % (n, resN, abs(resN-resExact)))

print()
print("resExact     = %18.12f" % resExact) # "exact" result from SymPy
print("resApproxInf = %18.12f" % resApproxInf) # result with approximate infinity
print("diff         = %18.10e" % abs(resExact-resApproxInf))
\end{pythoncode}

Hasil:
\begin{textcode}
Using Boole's rule
    1      38.8500245106   3.8476473783e+01   3.8476474283e+01
    2      19.4250984308   1.9051547703e+01   1.9051548203e+01
    4       9.7128865195   9.3393357916e+00   9.3393362916e+00
   10       3.8872647405   3.5137140126e+00   3.5137145126e+00
   50       0.8148676096   4.4131688175e-01   4.4131738175e-01
  100       0.4862953453   1.1274461740e-01   1.1274511740e-01
  200       0.3851574116   1.1606683700e-02   1.1607183700e-02
  300       0.3740278872   4.7715934605e-04   4.7765934605e-04
  500       0.3730326569   5.1807101260e-04   5.1757101260e-04
 1000       0.3735156871   3.5040805719e-05   3.4540805720e-05
10000       0.3735502279   5.0002079310e-07   2.0794199695e-11

Using integ_romberg
converged, iterConv =  13
iterConv =  13
resN =       0.3735503753   3.5260186448e-07   1.4739813442e-07

Using Newton-Cotes 6seg
    1       0.3743445981   7.9387016285e-04
    2       0.3735469160   3.8118722618e-06
    4       0.3735507339   5.9745146053e-09
   10       0.3735507279   3.8686998050e-11
   50       0.3735507279   3.5527136788e-15
  100       0.3735507279   7.7715611724e-16

resExact     =     0.373550727891
resApproxInf =     0.373550227891
diff         =   4.9999999890e-07
\end{textcode}

Program di atas menggunakan nilai 1000 sebagai \textit{practical infinity}.
Perbandingan antara integral eksak dengan batas atas tak hingga, \pyinline{resExact},
dengan integral eksak dengan batas atas \textit{practical infinity},
yaitu \pyinline{resApproxInf}, diberikan pada
bagian akhir. Perbedaannya berada pada orde $10^{-7}$, sehingga metode
yang menggunakan \textit{practical infinity} tidak akan memiliki ketelitian
yang lebih tingga dari ini.

Pada bagian awal diberikan perbandingan nilai yang diberikan dengan menggunakan
aturan Boole multi interval. Selain nilai aproksimasi integral, juga diberikan
perbedaan antara aproksimasi dengan \textit{resExact} dan \pyinline{resApproxInf}.
Dari hasil yang diberikan, kita melihat bahwa aturan Boole multi interval
kesulitan untuk mendapatkan hasil yang akurat.

Penggunaan aturan Romberg dapat mencapai orde kesalahan $10^{-7}$
dengan kurang lebih 13 level rekursi.

Aturan Newton-Cotes 6-segment multi-interval pada integral yang sudah
ditransformasi dapat mencapai orde kesalahan $10^{-11}$ hanya dengan 10 interval
(sekitar 50 evaluasi fungsi).

Perhatikan bahwa penggunaan metode transformasi variabel ini tidak selalu
memberikan hasil yang lebih baik. Hal ini sangat bergantung pada integran yang
terlibat.
Anda disarankan untuk membuat plot dari integran maupun
integran yang sudah ditransformasi untuk mendapatkan informasi lebih lanjut
sebelum mengaplikasikan metode yang sudah ada.

\end{document}
