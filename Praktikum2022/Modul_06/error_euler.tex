\subsection*{Analisis kesalahan metode Euler}

Kesalahan pada metode Euler terdiri dari dua kontribusi:
\begin{itemize}
\item Kesalahan pemotongan (truncation)
\item Kesalahan pembulatan (round-off)
\end{itemize}

Kesalahan pemotongan terdiri dari dua bagian:
\begin{itemize}
\item kesalahan pemotongan lokal yang berasal dari aplikasi metode pada satu
ukuran langkah.
\item kesalahan pemotongan perambatan (propagated truncation error) yang diakibatkan
aproksimasi pada langkah-langkah sebelumnya.
\end{itemize}
Penjumlahan dari kedua jenis kesalahan tersebut adalah kesalahan pemotongan global.

Misalkan, persamaan diferensial yang akan diintegralkan adalah:
\begin{equation*}
y' = \frac{\mathrm{d}y}{\mathrm{d}x} = f(x,y)
\end{equation*}
Jika solusi, yaitu fungsi yang mendeskripsikan $y$, memiliki turunan yang kontinu, maka
deret Taylor dapat digunakan untuk merepresentasikan fungsi di sekitar nilai awal
$(x_i, y_i)$:
\begin{equation*}
y_{i+1} = y_{i} + y'_{i} h + \frac{y''_{i}}{2!}h^2 + \cdots + \frac{y^{(n)}_{i}}{n!} h^n + R_n
\end{equation*}
dengan $h = x_{i+1} - x_{i}$ dan $R_n$ adalah suku sisa yang didefinisikan sebagai:
\begin{equation*}
R_n = \frac{y^{n+1}(\xi)}{(n+1)!} h^{n+1}
\end{equation*}
di mana $\xi$ berada di dalam interval $[x_i, x_{i+1}]$.
Alternatif penulisan:
\begin{equation*}
y_{i+1} = y_{i} + f(x_i, y_i) h + \frac{f'(x_i, y_i)}{2!}h^2 + \cdots +
\frac{f^{(n-1)}(x_i, y_i)}{n!} h^n + \mathcal{O}(h^{n+1})
\end{equation*}
di mana $\mathcal{O}(h^{n+1})$ menyatakan kontribusi kesalahan pemotongan lokal
yang sebanding dengan ukuran langkah pangkat $(n+1)$.

Dengan membandingkan persamaan ini dengan skema metode Euler, diperoleh
kesalahan pemotongan pada metode Euler sebagai berikut:
\begin{equation*}
E_{t} = \frac{f'(x_i, y_i)}{2!} h^2 + \cdots + \mathcal{O}(h^{n+1})
\end{equation*}
untuk nilai $h$ yang cukup kecil, dapat digunakan hanya satu suku saja:
\begin{equation*}
E_{a} = \frac{f'(x_i, y_i)}{2!} h^2
\end{equation*}
atau
\begin{equation*}
E_a = \mathcal{O}(h^2)
\end{equation*}
di mana $E_a$ adalah aproksimasi kesalahan pemotongan lokal.

Dengan loop, perhitungan error pemotongan
\begin{pythoncode}
# .... definisi deriv
# .... definisi ode_euler_1step
# .... definisi exact_sol

# For local truncation errors
from math import factorial
def trunc_err_second(x,y,h):
    return (-6*x**2 + 24*x - 20)*h**2/factorial(2)

def trunc_err_third(x,y,h):
    return (-12*x + 24)*h**3/factorial(3)

def trunc_err_fourth(x,y,h):
    return -12*h**4/factorial(4)

# initial cond
x0 = 0.0
y0 = 1.0

print("   x     y_true   y_Euler       ε_t   ε_t local")
print("-----------------------------------------------")

print("%5.1f  %8.5f  %8.5f" % (x0, y0, y0)) # Initial cond

x = x0
y = y0
h = 0.5
for i in range(0,8):
    xp1 = x + h
    yp1 = ode_euler_1step(deriv, x, y, h)
    y_true = exact_sol(xp1)
    ε_t = (y_true - yp1)/y_true * 100
    ε_t_local = (trunc_err_second(x,y,h) + trunc_err_third(x,y,h) + \
        trunc_err_fourth(x,y,h))/y_true * 100
    print("%5.1f  %8.5f  %8.5f  %8.1f%%  %8.1f%%" % (xp1, y_true, yp1, ε_t, ε_t_local))
    # Update x and y for the next step
    x = xp1
    y = yp1
\end{pythoncode}

Contoh keluaran:
\begin{textcode}
   x     y_true   y_Euler       ε_t   ε_t local
-----------------------------------------------
  0.0   1.00000   1.00000
  0.5   3.21875   5.25000     -63.1%     -63.1%
  1.0   3.00000   5.87500     -95.8%     -28.1%
  1.5   2.21875   5.12500    -131.0%      -1.4%
  2.0   2.00000   4.50000    -125.0%      20.3%
  2.5   2.71875   4.75000     -74.7%      17.2%
  3.0   4.00000   5.87500     -46.9%       3.9%
  3.5   4.71875   7.12500     -51.0%     -11.3%
  4.0   3.00000   7.00000    -133.3%     -53.1%
\end{textcode}