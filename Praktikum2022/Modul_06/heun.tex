\subsection{Metode Heun}

Metode Heun
\begin{equation*}
y'_{i} = f(x_i, y_i)
\end{equation*}
ekstrapolasi:
\begin{equation*}
y^{0}_{i+1} = y_{i} + f(x_i, y_i) h
\end{equation*}
Untuk metode Heun, $y^{0}_{i+1}$ digunakan sebagai prediksi intermediat.
Persamaan ini dikenal sebagai persamaan prediktor.
Persamaan ini memberikan estimasi $y_{i+1}$ pada $x_{i+1}$:
\begin{equation*}
y'_{i+1} = f(x_{i+1}, y^{0}_{i+1})
\end{equation*}
Dengan menggunakan rata-rata dari nilai kemiringan ini diperoleh:
\begin{equation*}
\overline{y}' = \frac{y'_{i} + y'_{i+1}}{2} = 
\frac{f(x_i, y_i) + f(x_{i+1},y^{0}_{i+1})}{2}
\end{equation*}
Kemiringan rata-rata ini kemudian digunakan untuk ekstrapolasi linear dari $y_{i}$ ke
$y_{i+1}$:
\begin{equation*}
y_{i+1} = y_{i} + \frac{f(x_i, y_i) + f(x_{i+1}, y^{0}_{i+1})}{2} h
\end{equation*}
Persamaan ini dikenal sebagai persamaan korektor.

Metode Heun termasuk ke dalam kelompok metode prediktor-korektor. Metode Heun dapat
dituliskan sebagai berikut.
\begin{align}
y^{0}_{i+1} & = y_{i} + f(x_i, y_i) h & \text{   Prediktor}\\
y_{i+1} & = y_{i} + \frac{f(x_i, y_i) + f(x_{i+1}, y^{0}_{i+1})}{2} h & \text{   Korektor}
\end{align}
Karena $y_{i+1}$ berada pada kedua ruas dari persamaan korektor, maka kita harus mengaplikasikan
persamaan tersebut secara iteratif.


\begin{soal}[Chapra Contoh 25.5]
Gunakan metode Heun untuk mengintegrasikan persamaan diferensial
\begin{equation*}
\frac{\mathrm{d}y}{\mathrm{d}x} = 4e^{0.8x} - 0.5y
\end{equation*}
dari $x=0$ sampai $x=4$ dengan ukuran langkah 1. Syarat awal adalah
$y(x=0) = 2$.
Bandingkan dengan solusi analitik:
\begin{equation*}
y = \frac{4}{1.3} \left( e^{0.8x} - e^{-0.5x} \right) + 2e^{-0.5x}
\end{equation*}
\end{soal}


\begin{pythoncode}
# One-step application of Heun's method for ODE
def ode_heun_1step(dfunc, xi, yi, h):
    y0ip1 = yi + dfunc(xi,yi)*h
    avg = 0.5*( dfunc(xi,yi) + dfunc(xi+h,y0ip1) )*h
    return yi + avg
\end{pythoncode}

\begin{pythoncode}
from math import exp

# .... import dan/atau definisi

def deriv(x, y):
    return 4*exp(0.8*x) - 0.5*y
    
def exact_sol(x):
    return 4.0/1.3*( exp(0.8*x) - exp(-0.5*x) ) + 2*exp(-0.5*x)
    
# Initial cond
x0 = 0.0
y0 = 2.0
xf = 4.0
h = 1.0
Nstep = int(xf/h)
    
x = x0
y = y0
for i in range(0,Nstep):
    xp1 = x + h
    yp1 = ode_heun_1step(deriv, x, y, h)
    y_true = exact_sol(xp1)
    ε_t = (y_true - yp1)/y_true * 100
    print("%f %12.7f %12.7f  %5.2f%%" % (xp1, y_true, yp1, abs(ε_t)))
    # For the next step
    x = xp1
    y = yp1    
\end{pythoncode}


Iteratif:
\begin{pythoncode}
# One-step application of Heun's method for ODE
# Using iterative steps to determine y0ip1
def ode_heun_1step_iterative(dfunc, xi, yi, h, NiterMax=100, Δ=1e-6):
    y0ip1 = yi + dfunc(xi,yi)*h
    y0ip1_old = y0ip1
    for i in range(NiterMax+1):
        avg = 0.5*( dfunc(xi,yi) + dfunc(xi+h,y0ip1) )*h
        y0ip1 = yi + avg
        diff = abs(y0ip1 - y0ip1_old)
        # Uncomment this to see the iteration process
        print("iter: %2d y0ip1 = %12.7f  diff = %12.7e" % (i+1, y0ip1, diff))
        if diff <= Δ:
            break
        y0ip1_old = y0ip1
    return y0ip1
\end{pythoncode}
