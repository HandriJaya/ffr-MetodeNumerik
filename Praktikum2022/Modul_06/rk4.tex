\section{Metode Runge-Kutta Orde 4}

\begin{pythoncode}
def ode_rk4_1step(dfunc, xi, yi, h):
    k1 = dfunc(xi, yi)
    k2 = dfunc(xi + 0.5*h, yi + 0.5*k1*h)
    k3 = dfunc(xi + 0.5*h, yi + 0.5*k2*h)
    k4 = dfunc(xi + h, yi + k3*h)
    yip1 = yi + (k1 + 2*k2 + 2*k3 + k4)*h/6
    return yip1
\end{pythoncode}


\section{Metode Runge-Kutta Orde 5}

\begin{pythoncode}
def ode_rk5_1step(dfunc, xi, yi, h):
    k1 = dfunc(xi, yi)
    k2 = dfunc(xi + h/4, yi + k1*h/4)
    k3 = dfunc(xi + h/4, yi + k1*h/8 + k2*h/8)
    k4 = dfunc(xi + h/2, yi - k2*h/2 + k3*h)
    k5 = dfunc(xi + 3*h/4, yi + 3*k1*h/16 + 9*k4*h/16)
    k6 = dfunc(xi + h, yi - 3*k1*h/7 + 2*k2*h/7 + 12*k3*h/7 - 12*k4*h/7 + 8*k5*h/7)
    yip1 = yi + (7*k1 + 32*k3 + 12*k4 + 32*k5 + 7*k6)*h/90
    return yip1
\end{pythoncode}


\section{Subrutin umum untuk metode RK}

\begin{pythoncode}
def ode_solve(dfunc, do_1step, x0, y0, h, Nstep):
    Nvec = len(y0)
    x = np.zeros(Nstep+1)
    y = np.zeros((Nstep+1,Nvec))
    # Start with initial cond
    x[0] = x0
    y[0,:] = y0[:]
    for i in range(0,Nstep):
        x[i+1] = x[i] + h
        y[i+1,:] = do_1step(dfunc, x[i], y[i,:], h)
    return x, y
\end{pythoncode}

