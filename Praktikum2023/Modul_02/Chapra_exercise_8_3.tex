\begin{soal}[Chapra Latihan 8.3]
Dalam proses kimia, uap air ($\mathrm{H}_{\mathrm{2}}\mathrm{O}$) dipanaskan sampai
suhu yang cukup tinggi sehingga sebagian besar dari air akan terdisosiasi membentuk oksigen
($\mathrm{O}_{2}$) dan hidrogen ($\mathrm{H}_{2}$) menurut persamaan reaksi
\begin{equation*}
\mathrm{H}_2\mathrm{O} \rightarrow \mathrm{H}_{2}\,+\,\mathrm{O}_{2}
\end{equation*}
Jika diasumsikan bahwa ini adalah satu-satunya reaksi yang terlibat, fraksi
mol $\ce{H2O}$ yang berdisosiasi, dilambangkan dengan $x$, dapat dihitung dari persamaan
\begin{equation*}
K = \frac{x}{1-x}\sqrt{\frac{2p_t}{2+x}}
\end{equation*}
dimana $K$ = kesetimbangan konstan reaksi dan $p_t$: tekanan total campuran
dalam satuan atm.
Jika diketahui bahwa $p_t = 3.5$ atm dan $K = 0.4$, tentukan nilai $x$ yang memenuhi
persamaan di atas.
\end{soal}