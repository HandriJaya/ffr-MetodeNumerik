\begin{soal}[Chapra Latihan 12.39]
Ditribusi temperatur pada keadaan tunak dari suatu plat yang dipanaskan
dapat dimodelkan dengan menggunakan persamaan Laplace:
\begin{equation*}
\frac{\partial^2 T}{\partial x^2} + \frac{\partial^2 T}{\partial y^2} = 0
\end{equation*}
Jika plat ini direpresentasikan menjadi kumpulan titik-titik yang berjarak
seragam, metode beda hingga dapat diaplikasikan untuk mencari solusi numerik
dari persamaan Laplace. Gunakan aproksimasi beda hingga (seperti pada soal
sebelumnya) untuk menurunkan sistem persamaan linear untuk 4 titik interior
seperti pada gambar berikut.

{\centering
\includegraphics[scale=0.3]{images_priv/Chapra_Fig_12_39.pdf}
\par}

Asumsikan $\Delta x = \Delta y$. Tulis juga persamaan linear yang diperoleh
jika ada 9 dan 16 titik interior. 
Implementasikan program untuk menyelesaikan sistem persamaan linear
ini dengan menggunakan metode Gauss-Seidel. Program Anda harus dapat menerima input
$N$ di mana $N = N$ adalah jumlah titik interior yang digunakan. Asumsikan juga bahwa $N$
merupakan bilangan kuadrat, artinya jumlah titik diskritisasi pada arah $x$ dan $y$
bernilai sama.
(Lihat Chapra Contoh 29.1 untuk informasi lebih lanjut, jika diperlukan)
\end{soal}

