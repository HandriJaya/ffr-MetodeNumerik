\section{Polinomial Interpolasi Lagrange}
\begin{soal}
Implementasikan fungsi atau subroutin dalam Python untuk implementasi algoritma
pada Gambar 18.11 untuk implementasi polinomial Lagrange.
Uji hasil yang Anda dapatkan dengan menggunakan data-data yang diberikan pada contoh
18.2 (polinomial kuadrat) dan 18.3 (polinomial kubik) pada Chapra.
\end{soal}

\begin{soal}
Kerjakan Contoh 18.7 pada Chapra dengan menggunakan fungsi/subrutin interpolasi Lagrange
yang sudah Anda buat pada soal sebelumnya sehingga Anda dapat mereproduksi Gambar 18.12
pada Chapra. Pada Contoh 18.17 Anda diminta untuk mengestimasi kecepatan penerjun
pada $t=10$ s, yang berada di antara dua titik data terakhir.
Untuk polinom orde-1, gunakan dua titik data terakhir, untuk orde-2 gunakan tiga titik
data terakhir, dan seterusnya sampai orde-4.
\end{soal}

\begin{pythoncode}
def lagrange_interp(x, y, xx):
    assert(len(x) == len(y))
    N = len(x) - 1  # length of array is (N + 1)
    ss = 0.0
    for i in range(0,N+1):
        pp = ....
        for j in range(0,N+1):
            if i != j:
                ....
        ss = ....
    return ss
\end{pythoncode}