\section{Aturan Simpson}

\begin{soal}
Hitung integral yang dikerjakan pada Soal Chapra 21.1 dengan menggunakan aturan 1/3 Simpson
untuk satu selang. Perhatikan bawah aturan 1/3 Simpson memerlukan 3 titik dasar sedangkan
aturan trapesium hanya memerlukan 2 titik.
Bandingkan hasilnya dengan hasil eksak dan aturan trapesium.
\end{soal}

\begin{pythoncode}
def my_func(x):
    return 0.2 + 25*x - 200*x**2 + 675*x**3 - 900*x**4 + 400*x**5

a = 0.0
b = 0.8
h = (b-a)/2
# Base points for Simpson 1/3 rule
x0 = a
x1 = a + h
x2 = b
# We have three points or two segments with equal length

I_exact = 1.640533

# Hitung aproksimasi dengan aturan 1/3 Simpson di sini
I = .... # LENGKAPI
# Anda dapat menggunakan Pers. 21.14 atau 21.15

E_t = (I_exact - I)
ε_t = E_t/I_exact * 100
print("Integral result = %.7f" % I)
print("True integral   = %.6f" % I_exact)
print("True error      = %.7f" % E_t)
print("ε_t             = %.1f%%" % ε_t)

import sympy
x = sympy.symbols("x")
f = 0.2 + 25*x - 200*x**2 + 675*x**3 - 900*x**4 + 400*x**5
d4f = f.diff(x,4)
avg_d4f_xi = sympy.integrate( d4f, (x,a,b) )/(b - a)
E_a = -1/2880*avg_d4f_xi*(b - a)**5 # Pers. 21.16
print("Approx error    = %.7f" % E_a)
\end{pythoncode}

\begin{soal}
Ulangi soal sebelumnya dengan menggunakan aturan 1/3 Simpson dengan
jumlah segmen 4 atau $n = 4$ pada Persamaan 21.18 di Chapra (ada 5 titik
dasar yang diperlukan). Coba variasikan jumlah segmen dan bandingkan
hasil yang Anda peroleh dengan hasil analitik.
\end{soal}

Anda dapat melengkapi kode berikut ini
\begin{pythoncode}
    import numpy as np

    def my_func(x):
        return 0.2 + 25*x - 200*x**2 + 675*x**3 - 900*x**4 + 400*x**5
    
# Using equation 21.18
def integ_simpson13_multiple( f, a, b, N ):
    # N: is number of segments
    assert N >= 2
    assert N % 2 == 0 # must be an even number

    x0 = a
    xN = b
    h = (b - a)/N
    x = np.linspace(a, b, N+1)
    # Total number of base points is N+1 (an odd number)

    ss_odd = 0.0
    for i in range(....):
        ss_odd = .... # Lengkapi
    
    ss_even = 0.0
    for i in range(....):
        ss_even = .... # Lengkapi

    I = (b - a)/(3*N) * ( f(x0) + 4*ss_odd + 2*ss_even + f(xN) )
    return I
    
a = 0.0
b = 0.8
I_exact = 1.640533
Nsegments = 4
    
I = integ_simpson13_multiple(my_func, a, b, Nsegments)
E_t = (I_exact - I)
ε_t = E_t/I_exact * 100
print("Nsegments = ", Nsegments)
print("Integral result = %.7f" % I)
print("True error      = %.7f" % E_t)
print("ε_t             = %.2f%%" % ε_t)
    
import sympy
x = sympy.symbols("x")
f = 0.2 + 25*x - 200*x**2 + 675*x**3 - 900*x**4 + 400*x**5
d4f = f.diff(x,4)
avg_d4f_xi = sympy.integrate( d4f, (x,a,b) )/(b - a)
E_a = -avg_d4f_xi*(b - a)**5 / (180*Nsegments**4)
print("Approx error    = %.7f" % E_a)    
\end{pythoncode}

\begin{soal}[Chapra Contoh 21.6a]
Gunakan aturan 3/8 Simpson untuk integral pada soal sebelumnya.
Perhatikan bahwa satu kali aplikasi aturan 3/8 Simpson (pada satu interval
$[a,b]$) memerlukan 4 titik atau 3 segmen dengan panjang yang sama.
\end{soal}

Anda dapat melengkapi kode berikut.
\begin{pythoncode}
import numpy as np

def my_func(x):
    return 0.2 + 25*x - 200*x**2 + 675*x**3 - 900*x**4 + 400*x**5

# One application of Simpson 3/8 rule
def integ_simpson38( f, a, b ):
    h = (b - a)/3
    # Four points are needed
    x0 = a
    x1 = a + h
    x2 = a + 2*h
    x3 = b
    #
    I = .... # Lengkapi
    return I
    
a = 0.0
b = 0.8
I_exact = 1.640533 # from the book
    
I = integ_simpson38(my_func, a, b)
E_t = (I_exact - I)
ε_t = E_t/I_exact * 100
print("Integral result = %.7f" % I)
print("True error      = %.7f" % E_t)
print("ε_t             = %.2f%%" % ε_t)
    
import sympy
x = sympy.symbols("x")
f = 0.2 + 25*x - 200*x**2 + 675*x**3 - 900*x**4 + 400*x**5
d4f = f.diff(x,4)
avg_d4f_xi = sympy.integrate( d4f, (x,a,b) )/(b - a)
E_a = -avg_d4f_xi*(b - a)**5 / 6480
print("Approx error    = %.7f" % E_a)    
\end{pythoncode}


\begin{soal}
Melanjutkan soal sebelumnya, menggunakan lima segmen (6 titik),
kombinasi antara aturan 1/3 dan 3/8 Simpson.
\end{soal}

Anda dapat melengkapi kode berikut.
\begin{pythoncode}
# Definisi fungsi yang akan dicari nilai integralnya

def integ_simpson13( f, a, b ):
    #
    h = (b - a)/2
    x0 = a
    x1 = a + h
    x2 = b
    #
    I = .... # lengkapi
    return I
    
# ... Definisi integ_simpson38
    
import numpy as np
a = 0.0
b = 0.8
Nsegments = 5
x = np.linspace(a, b, Nsegments+1)
I_exact = 1.640533

print(x)
# Tampilkan titik-titik yang digunakan, bandingkan dengan Chapra
    
I_13 = integ_simpson13(my_func, x[0], x[2]) # first two segments
print("I_13 = %.7f" % I_13 )
    
I_38 = integ_simpson38(my_func, x[2], x[Nsegments])
print("I_38 = %.7f" % I_38 )
    
I = I_13 + I_38
E_t = (I_exact - I)
ε_t = E_t/I_exact * 100
print("Integral result = %.7f" % I)
print("True error      = %.7f" % E_t)
print("ε_t             = %.2f%%" % ε_t)    
\end{pythoncode}