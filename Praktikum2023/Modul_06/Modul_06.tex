\documentclass[a4paper,10pt,bahasa]{extarticle} % screen setting
\usepackage[a4paper]{geometry}

%\documentclass[b5paper,11pt,bahasa]{article} % screen setting
%\usepackage[b5paper]{geometry}

%\geometry{verbose,tmargin=1.5cm,bmargin=1.5cm,lmargin=1.5cm,rmargin=1.5cm}

\geometry{verbose,tmargin=2.0cm,bmargin=2.0cm,lmargin=2.0cm,rmargin=2.0cm}

\setlength{\parskip}{\smallskipamount}
\setlength{\parindent}{0pt}

%\usepackage{cmbright}
%\renewcommand{\familydefault}{\sfdefault}

\usepackage{amsmath}
\usepackage{amssymb}

\usepackage[libertine]{newtxmath}

\usepackage[no-math]{fontspec}
\setmainfont{Linux Libertine O}

%\usepackage{fontspec}
%\usepackage{lmodern}

\setmonofont{JuliaMono-Regular}


\usepackage{hyperref}
\usepackage{url}
\usepackage{xcolor}
\usepackage{enumitem}
\usepackage{mhchem}
\usepackage{graphicx}
\usepackage{float}

\usepackage{minted}

\newminted{julia}{breaklines,fontsize=\footnotesize}
\newminted{python}{breaklines,fontsize=\footnotesize}

\newminted{bash}{breaklines,fontsize=\footnotesize}
\newminted{text}{breaklines,fontsize=\footnotesize}

\newcommand{\txtinline}[1]{\mintinline[breaklines,fontsize=\footnotesize]{text}{#1}}
\newcommand{\jlinline}[1]{\mintinline[breaklines,fontsize=\footnotesize]{julia}{#1}}
\newcommand{\pyinline}[1]{\mintinline[breaklines,fontsize=\footnotesize]{python}{#1}}

\newmintedfile[juliafile]{julia}{breaklines,fontsize=\footnotesize}
\newmintedfile[pythonfile]{python}{breaklines,fontsize=\footnotesize}
\newmintedfile[fortranfile]{fortran}{breaklines,fontsize=\footnotesize}

\usepackage{mdframed}
\usepackage{setspace}
\onehalfspacing

\usepackage{babel}
\usepackage{appendix}

\newcommand{\highlighteq}[1]{\colorbox{blue!25}{$\displaystyle#1$}}
\newcommand{\highlight}[1]{\colorbox{red!25}{#1}}

\newcounter{soal}%[section]
\newenvironment{soal}[1][]{\refstepcounter{soal}\par\medskip
   \noindent \textbf{Soal~\thesoal. #1} \sffamily}{\medskip}


\definecolor{mintedbg}{rgb}{0.95,0.95,0.95}
\BeforeBeginEnvironment{minted}{
    \begin{mdframed}[backgroundcolor=mintedbg,%
        topline=false,bottomline=false,%
        leftline=false,rightline=false]
}
\AfterEndEnvironment{minted}{\end{mdframed}}


\BeforeBeginEnvironment{soal}{
    \begin{mdframed}[%
        topline=true,bottomline=false,%
        leftline=true,rightline=false]
}
\AfterEndEnvironment{soal}{\end{mdframed}}

\begin{document}

\title{%
{\small TF2202 Komputasi Rekayasa}\\
Persamaan Diferensial Biasa
}
\author{Tim Praktikum Komputasi Rekayasa 2022\\
Teknik Fisika\\
Institut Teknologi Bandung}
\date{}
\maketitle


\section*{Teori}
% ----------------------------------
\subsection*{Menggunakan deret Taylor orde-tinggi}

Misalnya, menggunakan orde dua:
\begin{equation*}
y_{i+1} = y_{i} + f(x_i, y_i)h + \frac{f'(x_i, y_i)}{2!}h^2
\end{equation*}
dengan kesalahan pemotongan lokal:
\begin{equation*}
E_{a} = \frac{f''(x_i, y_i)}{6} h^3
\end{equation*}
Dengan menggunakan aturan rantai untuk turunan:
\begin{equation*}
f'(x_i, y_i) = \frac{\partial f(x,y)}{\partial x} +
\frac{\partial f(x,y)}{\partial y}\frac{\mathrm{d}y}{\mathrm{d}x}
\end{equation*}


\subsection*{Metode Runge-Kutta}

\begin{equation*}
y_{i+1} = y_{i} + \phi(x_i, y_i, h) h
\end{equation*}

$\phi(x_i, y_i, h)$: fungsi kenaikan (\textit{increment})

\begin{equation*}
\phi = a_1 k_1 + a_2 k_2 + \cdots + a_n k_n
\end{equation*}

di mana $a$ adalah konstanta dan $k$ adalah:
\begin{align*}
k_1 & = f(x_i, y_i) \\
k_2 & = f(x_i + p_1 h, y_i + q_{11} k_1 h ) \\
k_3 & = f(x_i + p_2 h, y_i + q_{21} k_1 h + q_{22} k_2 h) \\
\cdots & \cdots \\
k_n & = f(x_i + p_{n-1}h, y_i + q_{n-1,1} k_1 h + q_{n-1,2} k_2 h + \cdots + q_{n-1,n-1} k_{n-1} h)
\end{align*}
di mana $p$ dan $q$ adalah konstanta.
Perhatikan bahwa $k$ memiliki hubungan rekurensi (perulangan).


\subsection{Runge-Kutta orde 2}

\begin{equation*}
y_{i+1} = y_i + (a_1 k_1 + a_2 k_2) h
\end{equation*}
dengan
\begin{align*}
k_1 = f(x_i, y_i) \\
k_2 = f(x_i + p_1 h, y_i + q_{11} k_1 h)
\end{align*}
Nilai-nilai dari $a_1$, $a_2$, $p_1$, dan $q_{11}$ dapat diperoleh dari persamaan
berikut.
\begin{align*}
a_1 + a_2  & = 1 \\
a_2 p_1    & = \frac{1}{2} \\
a_2 q_{11} & = \frac{1}{2}
\end{align*}
Karena ada tiga persamaan dan empat variabel yang tidak diketahui, kita harus mengasumsikan satu
nilai dari variabel tersebut untuk mendapatkan tiga variabel yang lain.
Misalkan nilai $a_2$ telah dipilih, maka variabel-variabel yang lain
dapat ditentukan sebagai berikut.
\begin{align*}
a_1 = 1 - a_2
p_1 = q_{11} = \frac{1}{2a_2}
\end{align*}
Dengan kata lain, ada tak hingga versi dari metode Runge-Kutta orde-2.

Ada tiga versi yang populer:

Metode Heun dengan korektor tunggal, $a_2 = 1/2$
$a_1 = 1/2$ dan $p_1 = q_{11} = 1$. Dengan parameter tersebut, diperoleh skema
sebagai berikut:
\begin{equation*}
y_{i+1} = y_{i} + \left(
\frac{1}{2}k_1 + \frac{1}{2}k_2
\right)h
\end{equation*}
dengan
\begin{align*}
k_1 = f(x_i, y_i) \\
k_2 = f(x_i + h, y_i + k_1 h)
\end{align*}
Perhatikan bahwa $k_1$ adalah kemiringan pada awal interval dan $k_2$ adalah
kemiringan pada akhir interval. Oleh karena itu, metode Runge-Kutta orde-2 ini
tidak lain adalah metode Heun tanpa iterasi.


Metode titik tengah, $a_2 = 1$, $a_1 = 0$, $p_1 = q_{11} = 1/2$, diperoleh
skema sebagai berikut.
\begin{equation*}
y_{i+1} = y_i + k_2 h
\end{equation*}
dengan
\begin{align*}
k_1 = f(x_i, y_i) \\
k_2 = f\left( x_i + \frac{1}{2}h, y_i + \frac{1}{2} k_1 h \right)
\end{align*}
yang merupakan metode titik tengah.

Metode Ralston, dikembangkan oleh Ralston (1962) dan Ralston dan Rabinowitz (1978),
yang memilih parameter-parameter sehingga batas minimum untuk kesalahan pemotongan,
dengan parameter $a_2 = 2/3$, $a_1 = 1/3$, dan $p_1 = q_{11} = 3/4$, yang
memberikan skema sebagai berikut:
\begin{equation*}
y_{i+1} = y_i + \left( \frac{1}{3}k_1 + \frac{2}{3}k_2 \right) h
\end{equation*}
dengan
\begin{align*}
k_1 = f(x_i, y_i) \\
k_2 = f\left( x_i + \frac{3}{4}h, y_i + \frac{3}{4}k_1 h \right)
\end{align*}



\subsection*{Persamaan diferensial orde dua}

Contoh:
\begin{equation*}
m \frac{\mathrm{d}^2 x}{\mathrm{d}t^2} + c \frac{\mathrm{d}x}{\mathrm{d}t} + kx = 0
\end{equation*}
$c$: koefisien redaman, $k$ konstant pegas.

Definisikan $y_{1}(t) = x(t)$, $y_{2}(t) = x'(t)$, sehingga:
$y'_{1}(t) = x'(t) = y_{2}(t)$, dan
$y'_{2}(t) = x''(t)$:
\begin{align*}
m \frac{\mathrm{d}^2 x}{\mathrm{d}t^2} + c \frac{\mathrm{d}x}{\mathrm{d}t} + kx & = 0 \\
m y'_{2}(t) + c y_{2}(t) + k y_{1}(t) & = 0 \\
y'_{2}(t) & = -\frac{c y_{2}(t) + k y_{1}(t)}{m}
\end{align*}

\begin{align*}
y'_{1}(t) & = y_{2}(t) \\
y'_{2}(t) & = -\frac{c y_{2}(t) + k y_{1}(t)}{m}
\end{align*}
%m \frac{\mathrm{d}^2 x}{\mathrm{d}t^2} + c \frac{\mathrm{d}x}{\mathrm{d}t} + kx = 0


\subsection*{Metode multilangkah}

Pada metode satu langkah, kita menggunakan informasi sebelumnya pada satu titik $x_i$ untuk
memprediksi nilai dari variabel dependen $y_{i+1}$ pada titik depan $x_{i+1}$.

Pada metode multilangkah (\textit{multistep}), kita menggunakan informasi sebelumnya
yang diperoleh pada beberapa titik sebelumnya.

\subsection*{Metode Heun \textit{non-self-starting}}

Ingat bahwa metode Heun menggunakan metode Euler sebagai prediktor:
\begin{equation*}
y_{i+1}^{0} = y_{i} + f(x_i, y_i) h
\end{equation*}
dan aturan trapesium sebagai korektor:
\begin{equation*}
y_{i+1} = y_{i} + \frac{f(x_i,y_i) + f(x_{i+1},y^{0}_{i+1})}{2} h 
\end{equation*}
Prediktor dan korektor memiliki kesalahan pemotongan lokal $\mathcal{O(h^2)}$
dan $\mathcal{O}(h^3)$. Hal ini menyarankan bahwa prediktor merupakan
hubungan yang lemah pada metode ini karena memiliki kesalahan yang terbesar.
Kelemahan ini signifikan karena efisiensi dari korektor iteratif bergantung
pada akurasi dari prediksi awal.
Salah satu cara untuk memperbaiki metode Heun adalah dengan mengembangkan prediktor
yang memiliki kesalahan pemotongan lokal $\mathcal{O}(h^3)$.
Hal ini dapat dicapai dengan menggunakan metode Euler dan kemiringan pada $y_i$ dan
informasi tambahan dari titik sebelumnya $y_{i-1}$ sebagai berikut:
\begin{equation*}
y^{0}_{i+1} = y_{i-1} + f(x_i, y_i)2h
\end{equation*}
Perhatikan bahwa persamaan diatas memiliki kesalahan lokal $\mathcal{O}(h^3)$
dengan menggunakan ukuran langkah yang lebih besar $2h$. Selain itu, persamaan
ini tidak self-starting karena melibatkan nilai sebelumnya, $y_{i-1}$.
Informasi ini tidak tersedia pada permasalahan nilai awal. Oleh karena itu
skema yang dihasilkan disebut sebagai metode Heun non-self-starting.

\begin{align*}
y^{0}_{i+1} = y^{m}_{i-1} + f(x_i,y^{m}_i)2h \\
y^{j}_{i+1} = y^{m}_{i} + \frac{f(x_i,y^{m}_i) + f(x_{i+1},y^{j-1}_{i+1})}{2} h 
\end{align*}

$m$: indeks iterasi

Perhatikan bahwa $y^{m}_{i}$ dan $y^{m}_{i-1}$


\subsection*{Formula Adams-Bashforth}

Ekspansi Taylor:
\begin{equation*}
y_{i+1} = y_{i} + f_{i} h + \frac{f'_{i}}{2!} h^2 + \frac{f''_{i}}{3!} h^3 + \cdots
\end{equation*}
yang juga dapat dituliskan sebagai:
\begin{equation*}
y_{i+1} = y_{i} + h\left( f_i + \frac{h}{2}f'_{i} + \frac{h^2}{3!} + \cdots \right)
\end{equation*}

Backward difference:
\begin{equation*}
f'_{i} = \frac{f_i - f_{i-1}}{h} + \frac{f''_{i}}{2!} + \mathcal{O}(h^2)
\end{equation*}

Substitusi:
\begin{equation*}
y_{i+1} = y_{i} + h \left(
f_{i} + \frac{h}{2} \left[ \frac{f_i - f_{i-1}}{h} + \frac{f''_{i}}{2} + \mathcal{O}(h^2) \right]
+ \frac{h^2}{6} f''_{i} + \cdots \right)
\end{equation*}
atau:
\begin{equation*}
y_{i+1} = y_{i} + h \left( \frac{3}{2} f_{i} - \frac{1}{2} f_{i-1} \right) +
\frac{5}{12} h^3 f''_{i} + \mathcal{O}(h^4)
\end{equation*}
(2nd order open Adams formula)
or Adams-Bashforth formula.

Bentuk umum formula Adams-Bashforth:
\begin{equation*}
y_{i+1} = y_{i} + h \sum_{k=0}^{n-1} \beta_{k} f_{i-k} + \mathcal{O}(h^{n+1})
\end{equation*}



\subsection*{Formula Adams-Moulton}

Deret Taylor mundur disekitar $x_{i+1}$:
\begin{equation*}
y_{i} = y_{i+1} - f_{i+1}h + \frac{f'_{i+1}}{2!} h^2 - \frac{f''_{i+1}}{3!} h^3 + \cdots
\end{equation*}

Cari $y_{i+1}$:
\begin{equation*}
y_{i+1} = y_i + h \left( f_{i+1} - \frac{h}{2}f'_{i+1} + \frac{h^2}{6} f''_{i+1} + \cdots \right)
\end{equation*}

Aproksimasi turunan pertama:
\begin{equation*}
f'_{i+1} = \frac{f_{i+1} - f_{i}}{h} + \frac{f''_{i+1}}{2}h + \mathcal{O}(h^2)
\end{equation*}
Substitusi:
\begin{equation*}
y_{i+1} = y_{i} + h \left( \frac{1}{2}f_{i+1} + \frac{1}{2}f_{i} \right) - 
\frac{1}{12} h^3 f''_{i+1} - \mathcal{O}(h^4)
\end{equation*}

Formula Adams tertutup orde-2 atau
Formula Adams-Moulton orde-2

Bentuk umum:
\begin{equation*}
y_{i+1} = y_{i} + h \sum_{k=0}^{n-1} \beta_{k} f_{i+1-k} + \mathcal{O}(h^{n+1})
\end{equation*}


\subsection*{Metode Milne}

Menggunakan formula Newton-Cotes terbuka tiga-titik sebagai prediktor:
\begin{equation*}
y^{0}_{i+1} = y^{m}_{i-3} + \frac{4h}{3}\left(
2f^{m}_{i} - f^{m}_{i-1} + 2f^{m}_{i-2}
\right)
\end{equation*}
dan Newton-Cotes tertutup tiga-titik sebagai korektor:
\begin{equation*}
y^{j}_{i+1} = y^{m}_{i-1} + \frac{h}{3} \left(
f^{m}_{i-1} + 4f^{m}_{i} + f^{j-1}_{i+1}
\right)
\end{equation*}
$j$ adalah indeks iterasi.

\subsection*{Metode Adams orde-4}

Menggunakan formula Adams-Bashforth orde-4 sebagai prediktor:
\begin{equation*}
y^{0}_{i+1} = y^{m}_{i} + h \left(
\frac{55}{24} f^{m}_{i} - \frac{59}{24}f^{m}_{i-1} + \frac{37}{24}f^{m}_{i-2}
- \frac{9}{24} f^{m}_{i-3}
\right)
\end{equation*}
dan formula Adams-Moulton orde-4 sebagai korektor:
\begin{equation*}
y^{j}_{i+1} = y^{m}_{i} + h \left( 
\frac{9}{24} f^{j-1}_{i+1} + \frac{19}{24} f^{m}_{i} - \frac{5}{24} f^{m}_{i-1}
+ \frac{1}{24} f^{m}_{i-2}
\right)
\end{equation*}




\section*{Pendahuluan}
Pada modul ini kita akan mengeksplorasi beberapa metode yang
dapat digunakan untuk menyelesaikan persamaan diferensial biasa, dengan bentuk umum:
\begin{equation}
\frac{\mathrm{d}y}{\mathrm{d}x} = f(x,y)
\label{eq:dydx_umum}
\end{equation}
Untuk mendapatkan solusi yang unik, diperlukan informasi tambahan.
Informasi tambahan ini dapat berupa:
\begin{itemize}
\item Nilai fungsi yang ingin dicari dispesifikasikan pada satu titik $y(x=x_0)=y_0$.
Dalam konteks persoalan fisis, biasanya variabel independen ini adalah waktu, di mana
fungsi atau kuantitas yang ingin dicari solusinya diketahui pada suatu kondisi awal
dan kita ingin mengetahui evolusi atau dinamika dari kuantitas tersebut. Persoalan
ini dikenal sebagai persoalan nilai awal (\textit{initial valus problem}).
\item Nilai fungsi dispesifikasikan pada titik-titik batas domain fungsi.
Persoalan ini dikenal dengan persoalan nilai batas (\textit{boundary value problem}).
\end{itemize}

Kita akan mulai dengan membahas mengenai persoalan nilai awal. Salah satu metode yang
dapat digunakan adalah metode satu langkah. Metode ini termasuk metode eksplisit
di mana solusi aproksimasi memiliki bentuk sebagai berikut:
\begin{equation*}
y_{i+1} = y_{i} + \phi h
\end{equation*}
Estimasi kemiringan $\phi$ digunakan untuk mengekstrapolasi nilai lama
$y_{i}$ ke nilai baru $y_{i+1}$ pada jarak atau ukuran langkah $h$.
Perhatikan bahwa solusi pada titik ke-$(i+1)$ hanya bergantung pada informasi solusi
pada titik ke-$i$.


\section{Metode Satu Langkah}

\section{Metode Euler}

Metode Euler: menggunakan turunan pertama sebagai estimasi kemiringan pada
$x_i$:
\begin{equation*}
\phi = f(x_i, y_i)
\end{equation*}
sehingga diperoleh:
\begin{equation}
y_{i+1} = y_{i} + f(x_i, y_i) h
\label{eq:euler_1step}
\end{equation}


\begin{soal}[Chapra Contoh 25.1]
Gunakan metode Euler untuk menyelesaikan persamaan diferensial berikut:
\begin{equation*}
\frac{\mathrm{d}y}{\mathrm{d}x} = -2x^3 + 12x^2 - 20x + 8.5
\end{equation*}
dari $x=0$ sampai $x=4$ dengan ukuran langkah 0.5.
Syarat awal adalah $y(x=0) = 1$.
Bandingkan dengan solusi eksak:
\begin{equation*}
y(x) = -0.5x^4 + 4x^3 - 10x^2 + 8.5x + 1
\end{equation*}
\end{soal}

Implementasi Python, versi manual:
\begin{pythoncode}
# initial cond
x0 = 0.0
y0 = 1.0

# Using step of 0.5, starting from x0 and y0
x = x0
y = y0
h = 0.5
xp1 = x + h # we are searching for solution at x = 0.5
ϕ = -2*x**3 + 12*x**2 - 20*x + 8.5
yp1 = y + ϕ*h
y_true = -0.5*xp1**4 + 4*xp1**3 - 10*xp1**2 + 8.5*xp1 + 1
ε_t = (y_true - yp1)/y_true * 100
print("First step : x = %f y_true = %.5f y = %.5f ε_t = %.1f %%" % (xp1, y_true, yp1, ε_t))

# Second step
x = xp1 # x from the previous step
y = yp1 # y from the previoud step
h = 0.5
xp1 = x + h # we are searching for solution at x = 1.0
ϕ = -2*x**3 + 12*x**2 - 20*x + 8.5
yp1 = y + ϕ*h
y_true = -0.5*xp1**4 + 4*xp1**3 - 10*xp1**2 + 8.5*xp1 + 1
ε_t = (y_true - yp1)/y_true * 100
print("Second step: x = %f y_true = %.5f y = %.5f ε_t = %.1f %%" % (xp1, y_true, yp1, ε_t))
\end{pythoncode}

Contoh keluaran:
\begin{textcode}
First step : x = 0.500000 y_true = 3.21875 y = 5.25000 ε_t = -63.1 %
Second step: x = 1.000000 y_true = 3.00000 y = 5.87500 ε_t = -95.8 %
\end{textcode}


Karena skema metode Euler yang diberikan pada Persamaan \eqref{eq:euler_1step}
akan digunakan berulang kali, maka sebaiknya skema ini diimplementasikan dalam
suatu fungsi.
\begin{pythoncode}
# One-step aplication of Euler's method for ODE
def ode_euler_1step(dfunc, xi, yi, h):
    return yi + dfunc(xi,yi)*h
\end{pythoncode}

Kode program:
\begin{pythoncode}
# ... Tambahkan atau import fungsi-fungsi yang diperlukan

# the left hand side of dy/dx=... (in general depends on x and y)
# In the present case it only depends on x
def deriv(x, y):
    return -2*x**3 + 12*x**2 - 20*x + 8.5

def exact_sol(x):
    return -0.5*x**4 + 4*x**3 - 10*x**2 + 8.5*x + 1

# initial cond
x0 = 0.0
y0 = 1.0

# Using step of 0.5, starting from x0 and y0
x = x0
y = y0
h = 0.5
xp1 = x + h # we are searching for solution at x = 0.5
yp1 = ode_euler_1step(deriv, x, y, h)
y_true = exact_sol(xp1)
ε_t = (y_true - yp1)/y_true * 100
print("First step : x = %f y_true = %.5f y = %.5f ε_t = %.1f %%" % (xp1, y_true, yp1, ε_t))

# Second step
x = xp1 # x from the previous step
y = yp1 # y from the previoud step
h = 0.5
xp1 = x + h # we are searching for solution at x = 1.0
yp1 = ode_euler_1step(deriv, x, y, h)
y_true = exact_sol(xp1)
ε_t = (y_true - yp1)/y_true * 100
print("Second step: x = %f y_true = %.5f y = %.5f ε_t = %.1f %%" % (xp1, y_true, yp1, ε_t))
\end{pythoncode}
Keluaran yang diperoleh seharusnya sama dengan cara manual (tanpa fungsi).


\subsection*{Analisis kesalahan metode Euler}

Kesalahan pada metode Euler:
\begin{itemize}
\item Kesalahan pemotongan (truncation)
\item Kesalahan pembulatan (round-off)
\end{itemize}

Kesalahan pemotongan terdiri dari dua bagian:
\begin{itemize}
\item kesalahan pemotongan lokal yang berasal dari aplikasi metode pada satu
ukuran langkah.
\item kesalahan pemotongan perambatan (propagated truncation error) yang diakibatkan
aproksimasi pada langkah-langkah sebelumnya.
\end{itemize}
Penjumlahan dari kedua jenis kesalahan tersebut adalah kesalahan pemotongan global.

Misalkan, persamaan diferensial yang akan diintegralkan adalah:
\begin{equation*}
y' = \frac{\mathrm{d}y}{\mathrm{d}x} = f(x,y)
\end{equation*}
Jika solusi, yaitu fungsi yang mendeskripsikan $y$, memiliki turunan yang kontinu, maka
deret Taylor dapat digunakan untuk merepresentasikan fungsi di sekitar nilai awal
$(x_i, y_i)$:
\begin{equation*}
y_{i+1} = y_{i} + y'_{i} h + \frac{y''_{i}}{2!}h^2 + \cdots + \frac{y^{(n)}_{i}}{n!} h^n + R_n
\end{equation*}
dengan $h = x_{i+1} - x_{i}$ dan $R_n$ adalah suku sisa yang didefinisikan sebagai:
\begin{equation*}
R_n = \frac{y^{n+1}(\xi)}{(n+1)!} h^{n+1}
\end{equation*}
di mana $\xi$ berada di dalam interval $[x_i, x_{i+1}]$.
Alternatif penulisan:
\begin{equation*}
y_{i+1} = y_{i} + f(x_i, y_i) h + \frac{f'(x_i, y_i)}{2!}h^2 + \cdots +
\frac{f^{(n-1)}(x_i, y_i)}{n!} h^n + \mathcal{O}(h^{n+1})
\end{equation*}
di mana $\mathcal{O}(h^{n+1})$ menyatakan kontribusi kesalahan pemotongan lokal
yang sebanding dengan ukuran langkah pangkat $(n+1)$.

Dengan membandingkan persamaan ini dengan skema metode Euler, diperoleh
kesalahan pemotongan pada metode Euler sebagai berikut:
\begin{equation*}
E_{t} = \frac{f'(x_i, y_i)}{2!} h^2 + \cdots + \mathcal{O}(h^{n+1})
\end{equation*}
untuk nilai $h$ yang cukup kecil, dapat digunakan hanya satu suku saja:
\begin{equation*}
E_{a} = \frac{f'(x_i, y_i)}{2!} h^2
\end{equation*}
atau
\begin{equation*}
E_a = \mathcal{O}(h^2)
\end{equation*}
di mana $E_a$ adalah aproksimasi kesalahan pemotongan lokal.

Dengan loop, perhitungan error pemotongan
\begin{pythoncode}
# .... definisi deriv
# .... definisi ode_euler_1step
# .... definisi exact_sol

# For local truncation errors
from math import factorial
def trunc_err_second(x,y,h):
    return (-6*x**2 + 24*x - 20)*h**2/factorial(2)

def trunc_err_third(x,y,h):
    return (-12*x + 24)*h**3/factorial(3)

def trunc_err_fourth(x,y,h):
    return -12*h**4/factorial(4)

# initial cond
x0 = 0.0
y0 = 1.0

print("   x     y_true   y_Euler       ε_t   ε_t local")
print("-----------------------------------------------")

print("%5.1f  %8.5f  %8.5f" % (x0, y0, y0)) # Initial cond

x = x0
y = y0
h = 0.5
for i in range(0,8):
    xp1 = x + h
    yp1 = ode_euler_1step(deriv, x, y, h)
    y_true = exact_sol(xp1)
    ε_t = (y_true - yp1)/y_true * 100
    ε_t_local = (trunc_err_second(x,y,h) + trunc_err_third(x,y,h) + \
        trunc_err_fourth(x,y,h))/y_true * 100
    print("%5.1f  %8.5f  %8.5f  %8.1f%%  %8.1f%%" % (xp1, y_true, yp1, ε_t, ε_t_local))
    # Update x and y for the next step
    x = xp1
    y = yp1
\end{pythoncode}

Contoh keluaran:
\begin{textcode}
   x     y_true   y_Euler       ε_t   ε_t local
-----------------------------------------------
  0.0   1.00000   1.00000
  0.5   3.21875   5.25000     -63.1%     -63.1%
  1.0   3.00000   5.87500     -95.8%     -28.1%
  1.5   2.21875   5.12500    -131.0%      -1.4%
  2.0   2.00000   4.50000    -125.0%      20.3%
  2.5   2.71875   4.75000     -74.7%      17.2%
  3.0   4.00000   5.87500     -46.9%       3.9%
  3.5   4.71875   7.12500     -51.0%     -11.3%
  4.0   3.00000   7.00000    -133.3%     -53.1%
\end{textcode}
    




Chapra Contoh 25.3

\begin{pythoncode}
def ode_euler(dfunc, x0, y0, h, Nstep):
    x = np.zeros(Nstep+1)
    y = np.zeros(Nstep+1)
    # Start with initial cond
    x[0] = x0
    y[0] = y0
    for i in range(0,Nstep):
        x[i+1] = x[i] + h
        y[i+1] = ode_euler_1step(dfunc, x[i], y[i], h)
    return x, y
\end{pythoncode}

Kode program
\begin{pythoncode}
# .... Definisi dan/atau import fungsi-fungsi yang diperlukan

# initial cond
x0 = 0.0
y0 = 1.0
xf = 4.0

# Using h=0.5
h = 0.5
Nstep = int(xf/h)
xs1, ys1 = ode_euler(deriv, x0, y0, h, Nstep)

# Using h=0.25
h = 0.25
Nstep = int(xf/h)
xs2, ys2 = ode_euler(deriv, x0, y0, h, Nstep)

# True solution
x_true = np.linspace(0.0, 4.0, 200)
y_true = exact_sol(x_true)

# Plot
plt.clf()
plt.plot(xs1, ys1, label="h=0.5", marker="o")
plt.plot(xs2, ys2, label="h=0.5", marker="o")
plt.plot(x_true, y_true, label="true") # Do not show the markers
plt.legend()
plt.tight_layout()
plt.savefig("IMG_example_25_3.png", dpi=150)
plt.savefig("IMG_example_25_3.pdf")
\end{pythoncode}

\begin{figure}[h]
{\centering
\includegraphics[scale=0.7]{../../chapra_7th/ch25/IMG_chapra_example_25_3.pdf}
\par}
\end{figure}


\subsection{Metode Heun}

Pad metode Euler, turunan fungsi pada titik ke-$i$ 
\begin{equation*}
y'_{i} = f(x_i, y_i)
\end{equation*}
digunakan untuk mendapatkan solusi pada titik ke-$(i+1)$ (atau ekstrapolasi):
\begin{equation*}
y^{0}_{i+1} = y_{i} + f(x_i, y_i) h
\end{equation*}
Untuk metode Heun, $y^{0}_{i+1}$ tidak digunakan sebagai solusi (seperti pada metode Euler)
namun digunakan sebagai prediksi intermediat.
Persamaan ini dikenal sebagai persamaan prediktor.
Persamaan ini memberikan estimasi $y_{i+1}$ pada $x_{i+1}$:
\begin{equation*}
y'_{i+1} = f(x_{i+1}, y^{0}_{i+1})
\end{equation*}
Dengan menggunakan rata-rata dari nilai kemiringan ini diperoleh:
\begin{equation*}
\overline{y}' = \frac{y'_{i} + y'_{i+1}}{2} = 
\frac{f(x_i, y_i) + f(x_{i+1},y^{0}_{i+1})}{2}
\end{equation*}
Kemiringan rata-rata ini kemudian digunakan untuk ekstrapolasi linear dari $y_{i}$ ke
$y_{i+1}$:
\begin{equation*}
y_{i+1} = y_{i} + \frac{f(x_i, y_i) + f(x_{i+1}, y^{0}_{i+1})}{2} h
\end{equation*}
Persamaan ini dikenal sebagai persamaan korektor.

Metode Heun termasuk ke dalam kelompok metode prediktor-korektor. Metode Heun dapat
dituliskan sebagai berikut.
\begin{align}
y^{0}_{i+1} & = y_{i} + f(x_i, y_i) h & \text{   Prediktor}\\
y_{i+1} & = y_{i} + \frac{f(x_i, y_i) + f(x_{i+1}, y^{0}_{i+1})}{2} h & \text{   Korektor}
\end{align}
Karena $y_{i+1}$ berada pada kedua ruas dari persamaan korektor, maka kita harus mengaplikasikan
persamaan tersebut secara iteratif.

Jika tidak menggunakan skema iteratif, diperoleh metode Heun non-iteratif:
\begin{pythoncode}
# One-step application of Heun's method for ODE
def ode_heun_1step(dfunc, xi, yi, h):
    y0ip1 = yi + dfunc(xi,yi)*h
    avg = 0.5*( dfunc(xi,yi) + dfunc(xi+h,y0ip1) )*h
    return yi + avg
\end{pythoncode}


\begin{soal}[Chapra Contoh 25.5]
Gunakan metode Heun untuk mengintegrasikan persamaan diferensial
\begin{equation*}
\frac{\mathrm{d}y}{\mathrm{d}x} = 4e^{0.8x} - 0.5y
\end{equation*}
dari $x=0$ sampai $x=4$ dengan ukuran langkah 1. Syarat awal adalah
$y(x=0) = 2$.
Bandingkan dengan solusi analitik:
\begin{equation*}
y = \frac{4}{1.3} \left( e^{0.8x} - e^{-0.5x} \right) + 2e^{-0.5x}
\end{equation*}
\end{soal}


Berikut ini adalah implementasi Python menggunakan metode Heun non-iteratif:
\begin{pythoncode}
# .... import library dan/atau definisi fungsi

def deriv(x, y):
    return 4*exp(0.8*x) - 0.5*y
    
def exact_sol(x):
    return 4.0/1.3*( exp(0.8*x) - exp(-0.5*x) ) + 2*exp(-0.5*x)

x0 = 0.0; y0 = 2.0 # Initial cond
xf = 4.0
h = 1.0
Nstep = int(xf/h)
    
x = x0; y = y0
for i in range(0,Nstep):
    xp1 = x + h
    yp1 = ode_heun_1step(deriv, x, y, h) # or use the iterative one
    y_true = exact_sol(xp1)
    ε_t = (y_true - yp1)/y_true * 100
    print("%f %12.7f %12.7f  %5.2f%%" % (xp1, y_true, yp1, abs(ε_t)))
    # For the next step
    x = xp1
    y = yp1    
\end{pythoncode}



Berikut ini adalah hasil yang diperoleh jika kita menggunakan metode
non-iteratif:
\begin{textcode}
------------------------------------------
  x         y_true        y_Heun      ε_t
------------------------------------------
0.000000    2.0000000
1.000000    6.1946314    6.7010819   8.18%
2.000000   14.8439219   16.3197819   9.94%
3.000000   33.6771718   37.1992489  10.46%
4.000000   75.3389626   83.3377673  10.62%  
\end{textcode}


Berikut ini adalah implementasi metode Heun secara iteratif.
\begin{pythoncode}
# One-step application of Heun's method for ODE
# Using iterative steps to determine y0ip1
def ode_heun_1step_iterative(dfunc, xi, yi, h, NiterMax=100, Δ=1e-6):
    y0ip1 = yi + dfunc(xi,yi)*h
    y0ip1_old = y0ip1
    for i in range(NiterMax+1):
        avg = 0.5*( dfunc(xi,yi) + dfunc(xi+h,y0ip1) )*h
        y0ip1 = yi + avg
        diff = abs(y0ip1 - y0ip1_old)
        # Uncomment this to see the iteration process
        #print("iter: %2d y0ip1 = %12.7f  diff = %12.7e" % (i+1, y0ip1, diff))
        if diff <= Δ:
            break
        y0ip1_old = y0ip1
    return y0ip1
\end{pythoncode}

Berikut ini adalah hasil dari aplikasi metode Heun iteratif. Error yang dihasilkan lebih
kecil daripada metode Heun non-iteratif.
\begin{textcode}
------------------------------------------
  x         y_true      y_Heun_iter    ε_t
------------------------------------------
0.000000    2.0000000
1.000000    6.1946314    6.3608654   2.68%
2.000000   14.8439219   15.3022364   3.09%
3.000000   33.6771718   34.7432760   3.17%
4.000000   75.3389626   77.7350961   3.18%
\end{textcode}

\subsection{Metode titik tengah}

Metode titik tengah (midpoint atau improved polygon)
menggunakan metode Euler untuk memprediksi nilai $y$ pada titik tengah
interval:
\begin{equation*}
y_{i+1/2} = y_{i} + f(x_i, y_i) \frac{h}{2}
\end{equation*}
Kemudian nilai ini digunakan untuk menghitung kemiringan pada titik tengah:
\begin{equation*}
y'_{i+1/2} = f(x_{i+1/2}, y_{i+1/2})
\end{equation*}
yang diasumsikan sebagai aproksimasi dari rata-rata kemiringan untuk pada interval.
Kemiringan ini kemudian digunakan untuk mengekstrapolasi linear dari $x_i$ ke $x_{i+1}$:
\begin{equation}
y_{i+1} = y_{i} + f(x_{i+1/2}, y_{i+1/2}) h
\end{equation}

Implementasi:
\begin{pythoncode}
def ode_midpoint_1step(dfunc, xi, yi, h):
    yip12 = yi + deriv(xi,yi)*h/2  # midpoint value
    xip12 = xi + 0.5*h             # midpoint
    yip1 = yi + deriv(xip12,yip12)*h
    return yip1
\end{pythoncode}
\subsection{Metode Ralston}

Skema:
\begin{equation*}
y_{i+1} = y_i + \left( \frac{1}{3}k_1 + \frac{2}{3}k_2 \right) h
\end{equation*}
dengan
\begin{align*}
k_1 & = f(x_i, y_i) \\
k_2 & = f\left( x_i + \frac{3}{4}h, y_i + \frac{3}{4}k_1 h \right)
\end{align*}

\begin{pythoncode}
def ode_ralston_1step(dfunc, xi, yi, h):
    k1 = dfunc(xi, yi)
    k2 = dfunc(xi + 3*h/4, yi + 3*k1*h/4)
    yip1 = yi + (k1/3 + 2*k2/3)*h
    return yip1
\end{pythoncode}

\section{Metode Runge-Kutta Orde 4}

\begin{pythoncode}
def ode_rk4_1step(dfunc, xi, yi, h):
    k1 = dfunc(xi, yi)
    k2 = dfunc(xi + 0.5*h, yi + 0.5*k1*h)
    k3 = dfunc(xi + 0.5*h, yi + 0.5*k2*h)
    k4 = dfunc(xi + h, yi + k3*h)
    yip1 = yi + (k1 + 2*k2 + 2*k3 + k4)*h/6
    return yip1
\end{pythoncode}


\section{Metode Runge-Kutta Orde 5}

\begin{pythoncode}
def ode_rk5_1step(dfunc, xi, yi, h):
    k1 = dfunc(xi, yi)
    k2 = dfunc(xi + h/4, yi + k1*h/4)
    k3 = dfunc(xi + h/4, yi + k1*h/8 + k2*h/8)
    k4 = dfunc(xi + h/2, yi - k2*h/2 + k3*h)
    k5 = dfunc(xi + 3*h/4, yi + 3*k1*h/16 + 9*k4*h/16)
    k6 = dfunc(xi + h, yi - 3*k1*h/7 + 2*k2*h/7 + 12*k3*h/7 - 12*k4*h/7 + 8*k5*h/7)
    yip1 = yi + (7*k1 + 32*k3 + 12*k4 + 32*k5 + 7*k6)*h/90
    return yip1
\end{pythoncode}


\section{Subrutin umum untuk metode RK}

\begin{pythoncode}
def ode_solve(dfunc, do_1step, x0, y0, h, Nstep):
    Nvec = len(y0)
    x = np.zeros(Nstep+1)
    y = np.zeros((Nstep+1,Nvec))
    # Start with initial cond
    x[0] = x0
    y[0,:] = y0[:]
    for i in range(0,Nstep):
        x[i+1] = x[i] + h
        y[i+1,:] = do_1step(dfunc, x[i], y[i,:], h)
    return x, y
\end{pythoncode}



\section{Sistem Persamaan Diferensial Orde-1}

Skema satu langkah yang sudah dipelajari sebelumnya juga dapat digunakan
untuk sistem persamaan diferensial orde 1.

\begin{soal}[Chapra Contoh 25.9]
Cari solusi numerik dari sistem persamaan diferensial berikut:
\begin{align*}
\frac{\mathrm{d}y_1}{\mathrm{d}x} & = -0.5 y_1 \\
\frac{\mathrm{d}y_2}{\mathrm{d}x} & = 4 - 0.3 y_2 - 0.1y_1
\end{align*}
dengan syarat awal pada $x=0$, $y_1 = 4$ dan $y_2 = 6$.
Cari solusi sampai pada $x=2$ dengan ukuran langkah 0.5.
\end{soal}

Berikut ini adalah program yang dapat kita gunakan.
\begin{pythoncode}
import numpy as np

def deriv(x, y):
    Nvec = len(y)
    # Here we make an assertion to make sure that y is a 2-component vector
    # Uncomment this line if the code appears to be slow
    assert Nvec == 2
    # Output array
    dydx = np.zeros(Nvec)
    # remember that in Python the array starts at 0
    # y1 = y[0]
    # y2 = y[1]
    dydx[0] = -0.5*y[0]
    dydx[1] = 4 - 0.3*y[1] - 0.1*y[0]
    # 
    return dydx
    
# One-step application of Euler's method for ODE
def ode_euler_1step(dfunc, xi, yi, h):
    return yi + dfunc(xi,yi)*h
    
def ode_solve(dfunc, do_1step, x0, y0, h, Nstep):
    Nvec = len(y0)
    x = np.zeros(Nstep+1)
    y = np.zeros((Nstep+1,Nvec))
    # Start with initial cond
    x[0] = x0
    y[0,:] = y0[:]
    for i in range(0,Nstep):
        x[i+1] = x[i] + h
        y[i+1,:] = do_1step(dfunc, x[i], y[i,:], h)
    return x, y
    
# initial cond
x0 = 0.0
y0 = np.array([4.0, 6.0])

h = 0.5
Nstep = 4
x, y = ode_solve(deriv, ode_euler_1step, x0, y0, h, Nstep)
print("")
print("---------------------------")
print(" x         y1         y2")
print("---------------------------")
for i in range(len(x)):
    print("%5.1f %10.6f %10.6f" % (x[i], y[i,0], y[i,1]))    
\end{pythoncode}

Contoh keluaran:
\begin{textcode}
---------------------------
   x       y1         y2
---------------------------
  0.0   4.000000   6.000000
  0.5   3.000000   6.900000
  1.0   2.250000   7.715000
  1.5   1.687500   8.445250
  2.0   1.265625   9.094088
\end{textcode}

Beberapa catatan:
\begin{itemize}
\item Tidak ada perubahan pada definisi fungsi \pyinline{ode_euler_1step}
\item Fungsi yang mendefinisikan (sistem) persamaan diferensial sekarang
mengembalikan array satu dimensi atau vektor.
\item Kita telah mendefinisikan fungsi \pyinline{ode_solve} yang dapat dikombinasikan
dengan metode satu langkah yang lain seperti \pyinline{ode_rk4_1step}.
\end{itemize}

\begin{soal}[Chapra Contoh 25.10]
Gunakan metode Runge-Kutta orde-4 untuk sistem persamaan diferensial yang didefinisikan
pada soal sebelumnya (Chapra Contoh 25.9).
\end{soal}

Contoh keluaran:
\begin{textcode}
---------------------------
   x        y1         y2
---------------------------
  0.0   4.000000   6.000000
  0.5   3.115234   6.857670
  1.0   2.426171   7.632106
  1.5   1.889523   8.326886
  2.0   1.471577   8.946865
\end{textcode}



\begin{soal}[Chapra Contoh 25.11]
Cari solusi numerik dari sistem persamaan diferensial berikut:
\begin{align*}
\frac{\mathrm{d}y_1}{\mathrm{d}x} & = y_2 \\
\frac{\mathrm{d}y_2}{\mathrm{d}x} & = -16.1 y_1 \\
\frac{\mathrm{d}y_3}{\mathrm{d}x} & = y_4 \\
\frac{\mathrm{d}y_4}{\mathrm{d}x} & = -16.1 \sin(y_3)
\end{align*}
untuk kasus-kasus syarat awal ($x=0$) berikut
\begin{itemize}
\item Pergeseran kecil: $y_1 = y_3 = 0.1$ radian, $y_2 = y_4 = 0$
\item Pergeseran besar: $y_1 = y_3 = \pi/4$ radian, $y_2 = y_4 = 0$.
\end{itemize}
\end{soal}


Anda dapat melengkapi kode berikut:
\begin{pythoncode}
# .... import dan definisi fungsi

def pendulum_ode(x, y):
    Nvec = len(y)
    # Here we make an assertion to make sure that y is a 4-component vector
    # Uncomment this line if the code appears to be slow
    assert Nvec == 4
    # Output array
    dydx = np.zeros(Nvec)
    # remember that in Python the array starts at 0
    # y1 = y[0]
    # y2 = y[1], etc ...
    dydx[0] = y[1]
    dydx[1] = -16.1*y[0]
    # Nonlinear effect
    dydx[2] = .... # LENGKAPI
    dydx[3] = .... # LENGKAPI
    # 
    return dydx


# initial cond
x0 = 0.0
y0 = np.array([0.1, 0.0, 0.1, 0.0]) # Small displacement

h = 0.01 # try playing with this parameter
xf = 4.0
Nstep = int(xf/h)
x, y = ode_solve(pendulum_ode, ode_rk4_1step, x0, y0, h, Nstep)

plt.clf()
plt.plot(x, y[:,0], label="y1")
plt.plot(x, y[:,1], label="y2")
plt.plot(x, y[:,2], label="y3")
plt.plot(x, y[:,3], label="y4")
plt.title("Small displacement case")
plt.ylim(-4,4) # The same for both small and large displacement
plt.legend()
plt.tight_layout()
plt.grid(True)


# initial cond
x0 = 0.0
y0 = np.array([np.pi/4, 0.0, np.pi/4, 0.0]) # Large displacement

# .... same as before

plt.clf()
# .... same as before
plt.title("Large displacement case")
# .... same as before
\end{pythoncode}

Contoh hasil visualisasi dapat dilihat pada Gambar \ref{fig:chapra_example_25_11}.

\begin{figure}[h]
{\centering
\includegraphics[width=0.45\textwidth]{../../chapra_7th/ch25/IMG_chapra_example_25_11_small.pdf}
\includegraphics[width=0.45\textwidth]{../../chapra_7th/ch25/IMG_chapra_example_25_11_large.pdf}
\par}
\caption{Chapra Contoh 25.11}
\label{fig:chapra_example_25_11}
\end{figure}

\section{Persoalan Nilai Batas}

Sebagai contoh persoalan nilai batas kita akan meninjau masalah distribusi
panas pada suatu batang panjang. Kedua ujung dari batang terinsulasi dan
dijaga pada satu suhu tertentu. Bagian lain dari batang tidak terinsulasi
sehingga memungkikan terjadinya pertukaran energi.
Distribusi temperatur pada batang dapat dijelaskan dengan persamaan diferensial
orde 2 berikut.
\begin{equation}
\frac{\mathrm{d}^2 T}{\mathrm{d}x^2} + h'(T_a - T) = 0
\label{eq:chapra_eq_27_1}
\end{equation}
dengan syarat batas: $T(x=0) = T_1$ dan $T(x=L) = T_2$.


\subsection{Metode \textit{shooting}}

Pada metode \textit{shooting}, permasalahan nilai batas diubah menjadi permasalahan
nilai awal. Solusi kemudian dicari dengan menggunakan metode \textit{trial-and-error}
pada.

\begin{soal}[Chapra Contoh 27.1]
Untuk $L=10$, $T_a = 20$, $T_1 = 40$, $T_2 = 200$, dan $h' = 0.01$, solusi
dari Persamaan \eqref{eq:chapra_eq_27_1} adalah.
\begin{equation*}
T(x) = 73.4523 e^{0.1x} - 53.4523 e^{-0.1x} + 20
\end{equation*}
Gunakan metode \textit{shooting} untuk menyelesaikan persamaan ini secara numerik
\end{soal}


Kita mengubah persamaan diferensial orde-2 menjadi sistem persamaan diferensial
orde-1 sebagai berikut:
\begin{align*}
\frac{\mathrm{d}T}{\mathrm{d}x} & = z \\
\frac{\mathrm{d}z}{\mathrm{d}x} & = h'(T - T_a)
\end{align*}
Berikut ini adalah implementasi dari persamaan di atas:
\begin{pythoncode}
# T == y[0]
# dT/dx == y[1]
# d2T/dx2 == dydx[1]
def deriv(x, y):
    #
    Nvec = len(y)
    assert Nvec == 2
    dydx = np.zeros(Nvec)
    # Parameters
    h = 0.01
    T_a = 20.0
    #
    dydx[0] = y[1]
    dydx[1] = h*(y[0] - T_a)
    return dydx
\end{pythoncode}

Kita juga mendefinisikan beberapa variabel yang akan digunakan nanti.
\begin{pythoncode}
x0 = 0.0 # Initial cond
y0 = np.zeros(2) # y0[0] should be equal to T(x=) and y0[1] will be set later
xf = 10.0  # end interval
Tf = 200.0 # Boundary condition, T(10) = 200
\end{pythoncode}


Nilai $T(x=0)$ sudah diberikan. Akan tetapi untuk menyelesaikan persamaan nilai
awal, kita juga memerlukan nilai $\frac{\mathrm{d}T}{\mathrm{d}x} = z$ pada $x=0$.
Misalkan kita menebaknya dengan nilai $z(0) = 10$.
Dengan menggunakan metode Runge-Kutta orde-4 dengan ukuran langkah 2 kita dapat memperoleh
nilai $T(x=L)$. Potongan kode berikut ini dapat digunakan:
\begin{pythoncode}
y0[0] = 40.0 # from the boundary condition, T(0) = 40
z0_1 = 10.0 # Guess for z = dT/dx == y[1], save to it variable for later use
y0[1] = z0_1

h = 2.0 # Step size
Nstep = int( (xf-x0)/h )
x, y = ode_solve(deriv, ode_rk4_1step, x0, y0, h, Nstep)
# At the end of the interval
Tf_1 = y[-1,0]
print("First guess: T(10) = y[-1,0] = ", Tf_1)
\end{pythoncode}
Keluaran:
\begin{textcode}
First guess: T(10) = y[-1,0] =  168.37965867134406
\end{textcode}
Diperleh $T(x=L)$ sekitar 168.3797. Ini masih berbeda dengan nilai yang kita perlukan
yaitu 200.

Berikut ini adalah plot dari solusi yang diperoleh (belum memenuhi syarat batas).

{\centering
\includegraphics[scale=0.7]{../../chapra_7th/ch27/IMG_example_27_1_v2_1st.pdf}
\par}

Karena masih berbeda, kita akan menggunakan tebakan lain, misalnya $z=10$:
\begin{pythoncode}
# Integrate again, now with new guess for z(0) = y0[1]
z0_2 = 20.0
y0[1] = z0_2 # Guess for z = dT/dx == y[1]
x, y = ode_solve(deriv, ode_rk4_1step, x0, y0, h, Nstep)
# At the end of the interval
Tf_2 = y[-1,0]
print("Second guess: T(10) = y[-1,0] = ", Tf_2)
\end{pythoncode}
Diperoleh:
\begin{textcode}
Second guess: T(10) = y[-1,0] =  285.89795359537567
\end{textcode}
Hasil yang diperoleh masih belum sama dengan yang diperlukan, yaitu 200.

Berikut ini adalah plot dari solusi yang diperoleh (belum memenuhi syarat batas).

{\centering
\includegraphics[scale=0.7]{../../chapra_7th/ch27/IMG_example_27_1_v2_2nd.pdf}
\par}

Karena persamaan diferensial \eqref{eq:chapra_eq_27_1} adalah linear, maka
dua solusi tersebut saling terkait.
Kita dapat menggunakan interpolasi linear untuk mendapatkan nilai dari $z(0)$.
Berikut adalah data yang sudah diperoleh.

{\centering
\begin{tabular}{|c|c|}
\hline
$z(0) = 10$ & $T(10) = 168.3797$ \\
$z(0) = 20$ & $T(10) = 285.8980$ \\
$z(0) = ??$ & $T(10) = 200$ \\
\hline
\end{tabular}
\par}

Kode Python:
\begin{pythoncode}
# Using linear interp to guess what value of z(0) which gives T(10) = 200
z0_new = z0_1 + (z0_2 - z0_1)/(Tf_2 - Tf_1) * (Tf - Tf_1)
print("z0_new = ", z0_new)
\end{pythoncode}
Diperoleh:
\begin{textcode}
z0_new =  12.690673937117328
\end{textcode}

Nilai ini dapat digunakan sebagai syarat awal.
\begin{pythoncode}
# Now solve the IVP using z0_new
y0[1] = z0_new # Guess for z = dT/dx == y[1]
x, y = ode_solve(deriv, ode_rk4_1step, x0, y0, h, Nstep)
# At the end of the interval
Tf_3 = y[-1,0]
print("Third guess: T(10) = y[-1,0] = ", Tf_3)
\end{pythoncode}
Akhirnya kita mendpatkan nilai yang diinginkan:
\begin{textcode}
Third guess: T(10) = y[-1,0] =  200.0
\end{textcode}

Solusi yang diperoleh dapat diplot pada gambar berikut (sudah memenuhi syarat batas).

{\centering
\includegraphics[scale=0.7]{../../chapra_7th/ch27/IMG_example_27_1_v2_3rd.pdf}
\par}


Berikut ini perbandingannya dengan solusi eksak:
\begin{pythoncode}
def exact_sol(x):
    #return 73.4532*np.exp(0.1*x) - 53.4523*np.exp(-0.1*x) + 20.0 # from book
    # Using SymPy
    return 20*((1 - np.exp(2))*np.exp(x/10) + (1 - 9*np.e)*np.exp(x/5) + \
            np.e*(9 - np.e))*np.exp(-x/10)/(1 - np.exp(2))

T_exact = exact_sol(x)
T_num = y[:,0]
error = np.abs(T_exact - T_num)
for i in range(len(x)):
    print("%18.10f %18.10f %18.10f %15.10e" % (x[i], T_num[i], T_exact[i], error[i]))
\end{pythoncode}

Hasil keluaran ($x$, numerik, eksak, error):
\begin{textcode}
    0.0000000000      40.0000000000      40.0000000000 0.0000000000e+00
    2.0000000000      65.9518901934      65.9517913981 9.8795269608e-05
    4.0000000000      93.7479650466      93.7477895327 1.7551381092e-04
    6.0000000000     124.5037505132     124.5035454074 2.0510586421e-04
    8.0000000000     159.4535539523     159.4533954960 1.5845629972e-04
   10.0000000000     200.0000000000     200.0000000000 0.0000000000e+00
\end{textcode}


Untuk persamaan diferensial nonlinear, interpolasi linear biasanya
tidak cukup untuk mendapatkan solusi yang memenuhi nilai batas.
Kita perlu menggunakan metode pencarian akar seperti metode bagi
dua untuk mendapatkan solusi.

\begin{soal}
Dengan menggunakan parameter numerik yang sama dengan soal sebelumnya, namun
persamaan diferensial yang digunakan adalah:
\begin{equation*}
\frac{\mathrm{d}^2 T}{\mathrm{d}x^2} + h''(T_a - T)^4 = 0
\end{equation*}
dengan $h'' = 5\times 10^{-8}$
\end{soal}


\begin{pythoncode}
# .... import dan definisi fungsi yang digunakan

# T == y[0]
# dT/dx == y[1]
# d2T/dx2 == dydx[1]
def deriv(x, y):
    Nvec = len(y)
    assert Nvec == 2
    dydx = np.zeros(Nvec)
    h = 5e-8
    T_a = 20.0
    dydx[0] = y[1]
    dydx[1] = h*(y[0] - T_a)**4
    return dydx

def obj_func(z0_guess):
    # Initial cond
    x0 = 0.0
    y0 = np.zeros(2)
    y0[0] = 40.0 # from the boundary condition, T(0) = 40
    y0[1] = z0_guess
    xf = 10.0  # end interval
    Tf = 200.0 # Boundary condition, T(10) = 200
    #
    h = 2.0 # Step size
    Nstep = int( (xf-x0)/h )
    x, y = ode_solve(deriv, ode_rk4_1step, x0, y0, h, Nstep)
    # At the end of the interval
    Tf_guess = y[-1,0]
    return Tf_guess - Tf


# For testing values of z0_1 and z0_2 which brackets obj_func
z0_1 = 5.0
Tf_1 = obj_func(z0_1)
z0_2 = 11.0
Tf_2 = obj_func(z0_2)
print("Tf_1 = ", Tf_1)
print("Tf_2 = ", Tf_2)

z0 = root_bisection(obj_func, z0_1, z0_2, TOL=1.0e-9)

# Now solve the ODE with the obtained z0
x0 = 0.0
y0 = np.zeros(2)
y0[0] = 40.0 # from the boundary condition, T(0) = 40
y0[1] = z0
xf = 10.0  # end interval
#
h = 2.0 # Step size
Nstep = int( (xf-x0)/h )
x, y = ode_solve(deriv, ode_rk4_1step, x0, y0, h, Nstep)
print("Tf = ", y[-1,0])  # CHECK: Should give a value close to 200.0

# Now plot the solution
plt.clf()
plt.plot(x, y[:,0], marker="o", label="Temperature")
# ....
\end{pythoncode}


Contoh hasil keluaran:
\begin{textcode}
Tf_1 =  -101.16453273552762
Tf_2 =  98.71355087438758
             Iter      Estimated          f(x)
             ----      ---------          ----

bisection:     1       8.0000000000     4.37036e+01
bisection:     2       9.5000000000     6.70737e+00
......  # output removed
bisection:    36       9.3398893460     3.14060e-10

bisection is converged in 36 iterations
Tf =  200.00000000031406
\end{textcode}


Berikut ini adalah plot solusi yang diperoleh.

{\centering
\includegraphics[scale=0.7]{../../chapra_7th/ch27/IMG_example_27_2.pdf}
\par}

Anda dapat menggunakan jumlah titik yang lebih banyak (ukuran langkah yang lebih
kecil untuk mendapatkan solusi yang lebih akurat).

Implementasi \pyinline{root_bisection} (jika diperlukan)
\begin{pythoncode}
from math import ceil, log10

def root_bisection(f, x1, x2, TOL=1.0e-9, NiterMax=None ):
    f1 = f(x1)
    if abs(f1) <= TOL:
        return x1, 0.0
    f2 = f(x2)
    if abs(f2) <= TOL:
        return x2, 0.0
    if f1*f2 > 0.0:
        raise RuntimeError("Root is not bracketed")

    # No NiterMax is provided
    # We calculate the default value here.
    if NiterMax == None:
        NiterMax = int(ceil( log10(abs(x2-x1)/TOL) )/ log10(2.0) ) + 10
        # extra 10 iterations

    # For the purpose of calculating relative error
    x3 = 0.0
    x3_old = 0.0

    print(13*" "+"Iter      Estimated          f(x)")
    print(13*" "+"----      ---------          ----")
    print("")

    for i in range(1,NiterMax+1):

        x3_old = x3
        x3 = 0.5*(x1 + x2)
        f3 = f(x3)

        print("bisection: %5d %18.10f %15.5e" % (i, x3, abs(f3)))

        if abs(f3) <= TOL:
            print("")
            print("bisection is converged in %d iterations" % i)
            # return the result
            return x3

        if f2*f3 < 0.0:
            # sign of f2 and f3 is different
            # root is in [x2,x3]
            # change the interval bound of x1 to x3
            x1 = x3
            f1 = f3
        else:
            # sign of f1 and f3 is different
            # root is in [x1,x3]
            # change the interval bound of x2 to x3
            x2 = x3
            f2 = f3

    print("No root is found")
    return None
\end{pythoncode}


\subsection{Metode Beda Hingga}

\begin{soal}
Gunakan metode beda hingga untuk menyelesaikan persoalan nilai
batas pada Chapra Contoh 27.1
\end{soal}

\begin{pythoncode}
# Finite-difference method for linear BVP

# d2T/dx2 + h'(T_a - T) = 0
# Boundary condition:
#   T(0) = 40
#   T(10) = 200

import numpy as np

h = 0.01
T_a = 20.0

x0 = 0.0
T0 = 40.0 # Boundary condition, T(0) = 40
xf = 10.0
Tf = 200.0 # Boundary condition, T(10) = 200
Δx = 2.0 # segment length (or step size in shooting method)
Nstep = int( (xf-x0)/Δx )
Npoints = Nstep + 1

T = np.zeros(Npoints)
T[0] = T0
T[-1] = Tf

# Finite-difference operator of second derivative matrix
# In general we should use sparse matrix. However, because
# The size is rather small, we use full (dense) matrix
# Please refer to the left-hand-side of Eq. 27.3 for the matrix elements.
Npointsm2 = Npoints-2 # Number of interior points
d2dx2 =  np.zeros((Npointsm2,Npointsm2))
for i in range(Npointsm2):
    d2dx2[i,i] = 2 + h*Δx**2
    if i != 0:
        d2dx2[i-1,i] = -1.0
    if i != (Npointsm2-1):
        d2dx2[i+1,i] = -1.0
# Display the matrix
print("FD representation of second-derivative operator:")
print(d2dx2)

# The vector represented by the right hand side of Eq. 27.3
f = np.zeros(Npointsm2)
for i in range(1,Npointsm2-1):
    f[i] = h*Δx**2*T_a
# From the left BC
f[0] = h*Δx**2*T_a + T0
# From the right BC 
f[-1] = h*Δx**2*T_a + Tf
# Display
print("f = ", f)

# Solve the linear equations
T[1:Npoints-1] = np.linalg.solve(d2dx2,f)

def exact_sol(x):
    return 20*((1 - np.exp(2))*np.exp(x/10) + (1 - 9*np.e)*np.exp(x/5) + \
            np.e*(9 - np.e))*np.exp(-x/10)/(1 - np.exp(2))

x = np.zeros(Npoints)
for i in range(Npoints):
    x[i] = x0 + i*Δx
    T_exact = exact_sol(x[i])
    error = abs(T[i] - T_exact)
    print("%18.10f %18.10f %18.10f %18.10e" % (x[i], T[i], T_exact, error))

plt.clf()
plt.plot(x, T, marker="o", label="Temperature")
plt.xlabel("x")
plt.ylabel("T")
plt.legend()
\end{pythoncode}

Contoh keluaran
\begin{textcode}
FD representation of second-derivative operator:
[[ 2.04 -1.    0.    0.  ]
 [-1.    2.04 -1.    0.  ]
 [ 0.   -1.    2.04 -1.  ]
 [ 0.    0.   -1.    2.04]]
f =  [ 40.8   0.8   0.8 200.8]
      0.0000000000      40.0000000000      40.0000000000   0.0000000000e+00
      2.0000000000      65.9698343668      65.9517913981   1.8042968650e-02
      4.0000000000      93.7784621082      93.7477895327   3.0672575481e-02
      6.0000000000     124.5382283340     124.5035454074   3.4682926640e-02
      8.0000000000     159.4795236931     159.4533954960   2.6128197126e-02
     10.0000000000     200.0000000000     200.0000000000   0.0000000000e+00
\end{textcode}





\section{Soal Tambahan}

Chapra Latihan 25.16, 25.17, 25.18, 25.19

Chapra Latihan 27.28, 27.29, 27.30

\end{document}
