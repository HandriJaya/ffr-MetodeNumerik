\subsection{Propagasi galat (\textit{error propagation})}

\textbf{Chapra Contoh 4.6}

Defleksi $y$ dari bagian atas tiang kapal dapat dinyatakan
dengan
\begin{equation}
y = \frac{FL^{4}}{8EI}
\end{equation}
dengan $F$ menyatakan \textit{loading} seragam yang bekerja pada sisi
tiang (N/m), $L$ menyatakan tinggi tiang (m), dan $E$ menyatakan
modulus elastisitas tiang $\textrm{N/m}^{2}$, dan $I$ menyatakan
momen inersia ($\textrm{m}^{4}$). Kita ingin menghitung estimasi kesalahan
pada $y$ dengan menggunakan data-data berikut.
\begin{itemize}
\item $\tilde{F} = 750\,\textrm{N/m}$ dan $\Delta\tilde{F} = 30\,\textrm{N/m}$
\item $\tilde{L} = 9\,\textrm{m}$ dan $\Delta\tilde{L} = 0.03\,\textrm{m}$
\item $\tilde{E} = 7.5\times10^{9}\,\textrm{N/m}^{2}$ dan $\Delta\tilde{E} = 5\times10^{7}\,\textrm{N/m}^{2}$
\item $\tilde{I} = 0.0005\,\textrm{m}^{4}$ dan $\Delta\tilde{I} = 0.000005\,\textrm{m}^{4}$
\end{itemize}

Kita akan menggunakan hasil dari analisis error orde-1 (Persamaan 4.27 pada Chapra), kita
mendapatkan
\begin{equation*}
\Delta y(\tilde{F},\tilde{L},\tilde{E},\tilde{I}) =
\left| \frac{\partial y}{\partial F} \right| \Delta\tilde{F} +
\left| \frac{\partial y}{\partial L} \right| \Delta\tilde{L} +
\left| \frac{\partial y}{\partial E} \right| \Delta\tilde{E} +
\left| \frac{\partial y}{\partial I} \right| \Delta\tilde{I}
\end{equation*}

Kita akan menggunakan SymPy untuk melakukan perhitungan ini.

\begin{pythoncode}
from sympy import *

F, L, E, I = symbols("F L E I")
y = # ... lengkapi, tulis formula untuk menghitung y di sini
    
F_num = 750; ΔF = 30 # N/m
L_num = 9; ΔL = 0.03 # m
E_num = 7.5e9; ΔE = 5e7 # N/m^2
I_num = 0.0005; ΔI = 0.000005 # m^4

# Hitung nilai y dengan nilai numerik
dict_subs = {F: F_num, L: L_num, E: E_num, I: I_num}
y_num = y.subs(dict_subs)
print("y = ", y_num)

# Hitung nilai Δy dengan Pers. 4.27
Δy = abs(diff(y,F))*ΔF + ... # lengkapi
pprint(Δy) # bentuk simbolik
print("Δy = ", Δy.subs(dict_subs)) # evaluasi/substitusi nilai numerik

# Hitung nilai ekstrim

dict_subs = {
  F: F_num - ΔF, L: L_num - ΔL,
  E: E_num + ΔE, I: I_num + ΔI  # mengapa seperti ini?
} 
ymin = y.subs(dict_subs)
print("ymin = ", ymin)
    
dict_subs = .... # lengkapi 
ymax = y.subs(dict_subs)
print("ymax = ", ymax)    
\end{pythoncode}

\begin{soal}
Lengkapi dan/atau modifikasi kode di atas sehingga dapat menampilkan $\Delta y$
seperti yang dijelaskan pada Chapra Contoh 4.6.
\end{soal}