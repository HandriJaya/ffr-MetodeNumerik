\section{Metode \textit{secant}}

\textbf{Chapra Contoh 6.6}
Salah satu kekurangan dari metode Newton-Raphson
adalah perlunya menghitung turunan
pertama dari fungsi yang ingin dicari akarnya. Hal ini biasanya tidak dapat 
Metode \textit{secant} memiliki bentuk iterasi yang mirip dengan metode
Newton-Raphson, namun turunan fungsi diaproksimasi dengan menggunakan:
\begin{equation*}
f'(x_{i}) \approx \frac{f(x_{i-1}) - f(x_{i})}{x_{i-1} - x_{i}}
\end{equation*}
sehingga kita memperoleh bentuk iterasi sebagai berikut:
\begin{equation}
x_{i+1} = x_{i} - \frac{f(x_{i}) (x_{i-1} - x_{i})}{f(x_{i-1}) - f(x_{i})}
\end{equation}
Metode \textit{secant} memerlukan dua titik $x_{-1}$ dan $x_{0}$
sebagai tebakan akar awal. Berbeda
dengan metode \textit{bracketing}, nilai fungsi $f(x_{-1})$ dan $f(x_{0})$
tidak diharuskan memiliki tanda yang berbeda.

\begin{pythoncode}
# ... definisi f dan df sama dengan contoh Newton-Raphson
x0 = 0.0
x1 = 1.0
for i in range(1,6):
    # approximation of derivative of f(x)
    dfx = (f(x0) - f(x1))/(x0 - x1)
    #
    xnew = ... # lengkapi
    fxnew = f(xnew)
    print("%3d %18.10f %18.10e" % (i, xnew, fxnew))
    x0 = x1
    x1 = xnew    
\end{pythoncode}

\begin{soal}
Lengkapi program Python untuk metode \textit{secant} di atas.
Implementasikan program Anda sehingga
dapat melakukan iterasi sampai nilai kesalahan menjadi lebih kecil dari
suatu nilai tertentu yang diberikan. Anda dapat menggunakan loop \txtinline{while}
atau loop \txtinline{for} dengan jumlah iterasi maksimum tertentu.
\end{soal}