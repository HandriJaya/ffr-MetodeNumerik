\begin{soal}[Chapra Latihan 6.25]
\label{chapra_exe_6.25}
Kesetimbangan massa dari suatu polutan dalam suatu danau
dapat dinyatakan dengan persamaan diferensial:
\begin{equation*}
V\frac{\mathrm{d}c}{\mathrm{d}t} = W - Qc - kV\sqrt{c}
\end{equation*}
di mana $c$ adalah konsentrasi polutan dalam $\mathrm{g/m}^{3}$,
dengan parameter
$V = 1\times10^6\,\mathrm{m}^3$,
$Q = 1\times10^5\,\mathrm{m}^3/\mathrm{tahun}$,
$W = 1\times10^6\,\mathrm{g}/\mathrm{tahun}$, dan
$k = 0.25\,\mathrm{m}^{0.5}/\mathrm{g}^{0.5}/\mathrm{tahun}$.
Hitung konsentrasi polutan dalam keadaan tunak.
\end{soal}