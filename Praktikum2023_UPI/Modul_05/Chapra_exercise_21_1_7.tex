\begin{soal}[Chapra Latihan 21.1-21.7]
Untuk integral-integral berikut ini
\begin{align}
& \int_{0}^{\pi/2} (6 + 3\cos(x))\ \mathrm{d}x \\
& \int_{0}^{3} (1 - e^{-2x})\ \mathrm{d}x \\
& \int_{-2}^{4} (1 - x - 4x^3 + 2x^5)\ \mathrm{d}x \\
& \int_{1}^{2} (x - 2/x)^2\ \mathrm{d}x \\
& \int_{-3}^{5} (4x - 3)^3 \mathrm{d}x \\
& \int_{0}^{3} x^2 e^x \mathrm{d}x \\
& \int_{0}^{1} 14^{2x}\ \mathrm{d}x
\end{align}
dengan menggunakan metode:
\begin{enumerate}[label=(\alph*)]
\item analitik
\item aturan trapesium
\item aturan 1/3 Simpson
\item aturan 3/8 Simpson
\item aturan Boole
\end{enumerate}
Variasikan parameter numerik yang terkait, seperti jumlah titik yang dievaluasi,
untuk setiap metode yang digunakan.
Bandingkan hasil numerik yang diberikan dengan hasil analitik.
Anda dapat menggunakan SymPy untuk menghitung integral secara analitik.
\end{soal}