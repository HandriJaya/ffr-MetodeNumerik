\documentclass[a4paper,10pt,bahasa]{extarticle} % screen setting
\usepackage[a4paper]{geometry}

%\documentclass[b5paper,11pt,bahasa]{article} % screen setting
%\usepackage[b5paper]{geometry}

%\geometry{verbose,tmargin=1.5cm,bmargin=1.5cm,lmargin=1.5cm,rmargin=1.5cm}

\geometry{verbose,tmargin=2.0cm,bmargin=2.0cm,lmargin=2.0cm,rmargin=2.0cm}

\setlength{\parskip}{\smallskipamount}
\setlength{\parindent}{0pt}

%\usepackage{cmbright}
%\renewcommand{\familydefault}{\sfdefault}

\usepackage{amsmath}
\usepackage{amssymb}

\usepackage[libertine]{newtxmath}

\usepackage[no-math]{fontspec}
\setmainfont{Linux Libertine O}

%\usepackage{fontspec}
%\usepackage{lmodern}

\setmonofont{JuliaMono-Regular}


\usepackage{hyperref}
\usepackage{url}
\usepackage{xcolor}
\usepackage{enumitem}
\usepackage{mhchem}
\usepackage{graphicx}
\usepackage{float}

\usepackage{minted}

\newminted{julia}{breaklines,fontsize=\footnotesize}
\newminted{python}{breaklines,fontsize=\footnotesize}

\newminted{bash}{breaklines,fontsize=\footnotesize}
\newminted{text}{breaklines,fontsize=\footnotesize}

\newcommand{\txtinline}[1]{\mintinline[breaklines,fontsize=\footnotesize]{text}{#1}}
\newcommand{\jlinline}[1]{\mintinline[breaklines,fontsize=\footnotesize]{julia}{#1}}
\newcommand{\pyinline}[1]{\mintinline[breaklines,fontsize=\footnotesize]{python}{#1}}

\newmintedfile[juliafile]{julia}{breaklines,fontsize=\footnotesize}
\newmintedfile[pythonfile]{python}{breaklines,fontsize=\footnotesize}
\newmintedfile[fortranfile]{fortran}{breaklines,fontsize=\footnotesize}

\usepackage{mdframed}
\usepackage{setspace}
\onehalfspacing

\usepackage{babel}
\usepackage{appendix}

\newcommand{\highlighteq}[1]{\colorbox{blue!25}{$\displaystyle#1$}}
\newcommand{\highlight}[1]{\colorbox{red!25}{#1}}

\newcounter{soal}%[section]
\newenvironment{soal}[1][]{\refstepcounter{soal}\par\medskip
   \noindent \textbf{Soal~\thesoal. #1} \sffamily}{\medskip}


\definecolor{mintedbg}{rgb}{0.95,0.95,0.95}
\BeforeBeginEnvironment{minted}{
    \begin{mdframed}[backgroundcolor=mintedbg,%
        topline=false,bottomline=false,%
        leftline=false,rightline=false]
}
\AfterEndEnvironment{minted}{\end{mdframed}}


\BeforeBeginEnvironment{soal}{
    \begin{mdframed}[%
        topline=true,bottomline=false,%
        leftline=true,rightline=false]
}
\AfterEndEnvironment{soal}{\end{mdframed}}

\begin{document}

\title{%
{\small TF2202 Komputasi Rekayasa}\\
Integral dan Diferensiasi Numerik
}
\author{Tim Praktikum Komputasi Rekayasa 2023\\
Teknik Fisika\\
Institut Teknologi Bandung}
\date{}
\maketitle


\section{Aturan trapesium}

\begin{soal}[Chapra Contoh 21.1]
Gunakan aturan trapesium untuk aproksimasi integral berikut.
\begin{equation*}
f(x) = 0.2 + 25x - 200x^2 + 675x^3 - 900x^4 + 400x^5
\end{equation*}
dari $a = 0$ sampai $b = 0.8$. Bandingkan hasilnya dengan hasil analitik,
yaitu 1.640533. Hitung juga estimasi error dari aproksimasi yang digunakan.
\end{soal}

Anda dapat melengkapi kode berikut ini untuk Chapra Contoh 21.1
\begin{pythoncode}
def my_func(x):
    return 0.2 + 25*x - 200*x**2 + 675*x**3 - 900*x**4 + 400*x**5

a = 0.0
b = 0.8
f_a = my_func(a)
f_b = my_func(b)

I_exact = 1.640533 # from the book
# Alternatively, you can use SymPy to do the integration

# Use trapezoid rule here
I = .... # Lengkapi

E_t = I_exact - I
ε_t = E_t/I_exact * 100 # in percent
print("Integral result = %.6f" % I)
print("True integral   = %.6f" % I_exact)
print("True error      = %.6f" % E_t)
print("ε_t             = %.1f%%" % ε_t)

import sympy
x = sympy.symbols("x")
f = 0.2 + 25*x - 200*x**2 + 675*x**3 - 900*x**4 + 400*x**5
d2f = f.diff(x,2) # calculate 2nd derivative
avg_d2f_xi = sympy.integrate( d2f, (x,a,b) )/(b - a)
E_a = -1/12*avg_d2f_xi*(b - a)**3 # Persamaan 21.6
print("Approx error    = %.6f" % E_a)
\end{pythoncode}

Aturan trapesium (dan aturan integrasi Newton-Cotes lainnya) dapat digunakan
untuk lebih dari satu interval.

\begin{soal}
Gunakan aturan trapesium multi-interval dengan jumlah segmen 2 untuk menghitung
aproksimasi integral pada Soal sebelumnya (Chapra Contoh 21.1).
Coba juga ganti jumlah segmen yang digunakan dan bandingkan hasilnya
dengan hasil analitik.
\end{soal}

Anda dapat melengkapi kode berikut ini. Lihat juga persaman yang digunakan pada
buku Chapra.
\begin{pythoncode}
def my_func(x):
    return 0.2 + 25*x - 200*x**2 + 675*x**3 - 900*x**4 + 400*x**5

# N is number of segments
# Npoints is N + 1
# f is function
def integ_trapz_multiple( f, a, b, N ):
    x0 = a
    xN = b
    h = (b - a)/N
    ss = 0.0
    for i in range(1,N):
        xi = x0 + i*h
        ss = ss + f(xi)
    I = .... # lengkapi
    return I

Nsegments = 2
a = 0.0
b = 0.8
I_exact = 1.640533 # from the book

I = integ_trapz_multiple( my_func, a, b, Nsegments )

print("Nsegments = ", Nsegments)
print("Integral result = %.6f" % I)
E_t = I_exact - I
print("True integral   = %.6f" % I_exact)
print("True error      = %.6f" % E_t)
ε_t = E_t/I_exact * 100
print("ε_t             = %.1f%%" % ε_t)

import sympy
x = sympy.symbols("x")
f = 0.2 + 25*x - 200*x**2 + 675*x**3 - 900*x**4 + 400*x**5
d2f = f.diff(x,2)
#sympy.pprint(d2f)
avg_d2f_xi = sympy.integrate( d2f, (x,a,b) )/(b - a)
#print("avg_d2f_xi = ", avg_d2f_xi)
E_a = -1/12*avg_d2f_xi*(b - a)**3/Nsegments**2
print("Approx error    = %.6f" % E_a)
\end{pythoncode}

\begin{soal}[Chapra Contoh 21.3]
Hitung integral berikut ini dengan menggunakan aturan trapesium:
\begin{equation*}
d = \frac{gm}{c} \int_{0}^{t} (1 - e^{-(c/m)t}) \, \mathrm{d}t
\end{equation*}
Bandingkah hasil yang diperoleh dengan hasil analitik. Lihat buku Chapra untuk nilai-nilai
numerik yang diperlukan atau lihat pada kode Python di bawah.
\end{soal}

Anda dapat melengkapi kode berikut ini.
\begin{pythoncode}
from math import exp

def my_func(t):
    g = 9.8
    m = 68.1
    c = 12.5
    return g*m/c * (1 - exp(-(c/m)*t) )
    
# ... definisi integ_trapz_multiple
# ... LENGKAPI

t = 10.0
a = 0.0
b = t
    
d_exact = 289.43515 # dari buku
    
print("------------------------------------------------------")
print("   N          h          d          E_t         ε_t")
print("------------------------------------------------------")
for Nsegments in [10, 20, 50, 100, 200, 500, 1000, 2000, 5000, 5000, 10000]:
    h = (b - a)/Nsegments
    d = integ_trapz_multiple( my_func, a, b, Nsegments )
    E_t = d_exact - d
    ε_t = E_t/d_exact * 100
    print("%5d  %10.4f  %10.4f   %10.4e %10.2e%%" % (Nsegments, h, d, E_t, ε_t))


# Calculate "Exact" result using SymPy
import sympy
g = 9.8
m = 68.1
c = 12.5
t = sympy.symbols("t")
f = g*m/c * (1 - sympy.exp(-(c/m)*t) )
d_sympy = sympy.integrate( f, (t,0,10))
    
print()
print("SymPy result:")
print("d_sympy = ", d_sympy)    
\end{pythoncode}

Apakah hasil yang Anda peroleh sama dengan yang diberikan pada Chapra?
Jika ada perbedaan, apakah yang menyebabkan perbedaan tersebut?
\section{Aturan Simpson}


\begin{frame}
\frametitle{Aturan Simpson}

Aturan Simpson adalah formula integrasi Newton-Cotes yang menggunakan:
\begin{itemize}
\item tiga titik (aturan 1/3 Simpson)
\item empat titik (aturan 3/8 Simpson)
\end{itemize}

{\centering
\includegraphics[width=0.7\textwidth]{images_priv/Chapra_Fig_21_10.pdf}
\par}

\end{frame}




\begin{frame}
\frametitle{Aturan 1/3 Simpson}

Pada aturan Simpson, polinomial yang digunakan adalah orde 2, sehingga memerlukan tidak titik
dasar. Aplikasi aturan Simpson untuk satu selang dapat dituliskan sebagai berikut:
\begin{equation*}
I = \frac{h}{3}( f(x_0) + 4f(x_1) + f(x_2) )
\end{equation*}
di mana $h = (b-a)/3$. Aturan ini dikenal sebagai aturan 1/3 Simpson.
Aturan ini juga dapat dituliskan sebagai:
\begin{equation*}
I = \frac{b-a}{6}( f(x_0) + 4f(x_1) + f(x_2) )
\end{equation*}
Kesalahan pemotongan lokal dapat dituliskan sebagai:
\begin{equation*}
E_{t} = -\frac{(b-a)^5}{2880} f^{(4)}(\xi)
\end{equation*}
dengan $\xi$ berada di dalam interval $[a,b]$.

\end{frame}




\begin{frame}[fragile]
\frametitle{Contoh}

\begin{pythoncode}
def my_func(x):
    return 0.2 + 25*x - 200*x**2 + 675*x**3 - 900*x**4 + 400*x**5

a = 0.0; b = 0.8
h = (b-a)/2
# Base points for Simpson's 1/3 rule
x0 = a
x1 = a + h
x2 = b
# We use 3 points or 2 segments.

I_exact = 1.640533 # from the book

I = .... # LENGKAPI

E_t = (I_exact - I)
ε_t = E_t/I_exact * 100
print("Integral result = %.7f" % I)
print("True integral   = %.6f" % I_exact)
print("True error      = %.7f" % E_t)
print("ε_t             = %.1f%%" % ε_t)
\end{pythoncode}
\end{frame}


\begin{frame}[fragile]
\frametitle{Estimasi error}

\begin{equation*}
E_{t} = -\frac{(b-a)^5}{2880} f^{(4)}(\xi)
\end{equation*}

\begin{equation*}
f^{(4)}(\xi) \approx \dfrac{\int_{a}^{b} f^{(4)}(x) \, \mathrm{d}x}{b - a}
\end{equation*}

\begin{pythoncode}
import sympy
x = sympy.symbols("x")
f = 0.2 + 25*x - 200*x**2 + 675*x**3 - 900*x**4 + 400*x**5
d4f = f.diff(x,4)
avg_d4f_xi = sympy.integrate( d4f, (x,a,b) )/(b - a)
E_a = -1/2880*avg_d4f_xi*(b - a)**5
print("Approx error    = %.7f" % E_a)
\end{pythoncode}

\end{frame}




\begin{frame}[fragile]
\frametitle{Implementasi (dalam fungsi)}

\begin{equation*}
I = \frac{h}{3}( f(x_0) + 4f(x_1) + f(x_2) )
\end{equation*}

\begin{pythoncode}
def integ_simpson13( f, a, b ):
    h = (b - a)/2
    x0 = a
    x1 = a + h
    x2 = b
    I = h/3 * ( f(x0) + 4*f(x1) + f(x2) )
    return I
\end{pythoncode}

\end{frame}



\begin{frame}[fragile]
\frametitle{Contoh keluaran}

\begin{textcode}
Integral result = 1.3674667
True integral   = 1.640533
True error      = 0.2730663
ε_t             = 16.6%
Approx error    = 0.2730667    
\end{textcode}

Bandingkan dengan aturan trapesium.

\end{frame}




\begin{frame}
\frametitle{Aturan 1/3 Simpson Multi-interval}

Aturan 1/3 Simpson dapat digunakan untuk beberapa interval:
\begin{equation*}
\frac{b-a}{3n} \left(
f(x_0) + 4\sum_{i\text{ odd}}^{n-1} f(x_i) + 2 \sum_{j\text{ even}}^{n-2} f(x_j) + f(x_n)
\right)
\end{equation*}
dengan kesalahan pemotongan:
\begin{equation*}
E_{a} = -\frac{(b-a)^5}{180n^4} \overline{f}^{(4)}
\end{equation*}

\end{frame}


\begin{frame}[fragile]
\frametitle{Implementasi}

\begin{pythoncode}
def integ_simpson13_multiple( f, a, b, N ):
    # N: is number of segments
    assert N >= 2
    assert N % 2 == 0 # must be an even number
    
    x0 = a
    xN = b
    h = (b - a)/N
    x = np.linspace(a, b, N+1)
    # Total number of base points is N+1 (an odd number)
    
    ss_odd = 0.0
    for i in range(1,N,2):
        ss_odd = ss_odd + f(x[i])
        
    ss_even = 0.0
    for i in range(2,N-1,2):
        ss_even = ss_even + f(x[i])
    
    I = (b - a)/(3*N) * ( f(x0) + 4*ss_odd + 2*ss_even + f(xN) )
    return I
\end{pythoncode}

\end{frame}




\begin{frame}
\frametitle{Aturan 3/8 Simpson}

Dengan menggunakan 4 titik (polinomial kubik), dapat diturunkan aturan 3/8 Simpson:
\begin{equation*}
I = \frac{3h}{8} ( f(x_0) + 3f(x_1) + 3f(x_2) + f(x_3) )
\end{equation*}
dengan $h = (b-a)/3$.
Dapat juga dituliskan sebagai:
\begin{equation*}
I = \frac{b-a}{8} ( f(x_0) + 3f(x_1) + 3f(x_2) + f(x_3) )
\end{equation*}

Kesalahan pemotongan:
\begin{equation*}
E_{t} = -\frac{(b-a)^5}{6480} f^{(4)}(\xi)
\end{equation*}

\end{frame}



\begin{frame}[fragile]
\frametitle{Implementasi dalam Python}

\begin{pythoncode}
def integ_simpson38( f, a, b ):
    h = (b - a)/3
    x0 = a
    x1 = a + h
    x2 = a + 2*h
    x3 = b
    I = 3*h/8 * ( f(x0) + 3*f(x1) + 3*f(x2) + f(x3) )
    return I
\end{pythoncode}

\end{frame}


\begin{frame}
\frametitle{Simpson 1/3 vs Simpson 3/8}

\begin{itemize}
\item Aturan 3/8 Simpson sedikit lebih akurat karena penyebut pada kesalahan pemotongan
lebih besar dibandingkan aturan 1/3 Simpson.
\item Meskipun demikian, aturan 1/3 Simpson biasanya menjadi pilihan dibandingkan dengan aturan
3/8 karena orde kesalahannya sama meskipun hanya menggunakan tiga titik.
\item Aturan 3/8 Simpson dapat digunakan bersama-sama dengan aturan 1/3 Simpson apabila
jumlah segmen yang digunakan adalah ganjil (atau jumlah titik yang digunakan adalah genap).
\end{itemize}


\end{frame}






\section{Fungsi yang diberikan pada tabel}

Pada soal-soal sebelumnya, fungsi yang akan dihitung diberikan fungsi analitiknya.
Aturan integral yang telah digunakan sebelumnya dapat digunakan juga untuk
fungsi yang diberikan dalam bentuk tabulasi data. Dalam hal ini kita perlu
menyesuaikan kode yang telah kita buat agar menerima array (bukan fungsi).
Implementasi program cukup sederhana jika data yang diberikan didefinisikan
pada titik-titik dengan panjang segmen yang sama. Jika tidak maka kita perlu
memberikan perlakuan yang berbeda untuk setiap segmen.
Salah satu metode sederhana yang dapat kita lakukan adalah dengan cara mengaplikasikan
aturan trapesium untuk setiap segmen secara terpisah. Pada soal berikut kita akan
melakukan hal tersebut.

\begin{soal}[Chapra Contoh 21.7]
Gunakan data pada Tabel 21.3 untuk mencari integral fungsi yang didefinisikan pada
Contoh 21.1 pada Chapra. Bandingkan hasilnya dengan nilai eksak. Perhatikan
bahwa data pada Tabel 21.3 tidak memilik panjang segmen yang sama.
\end{soal}

Anda dapat melengkapi kode beriku.
\begin{pythoncode}
def integ_trapz_table( fa, fb, a, b ):
    I = .... # LENGKAPI
    return I
# fa adalah nilai fungsi pada x=a
# fb adalah nilai fungsi pada x=b

x = [0.0, 0.12, 0.22, 0.32, 0.36, 0.40,
     0.44, 0.54, 0.64, 0.70, 0.80]

fx = [0.200000, 1.309729, 1.305241, 1.743393, 2.074903, 2.456000, 
      2.842985, 3.507297, 3.181929, 2.363000, 0.232000]

I_exact = 1.640533

Ndata = len(x)
I = 0.0
for i in range(Ndata-1):
    I = I + integ_trapz_table( fx[i], fx[i+1], x[i], x[i+1] )

E_t = (I_exact - I)
ε_t = E_t/I_exact * 100
print("Integral result = %.6f" % I)
print("True error      = %.6f" % E_t)
print("ε_t             = %.1f%%" % ε_t)
\end{pythoncode}
\section{Integrasi Romberg}

\begin{soal}[Chapra Contoh 22.1 dan 22.2]
Tinjau kembali integral dari fungsi pada Chapra Contoh 21.1. Gunakan aturan trapesium
dengan jumlah segmen 1, 2, dan 4. Gunakan ekstrapolasi Richardson untuk
mendapatkan estimasi integral yang lebih akurat.
\end{soal}

Anda dapat melengkapi kode berikut.
\begin{pythoncode}
# Lengkapi definisi fungsi-fungsi yang diperlukan
# ....

a = 0.0
b = 0.8
I_exact = 1.640533

print("Using trapezoidal rule")

I_1 = integ_trapz_multiple(my_func, a, b, 1) # satu segmen
ε_t = (I_exact - I_1)/I_exact * 100
print("I_1  = %.6f ε_t = %2.1f%%" % (I_1, ε_t))

I_2 = integ_trapz_multiple(my_func, a, b, 2) # dua segmen
ε_t = (I_exact - I_2)/I_exact * 100
print("I_2  = %.6f ε_t = %2.1f%%" % (I_2, ε_t))

I_3 = integ_trapz_multiple(my_func, a, b, 3) # 3 segmen
ε_t = (I_exact - I_3)/I_exact * 100
print("I_3  = %.6f ε_t = %2.1f%%" % (I_3, ε_t))

I_4 = integ_trapz_multiple(my_func, a, b, 4) # 4 segmen
ε_t = (I_exact - I_4)/I_exact * 100
print("I_4  = %.6f ε_t = %2.1f%%" % (I_4, ε_t))

print()
print("Using Richardson's extrapolation:")

I_12 = .... # LENGKAPI: kombinasi I_2 dan I_1
ε_t = (I_exact - I_12)/I_exact * 100
print("I_12 = %.6f ε_t = %2.1f%%" % (I_12, ε_t))

I_24 = .... # LENGKAPI: kombinasi I_4 dan I_2
ε_t = (I_exact - I_24)/I_exact * 100
print("I_24 = %.6f ε_t = %2.1f%%" % (I_24, ε_t))

print()
print("Using Richardson's extrapolation (2nd iter):")

I_124 = .... # LENGKAPI: kombinasi I_24 dan I_12
ε_t = (I_exact - I_124)/I_exact * 100
print("I_124 = %.6f ε_t = %2.1f%%" % (I_124, ε_t))
\end{pythoncode}

\begin{soal}
Implementasikan algoritma pada Gambar 22.4 untuk integrasi Romberg.
\end{soal}

Anda dapat melengkapi kode berikut:
\begin{pythoncode}
def integ_romberg(f, a, b, es=1e-10, MAXIT=10):
    I = np.zeros( (MAXIT+2,MAXIT+2) )
    n = 1
    # We start from I[1,1], to follow the book's notation
    I[1,1] = integ_trapz_multiple(f, a, b, n)
    iterConv = 0
    for i in range(1,MAXIT+1):
        n = 2**i
        I[i+1,1] = integ_trapz_multiple(f, a, b, n)
        #
        for k in range(2,i+2):
            j = 2 + i - k
            I[j,k] = ....# LENGKAPI
        #
        ea = abs( (I[1,i+1] - I[2,i])/I[1,i+1] )*100 # in percent
        if ea <= es:
            iterConv = i
            # For debugging, please comment if not needed
            print("Converged)
            break
        # Here we set iterConv to i to make sure that the last value
        # will be returned in case of no convergence.
        iterConv = i
    # to make sure that we are use variable that is defined outside the loop
    # we use iterConv instead of i
    return I[1,iterConv+1]
\end{pythoncode}

Contoh pengujian (membandingkan dengan hasil dari SymPy):
\begin{pythoncode}
from math import cos, pi

import sympy
x = sympy.symbols("x")
func_symb = 6 + 3*sympy.cos(x)
resExact = sympy.N(sympy.integrate(func_symb, (x, 0, sympy.pi/2)))

# Definisikan atau import fungsi-fungsi yang diperlukan
# ...
    
def my_func(x):
    return 6 + 3*cos(x)
    
a = 0.0
b = pi/2
resN = integ_romberg(my_func, a, b)
print("resN = %18.12f" % resN)
print("res  = %18.12f" % resExact)
print("err  = %18.12e" % abs(resExact-resN))    
\end{pythoncode}

Contoh hasil keluaran:
\begin{textcode}
Converged
iterConv =  5
resN =    12.424777960769
res  =    12.424777960769
err  = 0.000000000000e+00    
\end{textcode}


\section{Kuadratur Gauss-Legendre}

\begin{soal}[Chapra Contoh 22.3 dan 22.4]
Aplikasikan kuadratur Gauss-Legendre dengan dua dan tiga titik untuk
menghitung integral dari fungsi pada Chapra Contoh 21.1
\end{soal}

Anda dapat melengkapi kode berikut.
\begin{pythoncode}
# .... Definisikan atau import fungsi-fungsi yang diperlukan

a = 0.0
b = 0.8

a_0 = (b + a)/2
a_1 = (b - a)/2

# mapping from x_d -> x
# NOTE: using a_0 and a_1 as global variables
def mapping_func(x_d):
    return a_0 + a_1*x_d
    
dx_dx_d = (b - a)/2 # Eq. (22.30)

NGaussPoints = 2
GAUSS2_c = [1.0, 1.0]
GAUSS2_x = [-1.0/sqrt(3), 1.0/sqrt(3)]
    
I_exact = 1.640533
    
I = 0.0
for i in range(NGaussPoints):
    x_d = GAUSS2_x[i]
    c = GAUSS2_c[i]
    I = .... # LENGKAPI
    
ε_t = (I_exact - I)/I_exact * 100
print("I = %.6f, ε_t = %.1f%%" % (I, ε_t))
\end{pythoncode}

Untuk tiga titik, Anda dapat mengganti daftar titik-titik Gauss sebagai berikut.
\begin{pythoncode}
# Can be found for example at:
# https://en.wikipedia.org/wiki/Gaussian_quadrature#Gauss%E2%80%93Legendre_quadrature
NGaussPoints = 3
GAUSS3_c = [5/9, 8/9, 5/9]
GAUSS3_x = [-sqrt(3/5), 0, sqrt(3/5)]
\end{pythoncode}

Untuk mendapatkan jumlah titik Gauss-Legendre yang lebih lengkap, kita dapat
menggunakan fungsi yang tersedia pada Numpy seperti pada contoh berikut.

\begin{soal}[Chapra Contoh 22.5]
Hitung integral berikut ini dengan menggunakan kuadratur Gauss-Legendre
\begin{equation*}
d = \frac{gm}{c} \int_{0}^{10} (1 - e^{-(c/m)t}) \, \mathrm{d}t
\end{equation*}
Variasikan jumlah titik yang digunakan dan
bandingkan hasil yang diperoleh dengan hasil analitik.
\end{soal}

Anda dapat melengkapi kode berikut.
\begin{pythoncode}
import numpy as np
from math import exp
    
def my_func(t):
    g = 9.8
    m = 68.1
    c = 12.5
    return g*m/c * (1 - exp(-(c/m)*t) )
    
def calc_exact_sympy():
    # Calculate "Exact" result using SymPy
    import sympy
    g = 9.8
    m = 68.1
    c = 12.5
    t = sympy.symbols("t")
    f = g*m/c * (1 - sympy.exp(-(c/m)*t) )
    d_sympy = sympy.integrate( f, (t,0,10))
    return d_sympy
    
I_exact = calc_exact_sympy()
    
a = 0.0
b = 10.0

a_0 = (b + a)/2
a_1 = (b - a)/2

def mapping_func(x_d):
    return a_0 + a_1*x_d

dx_dx_d = (b - a)/2
    
for NGaussPoints in range(2,7):
    GAUSS_x, GAUSS_c = np.polynomial.legendre.leggauss(NGaussPoints)
    I = 0.0
    for i in range(NGaussPoints):
        x_d = GAUSS_x[i]
        c = GAUSS_c[i]
        I = .... # LENGKAPI
    Δ = abs(I_exact-I)
    print("NGaussPoints = %d, I = %15.10f, err = %10.5e" % (NGaussPoints,I,Δ))
\end{pythoncode}

\section{Soal tambahan}

\begin{soal}[Chapra Latihan 21.1-21.7]
Untuk integral-integral berikut ini
\begin{align}
& \int_{0}^{\pi/2} (6 + 3\cos(x))\ \mathrm{d}x \\
& \int_{0}^{3} (1 - e^{-2x})\ \mathrm{d}x \\
& \int_{-2}^{4} (1 - x - 4x^3 + 2x^5)\ \mathrm{d}x \\
& \int_{1}^{2} (x - 2/x)^2\ \mathrm{d}x \\
& \int_{-3}^{5} (4x - 3)^3 \mathrm{d}x \\
& \int_{0}^{3} x^2 e^x \mathrm{d}x \\
& \int_{0}^{1} 14^{2x}\ \mathrm{d}x
\end{align}
dengan menggunakan metode:
\begin{enumerate}[label=(\alph*)]
\item analitik
\item aturan trapesium
\item aturan 1/3 Simpson
\item aturan 3/8 Simpson
\item aturan Boole
\end{enumerate}
Variasikan parameter numerik yang terkait, seperti jumlah titik yang dievaluasi,
untuk setiap metode yang digunakan.
Bandingkan hasil numerik yang diberikan dengan hasil analitik.
Anda dapat menggunakan SymPy untuk menghitung integral secara analitik.
\end{soal}
\begin{soal}[Chapra 21.10 dan 21.11]
    Hitung integral dari data berikut ini dengan menggunakan aturan trapesium dan aturan
    Simpson.
    
    Data Chapra 21.10
    
    {\centering
    \begin{tabular}{|c|cccccc|}
    \hline
    $x$    & 0 & 0.1 & 0.2 & 0.3 & 0.4 & 0.5 \\
    $f(x)$ & 1 & 8   & 4   & 3.5 & 5   & 1 \\
    \hline
    \end{tabular}
    \par}
    
    Data Chapra 21.11
    
    {\centering
    \begin{tabular}{|c|ccccccc|}
    \hline
    $x$    & -2 & 0 & 2 & 4 & 6 & 8 & 10 \\
    $f(x)$ & 35 & 5 & -10 & 2 & 5 & 3 & 20 \\
    \hline
    \end{tabular}
    \par}
    \end{soal}
\begin{soal}[Chapra Latihan 22.1, 22.2, dan 22.3]
Gunakan metode Romberg untuk menghitung integral berikut.
\begin{align}
& \int_{0}^{3} x e^{2x}\ \mathrm{d}x \\
& \int_{1}^{2} \left( x + \frac{1}{x} \right)^2 \ \mathrm{d}x \\
& \int_{0}^{2} \frac{e^{x}\sin(x)}{1 + x^2}\ \mathrm{d}x
\end{align}
Variasikan parameter numerik yang Anda gunakan, misalnya jumlah maksimal selang
yang digunakan.
Bandingkan hasil yang Anda dapatkan dengan hasil analitik atau dari SymPy.
\end{soal}
\begin{soal}[Chapra Latihan 22.8]
Gunakan formula Gauss-Legendre dengan dua sampai 6 titik untuk menghitung integral
\begin{equation}
\int_{-3}^{3} \frac{1}{1 + x^2}\ \mathrm{d}x
\end{equation}
Bandingkan hasil yang Anda dapatkan dengan hasil analitik atau dari SymPy.
\end{soal}
\begin{soal}[Chapra Latihan 22.9]
Gunakan metode integrasi numerik untuk menghitung integral:
\begin{align}
& \int_{2}^{\infty} \frac{1}{x(x + 2)}\ \mathrm{dx} \\
& \int_{0}^{\infty} e^{-x} \sin^{2} x\ \mathrm{dx}
\end{align}
Bandingkan hasil yang Anda peroleh dengan hasil analitik atau dari SymPy.
Gunakan metode standard (trapesium, Simpson, Romberg, dll) secara
langsung dan dengan melakukan penggantian variabel. Anda juga dapat menggunakan
kombinasi metode tersebut jika integral yang Anda lakukan dibagi menjadi dua
integral (Lihat Contoh 22.6 pada Chapra). Plot juga integran yang terlibat.
\end{soal}

\begin{soal}[Chapra Latihan 23.8]
Hitung turunan pertama dari fungsi-fungsi berikut dengan menggunakan metode beda hingga
tengah $\mathcal{O}(h^4)$:
\begin{itemize}
\item $y = x^3 + 4x - 15$ pada $x=0$ dengan $h=0.25$
\item $y = x^2 \cos(x)$ pada $x=0.4$ dengan $h=0.1$
\item $y = \tan(x/3)$ pada $x=3$ dengan $h=0.5$
\item $y = \sin(0.5\sqrt{x})/x$ pada $x=1$ dengan $h=0.2$
\item $y = e^{x} + x$ pada $x=2$ dengan $h=0.2$
\end{itemize}
Bandingkan hasil yang Anda peroleh dengan hasil analitik atau SymPy.
\end{soal}


\begin{soal}[Chapra Latihan 23.9]
Diketahui data jarak dan waktu yang ditempuh pada suatu roket.

{\centering
\begin{tabular}{|c|cccccc|}
\hline
t (s)  & 0 & 25 & 50 & 75 & 100 & 125 \\
y (km) & 0 & 32 & 58 & 78 &  92 & 100 \\
\hline
\end{tabular}
\par}

Gunakan metode numerik untuk untuk mengestimasi kecepatan dan percepatan roket
pada masing-masing waktu pada tabel (jika memungkinkan dengan metode yang Anda
gunakan).
\end{soal}


\section{Catatan Tambahan}

\subsection{Contoh penggunaan SymPy}
Contoh penggunaan SymPy untuk perhitungan integral tentu.
\begin{pythoncode}
import sympy
x = sympy.symbols("x")
func_symb = 6 + 3*sympy.cos(x)
resExact = sympy.N(sympy.integrate(func_symb, (x, 0, sympy.pi/2)))
# fungsi sympy.N digunakan untuk memaksa hasil dalam bentuk numerik.
\end{pythoncode}

Untuk batas integral yang melibatkan tak-hingga, Anda dapat menggunakan
\pyinline{sympy.oo}. Contoh:
\begin{pythoncode}
import sympy
x = sympy.symbols("x")
func_symb = sympy.exp(-x)*sympy.sin(x)**2
resExact = sympy.N(sympy.integrate(func_symb, (x, 0, sympy.oo)))
\end{pythoncode}


\subsection{Metode Newton-Cotes untuk interval terbuka}

Metode Newton-Cotes, interval terbuka (lihat Tabel 21.4 pada Chapra)
\begin{pythoncode}
# hanya untuk satu interval, f diberikan dalam bentuk analitik (bukan tabel)
def integ_newtoncotes_open6seg(f, a, b):
    # menggunakan 6 segmen atau 5 titik
    # silakan gunakan formula lain jika diperlukan
    h = (b-a)/6
    x1 = a + h
    x2 = a + 2*h
    x3 = a + 3*h
    x4 = a + 4*h
    x5 = a + 5*h
    #
    s = 11*f(x1) - 14*f(x2) + 26*f(x3) - 14*f(x4) + 11*f(x5)
    return (b-a)*s/20.0
\end{pythoncode}


\subsection{Fungsi Python untuk integral multi-interval yang seragam}

Mengaplikasikan aturan integrasi pada beberapa interval homogen
dengan menggunakan fungsi aturan integrasi satu interval, diberi nama
\pyinline{quad_func} pada argumen fungsi berikut.
\begin{pythoncode}
def apply_quadrature_multi_interval(quad_func, f, a, b, Ninterval):
    s = 0.0
    Δ = (b-a)/Ninterval
    for i in range(Ninterval):
        aa = a + i*Δ
        bb = a + (i+1)*Δ
        s = s + quad_func(f, aa, bb)
    return s
\end{pythoncode}

Contoh penggunaan:
\begin{pythoncode}
from math import cos, pi

import sympy
x = sympy.symbols("x")
func_symb = 6 + 3*sympy.cos(x)
resExact = sympy.N(sympy.integrate(func_symb, (x, 0, sympy.pi/2)))

def my_func(x):
    return 6 + 3*cos(x)

# Pastikan fungsi-fungsi yang terkait sudah di-import atau didefinisikan

a = 0.0
b = pi/2
resN = apply_quadrature_multi_interval(
    integ_newtoncotes_open6seg, my_func, a, b, 10
)
print("resN = %18.10f" % resN)
print("res  = %18.10f" % resExact)
print("err  = %18.10e" % abs(resExact-resN))

resN = apply_quadrature_multi_interval(
    integ_simpson38, my_func, a, b, 10
)
print("\nUsing integ_simpson38 10 intervals")
print("resN = %18.10f" % resN)
print("res  = %18.10f" % resExact)
print("err  = %18.10e" % abs(resExact-resN))

resN = apply_quadrature_multi_interval(
    integ_boole, my_func, a, b, 10
)
print("\nUsing integ_boole 10 intervals")
print("resN = %18.10f" % resN)
print("res  = %18.10f" % resExact)
print("err  = %18.10e" % abs(resExact-resN))
\end{pythoncode}


\subsection{Integral yang melibatkan batas tak-hingga}
Tinjau integral:
\begin{equation}
\int_{a}^{b} f(x)\ \mathrm{d}x
\label{eq:orig_integ}
\end{equation}
Jika $a$ dan/atau $b$ bernilai tak-hingga maka, kita dapat menggunakan suatu bilangan
yang cukup besar atau cukup kecil, atau \textit{practical infinity},
untuk mengaproksimasi tak-hingga dan metode
standard seperti aturan trapesium, Simpson, dan lainnya dapat digunakan.

Cara ini efektif apabila integran $f(x)$ mendekati nol pada tak-hingga dan nilai fungsi
sudah sangat kecil pada \textit{practical infinity}.
Pada beberapa fungsi atau integran yang terlokalisasi pada suatu titik, seperti Gaussian,
hal ini cukup efektif.

Cara ini tidak efektif apabila $f(x)$ mendekati nol dengan sangat lambat sehingga
\textit{practical infinity} yang digunakan menjadi sangat besar sehingga diperlukan
banyak selang untuk mengevaluasi integral. Metode adaptif dapat digunakan untuk
mengatasi hal ini.

Metode lain yang dapat digunakan adalah dengan melakukan penggantian variabel.
Kita akan menggunakan penggantian variabel $x \rightarrow \dfrac{1}{t}$ dan melakukan
integrasi pada domain $t$.

Dengan menggunakan substitusi variabel $x \rightarrow \dfrac{1}{t}$, sehingga
$\mathrm{d}x \rightarrow -\dfrac{1}{t^2} \mathrm{d}t$. Batas integrasi menjadi
$x=a \rightarrow t=1/a$ dan $x=b \rightarrow t=1/b$.
Integral pada Pers. \eqref{eq:orig_integ} menjadi:
\begin{align*}
\int_{a}^{b} f(x)\ \mathrm{d}x & = \int_{1/a}^{1/b}
\left( -\frac{1}{t^2} \right) f\left(\frac{1}{t}\right)\ \mathrm{dt} \\
& = \int_{1/b}^{1/a}
\left( \frac{1}{t^2} \right) f\left(\frac{1}{t}\right)\ \mathrm{dt}
\end{align*}
Metode integrasi Newton-Cotes selang terbuka biasanya digunakan
untuk menghitung integral ini secara numerik (mengapa?).

Sebagai contoh, kita akan menghitung integral:
\begin{equation}
\int_{1}^{\infty} \frac{1}{x^3 + 1}\ \mathrm{d}x
\end{equation}
SymPy mengalami kesulitan untuk mengevaluasi integral ini secara analitik
jika kita menggunakan \pyinline{sympy.oo} sebagai batas atas, namun kita
dapat melakukan aproksimasi dengan menggunakan suatu bilangan yang besar
sebagai pengganti dari tak-hingga. Untuk integral yang ada pada
Latihan Chapra 22.9, kita dapat menggunakan \pyinline{sympy.oo} secara
langsung.

Sebagai alternatif lain, Anda juga dapat menggunakan Wolfram Alpha
atau Mathematica.

Akses laman
{\scriptsize\url{https://www.wolframalpha.com/}} dan ketikkan teks
berikut pada input teks yang tersedia.
\begin{textcode}
N[Integrate[1/(x^3 +1), {x,1,Infinity}], 20]
\end{textcode}
Hasil diberikan dalam 20 digit. Anda dapat membandingkannya dengan hasil SymPy.

Berikut ini adalah kode Python yang akan kita gunakan.
\begin{pythoncode}
import sympy
x = sympy.symbols("x")
func_symb = 1/(x**3 + 1)
a = 1.0
# SymPy cannot evaluate this integral when we used sympy.oo
# We use a large number instead
resExact = sympy.N(sympy.integrate(func_symb, (x, a, 1e10)))

#
# Don't forget to import the needed functions or add their
# definitions
#

def my_func(x):
    return 1/(x**3 + 1)

b = 1000.0 # practical, approximate infinity

resApproxInf = sympy.N(sympy.integrate(func_symb, (x, a, b)))

print()
print("resExact     = %18.12f" % resExact) # "exact" result from SymPy
print("resApproxInf = %18.12f" % resApproxInf) # result with approximate infinity
print("diff         = %18.10e" % abs(resExact-resApproxInf))
\end{pythoncode}

Hasil
\begin{textcode}
resExact     =     0.373550727891
resApproxInf =     0.373550227891
diff         =   4.9999999890e-07    
\end{textcode}

Program di atas menggunakan nilai 1000 sebagai \textit{practical infinity}.
Perbandingan antara integral eksak dengan batas atas tak hingga, \pyinline{resExact},
dengan integral eksak dengan batas atas \textit{practical infinity},
yaitu \pyinline{resApproxInf}, diberikan pada
bagian akhir. Perbedaannya berada pada orde $10^{-7}$, sehingga metode
yang menggunakan \textit{practical infinity} tidak akan memiliki ketelitian
yang lebih tingga dari ini.

Berikutnya, kita akan menggunakan aturan Boole
\begin{pythoncode}
print("\nUsing Boole's rule") # naive
for n in [1, 10, 50, 100, 200, 500, 1000, 2000, 3000, 5000, 10000]:
    resN = apply_quadrature_multi_interval(
        integ_boole, my_func, a, b, n
    )    
    print("%5d %18.10f %18.10e %18.10e" % (n,
        resN, abs(resN-resExact), abs(resN-resApproxInf)))
\end{pythoncode}

Hasil:
\begin{textcode}
Using Boole's rule
    1      38.8500245106   3.8476473783e+01   3.8476474283e+01
   10       3.8872647405   3.5137140126e+00   3.5137145126e+00
   50       0.8148676096   4.4131688175e-01   4.4131738175e-01
  100       0.4862953453   1.1274461740e-01   1.1274511740e-01
  200       0.3851574116   1.1606683700e-02   1.1607183700e-02
  500       0.3730326569   5.1807101260e-04   5.1757101260e-04
 1000       0.3735156871   3.5040805719e-05   3.4540805720e-05
 2000       0.3735497047   1.0231788408e-06   5.2317884186e-07
 3000       0.3735501911   5.3678128181e-07   3.6781282908e-08
 5000       0.3735502264   5.0144397934e-07   1.4439804352e-09
10000       0.3735502279   5.0002079310e-07   2.0794199695e-11
\end{textcode}
Pada hasil di atas, selain nilai aproksimasi integral, juga diberikan
perbedaan antara aproksimasi dengan \pyinline{resExact} dan
\pyinline{resApproxInf} (karena nilai batas atas integral yang digunakan
pada aturan Boole di atas
adalah \textit{approximate infinity}).
Dari hasil yang diberikan, kita melihat bahwa aturan Boole multi interval
kesulitan untuk mendapatkan hasil yang akurat. Dengan menggunakan 10000 subinterval,
aturan Boole mendapatkan akurasi sekitar $10^{-11}$ (dibandingkan dengan hasil eksak
yang diperoleh dengan menggunakan \textit{approximate infinity}).

Berikutnya akan dicoba menggunakan metode Romberg.
\begin{pythoncode}
print("\nUsing integ_romberg")
resN = integ_romberg(my_func, a, b, MAXIT=14)
print("resN = %18.10f %18.10e %18.10e" % (resN,
    abs(resN-resExact), abs(resN-resApproxInf)))
\end{pythoncode}

Hasil:
\begin{textcode}
Using integ_romberg
Converged
iterConv =  13
resN =       0.3735503753   3.5260186448e-07   1.4739813442e-07
\end{textcode}
Penggunaan aturan Romberg dapat mencapai orde kesalahan $10^{-7}$
dengan kurang lebih 13 level rekursi.

Berikutnya akan dicoba menggunakan aturan Newton-Cotes untuk interval terbuka
dan penggantian variabel.
\begin{pythoncode}
def my_func2(t):
    x = 1/t
    return (1/t**2)*my_func(x)

# x -> a, t = 1/x -> 1/a
# x -> oo, t = 1/x -> 0
t1 = 0
t2 = 1/a

print("\nUsing Newton-Cotes 6seg")
for n in [1, 2, 4, 10, 50, 100]:
    resN = apply_quadrature_multi_interval(
        integ_newtoncotes_open6seg, my_func2, t1, t2, n
    )
    print("%5d %18.10f %18.10e" % (n, resN, abs(resN-resExact)))
\end{pythoncode}

Hasil:
\begin{textcode}
Using Newton-Cotes 6seg
    1       0.3743445981   7.9387016285e-04
    2       0.3735469160   3.8118722618e-06
    4       0.3735507339   5.9745146053e-09
   10       0.3735507279   3.8686998050e-11
   50       0.3735507279   3.5527136788e-15
  100       0.3735507279   7.7715611724e-16
\end{textcode}

Aturan Newton-Cotes 6-segment multi-interval pada integral yang sudah
ditransformasi dapat mencapai orde kesalahan $10^{-11}$ hanya dengan 10 interval
(sekitar 50 evaluasi fungsi).

Perhatikan bahwa penggunaan metode transformasi variabel ini tidak selalu
memberikan hasil yang lebih baik. Hal ini sangat bergantung pada integran yang
terlibat.
Anda disarankan untuk membuat plot dari integran maupun
integran yang sudah ditransformasi untuk mendapatkan informasi lebih lanjut
sebelum mengaplikasikan metode yang sudah ada.


\subsection{Diferensiasi numerik}

Chapra Contoh 23.1
\begin{pythoncode}
def forward_diff_Oh(f, x, h):
    return ( f(x+h) - f(x) )/h

def forward_diff_Oh2(f, x, h):
    return ( -f(x+2*h) + 4*f(x+h) - 3*f(x) )/(2*h)

def backward_diff_Oh(f, x, h):
    return ( f(x) - f(x-h) )/h

def backward_diff_Oh2(f, x, h):
    return ( 3*f(x) - 4*f(x-h) + f(x-2*h) )/(2*h)

def centered_diff_Oh2(f, x, h):
    return ( f(x+h) - f(x-h) )/(2*h)

def centered_diff_Oh4(f, x, h):
    return ( -f(x+2*h) + 8*f(x+h) - 8*f(x-h) + f(x-2*h) )/(12*h)

def f(x):
    return -0.1*x**4 - 0.15*x**3 - 0.5*x**2 - 0.25*x + 1.2

x = 0.5
h = 0.25
true_val = -0.9125

print("Using h = ", h)

print()
print("Forward diff")
#
df = forward_diff_Oh(f, x, h)
print("Oh  = %18.10f %18.10e" % (df, abs(df-true_val)) )
#
df = forward_diff_Oh2(f, x, h)
print("Oh2 = %18.10f %18.10e" % (df, abs(df-true_val)) )


print()
print("Backward diff")
#
df = backward_diff_Oh(f, x, h)
print("Oh  = %18.10f %18.10e" % (df, abs(df-true_val)) )
#
df = backward_diff_Oh2(f, x, h)
print("Oh2 = %18.10f %18.10e" % (df, abs(df-true_val)) )

print()
print("Centered diff")
#
df = centered_diff_Oh2(f, x, h)
print("Oh2 = %18.10f %18.10e" % (df, abs(df-true_val)) )
#
df = centered_diff_Oh4(f, x, h)
print("Oh4 = %18.10f %18.10e" % (df, abs(df-true_val)) )
\end{pythoncode}

Hasil:
\begin{textcode}
Using h =  0.25

Forward diff
Oh  =      -1.1546875000   2.4218750000e-01
Oh2 =      -0.8593750000   5.3125000000e-02
    
Backward diff
Oh  =      -0.7140625000   1.9843750000e-01
Oh2 =      -0.8781250000   3.4375000000e-02
    
Centered diff
Oh2 =      -0.9343750000   2.1875000000e-02
Oh4 =      -0.9125000000   2.2204460493e-1
\end{textcode}

Chapra Contoh 23.2
\begin{pythoncode}
# Tambahkan kode-kode yang diperlukan di sini
# (sama dengan Chapra Contoh 23.1)

x = 0.5
h1 = 0.5
h2 = 0.25
true_val = -0.9125

Dh1 = centered_diff_Oh2(f, x, h1)
Dh2 = centered_diff_Oh2(f, x, h2)

# Richardson extrapolation
Dh12 = 4*Dh2/3 - Dh1/3

print("Dh1  = %18.10f %18.10e" % (Dh1, abs(Dh1-true_val)) )
print("Dh2  = %18.10f %18.10e" % (Dh2, abs(Dh2-true_val)) )
print("Dh12 = %18.10f %18.10e" % (Dh12, abs(Dh12-true_val)) )
\end{pythoncode}

Hasil:
\begin{textcode}
Dh1  =      -1.0000000000   8.7500000000e-02
Dh2  =      -0.9343750000   2.1875000000e-02
Dh12 =      -0.9125000000   3.3306690739e-16    
\end{textcode}

\end{document}
