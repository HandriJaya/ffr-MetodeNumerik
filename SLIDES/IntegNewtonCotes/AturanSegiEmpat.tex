\section{Aturan segi empat}

\begin{frame}
\frametitle{Aturan Segi empat}

\begin{columns}

\begin{column}{0.5\textwidth}
{\centering
\includegraphics[width=\textwidth]{images_priv/Chapra_Fig_Pt6_6.pdf}
}
\end{column}

\begin{column}{0.5\textwidth}
Aproksimasi aturan segi-empat
\begin{equation*}
I = \int_{a}^{b} f(x) \,\mathrm{d}x \approx \sum_{i} f(x_{i}) \Delta x
\end{equation*}
\end{column}

\end{columns}
  
\end{frame}


\begin{frame}[fragile]
\frametitle{Contoh}

\begin{columns}

\begin{column}{0.5\textwidth}
Hitung aproksimasi dari integral berikut dengan menggunakan aturan segiempat:
\begin{equation*}
\int_{0}^{1} \sin(x) \, \mathrm{d}x
\end{equation*}
Bandingkan hasilnya dengan hasil analitik.  
\end{column}

\begin{column}{0.5\textwidth}
Perhitungan analitik dengan menggunakan SymPy:
\begin{pythoncode}
from sympy import *
x = symbols("x")
f = sin(x)
a = 0
b = 1
res = integrate(f, (x,0,1))
pprint(res)
print(N(res))
\end{pythoncode}
Hasil:
\begin{equation*}
1 - \cos(1) \approx 0.459697694131860
\end{equation*}
\end{column}

\end{columns}

\end{frame}


\begin{frame}[fragile]

\begin{columns}

\begin{column}{0.5\textwidth}
\begin{minted}[fontsize=\scriptsize]{python}
import numpy as np
    
def f(x):
    return np.sin(x)
  
a = 0.0
b = 1.0  
Npoints = 100
xgrid = np.linspace(a, b, Npoints)
Δx = xgrid[1] - xgrid[0]
fgrid = f(xgrid)
I_exact = 1.0 - np.cos(1.0)
I = sum(fgrid)*Δx
print("Npoints = ", Npoints)
print("Integral approx = %18.10f" % I)
print("Error = %18.10e" % abs(I - I_exact))  
\end{minted}
\end{column}

\begin{column}{0.5\textwidth}
Contoh keluaran:
\begin{textcode}
Npoints =  100
Integral approx = 0.4639436390
Error = 4.2459448568e-03
\end{textcode}
\end{column}

\end{columns}

\end{frame}


\begin{frame}[fragile]
\frametitle{Variasi jumlah titik}

Coba variasikan jumlah titik yang digunakan dan bandingkan hasilnya dengan
hasil analitik.

Contoh keluaran: (jumlah titik, aproksimasi integral, error absolut)
\begin{textcode}
   10    0.5059730448 4.62754e-02
  100    0.4639436390 4.24594e-03
 1000    0.4601188124 4.21118e-04
 5000    0.4597818565 8.41624e-05
10000    0.4597397715 4.20774e-05
\end{textcode}

\end{frame}



