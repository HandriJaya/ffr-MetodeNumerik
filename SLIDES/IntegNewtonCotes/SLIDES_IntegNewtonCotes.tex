\documentclass[10pt,fleqn,aspectratio=169]{beamer}
%\documentclass[fleqn,aspectratio=169]{beamer}

\usepackage{amsmath, amssymb}
\usepackage{fancyvrb, color, graphicx, hyperref, url}

%\setbeamersize{text margin left=5pt, text margin right=5pt}

\setlength{\parskip}{\smallskipamount}
\setlength{\parindent}{0pt}

\usepackage{fontspec}
%\setmonofont{DejaVu Sans Mono}
\setmonofont{JuliaMono-Regular}

\usefonttheme[onlymath]{serif}

\usepackage{minted}
\newminted{python}{breaklines,fontsize=\footnotesize}
\newminted{julia}{breaklines,fontsize=\footnotesize}
\newminted{bash}{breaklines,fontsize=\footnotesize}
\newminted{text}{breaklines,fontsize=\footnotesize}

\newcommand{\txtinline}[1]{\mintinline[breaklines,fontsize=\footnotesize]{text}{#1}}
\newcommand{\pyinline}[1]{\mintinline[breaklines,fontsize=\footnotesize]{python}{#1}}
\newcommand{\jlinline}[1]{\mintinline[breaklines,fontsize=\footnotesize]{julia}{#1}}

\definecolor{mintedbg}{rgb}{0.95,0.95,0.95}
\usepackage{mdframed}

\BeforeBeginEnvironment{minted}{\begin{mdframed}[backgroundcolor=mintedbg,%
  rightline=false,leftline=false,topline=false,bottomline=false]}
\AfterEndEnvironment{minted}{\end{mdframed}}

% https://tex.stackexchange.com/questions/33969/changing-font-size-of-selected-slides-in-beamer

\usepackage{environ}
%
% Custom font for a frame.
%
\newcommand{\customframefont}[1]{
  \setbeamertemplate{itemize/enumerate body begin}{#1}
  \setbeamertemplate{itemize/enumerate subbody begin}{#1}
}

\NewEnviron{framefont}[1]{
  \customframefont{#1} % for itemize/enumerate
  {#1 % For the text outside itemize/enumerate
    \BODY
  }
  \customframefont{\normalsize}
}



\begin{document}

\title{Metode Numerik}
\subtitle{Integrasi Numerik: Formula Newton-Cotes}
%\author{Fadjar Fathurrahman}
%\institute{
%Teknik Fisika \\
%Institut Teknologi Bandung
%}
\date{}


\frame{\titlepage}


\section{Aturan segi empat}

\begin{frame}
\frametitle{Aturan Segi empat}

\begin{columns}

\begin{column}{0.5\textwidth}
{\centering
\includegraphics[width=\textwidth]{images_priv/Chapra_Fig_Pt6_6.pdf}
}
\end{column}

\begin{column}{0.5\textwidth}
Aproksimasi aturan segi-empat
\begin{equation*}
I = \int_{a}^{b} f(x) \,\mathrm{d}x \approx \sum_{i} f(x_{i}) \Delta x
\end{equation*}
\end{column}

\end{columns}
  
\end{frame}


\begin{frame}[fragile]
\frametitle{Contoh}

\begin{columns}

\begin{column}{0.5\textwidth}
Hitung aproksimasi dari integral berikut dengan menggunakan aturan segiempat:
\begin{equation*}
\int_{0}^{1} \sin(x) \, \mathrm{d}x
\end{equation*}
Bandingkan hasilnya dengan hasil analitik.  
\end{column}

\begin{column}{0.5\textwidth}
Perhitungan analitik dengan menggunakan SymPy:
\begin{pythoncode}
from sympy import *
x = symbols("x")
f = sin(x)
a = 0
b = 1
res = integrate(f, (x,0,1))
pprint(res)
print(N(res))
\end{pythoncode}
Hasil:
\begin{equation*}
1 - \cos(1) \approx 0.459697694131860
\end{equation*}
\end{column}

\end{columns}

\end{frame}


\begin{frame}[fragile]

\begin{columns}

\begin{column}{0.5\textwidth}
\begin{minted}[fontsize=\scriptsize]{python}
import numpy as np
    
def f(x):
    return np.sin(x)
  
a = 0.0
b = 1.0  
Npoints = 100
xgrid = np.linspace(a, b, Npoints)
Δx = xgrid[1] - xgrid[0]
fgrid = f(xgrid)
I_exact = 1.0 - np.cos(1.0)
I = sum(fgrid)*Δx
print("Npoints = ", Npoints)
print("Integral approx = %18.10f" % I)
print("Error = %18.10e" % abs(I - I_exact))  
\end{minted}
\end{column}

\begin{column}{0.5\textwidth}
Contoh keluaran:
\begin{textcode}
Npoints =  100
Integral approx = 0.4639436390
Error = 4.2459448568e-03
\end{textcode}
\end{column}

\end{columns}

\end{frame}


\begin{frame}[fragile]
\frametitle{Variasi jumlah titik}

Coba variasikan jumlah titik yang digunakan dan bandingkan hasilnya dengan
hasil analitik.

Contoh keluaran: (jumlah titik, aproksimasi integral, error absolut)
\begin{textcode}
   10    0.5059730448 4.62754e-02
  100    0.4639436390 4.24594e-03
 1000    0.4601188124 4.21118e-04
 5000    0.4597818565 8.41624e-05
10000    0.4597397715 4.20774e-05
\end{textcode}

\end{frame}




\section{Formula Newton-Cotes}

\begin{frame}
\frametitle{Formula Newton-Cotes}
\fontsize{9pt}{8.0}\selectfont
% 7.2 baselineskip

\begin{columns}

\begin{column}{0.5\textwidth}
Formula Newton-Cotes merupakan skema integrasi numerik yang paling umum.
Formula ini menggunakan strategi dengan cara mengganti fungsi atau tabel data
yang ingin dicari nilai integralnya dengan suatu fungsi pendekatan yang
mudah untuk diintegralkan:
\begin{equation*}
I = \int_{a}^{b} f(x) \, \mathrm{d}x \approx \int_{a}^{b} f_{n}(x) \, \mathrm{d}x
\end{equation*}
di mana $f_{n}(x)$ adalah sebuah polinomial:
\begin{equation*}
f_{n}(x) = a_{0} + a_{1}x + \cdots + a_{n-1} x^{n-1} + a_{n} x^{n}
\end{equation*}
di mana $n$ adalah orde dari polinomial.    
\end{column}

\begin{column}{0.5\textwidth}
{\centering
\includegraphics[width=\textwidth]{images_priv/Chapra_Fig_21_1.pdf}
\par}

(a) polinomial orde 1 (b) polinomial orde 2
\end{column}

\end{columns}



\end{frame}
\section{Aturan trapesium}

\begin{soal}[Chapra Contoh 21.1]
Gunakan aturan trapesium untuk aproksimasi integral berikut.
\begin{equation*}
f(x) = 0.2 + 25x - 200x^2 + 675x^3 - 900x^4 + 400x^5
\end{equation*}
dari $a = 0$ sampai $b = 0.8$. Bandingkan hasilnya dengan hasil analitik,
yaitu 1.640533. Hitung juga estimasi error dari aproksimasi yang digunakan.
\end{soal}

Anda dapat melengkapi kode berikut ini untuk Chapra Contoh 21.1
\begin{pythoncode}
def my_func(x):
    return 0.2 + 25*x - 200*x**2 + 675*x**3 - 900*x**4 + 400*x**5

a = 0.0
b = 0.8
f_a = my_func(a)
f_b = my_func(b)

I_exact = 1.640533 # from the book
# Alternatively, you can use SymPy to do the integration

# Use trapezoid rule here
I = .... # Lengkapi

E_t = I_exact - I
ε_t = E_t/I_exact * 100 # in percent
print("Integral result = %.6f" % I)
print("True integral   = %.6f" % I_exact)
print("True error      = %.6f" % E_t)
print("ε_t             = %.1f%%" % ε_t)

import sympy
x = sympy.symbols("x")
f = 0.2 + 25*x - 200*x**2 + 675*x**3 - 900*x**4 + 400*x**5
d2f = f.diff(x,2) # calculate 2nd derivative
avg_d2f_xi = sympy.integrate( d2f, (x,a,b) )/(b - a)
E_a = -1/12*avg_d2f_xi*(b - a)**3 # Persamaan 21.6
print("Approx error    = %.6f" % E_a)
\end{pythoncode}

Aturan trapesium (dan aturan integrasi Newton-Cotes lainnya) dapat digunakan
untuk lebih dari satu interval.

\begin{soal}
Gunakan aturan trapesium multi-interval dengan jumlah segmen 2 untuk menghitung
aproksimasi integral pada Soal sebelumnya (Chapra Contoh 21.1).
Coba juga ganti jumlah segmen yang digunakan dan bandingkan hasilnya
dengan hasil analitik.
\end{soal}

Anda dapat melengkapi kode berikut ini. Lihat juga persaman yang digunakan pada
buku Chapra.
\begin{pythoncode}
def my_func(x):
    return 0.2 + 25*x - 200*x**2 + 675*x**3 - 900*x**4 + 400*x**5

# N is number of segments
# Npoints is N + 1
# f is function
def integ_trapz_multiple( f, a, b, N ):
    x0 = a
    xN = b
    h = (b - a)/N
    ss = 0.0
    for i in range(1,N):
        xi = x0 + i*h
        ss = ss + f(xi)
    I = .... # lengkapi
    return I

Nsegments = 2
a = 0.0
b = 0.8
I_exact = 1.640533 # from the book

I = integ_trapz_multiple( my_func, a, b, Nsegments )

print("Nsegments = ", Nsegments)
print("Integral result = %.6f" % I)
E_t = I_exact - I
print("True integral   = %.6f" % I_exact)
print("True error      = %.6f" % E_t)
ε_t = E_t/I_exact * 100
print("ε_t             = %.1f%%" % ε_t)

import sympy
x = sympy.symbols("x")
f = 0.2 + 25*x - 200*x**2 + 675*x**3 - 900*x**4 + 400*x**5
d2f = f.diff(x,2)
#sympy.pprint(d2f)
avg_d2f_xi = sympy.integrate( d2f, (x,a,b) )/(b - a)
#print("avg_d2f_xi = ", avg_d2f_xi)
E_a = -1/12*avg_d2f_xi*(b - a)**3/Nsegments**2
print("Approx error    = %.6f" % E_a)
\end{pythoncode}

\begin{soal}[Chapra Contoh 21.3]
Hitung integral berikut ini dengan menggunakan aturan trapesium:
\begin{equation*}
d = \frac{gm}{c} \int_{0}^{t} (1 - e^{-(c/m)t}) \, \mathrm{d}t
\end{equation*}
Bandingkah hasil yang diperoleh dengan hasil analitik. Lihat buku Chapra untuk nilai-nilai
numerik yang diperlukan atau lihat pada kode Python di bawah.
\end{soal}

Anda dapat melengkapi kode berikut ini.
\begin{pythoncode}
from math import exp

def my_func(t):
    g = 9.8
    m = 68.1
    c = 12.5
    return g*m/c * (1 - exp(-(c/m)*t) )
    
# ... definisi integ_trapz_multiple
# ... LENGKAPI

t = 10.0
a = 0.0
b = t
    
d_exact = 289.43515 # dari buku
    
print("------------------------------------------------------")
print("   N          h          d          E_t         ε_t")
print("------------------------------------------------------")
for Nsegments in [10, 20, 50, 100, 200, 500, 1000, 2000, 5000, 5000, 10000]:
    h = (b - a)/Nsegments
    d = integ_trapz_multiple( my_func, a, b, Nsegments )
    E_t = d_exact - d
    ε_t = E_t/d_exact * 100
    print("%5d  %10.4f  %10.4f   %10.4e %10.2e%%" % (Nsegments, h, d, E_t, ε_t))


# Calculate "Exact" result using SymPy
import sympy
g = 9.8
m = 68.1
c = 12.5
t = sympy.symbols("t")
f = g*m/c * (1 - sympy.exp(-(c/m)*t) )
d_sympy = sympy.integrate( f, (t,0,10))
    
print()
print("SymPy result:")
print("d_sympy = ", d_sympy)    
\end{pythoncode}

Apakah hasil yang Anda peroleh sama dengan yang diberikan pada Chapra?
Jika ada perbedaan, apakah yang menyebabkan perbedaan tersebut?

\begin{frame}
\frametitle{Latihan}

Hitung aproksimasi dari integral-integral berikut dengan aturan segiempat dan trapesium.

\begin{columns}

\begin{column}{0.5\textwidth}
\begin{align*}
& \int_{0}^{\pi/2} (6 + 3\cos(x)) \, \mathrm{d}x \\
& \int_{0}^{3} (1 - e^{-2x}) \, \mathrm{d}x \\
& \int_{-2}^{4} (1 - x - 4x^3 + 2x^5) \, \mathrm{d}x \\
& \int_{1}^{2} (x - 2/x)^2 \, \mathrm{d}x
\end{align*}      
\end{column}

\begin{column}{0.5\textwidth}
\begin{align*}
& \int_{-3}^{5} (4x - 3)^3 \, \mathrm{d}x \\
& \int_{0}^{3} x^2 e^x \, \mathrm{d}x \\
& \int_{0}^{1} 14^{2x} \, \mathrm{d}x \\
& \int_{1}^{5} \frac{\sin(x)}{x}\,\mathrm{d}x
\end{align*}
\end{column}

\end{columns}

\end{frame}


\end{document}
