\documentclass[10pt,fleqn,aspectratio=169]{beamer}
%\documentclass[fleqn,aspectratio=169]{beamer}

\usepackage{amsmath, amssymb}
\usepackage{fancyvrb, color, graphicx, hyperref, url}

%\setbeamersize{text margin left=5pt, text margin right=5pt}

\setlength{\parskip}{\smallskipamount}
\setlength{\parindent}{0pt}

\usepackage{fontspec}
%\setmonofont{DejaVu Sans Mono}
\setmonofont{JuliaMono-Regular}

\usefonttheme[onlymath]{serif}

\usepackage{minted}
\newminted{python}{breaklines,fontsize=\footnotesize}
\newminted{julia}{breaklines,fontsize=\footnotesize}
\newminted{bash}{breaklines,fontsize=\footnotesize}
\newminted{text}{breaklines,fontsize=\footnotesize}

\newcommand{\txtinline}[1]{\mintinline[breaklines,fontsize=\footnotesize]{text}{#1}}
\newcommand{\pyinline}[1]{\mintinline[breaklines,fontsize=\footnotesize]{python}{#1}}
\newcommand{\jlinline}[1]{\mintinline[breaklines,fontsize=\footnotesize]{julia}{#1}}

\definecolor{mintedbg}{rgb}{0.95,0.95,0.95}
\usepackage{mdframed}

\BeforeBeginEnvironment{minted}{\begin{mdframed}[backgroundcolor=mintedbg,%
  rightline=false,leftline=false,topline=false,bottomline=false]}
\AfterEndEnvironment{minted}{\end{mdframed}}

% https://tex.stackexchange.com/questions/33969/changing-font-size-of-selected-slides-in-beamer

\usepackage{environ}
%
% Custom font for a frame.
%
\newcommand{\customframefont}[1]{
  \setbeamertemplate{itemize/enumerate body begin}{#1}
  \setbeamertemplate{itemize/enumerate subbody begin}{#1}
}

\NewEnviron{framefont}[1]{
  \customframefont{#1} % for itemize/enumerate
  {#1 % For the text outside itemize/enumerate
    \BODY
  }
  \customframefont{\normalsize}
}



\begin{document}

\title{Numerical Methods}
\subtitle{Ordinary Differential Equation}
\author{Fadjar Fathurrahman}
\institute{
Teknik Fisika \\
Institut Teknologi Bandung
}
\date{}


\frame{\titlepage}


\begin{frame}
\frametitle{Examples}

Newton's equation of motion
\begin{equation*}
\frac{\mathrm{d}v}{\mathrm{d}t} = \frac{F}{m}
\end{equation*}

Fourier's heat law:
\begin{equation*}
q = -k \frac{\mathrm{d}T}{\mathrm{d}x}
\end{equation*}

Fick's law of diffusion:
\begin{equation*}
J = -D \frac{\mathrm{d}c}{\mathrm{d}x}
\end{equation*}

Faraday's law:
\begin{equation*}
\Delta V_{L} = L \frac{\mathrm{d}i}{\mathrm{d}t}
\end{equation*}

\end{frame}


% ----------------------------------------------------
\begin{frame}
\frametitle{General form of initial value problem}

Solve the ODE:
\begin{equation}
\frac{\mathrm{d}y}{\mathrm{d}x} = f(x,y)
\label{eq:ode_gen01}
\end{equation}
given the initial condition $y(x=x_0) = y_0$.

$y(x)$ is the dependent variable (or the function that we search for)
and $x$ is the independent variable.

For initial value problem it is usually
more common to use $t$ as the independent variable:
\begin{equation}
\frac{\mathrm{d}y}{\mathrm{d}t} = f(t,y)
\label{eq:ode_gen02}
\end{equation}

\end{frame}

% -----------------------

\begin{frame}
\frametitle{Methods to solve IVP}
  
\begin{itemize}
\item one-step methods
\item multi-step methods
\item implicit methods for stiff ODEs
\end{itemize}

\end{frame}



% -----------------------

\begin{frame}
\frametitle{One-step methods}

A simple method:
\begin{equation*}
y_{i+1} = y_{i} + \phi h
\end{equation*}

$y_{i}$: old value

$y_{i+1}$: new value

$\phi$: slope estimate

\end{frame}


\begin{frame}
\frametitle{Euler's method}

\begin{columns}

\begin{column}{0.5\textwidth}
  {\centering
  \includegraphics[width=0.9\textwidth]{images_priv/Chapra_Fig_25_1.pdf}
  \par}    
\end{column}

\begin{column}{0.5\textwidth}
  Use the 1st derivative as direct estimate of slope at $x_i$:
  \begin{equation*}
  \phi = f(x_i, y_i)
  \end{equation*}
  or
  \begin{equation*}
  y_{i+1} = y_{i} + f(x_i, y_i) h
  \end{equation*}
\end{column}

\end{columns}

\end{frame}


% ------------------------------------

\begin{frame}
\frametitle{Example}
Solve
\begin{equation*}
\frac{\mathrm{d}y}{\mathrm{d}x} = -2x^3 + 12x^2 - 20x + 8.5
\end{equation*}
from $x = 0$ to $x = 4$ with a step size $h=0.5$.
Initial condition: $y(x=0) = 1$.

Comparing with general form of ODE \eqref{eq:ode_gen01}, we have:
\begin{equation*}
f(x,y) = -2x^3 + 12x^2 - 20x + 8.5
\end{equation*}
Note that in this case $f(x,y)$ does not depend of $y$.

Compare the obtained numerical solution with exact solution:
\begin{equation*}
y(x) = -0.5x^4 + 4x^3 - 10x^2 + 8.5x + 1
\end{equation*}

\end{frame}


\begin{frame}[fragile]
\frametitle{Implementation}

Let's make define several functions that we will use later.

First the one-step application of Euler's method.

This code fragment implements one-step application of Euler's method for ODE
\begin{equation*}
y_{i+1} = y_{i} + f(x_i, y_i) h
\end{equation*}

\begin{pythoncode}
def ode_euler_1step(dfunc, xi, yi, h):
    return yi + dfunc(xi,yi)*h
\end{pythoncode}

The passed function \pyinline{dfunc} implement the RHS $f(x,y)$. It should be defined like:
\begin{pythoncode}
def dfunc(x,y):
    return ... # implement f(x,y)
\end{pythoncode}

We can call \pyinline{ode_euler_1step} multiple times to advance the solution from initial
condition until the required $x$ (independent variable).

\end{frame}


\begin{frame}[fragile]
The previous code fragments are quite general. Now let's code specific
details for our current problem, i.e:
\begin{equation*}
\frac{\mathrm{d}y}{\mathrm{d}x} = -2x^3 + 12x^2 - 20x + 8.5
\end{equation*}

The following implement the RHS function $f(x,y)$:
\begin{pythoncode}
def deriv(x, y):
    return -2*x**3 + 12*x**2 - 20*x + 8.5
\end{pythoncode}
and the exact solution:
\begin{pythoncode}
def exact_sol(x):
    return -0.5*x**4 + 4*x**3 - 10*x**2 + 8.5*x + 1
\end{pythoncode}
\end{frame}


\begin{frame}[fragile]

Test \pyinline{ode_euler_1step}:
\begin{pythoncode}
x0 = 0.0; y0 = 1.0 # Initial condition
x = x0; y = y0; h = 0.5 # Using step of 0.5, starting from x0 and y0
xp1 = x + h # we are searching for solution at x = 0.5, increment by step size
yp1 = ode_euler_1step(deriv, x, y, h)
y_true = exact_sol(xp1)
ε_t = (y_true - yp1)/y_true * 100 # relative error in percent
print("First step : x = %f y_true = %.5f y = %.5f ε_t = %.1f %%" %
  (xp1, y_true, yp1, ε_t))
\end{pythoncode}

Your task: continue this for the next steps.
You can use loop.

Example output (including 2nd step):
\begin{textcode}
First step : x = 0.500000 y_true = 3.21875 y = 5.25000 ε_t = -63.1 %
Second step: x = 1.000000 y_true = 3.00000 y = 5.87500 ε_t = -95.8 %
\end{textcode}

\end{frame}



\begin{frame}[fragile]

Example of calling \pyinline{ode_euler_1step} several times.
\begin{pythoncode}
x = x0; y = y0; h = 0.5
for i in range(0,Nsteps):
    xp1 = x + h
    yp1 = ode_euler_1step(deriv, x, y, h)
    y_true = exact_sol(xp1) # calculate exact solution if available
    # relative error in percent, you can use other
    ε_t = (y_true - yp1)/y_true * 100
    # print the results ...
    # Update x and y for the next step
    x = xp1; y = yp1
\end{pythoncode}

Your task: Save the calculation results (x and y) to array and make a plot.

You also can wrap this in a function.

\end{frame}



\end{document}
