%\documentclass[a4paper,11pt]{article} % print setting
\documentclass[a4paper,12pt]{article} % screen setting

\usepackage[a4paper]{geometry}
\geometry{verbose,tmargin=1.5cm,bmargin=1.5cm,lmargin=1.5cm,rmargin=1.5cm}

\setlength{\parskip}{\smallskipamount}
\setlength{\parindent}{0pt}

%\usepackage{cmbright}
%\renewcommand{\familydefault}{\sfdefault}

\usepackage[libertine]{newtxmath}
\usepackage[no-math]{fontspec}
%
\setmainfont{Linux Libertine O}
%\setmainfont{STIX}

%\usepackage{sfmath}
%\usepackage{mathptmx}
%\usepackage{mathpazo}

\usepackage{hyperref}
\usepackage{url}
\usepackage{xcolor}

\usepackage{amsmath}
\usepackage{amssymb}

\usepackage{graphicx}
\usepackage{float}

\usepackage{minted}
\newminted{julia}{breaklines,fontsize=\small}
\newminted{bash}{breaklines,fontsize=\small}
\newminted{text}{breaklines,fontsize=\small}

\newcommand{\txtinline}[1]{\mintinline{text}{#1}}
\newcommand{\jlinline}[1]{\mintinline{julia}{#1}}

\newmintedfile[juliafile]{julia}{breaklines,fontsize=\small}

\definecolor{mintedbg}{rgb}{0.90,0.90,0.90}
\usepackage{mdframed}

\BeforeBeginEnvironment{minted}{\begin{mdframed}[backgroundcolor=mintedbg]}
\AfterEndEnvironment{minted}{\end{mdframed}}

\usepackage{setspace}

\onehalfspacing

\usepackage{appendix}


\begin{document}


\title{Tugas Besar TF4062 - 2021}
\author{Iwan Prasetyo \\
Fadjar Fathurrahman}
\date{}
\maketitle


Buat implementasi solusi persamaan diferensial parsial pada 2d atau 3d.
\begin{itemize}
\item Dua contoh kasus wajib: \txtinline{diffusion2d} dan \txtinline{elliptic2d}
\item Satu contoh kasus pilihan: \txtinline{diffusion3d}, \txtinline{rxndiffn2d}, atau
contoh kasus lain.
\end{itemize}

Untuk setiap kasus berikan penjelasan sebagai berikut.
\begin{itemize}
\item Persamaan diferensial yang diselesaikan (dengan kondisi batas) dan penurunan
weak form.
\item Jenis elemen dan fungsi basis atau \textit{shape functions} yang digunakan
\item Visualisasi hasil
\item Penjelasan mengenai program yang ditulis: struktur data, algorithma, dll.
\end{itemize}

Tugas dikumpulkan melalui email ke \texttt{fadjar@tf.itb.ac.id}.

\end{document}
