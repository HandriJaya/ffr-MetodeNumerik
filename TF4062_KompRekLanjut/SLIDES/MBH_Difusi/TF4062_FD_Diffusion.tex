\documentclass[9pt]{beamer}

\usepackage{amsmath, amssymb}
\usepackage{fancyvrb, color, graphicx, hyperref, url}

\setbeamersize{text margin left=5pt, text margin right=5pt}

\setlength{\parskip}{\smallskipamount}
\setlength{\parindent}{0pt}

\usepackage{fontspec}
\setmonofont{DejaVu Sans Mono}

\usefonttheme[onlymath]{serif}

\usepackage{minted}
\newminted{python}{breaklines}
\newminted{julia}{breaklines}
\newminted{bash}{breaklines}
\newminted{text}{breaklines}

\newcommand{\txtinline}[1]{\mintinline{text}{#1}}
\newcommand{\pyinline}[1]{\mintinline{python}{#1}}
\newcommand{\jlinline}[1]{\mintinline{julia}{#1}}

\definecolor{mintedbg}{rgb}{0.90,0.90,0.90}
\usepackage{mdframed}

\BeforeBeginEnvironment{minted}{\begin{mdframed}[backgroundcolor=mintedbg,%
  rightline=false,leftline=false,topline=false,bottomline=false]}
\AfterEndEnvironment{minted}{\end{mdframed}}


\begin{document}

\title{Persamaan Difusi}
\subtitle{TF4062: Komputasi Rekayasa Lanjut}
\author{Iwan Prasetyo\\
Fadjar Fathurrahman}
\institute{
Teknik Fisika \\
Institut Teknologi Bandung
}
\date{}


\frame{\titlepage}


\begin{frame}
\frametitle{Persamaan Difusi Kalor 1d}

Pada satu dimensi spasial, misalkan $x$, dapat dituliskan
%
\begin{equation}
\alpha \frac{\partial^{2}}{\partial x^2} u(x,t) = \frac{\partial}{\partial t} u(x,t)
\end{equation}

Solusi $x(x,t)$ akan dicari pada domain spasial: $0 \leq x \leq x_{f}$
dan domain temporal: $0 \leq t \leq t_{f}$.
dengan syarat batas: $u(0,t) = b_{0}(t)$ and $u(x_{f},t) = b_{x_f}(t)$
dan syarat awal: $u(x,0) = u_{0}(x)$

\end{frame}



\begin{frame}
\frametitle{Metode Euler Eksplisit}

Domain spasial dibagi menjadi $N_{x}$ segmen dengan $\Delta x = x_{f}/(N_{x}-1)$.
Domain temporal dibagi menjadi $N_{t}$ segmen dengan $\Delta t = t_{f}/(N_{t}-1)$.
Turunan parsial kedua terhadap $x$ diaproksimasi dengan menggunakan central difference.
Turunan parsial pertama terhadap $t$ diaproksimasi dengan forward difference.

Dengan menggunakan notasi berikut: $u(x,t) = u^{k}_{i}$, $u(x+\Delta x,t) = u^{k}_{i+1}$
$u(x-\Delta x,t) = u^{k}_{i-1}$, $u(x,t+\Delta t) = u^{k+1}_{i}$,
and $u(x,t-\Delta t) = u^{k-1}_{i}$
%
dapat dituliskan:
\begin{equation}
\alpha \frac{u^{k}_{i+1} - 2u^{k}_{i} + u^{k}_{i-1}}{(\Delta x)^2} =
\frac{u^{k+1}_{i} - u^{k}_{i}}{\Delta t}
\end{equation}

Dengan menggunakan definisi:
\begin{equation}
r = \alpha \frac{\Delta t}{(\Delta x)^2}
\end{equation}
Persamaan ini dapat dipecahkan untuk mendapatkan $u^{k+1}_{i}$
\begin{equation}
u^{k+1}_{i} = r \left( u^{k}_{i+1} + u^{k}_{i-1} \right) + (1 - 2r) u^{k}_{i}
\end{equation}
untuk $i = 1, 2, \ldots, N_{x}-1$.

Dapat ditunjukkan bahwa skema ini akan stabil jika:
\begin{equation}
r = \alpha \frac{\Delta t}{(\Delta x)^2} \leq \frac{1}{2}
\end{equation}

\end{frame}


\begin{frame}
\frametitle{An example}

Cari solusi numerik persamaan kalor:
\begin{equation*}
\frac{\partial^{2}}{\partial x^2} u(x,t) = \frac{\partial}{\partial t} u(x,t)
\end{equation*}
pada domain spasial: $0 \leq x \leq 1$ and $0 \leq t \leq 0.1$.

Syarat batas: $u(0,t) = 0$ $u(1,t) = 0$

Syarat awal: $u(x,0) = \sin(\pi x)$

%
Bandingkan dengan solusi analitik:
\begin{equation}
u(x,t) = \sin(\pi x) \exp(-\pi^2 t)
\end{equation}

\end{frame}


\begin{frame}
\frametitle{Metode Euler implisit}

Turunan parsial kedua terhadap $x$ diaproksimasi dengan menggunakan central difference.
Turunan parsial pertama terhadap $t$ diaproksimasi dengan backward difference.

\begin{equation}
\alpha \frac{u^{k}_{i+1} - 2u^{k}_{i} + u^{k}_{i-1}}{(\Delta x)^2} = \frac{u^{k}_{i} - u^{k-1}_{i}}{\Delta t}
\end{equation}

Dengan menggunakan definisi
\begin{equation}
r = \alpha \frac{\Delta t}{(\Delta x)^2}
\end{equation}
Diperoleh persamaan implisit untuk $i = 2, 3, \ldots, N_{x}-2$:
\begin{equation}
-ru^{k}_{i-1}+ (1 + 2r)u^{k}_{i} - ru^{k}_{i+1} = u^{k-1}_{i}
\end{equation}

Dalam bentuk matriks:
\begin{equation}
\begin{bmatrix}
1 + 2r & -r & 0 & \cdots & 0 & 0 \\
-r & 1 + 2r & -r & \cdots & 0 & 0 \\
0 & -r & 1 + 2r & \cdots & 0 & 0 \\
\cdots & \cdots & \cdots & \cdots & \cdots & \cdots \\
0 & 0 & 0 & \cdots & 1 + 2r & -r \\
0 & 0 & 0 & \cdots & -r & 1 + 2r
\end{bmatrix}
\begin{bmatrix}
u^{k}_{1} \\
u^{k}_{2} \\
u^{k}_{3} \\
\cdots \\
u^{k}_{N_{x}-2} \\
u^{k}_{N_{x}-1}
\end{bmatrix} =
\begin{bmatrix}
u^{k-1}_{1} + ru^{k}_{0} \\
u^{k-1}_{2} \\
u^{k-1}_{3} \\
\cdots \\
u^{k-1}_{N_{x}-2} \\
u^{k-1}_{N_{x}-1} + ru^{k}_{N_{x}}
\end{bmatrix}
\end{equation}

\end{frame}


\begin{frame}
\frametitle{Metode Crank-Nicholson}

Metode Crank-Nicholson diperoleh dengan menggunakan rata-rata aproksimasi central difference
antara titik waktu $k + 1$ dan $k$ sehingga diperoleh:
\begin{equation}
\frac{\alpha}{2} \left(
\frac{u^{k+1}_{i+1} - 2u^{k+1}_{i} + u^{k+1}_{i-1}}{(\Delta x)^2} +
\frac{u^{k}_{i+1} - 2u^{k}_{i} + u^{k}_{i-1}}{(\Delta x)^2}
\right) =
\frac{u^{k+1}_{i} - u^{k}_{i}}{\Delta t}
\end{equation}
atau:
\begin{equation}
ru^{k+1}_{i+1} - 2ru^{k+1}_{i} + ru^{k+1}_{i-1} + ru^{k}_{i+1} - 2ru^{k}_{i} + ru^{k}_{i-1} = 2u^{k+1}_{i} - 2u^{k}_{i}
\end{equation}

\begin{equation}
-ru^{k+1}_{i+1} + 2(1 + r)u^{k+1}_{i} - ru^{k+1}_{i-1} = ru^{k}_{i+1} + 2(1 - r)u^{k}_{i} + ru^{k}_{i-1}
\end{equation}

Dalam bentuk matriks:
\begin{equation}
\mathbf{A}\mathbf{u}^{k+1} = \mathbf{B}\mathbf{u}^{k}
\end{equation}

dengan matriks sebagai berikut.

$$
\mathbf{A} =
\begin{bmatrix}
2(1 + r) & -r & 0 & \cdots & 0 & 0 \\
-r & 2(1 + r) & -r & \cdots & 0 & 0 \\
0 & -r & 2(1 + r) & \cdots & 0 & 0 \\
\cdots & \cdots & \cdots & \cdots & \cdots & \cdots \\
0 & 0 & 0 & \cdots & 2(1 + r) & -r \\
0 & 0 & 0 & \cdots & -r & 2(1 + r)
\end{bmatrix}
$$

\end{frame}



\begin{frame}
\frametitle{Metode Crank-Nicolson}

$$
\mathbf{B} =
\begin{bmatrix}
2(1 - r) & r & 0 & \cdots & 0 & 0 \\
r & 2(1 - r) & r & \cdots & 0 & 0 \\
0 & r & 2(1 - r) & \cdots & 0 & 0 \\
\cdots & \cdots & \cdots & \cdots & \cdots & \cdots \\
0 & 0 & 0 & \cdot & 2(1 - r) & r \\
0 & 0 & 0 & \cdot & r & 2(1 - r)
\end{bmatrix}
$$

$$
\mathbf{u}^{k} =
\begin{bmatrix}
u^{k}_{1} \\
u^{k}_{2} \\
u^{k}_{3} \\
\cdot \\
u^{k}_{M-1} \\
u^{k}_{M}
\end{bmatrix}
$$


\end{frame}


\begin{frame}
\frametitle{Metode eksplisit}

\begin{equation}
\frac{\partial u}{\partial t} = \alpha \frac{\partial^2 u}{\partial x^2} + f
\end{equation}
%
pada domain: $x \in (0,L)$ dan $t \in (0,T]$

Syarat batas: $x(x,0) = I(x)$, $x \in [0,L]$

$u(0,t) = 0$ dan $u(L,t) = 0$ pada $t > 0$

Diskritisasi
\begin{equation}
x_{i} = (i-1)\Delta x,\,\,\, i = 1,\ldots,N_{x}
\end{equation}

\begin{equation}
t_{n} = (n-1)\Delta t,\,\,\, n= 1,\ldots,N_{t}
\end{equation}

\end{frame}


\begin{frame}

\begin{equation}
\frac{\partial}{\partial t} u(x_{i}, t_{n}) =
\alpha \frac{\partial^2}{\partial x^2} u(x_{i}, t_{n}) + f(x_{i}, t_{n})
\end{equation}

Forward difference in time and central difference in space:
\begin{equation}
\frac{u^{n+1}_{i} - u^{n}_{i}}{\Delta t} = 
\alpha \frac{u^{n}_{i+1} - 2u^{n}_{i} + u^{n}_{i-1}}{(\Delta x)^2} + f^{n}_{i}
\end{equation}

\begin{equation*}
u^{n+1}_{i} = u^{n}_{i} +
F \left(
u^{n}_{i+1} - 2u^{n}_{i} + u^{n}_{i-1}
\right) + f^{n}_{i} \Delta t
\end{equation*}

\begin{equation}
F = \alpha \frac{\Delta t}{(\Delta x)^2}
\end{equation}

\end{frame}



\begin{frame}
\frametitle{Aturan $\theta$}

\begin{equation}
\frac{\partial u}{\partial t} = G(u)
\end{equation}
di mana $G(u)$ adalah suatu operator diferensial spasial

\begin{equation}
\frac{u_{i}^{n+1} - u_{i}^{n}}{\Delta t} = \theta G(u_{i}^{n+1}) + (1 - \theta) G(u_{i}^{n})
\end{equation}

Untuk persamaan difusi:
\begin{multline}
\frac{u_{i}^{n+1} - u_{i}^{n}}{\Delta t} = \\
\alpha \left(
\theta \frac{u_{i+1}^{n+1} - 2u_{i}^{n+1} + u_{i-1}^{n+1}}{(\Delta x)^2} +
(1 - \theta) \frac{u_{i+1}^{n} - 2u_{i}^{n} + u_{i-1}^{n}}{(\Delta x)^2}
\right) +
\theta f_{i}^{n+1} + (1 - \theta)f_{i}^{n}
\end{multline}

Sistem matriks:
\begin{equation}
A_{i,i-1} = -F\theta, \,\, A_{i,i} = 1 + 2F\theta, \,\, A_{i,i+1} = -F\theta
\end{equation}



\end{frame}




\end{document}
