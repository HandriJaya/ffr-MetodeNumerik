\documentclass[10pt,fleqn,aspectratio=169]{beamer}
%\documentclass[fleqn,aspectratio=169]{beamer}

\usepackage{amsmath, amssymb}
\usepackage{fancyvrb, color, graphicx, hyperref, url}

%\setbeamersize{text margin left=5pt, text margin right=5pt}

\setlength{\parskip}{\smallskipamount}
\setlength{\parindent}{0pt}

\usepackage{fontspec}
%\setmonofont{DejaVu Sans Mono}
\setmonofont{JuliaMono-Regular}

\usefonttheme[onlymath]{serif}

\usepackage{minted}
\newminted{python}{breaklines,fontsize=\footnotesize}
\newminted{julia}{breaklines,fontsize=\footnotesize}
\newminted{bash}{breaklines,fontsize=\footnotesize}
\newminted{text}{breaklines,fontsize=\footnotesize}

\newcommand{\txtinline}[1]{\mintinline[breaklines,fontsize=\footnotesize]{text}{#1}}
\newcommand{\pyinline}[1]{\mintinline[breaklines,fontsize=\footnotesize]{python}{#1}}
\newcommand{\jlinline}[1]{\mintinline[breaklines,fontsize=\footnotesize]{julia}{#1}}

\definecolor{mintedbg}{rgb}{0.95,0.95,0.95}
\usepackage{mdframed}

\BeforeBeginEnvironment{minted}{\begin{mdframed}[backgroundcolor=mintedbg,%
  rightline=false,leftline=false,topline=false,bottomline=false]}
\AfterEndEnvironment{minted}{\end{mdframed}}

% https://tex.stackexchange.com/questions/33969/changing-font-size-of-selected-slides-in-beamer

\usepackage{environ}
%
% Custom font for a frame.
%
\newcommand{\customframefont}[1]{
  \setbeamertemplate{itemize/enumerate body begin}{#1}
  \setbeamertemplate{itemize/enumerate subbody begin}{#1}
}

\NewEnviron{framefont}[1]{
  \customframefont{#1} % for itemize/enumerate
  {#1 % For the text outside itemize/enumerate
    \BODY
  }
  \customframefont{\normalsize}
}



\begin{document}

\title{Metode Numerik}
\subtitle{Persamaan Diferensial Biasa}
\author{Fadjar Fathurrahman}
\institute{
Teknik Fisika \\
Institut Teknologi Bandung
}
\date{}


\frame{\titlepage}


\begin{frame}
\frametitle{Contoh}

Persamaan gerak Newton:
\begin{equation*}
\frac{\mathrm{d}v}{\mathrm{d}t} = \frac{F}{m}
\end{equation*}

Persamaan kalor Fourier:
\begin{equation*}
q = -k \frac{\mathrm{d}T}{\mathrm{d}x}
\end{equation*}

Hukum difusi Fick:
\begin{equation*}
J = -D \frac{\mathrm{d}c}{\mathrm{d}x}
\end{equation*}

Hukum Faraday:
\begin{equation*}
\Delta V_{L} = L \frac{\mathrm{d}i}{\mathrm{d}t}
\end{equation*}

\end{frame}


% ----------------------------------------------------
\begin{frame}
\frametitle{Bentuk umum permasalahan nilai awal}

Permasalahan nilai awal: \textit{initial value problem} (IVP)

Selesaikan persamaan diferensial
\begin{equation}
\frac{\mathrm{d}y}{\mathrm{d}x} = f(x,y)
\label{eq:ode_gen01}
\end{equation}
diberikan syarat awal: $y(x=x_0) = y_0$.

$y(x)$ adalah variabel dependen (atau fungsi yang ingin dicari)
dan $x$ adalah variabel independen.

Untuk masalah nilai awal, biasanya lebih sering menggunakan $t$ sebagai
variabel independen:
\begin{equation}
\frac{\mathrm{d}y}{\mathrm{d}t} = f(t,y)
\label{eq:ode_gen02}
\end{equation}

\end{frame}

% -----------------------

\begin{frame}
\frametitle{Metode-metode untuk IVP}
  
\begin{itemize}
\item Metode satu-langkah
\item Metode multi-langkah
\item Metode implisit
\end{itemize}

\end{frame}



% -----------------------

\begin{frame}
\frametitle{Metode satu langkah}

Bentuk umum:
\begin{equation*}
y_{i+1} = y_{i} + \phi h
\end{equation*}

$y_{i}$: nilai lama

$y_{i+1}$: nilai baru

$\phi$: estimasi kemiringan (\textit{slope})

\end{frame}


\begin{frame}
\frametitle{Metode Euler}

\begin{columns}

\begin{column}{0.5\textwidth}
  {\centering
  \includegraphics[width=0.9\textwidth]{../../images_priv/Chapra_Fig_25_1.pdf}
  \par}    
\end{column}

\begin{column}{0.5\textwidth}
  Menggunakan turunan pertama pada $x_i$ sebagai estimasi langsung dari kemiringan $\phi$:
  \begin{equation*}
  \phi = f(x_i, y_i)
  \end{equation*}
  sehingga diperoleh:
  \begin{equation*}
  y_{i+1} = y_{i} + f(x_i, y_i) h
  \end{equation*}
\end{column}

\end{columns}

\end{frame}


% ------------------------------------

\begin{frame}
\frametitle{Contoh}
Cari solusi numerik dari IVP
\begin{equation*}
\frac{\mathrm{d}y}{\mathrm{d}x} = -2x^3 + 12x^2 - 20x + 8.5
\end{equation*}
dari $x = 0$ sampai $x = 4$ dengan ukuran langkah $h=0.5$.
Syarat awal: $y(x=0) = 1$.

Dengan membandingkan bentuk umum dari ODE \eqref{eq:ode_gen01}, diperoleh:
\begin{equation*}
f(x,y) = -2x^3 + 12x^2 - 20x + 8.5
\end{equation*}
Perhatikan bahwa pada kasus pada contoh ini, fungsi $f(x,y)$
tidak bergantung dari $y$.

Bandingkan solusi numerik yang diperoleh dengan solusi eksak:
\begin{equation*}
y(x) = -0.5x^4 + 4x^3 - 10x^2 + 8.5x + 1
\end{equation*}

\end{frame}



\begin{frame}[fragile]
\frametitle{Implementasi}

Pertama, kita akan mendefinisikan beberapa fungsi yang akan digunakan nantinya.

Pertama, kita akan mendefinisikan fungsi untuk aplikasi metode Euler satu langkah:
\begin{equation*}
y_{i+1} = y_{i} + f(x_i, y_i) h
\end{equation*}
dengan menggunakan fungsi berikut:
\begin{pythoncode}
def ode_euler_1step(dfunc, xi, yi, h):
    return yi + dfunc(xi,yi)*h
\end{pythoncode}

Fungsi yang dilemparkan \pyinline{dfunc} mengimplementasikan fungsi pada ruas kanan,
yaitu $f(x,y)$. Fungsi ini harus diimplementasikan dalam bentuk sebagai berikut.
\begin{pythoncode}
def dfunc(x,y):
    return ... # implement f(x,y)
\end{pythoncode}

Kita dapat memanggil \pyinline{ode_euler_1step} beberapa kali dari kondisi awal
sampai jumlah langkah yang diperlukan.

\end{frame}


\begin{frame}[fragile]
Potongan kode yang diberikan sebelumnya cukup umum. Sekarang kita akan
memberikan kode untuk kasus yang diberikan pada contoh, yaitu:
\begin{equation*}
\frac{\mathrm{d}y}{\mathrm{d}x} = -2x^3 + 12x^2 - 20x + 8.5
\end{equation*}

Fungsi berikut ini mendefinisikan fungsi pada ruas kanan $f(x,y)$
\begin{pythoncode}
def deriv(x, y):
    return -2*x**3 + 12*x**2 - 20*x + 8.5
\end{pythoncode}
Solusi eksak didefinisikan pada fungsi berikut.
\begin{pythoncode}
def exact_sol(x):
    return -0.5*x**4 + 4*x**3 - 10*x**2 + 8.5*x + 1
\end{pythoncode}
\end{frame}


\begin{frame}[fragile]

Pengujian \pyinline{ode_euler_1step}:
\begin{pythoncode}
x0 = 0.0; y0 = 1.0 # Initial condition
x = x0; y = y0; h = 0.5 # Using step of 0.5, starting from x0 and y0
xp1 = x + h # we are searching for solution at x = 0.5, increment by step size
yp1 = ode_euler_1step(deriv, x, y, h)
y_true = exact_sol(xp1)
ε_t = (y_true - yp1)/y_true * 100 # relative error in percent
print("First step : x = %f y_true = %.5f y = %.5f ε_t = %.1f %%" %
  (xp1, y_true, yp1, ε_t))
\end{pythoncode}

Tugas: lanjutkan sampai langkah-langkah selanjutnya.
Anda dapat menggunakan loop.

Contoh keluaran (dua langkah):
\begin{textcode}
First step : x = 0.500000 y_true = 3.21875 y = 5.25000 ε_t = -63.1 %
Second step: x = 1.000000 y_true = 3.00000 y = 5.87500 ε_t = -95.8 %
\end{textcode}

\end{frame}



\begin{frame}[fragile]

Contoh pemanggilan \pyinline{ode_euler_1step} beberapa kali.
\begin{pythoncode}
x = x0; y = y0; h = 0.5
for i in range(0,Nsteps):
    xp1 = x + h
    yp1 = ode_euler_1step(deriv, x, y, h)
    y_true = exact_sol(xp1) # calculate exact solution if available
    # relative error in percent, you can use other
    ε_t = (y_true - yp1)/y_true * 100
    # print the results ...
    # Update x and y for the next step
    x = xp1; y = yp1
\end{pythoncode}

Tugas: simpan hasil kalkulasi, yaitu \pyinline{x} dan
\pyinline{y} ke array dan buat plot.

Anda juga dapat mengimplementasikan ini dalam suatu fungsi.

\end{frame}



\end{document}
